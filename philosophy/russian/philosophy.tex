% !TeX spellcheck = ru_RU
\documentclass[a4paper, 14pt]{extarticle}

\usepackage{fancyhdr}
\usepackage{lastpage}

\usepackage[utf8]{inputenc}
\usepackage{polyglossia}
\setmainlanguage{russian}
\setotherlanguage{english}
\usepackage[htt]{hyphenat}
\usepackage{setspace}
\usepackage[hidelinks]{hyperref}
\setlength{\emergencystretch}{30pt}

\setmainfont{Liberation Serif}
\newfontfamily\cyrillicfont{Liberation Serif}
\let\cyrillicfonttt\ttfamily
\setmonofont{Liberation Mono}
\setsansfont{Liberation Sans}

\pagestyle{fancy}
\fancyhead[]{}
\fancyhead[L]{\textmd{Предварительная версия 1}}
\fancyhead[C]{}
\fancyhead[R]{\textmd{Страница \thepage{}-я из \pageref{LastPage}}}
\fancyfoot[]{}
\renewcommand{\headrulewidth}{0pt}

\usepackage{indentfirst}
\setlength{\parindent}{1.25cm}

\usepackage{enumitem}
\newenvironment{icItems}
{ \begin{itemize} [noitemsep,nolistsep] }
	{ \end{itemize} }

\begin{document}

\title{Философия icL (Предварительная версия)}
\author{Лелицак Василе}

\maketitle 

\pagebreak
\tableofcontents

\pagebreak
\section{Ведение}

Философия icL определяет чёткий курс развития проекта. Она предназначена для решения возможных конфликтов. Сделать изменений противоречивы философии icL строго запрещено. 

Главная задачи философии icL - сохранить ту искру, которая появилось в 2016 году и защитить проект от неправильных решениях.

\

Философия icL определяет:
\begin{icItems}
	\item какой функционалом должен присутствовать в инструментах тестировании, а какой нет;
	\item каким должен быть язык описания сценариев тестирования;
	\item процесс создания и редактирования скриптов;
	\item каким должен быть код icL.
\end{icItems}

% !TeX spellcheck = ru_RU
\section{Функционал}

\paragraph{Свобода выбора}

Пользователь имеет право выбирать на каком браузере будет тестировать своё приложение.

\paragraph{Всё и сразу}

Выполнить всё тесты используя каждый браузер не должно составить трудности.

\paragraph{Время}

Инструмент тестирования должен сохранить время пользователя если это возможно.

\paragraph{Работа}

Если компьютер не справляется с задачи, подключения других компьютеров к выполнению задачи не должно составить трудность.

\paragraph{Единое ПО}

Для решения всех задач тестирование достаточно установить и настроить одну программу.

\paragraph{Интеграции}

Инструмент может быть интегрирован с другими решениями только если они не дублирует его функционал.

\paragraph{Протокол}

Предпочитается использовать протокол WebSocket, а не HTTP.

\paragraph{Программирование}

Знания языков программирования является желательным фактором, но не обязательным.

\paragraph{Текст}

Описания всех команд в текстовом виде, а также изменения файл настроек вручную имеет предпочтения над графическим пользовательским интерфейсом.

\paragraph{Автоматизации}

Если процесс может быть автоматизирован, он должен быть автоматизирован.

% !TeX spellcheck = ru_RU
\section{Язык}

\paragraph{Синтаксис}

Синтаксис должен быть простым и привычным, то есть C-подобный.

\paragraph{Переменные}

Переменные должны чётка делится на локальные и глобальные.

\paragraph{ООП}

В языке могут присутствовать классы, но определить их нельзя.

\paragraph{Функции}

Т.к. классы определить нельзя, функция является единственной исполнительной пользовательской единицы.

\paragraph{Названия}

Названий встроенных классов/методов должны быть максимально короткими, но при этом максимально понятным.

\paragraph{Errorless}

Код не должен генерировать слишком много исключений, но при этом не должен допускать абсурдные выражения.

\paragraph{Преобразование}

Любое преобразование выполняется через оператор преобразования, а не через функции.

\paragraph{Скобки}

Пропуск скобок при вложении выражений является нормальной практике.

\paragraph{Разделитель}

Разделитель команд можно пропустить, разделитель значений нет.

\paragraph{Интеграции}

Код на других языков (JavaScript и SQL) интегрируется как часть исходного кода, а не как строки.
% !TeX spellcheck = ru_RU
\section{Среда разработки}

\paragraph{Языки}

Подсветка синтаксиса должна быть поддержана для всех языков использованных в сценариях.

\paragraph{Автозаполнение}

Автозаполнение должна предлагать не слова и идентификаторы, а данные и конструкции языка с объяснениями.

\paragraph{Стиль}

Поддержка кода в едином стиле - задача среды разработки.

\paragraph{Обновления}

За обновлениями файлами проекта должна следить среда разработки.

\paragraph{Синхронизация}

Набор тестов может быть разработан несколькими пользователи одновременно.

\paragraph{Блокировки}

Один тест в одно и то же время может быть редактирован только одним пользователем.

\paragraph{Контроль}

Среда разработки не должна допускать выполнение кода, который не прошел статический анализ успешно.

\paragraph{Успех}

Пользователь должен иметь возможность убедится в успех прохождения теста без его полное выполнение.

\paragraph{Гибкость}

Код может быть изменён во время его выполнении.

\paragraph{Устойчивость}

Работа командного процессора может быть полностью восстановлена после сбоя.

\paragraph{Прозрачность}

При необходимости тесты могут быть выполнены сторонним лицом, без затруднений.

% !TeX spellcheck = ru_RU
\section{Код}

\paragraph{Документация}

Все классы и методы должны быть документированы \textit{doxygen}.

\paragraph{OpenSource}

Используется только библиотеки с открытым исходным кодом. Код может быть использован только в решении с открытым исходным кодом.

\paragraph{Автоформатирование}

Весь код форматируется автоматически используя \textit{clang-format}.

\paragraph{Модули}

Весь код компилируется как отдельные разделяемые библиотеки, количество классов в одном модуле не должна превышать 20.

\paragraph{Ветки}

Для каждой изменений в коде создаётся отдельная ветка.

\paragraph{Минимум}

Количество зависимость должна быть минимальной.

\paragraph{Исходники}

Количество строк в каждом исходном файле не должна превышать барьер в 1000 строк.

\paragraph{Функции}

Количество строк в одной функций не должна превышать барьер в 50 строк, для функций отрисовки графических элементов барьер увеличен до 150 строк.

\paragraph{Названия}

Названия переменных/функций/классов/методов должны быть понятными и состоять не больше чем из 4-х слов.


\end{document}