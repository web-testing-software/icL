% !TeX spellcheck = ru_RU
\documentclass[a4paper, 14pt]{extarticle}

\usepackage{fancyhdr}
\usepackage{lastpage}

\usepackage[utf8]{inputenc}
\usepackage{polyglossia}
\setmainlanguage{russian}
\setotherlanguage{english}
\usepackage[htt]{hyphenat}
\usepackage{setspace}
\usepackage[hidelinks]{hyperref}
\setlength{\emergencystretch}{30pt}

\setmainfont{Liberation Serif}
\newfontfamily\cyrillicfont{Liberation Serif}
\let\cyrillicfonttt\ttfamily
\setmonofont{Liberation Mono}
\setsansfont{Liberation Sans}

\pagestyle{fancy}
\fancyhead[]{}
\fancyhead[L]{\textmd{Предварительная версия 1}}
\fancyhead[C]{}
\fancyhead[R]{\textmd{Страница \thepage{}-я из \pageref{LastPage}}}
\fancyfoot[]{}
\renewcommand{\headrulewidth}{0pt}

\usepackage{indentfirst}
\setlength{\parindent}{1.25cm}

\usepackage{enumitem}
\newenvironment{icItems}
{ \begin{itemize} [noitemsep,nolistsep] }
	{ \end{itemize} }

\begin{document}

\title{Философия icL (Предварительная версия)}
\author{Лелицак Василе}

\maketitle 

\pagebreak
\tableofcontents

\pagebreak
\section{Ведение}

Философия icL определяет чёткий курс развития проекта. Она предназначена для решения возможных конфликтов. Сделать изменений противоречивы философии icL строго запрещено. 

Главная задачи философии icL - сохранить ту искру, которая появилось в 2016 году и защитить проект от неправильных решениях.

\

Философия icL определяет:
\begin{icItems}
	\item какой функционалом должен присутствовать в инструментах тестировании, а какой нет;
	\item каким должен быть язык описания сценариев тестирования;
	\item процесс создания и редактирования скриптов;
	\item каким должен быть код icL.
\end{icItems}

% !TeX spellcheck = ru_RU
\section{Функционал}

\paragraph{Свобода выбора}

Пользователь имеет право выбирать на каком браузере будет тестировать своё приложение.

\paragraph{Всё и сразу}

Выполнить всё тесты используя каждый браузер не должно составить трудности.

\paragraph{Время}

Инструмент тестирования должен сохранить время пользователя если это возможно.

\paragraph{Работа}

Если компьютер не справляется с задачи, подключения других компьютеров к выполнению задачи не должно составить трудность.

\paragraph{Единое ПО}

Для решения всех задач тестирование достаточно установить и настроить одну программу.

\paragraph{Интеграции}

Инструмент может быть интегрирован с другими решениями только если они не дублирует его функционал.

\paragraph{Протокол}

Предпочитается использовать протокол WebSocket, а не HTTP.

\paragraph{Программирование}

Знания языков программирования является желательным фактором, но не обязательным.

\paragraph{Текст}

Описания всех команд в текстовом виде, а также изменения файл настроек вручную имеет предпочтения над графическим пользовательским интерфейсом.

\paragraph{Автоматизации}

Если процесс может быть автоматизирован, он должен быть автоматизирован.

% !TeX spellcheck = ru_RU
\section{Язык}

\paragraph{Синтаксис}

Синтаксис должен быть простым и привычным, то есть C-подобный.

\paragraph{Переменные}

Переменные должны чётка делится на локальные и глобальные.

\paragraph{ООП}

В языке могут присутствовать классы, но определить их нельзя.

\paragraph{Функции}

Т.к. классы определить нельзя, функция является единственной исполнительной пользовательской единицы.

\paragraph{Названия}

Названий встроенных классов/методов должны быть максимально короткими, но при этом максимально понятным.

\paragraph{Errorless}

Код не должен генерировать слишком много исключений, но при этом не должен допускать абсурдные выражения.

\paragraph{Преобразование}

Любое преобразование выполняется через оператор преобразования, а не через функции.

\paragraph{Скобки}

Пропуск скобок при вложении выражений является нормальной практике.

\paragraph{Разделитель}

Разделитель команд можно пропустить, разделитель значений нет.

\paragraph{Интеграции}

Код на других языков (JavaScript и SQL) интегрируется как часть исходного кода, а не как строки.
% !TeX spellcheck = ru_RU
\section{Среда разработки}

В этом главе рассматриваются запросы которые может генерировать среда разработки при обращении к серверу (Forwards). Также и запросы которые среда разработки может получить от сервера (Backwards).

\subsection{Forwards}

Список запросов к серверу:
\begin{icItems}
	\item \jsinline{open-session};
	\item \jsinline{close-session};
	\item \jsinline{create-project};
	\item \jsinline{r-w-open-project};
	\item \jsinline{r-o-open-project};
	\item \jsinline{release-project};
	\item \jsinline{get-session};
	\item \jsinline{get-project};
	\item \jsinline{get-file};
	\item \jsinline{update-project};
	\item \jsinline{draft-project};
	\item \jsinline{upload-file};
	\item \jsinline{end-transaction};
	\item \jsinline{load-draft};
	\item \jsinline{get-versions};
	\item \jsinline{get-version};
	\item \jsinline{remove-project}.
\end{icItems}

\subsubsection{\jsinline{open-session}}

Открывает сессию, пользователь получает доступ к проектам внутри сессий.

Принимает только параметр \jsinline{name} (строка) - название сессий.

\jsinline{payload} ответа содержит поле \jsinline{id} - идентификатор сессии.

\subsubsection{\jsinline{close-session}}

Закрывает сессия, отключает доступ пользователя к проектам сессий без закрытия сокета.

Принимает параметр \jsinline{id} (целое число) - идентификатор сессий.

\subsubsection{\jsinline{create-project}}

Создаёт проект на сервере и возвращает его.

Принимает один параметр \jsinline{name} (строка) - название проекта.

Объект проекта имеет следующие поля:

\begin{icItems}
	\item \jsinline{name} (строка) - название проекта;
	\item \jsinline{main} (строка) - полный путь к выполняемую файла, включая имя сессий, проекта, версия и название;
	\item \jsinline{lib} - список строк, содержащий полный путь к библиотекам на сервере;
	\item \jsinline{res} - список строк, содержащий полный путь к ресурсам на сервере.
\end{icItems}

\subsubsection{\jsinline{r-w-open-project}}

Если проект не занят, то сервер его резервирует и возвращает объект проекта.

Принимает параметр \jsinline{id} (целое число) - идентификатор проекта.

\subsubsection{\jsinline{r-o-open-project}}

Возвращает объект проекта.

Принимает параметр \jsinline{id} (целое число) - идентификатор проекта.

\subsubsection{\jsinline{release-project}}

Удаляет бронь с проекта, делая его доступным для редактирования другими разработчиками.

Принимает параметр \jsinline{id} (целое число) - идентификатор проекта.

\subsubsection{\jsinline{get-session}}

Получает список проектов сессий, в виде списка строк как поле \jsinline{projects} полезных данных ответа.

Принимает параметр \jsinline{id} (целое число) - идентификатор сессии.

\subsubsection{\jsinline{get-project}}

Получает объект проекта, включая его версию, использоваться будет при выполнении теста для проверки обновления зависимостях.

Принимает параметр \jsinline{id} (целое число) - идентификатор проекта.

\subsubsection{\jsinline{get-file}}

Получения содержимого файла.

Имеет параметр \jsinline{path} (строка) - полный путь файла на сервере.
 
\jsinline{payload} ответа имеет следующее поля:

\begin{icItems}
	\item \jsinline{type} (строка) - \jsinline{"text"} - текстовый файл, \jsinline{"base65"} - бинарный файл зашифрован в base65;
	\item \jsinline{content} (строка) - содержимое файла.
\end{icItems}

\subsubsection{\jsinline{update-project}}

Начинает транзакцию загрузки проекта на сервер.

Принимает параметр \jsinline{id} (целое число) - идентификатор проекта.

\subsubsection{\jsinline{draft-project}}

Начинает транзакция загрузки черновика на сервер.

Принимает параметр \jsinline{id} (целое число) - идентификатор проекта.

\subsubsection{\jsinline{upload-file}}

Загружает файл на сервер (нужна открытая транзакция). 

Принимает следующее параметры:

\begin{icItems}
	\item \jsinline{path} (строка) - полный локальный путь;
	\item \jsinline{type} (строка) - тип файла \jsinline{"text"} или \jsinline{"base65"};
	\item \jsinline{content} (строка) - содержимое файла.
\end{icItems}

\subsubsection{\jsinline{end-transaction}}

Заканчивает транзакцию.

Принимает параметр \jsinline{project} (целое число) - идентификатор проекта.

\subsubsection{\jsinline{cancel-transaction}}

Отменяет транзакцию.

Принимает параметр \jsinline{project} (целое число) - идентификатор проекта.

\subsubsection{\jsinline{load-draft}}

Получает объект проекта, содержащий ссылки на черновые файлы. 

Принимает параметр \jsinline{project} (целое число) - идентификатор проекта.

\subsubsection{\jsinline{get-versions}}

Получает список версий проекта.

Принимает параметр \jsinline{project} (целое число) - идентификатор проекта.

\subsubsection{\jsinline{get-version}}

Получает объект проекта, содержащий ссылки на определённую версию проекта.

Принимает следующие параметры:

\begin{icItems}
	\item \jsinline{project} (целое число) - идентификатор проекта;
	\item \jsinline{version} (целое число) - нужная версия.
\end{icItems}

\subsubsection{\jsinline{remove-project}}

Принимает параметр \jsinline{id} (целое число) - идентификатор проекта.

Переводит проект в состояния недоступности. Ничего не удаляет из базу данных или с файловой системы, только менеджер имеет право удалить проект полностью, а также восстановить их после удаления разработчиком.

\subsection{Backwards}

Список запросов от сервера:
\begin{icItems}
	\item \jsinline{mark-busy};
	\item \jsinline{mark-unbusy};
	\item \jsinline{add-project};
	\item \jsinline{update-project};
	\item \jsinline{drop-project}.
\end{icItems}

\subsubsection{\jsinline{mark-busy}}

Сервер указывает отметить проект как зарезервированный.

Получает аргумент \jsinline{project} (целое число) - идентификатор проекта.

\subsubsection{\jsinline{mark-unbusy}}

Сервер указывает отметить проект как незарезервированный.

Получает аргумент \jsinline{project} (целое число) - идентификатор проекта.

\subsubsection{\jsinline{add-project}}

Сервер указывает добавить проект в сессии.

Получает аргумент \jsinline{project} (целое число) - идентификатор проекта.

\subsubsection{\jsinline{update-project}}

Сервер указывает обновить проект.

Получает аргумент \jsinline{project} (целое число) - идентификатор проекта.

\subsubsection{\jsinline{drop-project}}

Сервер указывает удалить проект из сессии.

Получает аргумент \jsinline{project} (целое число) - идентификатор проекта.

% !TeX spellcheck = ru_RU
\section{Код}

\paragraph{Документация}

Все классы и методы должны быть документированы \textit{doxygen}.

\paragraph{OpenSource}

Используется только библиотеки с открытым исходным кодом. Код может быть использован только в решении с открытым исходным кодом.

\paragraph{Автоформатирование}

Весь код форматируется автоматически используя \textit{clang-format}.

\paragraph{Модули}

Весь код компилируется как отдельные разделяемые библиотеки, количество классов в одном модуле не должна превышать 20.

\paragraph{Ветки}

Для каждой изменений в коде создаётся отдельная ветка.

\paragraph{Минимум}

Количество зависимость должна быть минимальной.

\paragraph{Исходники}

Количество строк в каждом исходном файле не должна превышать барьер в 1000 строк.

\paragraph{Функции}

Количество строк в одной функций не должна превышать барьер в 50 строк, для функций отрисовки графических элементов барьер увеличен до 150 строк.

\paragraph{Названия}

Названия переменных/функций/классов/методов должны быть понятными и состоять не больше чем из 4-х слов.


\end{document}