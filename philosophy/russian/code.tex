% !TeX spellcheck = ru_RU
\section{Код}

\paragraph{Документация}

Все классы и методы должны быть документированы \textit{doxygen}.

\paragraph{OpenSource}

Используется только библиотеки с открытым исходным кодом. Код может быть использован только в решении с открытым исходным кодом.

\paragraph{Автоформатирование}

Весь код форматируется автоматически используя \textit{clang-format}.

\paragraph{Модули}

Весь код компилируется как отдельные разделяемые библиотеки, количество классов в одном модуле не должна превышать 20.

\paragraph{Ветки}

Для каждой изменений в коде создаётся отдельная ветка.

\paragraph{Минимум}

Количество зависимость должна быть минимальной.

\paragraph{Исходники}

Количество строк в каждом исходном файле не должна превышать барьер в 1000 строк.

\paragraph{Функции}

Количество строк в одной функций не должна превышать барьер в 50 строк, для функций отрисовки графических элементов барьер увеличен до 150 строк.

\paragraph{Названия}

Названия переменных/функций/классов/методов должны быть понятными и состоять не больше чем из 4-х слов.
