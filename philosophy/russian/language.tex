% !TeX spellcheck = ru_RU
\section{Язык}

\paragraph{Синтаксис}

Синтаксис должен быть простым и привычным, то есть C-подобный.

\paragraph{Переменные}

Переменные должны чётка делится на локальные и глобальные.

\paragraph{ООП}

В языке могут присутствовать классы, но определить их нельзя.

\paragraph{Функции}

Т.к. классы определить нельзя, функция является единственной исполнительной пользовательской единицы.

\paragraph{Названия}

Названий встроенных классов/методов должны быть максимально короткими, но при этом максимально понятным.

\paragraph{Errorless}

Код не должен генерировать слишком много исключений, но при этом не должен допускать абсурдные выражения.

\paragraph{Преобразование}

Любое преобразование выполняется через оператор преобразования, а не через функции.

\paragraph{Скобки}

Пропуск скобок при вложении выражений является нормальной практике.

\paragraph{Разделитель}

Разделитель команд можно пропустить, разделитель значений нет.

\paragraph{Интеграции}

Код на других языков (JavaScript и SQL) интегрируется как часть исходного кода, а не как строки.