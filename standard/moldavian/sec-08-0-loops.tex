% !TeX spellcheck = ro_RO
\section{Cicluri}

Poate să apară necesitatea de a folosi unul și același grup de comenzi de mai multe ori. Nu e numaidecît de scris același cod cîteva ori, e mai simplu a folosi un ciclu.

În limbajul icL sunt prezente următoarele tipuri de cicluri:

\begin{icItems}
\item
	\code{for} - ciclu universal;
\item
	\code{while} - repetarea condiționată a codului;
\item
	\code{do while} - ciclu cu condiție posterioară;
\item
	\code{for} - răsfoirea colecției;
\item
	\code{filter} - răsfoirea selectivă colecției;
\item
	\code{range} - răsfoirea parțială colecției.
\end{icItems}


\subsubsection{\lstinline`for` - ciclu universal}

{\bf Ciclul universal} permite a controla fiecare delatiu al executării. El conține:

\begin{icItems}
\item
	\code{initialization} - inițializare.
\item
	\code{condition} - condiție, care se controlează înainte de executare;
\item
	\code{step} - cod, care răspunde de trecerea la următoarea iterație.
\item
	\code{commands} - codul care fi executat repetat.
\end{icItems}

Ciclul universal are următoarea sintaxă -
\begin{lstlisting}[numbers=none]
for (initialization, condition, step) {
	commands
};
\end{lstlisting}

Pe foaia \ref{uniloopex} este prezentat un exemplu de folosire al ciclului \code{for}, ciclu va printa în consolă numere de la zero la patru.

\begin{lstlisting}[caption=Ciclu universal, label=uniloopex]
for (@i = 0, i < 5, @i = @i + 1) {
	_log.out @i;
};
\end{lstlisting}

\subsubsection{\lstinline`while` - repetare condiționată a codului}

{\bf Repetarea condiționată a codului} va executa codul pînă cînd condiția nu va deveni falsă. Condiția se verifică înainte de a executa ciclul, dacă este inițial falsă, ciclul nu va fi executat niciodată.

Avantajul lui principal este simplitatea. În comparație cu ciclul universal el nu are inițializare și trece la următoarea iterație. El are doar condiție și grup de comenzi -
\begin{lstlisting}[numbers=none]
while (condition) {
	commands
};
\end{lstlisting}

Pe foaia \ref{whileex} este prezentat cod care folosește ciclu \code{while}. Variabila \code{@number} este declarată pînă la ciclu și va fi disponibilă și după ciclu. După finalizarea ciclului variabila \code{@number} va avea valoarea 6.

\begin{lstlisting}[caption=Ciclu while, label=whileex]
@number = 1;

while (@number <= 5) {
	@number = @number + 1;
};
\end{lstlisting}

\subsubsection{\lstinline`do while` - ciclu cu condiție posterioară}

{\bf Ciclu cu condiție posterioară} nu tare se doasebește de \code{while}, dar el garantează că grupul de comenzi va fi executat cel puțin o dată. De asemenea condiția se declară după grupul de comenzi, arătînd la ceea că ea va fi verificată după prima iterație. Sintaxă -
\begin{lstlisting}[numbers=none]
do {
	commands
} while (condition);
\end{lstlisting}

Pe foaia \ref{dowhileex} este prezentat cod care folosește ciclul \code{do while}. Necătînd la aceea că condiția este implicit falsă, o iterație va avea loc, în rezultat variabila \code{@number} va deveni egală cu 7.

\begin{lstlisting}[caption=Ciclu do while, label=dowhileex]
@number = 6;

do {
	@number = @number + 1;
} while (@number < 5);
\end{lstlisting}

\subsubsection{\lstinline`for` - răsfoirea colecție}

{Colecția} este un obiect, care conține mai multe valori. Cel mai simplu exemplu de colecție este lista, ea conține mai multe șiruri de caractere. De exemplu, trebuie de executat o anumită operație cu fiecare șir de caractere din listă. Se poate de folosit ciclul universal (foaia \ref{colluniloop}) sau de {\bf răsfoit colecția} (foaia \ref{collsimple}). Sub numele de colecție se unesc următoarele tipuri de date \code{list}, \code{set} și \code{element}.
\begin{lstlisting}[caption=Răsfoirea colecție prin ciclu universal, label=colluniloop]
@list = ["apple" "banana" "fruit"];

for (@i = 0, @i < @list'length, @i = @i + 1) {
	@list.at @i; `` get i-th value
}
\end{lstlisting}

\begin{lstlisting}[caption=Răsfoirea simplificată a colecției, label=collsimple]
@list = ["apple" "banana" "fruit"];

for (@list) {
	@; `` get i-th value
}
\end{lstlisting}

\subsubsection{\lstinline`filter` - răsfoirea selectivă a colecție}

{\bf Construcția \code{filter}} permite a condiționa, pentru care elemente se va executa ciclu, pentru care nu.

În condiție se poate de folosit următoarele variabile: \lstinline|@| - valoarea elementului colecției și \lstinline|#| - indexul valorii.

Filtrul are următoarea sintaxă -
\begin{lstlisting}[numbers=none]
filter (collection, condition) {
	commands
};
\end{lstlisting}

A filtra elementele se poate pe bază de valoare (foaia \ref{filterdataex}, se filtreză cele care conțin litera a) sau după index (foaia \ref{filterindexex}, se aleg elementele pare a listei).

\begin{lstlisting}[caption=Filtrare pe bază de valoare, label=filterdataex]
@list = ["apple" "banana" "fruit"];

filter (@list, @ << "a") {
	@; `` apple, banana
}
\end{lstlisting}

\subsubsection{\lstinline`range` - răsfoirea selectivă a colecției}

{\bf Răsfoirea selectivă} permite a executa codul pentru un fragment de colecție. Intervalul de definește pe bază de o valoare sau două. Prima condiție se folosește pentru a defini începutul fragmentului, pe cînd a doua pentru sfîrșit, dacă a doua lipsește va fi folosită prima. În lipsă de un interval corespunzător, ciclul nu va fi executat. Pentru a selecta primul sau ultimul element (în cauză de destinația condiției), folosiți expresia \true{}.

Sintaxă cu o singură condiție -
\begin{lstlisting}[numbers=none]
range (collection, condition) {
	commands
};
\end{lstlisting}

Sintaxă cu două condiții (pentru început și sfîrșit de o parte) -
\begin{lstlisting}[numbers=none]
range (collection, condition_for_begin, condition_for_end) {
	commands
};
\end{lstlisting}

\begin{lstlisting}[caption=Filtare după index, label=filterindexex]
@list = ["apple" "banana" "fruit"];

filter (@list, # \ 2 == 0) {
@; `` apple, fruit
}
\end{lstlisting}

A alege intervalul se poate după valoare (foaia \ref{bananakiwiex}, interval de la \code{banana} pînă la \code{kiwi}) sau după index (foaia \ref{seclastex}, de la a doilea pînă la ultimul).

\begin{lstlisting}[caption=Interval: de la banana pîna la kiwi, label=bananakiwiex]
@list = ["apple" "banana" "fruit" "kiwi"];

range (@list, @ == "banana", @ == "kiwi") {
	@; `` banana, fruit, kiwi
}
\end{lstlisting}

\begin{lstlisting}[caption=Intervla: de la al doilea pînă la ultimul, label=seclastex]
@list = ["apple" "banana" "fruit" "kiwi"];

range (@list, # == 2, true) {
	@; `` fruit, kiwi
}
\end{lstlisting}

%\newpage
