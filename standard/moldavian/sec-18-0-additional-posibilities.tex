% !TeX spellcheck = ro_RO
% !TeX spellcheck = ro_RO
\part{Material pentru programatori}

Pentru a înțelege acest material trebuiesc cunoștințe în programare. În ea nu vor fi explicați termini tehnici. Necătînd la aceea că limbajul icL este simplu, în el se poate de realizat scripturi destul de complicate, de lucrat cu baze de date, expresii regulare și multe altele, se așteaptă dezvoltare intensă a acestei părți în următoarele versiuni ale limbajului icL.


\section{Posibilități adiționale}

{\bf Posibilitățile adiționale} sunt prezente în icL pentru acoperirea completă a standardului W3C WebDriver și pentru rezolvarea unor probleme care pot apărea la practică.

Aceste posibilități nu trebuiesc numaidecît, se poate de se descurcat și fără ele. Pentru că la executarea scriptului automat se deschide o nouă sesie și un nou tab, iar la oprire sesia se închide automat.

\subsection{{\color{orange} Sessions}}

Obiectul \sessions{} are următoarele proprietăți:
\begin{icItems}
	\item \mintinline{icl}{[r/o] Sessions'length : int};
	\item \mintinline{icl}{[r/o] Sessions'(i : int) : Session}.
\end{icItems}

Și următoarele metode:
\begin{icItems}
	\item \mintinline{icl}{Sessions.closeAll () : void};
	\item \mintinline{icl}{Sessions.get (i : int) : Session};
	\item \mintinline{icl}{Sessions.new () : Session}.
\end{icItems}

\subsubsection{\mintinline{icl}{[r/o] Sessions'length : int}}

Returnează numărul de sesiuni deschise.

\subsubsection{\mintinline{icl}{[r/o] Sessions'(i : int) : Session}}

Returnează a \mintinline{icl}{i}-a sesiune.

Excepții posibile: \ferror{OutOfBounds} (pr. tab. \ref{errors}).

\subsubsection{\mintinline{icl}{Sessions.closeAll () : void}}

Închide toate sesiunile.

\subsubsection{\mintinline{icl}{Sessions.get (i : int) : Session}}

Returnează a \mintinline{icl}{i}-a sesie.

Excepții posibile: \ferror{OutOfBounds} (pr. tab. \ref{errors}).

\subsubsection{\mintinline{icl}{Sessions.new () : Session}}

Deschide o sesiune nouă.

Excepții posibile: \ferror{SessionNotCreated} (pr. tab. \ref{errors}).

\subsection{{\color{orange} Session}}

Comanda \session{} returnează un link la sesiunea curentă.

Obiectul \session{} are următoarele proprietăți:
\begin{icItems}
	\item \mintinline{icl}{[r/w] Session'implicitTimeout : int};
	\item \mintinline{icl}{[r/w] Session'pageLoadTimeout : int};
	\item \mintinline{icl}{[r/w] Session'scriptTimeout : int};
	\item \mintinline{icl}{[r/o] Session'source : string};
	\item \mintinline{icl}{[r/o] Session'title : string};
	\item \mintinline{icl}{[r/w] Session'url : string}.
\end{icItems}

Și următoarele metode:
\begin{icItems}
	\item \mintinline{icl}{Session.back () : Session};
	\item \mintinline{icl}{Session.close () : void};
	\item \mintinline{icl}{Session.forward () : Session};
	\item \mintinline{icl}{Session.refresh () : Session};
	\item \mintinline{icl}{Session.screenshot () : string};
	\item \mintinline{icl}{Session.switchTo () : Session}.
\end{icItems}

\subsubsection{\mintinline{icl}{[r/w] Session'implicitTimeout : int}}

Tipul în milisecunde de așteptare a găsirii elementului sau al trecerii în stare interactivă. Implicit egal cu 0.

Excepții posibile: \ferror{NoSessions}, \ferror{InvalidArgument} (valoarea pentru instalare este mai mare sau mai mică decît intervalul sigur al numerelor întregi) (pr. tab. \ref{errors}).

\subsubsection{\mintinline{icl}{[r/w] Session'pageLoadTimeout : int}}

Timpul de așteptare al încărcării paginii web. Implicit 300 000 de milisecunde.

Excepții posibile: \ferror{NoSessions}, \ferror{InvalidArgument} (pr. tab. \ref{errors}).

\subsubsection{\mintinline{icl}{[r/w] Session'scriptTimeout : int}}

Timpul de așteptare al executării codului Javascript. Implicit 30 000 de milisecunde.

Excepții posibile: \ferror{NoSessions}, \ferror{InvalidArgument} (pr. tab. \ref{errors}).

\subsubsection{\mintinline{icl}{[r/o] Session'source : string}}

Codul sursă al paginii.

Excepții posibile: \ferror{NoSessions}, \ferror{NoSuchWindow} (pr. tab. \ref{errors}).

\subsubsection{\mintinline{icl}{[r/o] Session'title : string}}

Titlu paginii.

Excepții posibile: \ferror{NoSessions}, \ferror{NoSuchWindow} (pr. tab. \ref{errors}).

\subsubsection{\mintinline{icl}{[r/w] Session'url : string}}

Adresa URL a pagini curente.

Excepții posibile: \ferror{NoSessions}, \ferror{NoSuchWindow} (pr. tab. \ref{errors}).

\subsubsection{\mintinline{icl}{Session.back () : Session}}

Trece la pagina precedentă, cum ar fi să faceți clic pe butonul \textit{Înapoi} în browser.

Excepții posibile: \ferror{NoSessions}, \ferror{NoSuchWindow}, \ferror{Timeout} (pr. tab. \ref{errors}).

\subsubsection{\mintinline{icl}{Session.close () : void}}

Închide sesia. Toate taburile vor fi închise.

Excepții posibile: \ferror{NoSessions}, \ferror{NoSessions}, \ferror{NoSuchSession} (pr. tab. \ref{errors}).

\subsubsection{\mintinline{icl}{Session.forward () : Session}}

Trece la pagina precedentă, cum ar fi să faceți clic pe butonul \textit{Înainte} în browser.

Excepții posibile: \ferror{NoSessions}, \ferror{NoSuchWindow}, \ferror{Timeout} (pr. tab. \ref{errors}).

\subsubsection{\mintinline{icl}{Session.refresh () : Session}}

Reîncarcă pagina.

Excepții posibile: \ferror{NoSessions}, \ferror{NoSuchWindow}, \ferror{Timeout} (pr. tab. \ref{errors}).

\subsubsection{\mintinline{icl}{Session.screenshot () : string}}

Returnează un șir de caracter, care reprezintă captură de ecran codată în base64.

Excepții posibile: \ferror{NoSessions}, \ferror{NoSuchWindow}, \ferror{UnableToCaptureScreen} (pr. tab. \ref{errors}).

\subsubsection{\mintinline{icl}{Session.switchTo () : Session}}

Schimbă focusul la altă sesie.

Excepții posibile: \ferror{NoSuchSession} (pr. tab. \ref{errors}).

%\subsubsection{}

\subsection{{\color{orange} Windows}}

Obiectul \windows{} are următoarele proprietăți:
\begin{icItems}
	\item \mintinline{icl}{[r/o] Windows'length : int};
	\item \mintinline{icl}{[r/o] Windows'(i : int) : Window}.
\end{icItems}

Și următoarea metodă \mintinline{icl}{Windows.get (i : int) : Window}.

\subsubsection{\mintinline{icl}{[r/o] Windows'length : int}}

Returnează numărul de ferestre în sesiunea curentă.

Excepții posibile: \ferror{NoSessions} (pr. tab. \ref{errors}).

\subsubsection{\mintinline{icl}{[r/o] Windows'(i : int) : Window}}

Returnează a \mintinline{icl}{i}-a fereastră.

Excepții posibile: \ferror{NoSessions}, \ferror{OutOfBounds} (pr. tab. \ref{errors}).

\subsubsection{\mintinline{icl}{Windows.get (i : int) : Window}}

Returnează a \mintinline{icl}{i}-a fereastră.

Excepții posibile: \ferror{NoSessions}, \ferror{OutOfBounds} (pr. tab. \ref{errors}).

\subsection{{\color{orange} Window}}

Obiectul \window{} are următoarele proprietăți:
\begin{icItems}
	\item \mintinline{icl}{[r/w] Window'height : int};
	\item \mintinline{icl}{[r/w] Window'width : int};
	\item \mintinline{icl}{[r/w] Window'x : int};
	\item \mintinline{icl}{[r/w] Window'y : int}.
\end{icItems}

Și următoarele metode:
\begin{icItems}
	\item \mintinline{icl}{Window.close () : void};
	\item \mintinline{icl}{Window.focus () : Window};
	\item \mintinline{icl}{Window.fullscreen () : Window};
	\item \mintinline{icl}{Window.maximize () : Window};
	\item \mintinline{icl}{Window.minimize () : Window};
	\item \mintinline{icl}{Window.restore () : Window};
	\item \mintinline{icl}{Window.switchToDefault () : Window};
	\item \mintinline{icl}{Window.switchToFrame (i : int) : Window};
	\item \mintinline{icl}{Window.switchToFrame (el : element) : Window};
	\item \mintinline{icl}{Window.switchToParent () : Window}.
\end{icItems}

\subsubsection{\mintinline{icl}{[r/w] Window'height : int}}

Înălțimea ferestrei curente în pixeli.

Excepții posibile: \ferror{NoSessions}, \ferror{NoSuchWindow}, \ferror{InvalidArgument}, \ferror{UnsupportedOperation} (pr. tab. \ref{errors}).

\subsubsection{\mintinline{icl}{[r/w] Window'width : int}}

Lățimea ferestrei curente în pixeli.

Excepții posibile: \ferror{NoSessions}, \ferror{NoSuchWindow}, \ferror{InvalidArgument}, \ferror{UnsupportedOperation} (pr. tab. \ref{errors}).

\subsubsection{\mintinline{icl}{[r/w] Window'x : int}}

Coordonata $x$ a ferestrei curente în pixeli.

Excepții posibile: \ferror{NoSessions}, \ferror{NoSuchWindow}, \ferror{InvalidArgument}, \ferror{UnsupportedOperation} (pr. tab. \ref{errors}).

\subsubsection{\mintinline{icl}{[r/w] Window'y : int}}

Coordinata $y$ a ferestrei curente în pixeli.

Excepții posibile: \ferror{NoSessions}, \ferror{NoSuchWindow}, \ferror{InvalidArgument}, \ferror{UnsupportedOperation} (pr. tab. \ref{errors}).

\subsubsection{\mintinline{icl}{Window.close () : void}}

Închide fereastra curentă, dacă ea este ultima se închide și sesia.

Excepții posibile: \ferror{NoSessions}, \ferror{NoSuchWindow} (pr. tab. \ref{errors}).

\subsubsection{\mintinline{icl}{Window.focus () : Window}}

Schimbă focusul la fereastra aceasta.

Excepții posibile: \ferror{NoSessions}, \ferror{NoSuchWindow} (pr. tab. \ref{errors}).

\subsubsection{\mintinline{icl}{Window.fullscreen () : Window}}

Schimbă regimul de afișare a ferestrei browserului la \textit{ecran complet}.

Excepții posibile: \ferror{NoSessions}, \ferror{NoSuchWindow} (pr. tab. \ref{errors}).

\subsubsection{\mintinline{icl}{Window.maximize () : Window}}

Maximizează fereastra browserului.

Excepții posibile: \ferror{NoSessions}, \ferror{NoSuchWindow} (pr. tab. \ref{errors}).

\subsubsection{\mintinline{icl}{Window.minimize () : Window}}

Minimizează fereastra browserului.

Excepții posibile: \ferror{NoSessions}, \ferror{NoSuchWindow} (pr. tab. \ref{errors}).

\subsubsection{\mintinline{icl}{Window.restore () : Window}}

Restaurează fereastra în regim normal.

Excepții posibile: \ferror{NoSessions}, \ferror{NoSuchWindow} (pr. tab. \ref{errors}).

\subsubsection{\mintinline{icl}{Window.switchToDefault () : Window}}

Schimbă focusul la frame-ul principal.

Excepții posibile: \ferror{NoSessions}, \ferror{NoSuchWindow} (pr. tab. \ref{errors}).

\subsubsection{\mintinline{icl}{Window.switchToFrame (i : int) : Window}}

Schimbă focusul la frame-ul cu indexul \mintinline{icl}{i}.

Excepții posibile: \ferror{NoSessions}, \ferror{NoSuchWindow}, \ferror{NoSuchFrame} (pr. tab. \ref{errors}).

\subsubsection{\mintinline{icl}{Window.switchToFrame (el : element) : Window}}

Schimbă focusul la frame-ul elementului \mintinline{icl}{el}. Elementul trebuie să fir numaidecît tag frame sau iframe.

Excepții posibile: \ferror{NoSessions}, \ferror{NoSuchWindow}, \ferror{NoSuchFrame}, \ferror{StaleElementReference} (pr. tab. \ref{errors}).

\subsubsection{\mintinline{icl}{Window.switchToParent () : Window}}

Schimbă focusul la frame-ul ascensor.

Excepții posibile: \ferror{NoSessions}, \ferror{NoSuchWindow} (pr. tab. \ref{errors}).

\subsection{{\color{orange} Cookies}}

Obiectul \cookies{} are următoarea proprietate: \mintinline{icl}{[r/o] Cookies'(name : string) :} \\*\mintinline{icl}{Cookie}.

Și următoarele metode: 
\begin{icItems}
	\item \mintinline{icl}{Cookies.deleteAll () : void};
	\item \mintinline{icl}{Cookies.get (name : string) : Cookie}.
\end{icItems}

\subsubsection{\mintinline{icl}{[r/o] Cookies'(name : string) : Cookie}}

Returnează \cookie{} cu numele \mintinline{icl}{name}.

Excepții posibile: \ferror{NoSessions}, \ferror{NoSuchWindow}, \ferror{NoSuchCookie} (pr. tab. \ref{errors}).

\subsubsection{\mintinline{icl}{Cookies.deleteAll : void}}

Șterge toate fișierele cookie.

\subsubsection{\mintinline{icl}{Cookies.get (name : string) : Cookie}}

Returnează \cookie{} cu numele \mintinline{icl}{name}.

Excepții posibile: \ferror{NoSessions}, \ferror{NoSuchWindow}, \ferror{NoSuchCookie} (pr. tab. \ref{errors}).

\subsection{{\color{orange} Cookie}}

Obiectul \cookie{} are următoarele proprietăți:
\begin{icItems}
	\item \mintinline{icl}{[r/w] Cookie'domain : string};
	\item \mintinline{icl}{[r/w] Cookie'expiry : int};
	\item \mintinline{icl}{[r/w] Cookie'httpOnly : bool};
	\item \mintinline{icl}{[r/w] Cookie'name : string};
	\item \mintinline{icl}{[r/w] Cookie'path : string};
	\item \mintinline{icl}{[r/w] Cookie'secure : bool};
	\item \mintinline{icl}{[r/w] Cookie'value : string}.
\end{icItems}

Și următoarele metode:
\begin{icItems}
	\item \mintinline{icl}{Cookie.add (years : int, months : int, days : int, hours = 0, minutes = 0,}\\* \mintinline{icl}{seconds = 0) : Cookie};
	\item \mintinline{icl}{Cookie.load () : Cookie};
	\item \mintinline{icl}{Cookie.resetTime () : Cookie};
	\item \mintinline{icl}{Cookie.save () : Cookie};
	\item \mintinline{icl}{Cookie.delete () : void}.
\end{icItems}

\subsubsection{\mintinline{icl}{[r/w] Cookie'domain : string}}

Numele de domeniu, pe care \cookie{} este disponibil.

\subsubsection{\mintinline{icl}{[r/w] Cookie'expiry : int}}

Timpul de expirare al fișierului cookie. Implicit -1, indică că fișierul cookie va fi șters la sfîrșitul sesiei.

\subsubsection{\mintinline{icl}{[r/w] Cookie'httpOnly : bool}}

Doar pentru protocolul HTTP. Implicit \false.

\subsubsection{\mintinline{icl}{[r/w] Cookie'name : string}}

Denumire, obligatorie pentru completare.

\subsubsection{\mintinline{icl}{[r/w] Cookie'path : string}}

Calea pe care fișierul cookie este diponibil, implicit \mintinline{icl}{"/"}.

\subsubsection{\mintinline{icl}{[r/w] Cookie'secure : bool}}

Securizat, implicit \false.

\subsubsection{\mintinline{icl}{[r/w] Cookie'value : string}}

Valoarea fișierului cookie, obligatorie pentru completare.

\subsubsection{\mintinline{icl}{Cookie.add (years : int, months : int, days : int, hours = 0, minutes = 0}\\*\noindent\mintinline{icl}{seconds = 0) : Cookie}}

Adaugă la timpul de expirare numărul necesar de ani, luni, zile, ore, minute și secunde.

\subsubsection{\mintinline{icl}{Cookie.load () : Cookie}}

Încarcă datele despre \cookie{} din browser.

Excepții posibile: \ferror{NoSessions}, \ferror{NoSuchWindow}, \ferror{NoSuchCookie} (pr. tab. \ref{errors}).

\subsubsection{\mintinline{icl}{Cookie.resetTime () : Cookie}}

Setează timpul de expirare la timpul curent. De exemplu pentru ca fișierul să expire peste un an se folosește comanda următoare: \mintinline{icl}{Cookie.resetTime().add(1, 0, 0)}.

\subsubsection{\mintinline{icl}{Cookie.save () : Cookie}}

Transmite schimbările fișierului cookie în browser.

Excepții posibile: \ferror{NoSessions}, \ferror{NoSuchWindow}, \ferror{InvalidArgument}, \ferror{UnableToSetCookie}, \ferror{InvalidCookieDomain} (pr. tab. \ref{errors}).

\subsubsection{\mintinline{icl}{Cookie.delete () : void}}

Șterge fișierul cookie din browser.

Excepții posibile: \ferror{NoSessions}, \ferror{NoSuchWindow}.

\subsubsection{Crearea noilor fișiere cookie}

Pe foaia \ref{newcookies}, este prezentată metoda corectă de a crea fișiere cookie noi.

\begin{sourcecode}
    \captionof{listing}{Crearea noilor fișiere cookie}
    \label{newcookies}
    \inputminted[linenos]{icl}{../sources/newcookies.icL}
\end{sourcecode}


\subsection{{\color{orange} Alert}}

Obiectul \alert{} are următoarea proprietate: \mintinline{icl}{[r/o] Alert'text : string}.

Și următoarele metode:
\begin{icItems}
	\item \mintinline{icl}{Alert.accept () : void};
	\item \mintinline{icl}{Alert.dismiss () : void};
	\item \mintinline{icl}{Alert.sendKeys (keys : string) : void}.
\end{icItems}

\subsubsection{\mintinline{icl}{[r/o] Alert'text : string}}

Returnează textul avertizării.

Excepții posibile: \ferror{NoSessions}, \ferror{NoSuchWindow}, \ferror{NoSuchAlert} (pr. tab. \ref{errors}).

\subsubsection{\mintinline{icl}{Alert.accept () : void}}

Acceptă avertizarea.

Excepții posibile: \ferror{NoSessions}, \ferror{NoSuchWindow}, \ferror{NoSuchAlert} (pr. tab. \ref{errors}).

\subsubsection{\mintinline{icl}{Alert.dismiss () : void}}

Refuză avertizarea.

Excepții posibile: \ferror{NoSessions}, \ferror{NoSuchWindow}, \ferror{NoSuchAlert} (pr. tab. \ref{errors}).

\subsubsection{\mintinline{icl}{Alert.sendKeys (keys : string) : void}}

Completează formularul cu textul \mintinline{icl}{keys} și confirmă inserarea.

Excepții posibile: \ferror{NoSessions}, \ferror{NoSuchWindow}, \ferror{NoSuchAlert}, \ferror{ElementNotInteractable} (pr. tab. \ref{errors}).

\subsection{{\color{orange} Tabs}}

Obiectul \tabs{} are următoarele proprietăți:
\begin{icItems}
	\item \mintinline{icl}{[r/o] Tabs'current : Tab};
	\item \mintinline{icl}{[r/o] Tabs'first : Tab};
	\item \mintinline{icl}{[r/o] Tabs'last : Tab};
	\item \mintinline{icl}{[r/o] Tabs'length : int};
	\item \mintinline{icl}{[r/o] Tabs'next : Tab};
	\item \mintinline{icl}{[r/o] Tabs'previous : Tab};
	\item \mintinline{icl}{[r/o] Tabs'(i : int) : Tab}.
	% \item \mintinline{icl}{Tabs'};
\end{icItems}

Și următoarele metode:
\begin{icItems}
	\item \mintinline{icl}{Tabs.close (template : string) : int};
	\item \mintinline{icl}{Tabs.close (url : regex) : int};
	\item \mintinline{icl}{Tabs.closeByTitle (template : string) : int};
	\item \mintinline{icl}{Tabs.closeByTitle (title : regex) : int};
	\item \mintinline{icl}{Tabs.closeOthers () : int};
	\item \mintinline{icl}{Tabs.closeToLeft () : int};
	\item \mintinline{icl}{Tabs.closeToRight () : int};
	\item \mintinline{icl}{Tabs.find (template : string) : Tab};
	\item \mintinline{icl}{Tabs.find (url : regex) : Tab};
	\item \mintinline{icl}{Tabs.findByTitle (template : string) : Tab};
	\item \mintinline{icl}{Tabs.findByTitle (title : regex) : Tab}.
	% \item \mintinline{icl}{Tabs.};
\end{icItems}

În regimul de testare taburile vor fi în ordine aleatorie. În regim de automatizare în ordine strictă.

\subsubsection{\mintinline{icl}{[r/o] Tabs'current : Tab}}

Tabul curent.

Excepții posibile: \ferror{NoSessions} (pr. tab. \ref{errors}).

\subsubsection{\mintinline{icl}{[r/o] Tabs'first : Tab}}

Primul tab.

Excepții posibile: \ferror{NoSessions} (pr. tab. \ref{errors}).

\subsubsection{\mintinline{icl}{[r/o] Tabs'last : Tab}}

Ultimul tab.

Excepții posibile: \ferror{NoSessions} (pr. tab. \ref{errors}).

\subsubsection{\mintinline{icl}{[r/o] Tabs'length : int}}

Numărul de taburi în sesie.

Excepții posibile: \ferror{NoSessions} (pr. tab. \ref{errors}).

\subsubsection{\mintinline{icl}{[r/o] Tabs'next : Tab}}

Următorul tab.

Excepții posibile: \ferror{NoSessions}, \ferror{NoSuchTab} (pr. tab. \ref{errors}).

\subsubsection{\mintinline{icl}{[r/o] Tabs'previous : Tab}}

Tabul precedent.

Excepții posibile: \ferror{NoSessions}, \ferror{NoSuchTab} (pr. tab. \ref{errors}).

\subsubsection{\mintinline{icl}{[r/o] Tabs'(i : int) : Tab}}

Al \mintinline{icl}{i}-lea tab.

Excepții posibile: \ferror{NoSessions}, \ferror{OutOfBounds} (pr. tab. \ref{errors}).

\subsubsection{\mintinline{icl}{Tabs.close (url : regex) : int}}

Închide toate taburile, la care adresa URL convine șablonului. Returnează numărul de taburi închise.

Excepții posibile: \ferror{NoSessions} (pr. tab. \ref{errors}).

\subsubsection{\mintinline{icl}{Tabs.close (url : regex) : int}}

Închide toate taburile, la care adresa URL convine expresii regulare \mintinline{icl}{url}. Returnează numărul de taburi închise.

Excepții posibile: \ferror{NoSessions} (pr. tab. \ref{errors}).

\subsubsection{\mintinline{icl}{Tabs.closeByTitle (template : string) : int}}

Închide toate taburile, la care titlul convine șablonului. Returnează numărul de taburi închise.

Excepții posibile: \ferror{NoSessions} (pr. tab. \ref{errors}).

\subsubsection{\mintinline{icl}{Tabs.closeByTitle (title : regex) : int}}

Închide toate taburile, la care titlul convine expresii regulare \mintinline{icl}{url}. Returnează numărul de taburi închise.

Excepții posibile: \ferror{NoSessions} (pr. tab. \ref{errors}).

\subsubsection{\mintinline{icl}{Tabs.closeOthers () : int}}

Închide toate taburile în afară de tabul curent. Returnează numărul de taburi închise.

Excepții posibile: \ferror{NoSessions} (pr. tab. \ref{errors}).

\subsubsection{\mintinline{icl}{Tabs.closeToLeft () : int}}

Închide toate taburile care se află la stînga de tabul curent. Returnează numărul de taburi închise.

Excepții posibile: \ferror{NoSessions} (pr. tab. \ref{errors}).

\subsubsection{\mintinline{icl}{Tabs.closeToRight () : int}}

Închide toate taburile care se află la dreapta de tabul curent. Returnează numărul de taburi închise.

Excepții posibile: \ferror{NoSessions} (pr. tab. \ref{errors}).

\subsubsection{\mintinline{icl}{Tabs.find (template : string) : Tab}}

Returnează primul tab, la care adresa URL convine șablonului.

Excepții posibile: \ferror{NoSessions} (pr. tab. \ref{errors}).

\subsubsection{\mintinline{icl}{Tabs.find (url : regex) : Tab}}

Returnează primul tab, la care adresa URL convine expresii regulare.

Excepții posibile: \ferror{NoSessions} (pr. tab. \ref{errors}).

\subsubsection{\mintinline{icl}{Tabs.findByTitle (template : string) : Tab}}

Returnează primul tab, la care titlul convine șablonului.

Excepții posibile: \ferror{NoSessions} (pr. tab. \ref{errors}).

\subsubsection{\mintinline{icl}{Tabs.findByTitle (title : regex) : Tab}}

Returnează primul tab, la care titlul convine expresii regulare.

Excepții posibile: \ferror{NoSessions} (pr. tab. \ref{errors}).

\subsection{{\color{orange} Tab}}

Obiectul \tab{} are următoarele proprietăți:
\begin{icItems}
	\item \mintinline{icl}{[icL] [r/o] Tab'canGoBack : bool};
	\item \mintinline{icl}{[icL] [r/o] Tab'canGoForward : bool};
	\item \mintinline{icl}{[r/o] Tab'screenshot : string};
	\item \mintinline{icl}{[r/o] Tab'source : string};
	\item \mintinline{icl}{[r/*] Tab'title};
	\item \mintinline{icl}{[r/w] Tab'url}.
\end{icItems}

Și următoarele metode:
\begin{icItems}
	\item \mintinline{icl}{Tab.back () : void};
	\item \mintinline{icl}{Tab.close () : void};
	\item \mintinline{icl}{Tab.focus () : void};
	\item \mintinline{icl}{Tab.forward () : void};
	\item \mintinline{icl}{Tab.get (url : string) : bool};
	\item \mintinline{icl}{Tab.load (url : string) : bool}.
\end{icItems}

\subsubsection{\mintinline{icl}{[icL] [r/o] Tab'canGoBack : bool}}

Returnează \true, dacă butonul \textit{Înapoi} e disponibil, în caz contrar \false.

\subsubsection{\mintinline{icl}{[icL] [r/o] Tab'canGoForward : bool}}

Returnează \true, dacă butonul \textit{Înainte} e disponibil, în caz contrar \false.

\subsubsection{\mintinline{icl}{[r/o] Tab'screenshot : string}}

Returnează captura tabului, codată în base64.

Excepții posibile: \ferror{NoSessions}, \ferror{NoSuchWindow}, \ferror{UnableToCaptureScreen} (pr. tab. \ref{errors}).

\subsubsection{\mintinline{icl}{[r/o] Tab'source : string}}

Codul sursă al paginii deschise în tab.

Excepții posibile: \ferror{NoSessions}, \ferror{NoSuchWindow} (pr. tab. \ref{errors}).

\subsubsection{\mintinline{icl}{[r/*] Tab'title}}

Titlu paginii deschise în tab.

Excepții posibile: \ferror{NoSessions}, \ferror{NoSuchWindow} (pr. tab. \ref{errors}).

\subsubsection{\mintinline{icl}{[r/w] Tab'url}}

Adresa URL a paginii deschise în tab.

Excepții posibile: \ferror{NoSessions}, \ferror{NoSuchWindow} (pr. tab. \ref{errors}).

\subsubsection{\mintinline{icl}{Tab.back () : void}}

Trece la pagina precedentă, cum ar fi cînd apăsați pe butonul \textit{Înapoi} în browser.

Excepții posibile: \ferror{NoSessions}, \ferror{NoSuchWindow}, \ferror{Timeout} (pr. tab. \ref{errors}).

\subsubsection{\mintinline{icl}{Tab.close () : void}}

Închide tabul, dacă e ultimul închide și sesia.

Excepții posibile: \ferror{NoSessions}, \ferror{NoSuchWindow} (pr. tab. \ref{errors}).

\subsubsection{\mintinline{icl}{Tab.focus () : void}}

Schimbă focusul la tab. Focusul se schimbă între taburile din sesiune.

Excepții posibile: \ferror{NoSessions} (pr. tab. \ref{errors}).

\subsubsection{\mintinline{icl}{Tab.forward () : void}}

Trece la pagina precedentă, cum ar fi cînd apăsați pe butonul \textit{Înainte} în browser.


Excepții posibile: \ferror{NoSessions}, \ferror{NoSuchWindow}, \ferror{Timeout} (pr. tab. \ref{errors}).

\subsubsection{\mintinline{icl}{Tab.get (url : string) : bool}}

Trece la pagină, URL trebuie să fie absolut. Returnează \true{} dacă pagina a fost descărcată cu succes, în caz contrar \false.

\subsubsection{\mintinline{icl}{Tab.load (url : string) : void}}

Trece la pagină, URL trebuie să fie absolut. În caz de eroare generează excepție.

Excepții posibile: \ferror{NoSessions}, \ferror{NoSuchWindow}, \ferror{InvalidArgument}, \ferror{Timeout}, \ferror{InsecureCertificate} (pr. tab. \ref{errors}).

\subsection{{\color{orange} Doc}}

Obiectul \dom{} are următoarele metode:
\begin{icItems}
	\item \mintinline{icl}{Doc.query (by = By'cssSelector, selector : string) : element};
	\item \mintinline{icl}{Doc.queryAll (by = By'cssSelector, selector : string) : element};
	\item \mintinline{icl}{Doc.queryAllByXPath (xpath : string) : element};
	\item \mintinline{icl}{Doc.queryByXPath (xpath : string) : element};
	\item \mintinline{icl}{Doc.queryLink (name : string, isFragment = false) : element};
	\item \mintinline{icl}{Doc.queryLinks (name : string, isFragment = false) : element};
	\item \mintinline{icl}{Doc.queryTag (name : string) : element};
	\item \mintinline{icl}{Doc.queryTags (name : string) : element}.
\end{icItems}

\subsubsection{\mintinline{icl}{Doc.query (by = By'cssSelector, selector : string) : element}}

Primește aceiași parametri ca și  \mintinline{icl}{element.query}, numai că această funcție va căuta în tot documentul.

Excepții posibile: \ferror{NoSessions}, \ferror{Timeout} (pr. tab. \ref{errors}).

\subsubsection{\mintinline{icl}{Doc.queryAll (by = By'cssSelector, selecor : string) : element}}

Primește aceiași parametri ca și \mintinline{icl}{element.queryAll}, numai că această funcție va căuta în tot documentul.

Excepții posibile: \ferror{NoSessions} (pr. tab. \ref{errors}).

\subsubsection{\mintinline{icl}{Doc.queryAllByXPath (xpath : string) : element}}

Acronim pentru \mintinline{icl}{Doc.queryAll (By'xPath, @xpath)}.

\subsubsection{\mintinline{icl}{Doc.queryByXPath (xpath : string) : element}}

Acronim pentru \mintinline{icl}{Doc.query (By'xPath, @xpath)}.

\subsubsection{\mintinline{icl}{Doc.queryLink (name : string, isFragment = false) : element}}

Acronim pentru:
\begin{icItems}
	\item \mintinline{icl}{Doc.query (By'linkText, @name)};
	\item \mintinline{icl}{Doc.query (By'partialLinkText, @name)};
\end{icItems}

\subsubsection{\mintinline{icl}{Doc.queryLinks (name : string, isFragment = false) : element}}

Acronim pentru:
\begin{icItems}
	\item \mintinline{icl}{Doc.queryAll (By'linkText, @name)};
	\item \mintinline{icl}{Doc.queryAll (By'partialLinkText, @name)};
\end{icItems}

\subsubsection{\mintinline{icl}{Doc.queryTag (name : string) : element}}

Acronim pentru \mintinline{icl}{Doc.query (By'tagName, @name)}.

\subsubsection{\mintinline{icl}{Doc.queryTags (name : string) : element}}

Acronim pentru \mintinline{icl}{Doc.queryAll (By'tagName, @name)}.

\subsection{{\color{orange} Files}}

Obiectul \files{} are următoarele metode:
\begin{icItems}
	\item \mintinline{icl}{Files.open (path : string) : File};
	\item \mintinline{icl}{Files.create (path : string) : File};
	\item \mintinline{icl}{Files.createDir (path : string) : void};
	\item \mintinline{icl}{Files.createPath (path : string) : void}.
\end{icItems}

\subsubsection{\mintinline{icl}{Files.open (path : string) : File}}

Deschide fișierul.

Excepții posibile: \ferror{FileNotFound} (pr. tab. \ref{errors}).

\subsubsection{\mintinline{icl}{Files.create (path : string) : File}}

Deschide fișierul, dacă el nu există se creează.

Excepții posibile: \ferror{FolderNotFound} (pr. tab. \ref{errors}).

\subsubsection{\mintinline{icl}{Files.createDir (path : string) : void}}

Creează folder, folderul ascendent trebuie să existe deja.

Excepții posibile: \ferror{FolderNotFound} (pr. tab. \ref{errors}).

\subsubsection{\mintinline{icl}{Files.createPath (path : string) : void}}

Creează toate folderele care nu există în calea indicată.

\subsection{{\color{orange} File}}

Obiectul \file{} are următoarele proprietăți:
\begin{icItems}
	\item \mintinline{icl}{[r/o] File'csv : 1};
	\item \mintinline{icl}{[r/w] File'format : int};
	\item \mintinline{icl}{[r/o] File'none : 0};
	\item \mintinline{icl}{[r/o] File'tsv : 2};
	\item \mintinline{icl}{[r/o] File'valid : bool}.
\end{icItems}

Și următoarele metode:
\begin{icItems}
	\item \mintinline{icl}{File.close () : void};
	\item \mintinline{icl}{File.delete () : void}.
\end{icItems}

\subsubsection{\mintinline{icl}{[r/o] File'csv : 1}}

Format CSV.

\subsubsection{\mintinline{icl}{[r/w] File'format : int}}

Returnează formatul fișierului.

\subsubsection{\mintinline{icl}{[r/o] File'none : 0}}

File neinițializat.

\subsubsection{\mintinline{icl}{[r/o] File'tsv : 2}}

Format TSV.

\subsubsection{\mintinline{icl}{[r/o] File'valid : bool}}

Returnează \true, dacă fișierul este inițializat, în caz contrar \false.

\subsubsection{\mintinline{icl}{File.close () : void}}

Închide fișierul.

\subsubsection{\mintinline{icl}{File.delete () : void}}

Șterge fișierul.

\subsection{{\color{orange} Make}}

Obiectul \make{} deține următoarea metodă: \mintinline{icl}{Make.image (base64 : string, path :}\\*\mintinline{icl}{string) : void}.

\subsubsection{\mintinline{icl}{Make.image (base64 : string, path : string) : void}}

Salvează captura de ecran pe disc.

\subsection{{\color{orange} Log}}

Obiectul \logtype{} deține următoarele metode:
\begin{icItems}
	\item \mintinline{icl}{Log.error (message : string) : void};
	\item \mintinline{icl}{Log.info (message : string) : void};
	\item \mintinline{icl}{Log.out (args : any ...) : void};
	\item \mintinline{icl}{Log.stack (var : any) : void};
	\item \mintinline{icl}{Log.state (var : any) : void}.
\end{icItems}

\subsubsection{\mintinline{icl}{Log.error (message : string) : void}}

Printează un mesaj de eroare.

\subsubsection{\mintinline{icl}{Log.info (message : string) : void}}

Printează un mesaj informațional.

\subsubsection{\mintinline{icl}{Log.out (args : any ...) : void}}

Printează informație pentru debug, primește cîțeva argumente de orice tip. La transmiterea variabilelor se printează containerul, numele variabilei, tipul de date și valoarea. La transmiterea constantelor numai valoarea. La transmiterea rezultatului funcției se printează tipul și valoarea.

\subsubsection{\mintinline{icl}{Log.stack (var : any) : void}}

Printează lista de stive, arătînd în care din ele se întilnește așa variabilă și ce valori are.

\subsubsection{\mintinline{icl}{Log.state (var : any) : void}}

Printează lista tuturor stărilor, arătînd în care din ele se întilnește așa variabilă și ce valori are.

\subsection{{\color{orange} Numbers}}

Obiectul \numbers{} are următoarele proprietăți:
\begin{icItems}
	\item \mintinline{icl}{[r/o] Numbers'max : 4};
	\item \mintinline{icl}{[r/o] Numbers'min : 3};
	\item \mintinline{icl}{[r/o] Numbers'product : 2};
	\item \mintinline{icl}{[r/o] Numbers'process : int};
	\item \mintinline{icl}{[r/o] Numbers'sum : 1}.
\end{icItems}

Și următoarele metode:
\begin{icItems}
	\item \mintinline{icl}{Numbers.process (a : int, b : int) : int};
	\item \mintinline{icl}{Numbers.process (a : double, b : double) : double};
	\item \mintinline{icl}{Numbers.restoreProcess () : void};
	\item \mintinline{icl}{Numbers.setProcess (proc : int) : void}.
\end{icItems}

\subsubsection{\mintinline{icl}{[r/o] Numbers'max : 4}}

A alege maximum.

\subsubsection{\mintinline{icl}{[r/o] Numbers'min : 3}}

A alege minimum.

\subsubsection{\mintinline{icl}{[r/o] Numbers'product : 2}}

A înmulți numerele.

\subsubsection{\mintinline{icl}{[r/o] Numbers'process : int}}

Modul curent de a prelucra numerele.

\subsubsection{\mintinline{icl}{[r/o] Numbers'sum : 1}}

A aduna numerele.

\subsubsection{\mintinline{icl}{Numbers.process (a : int, b : int) : int}}

Prelucrează numerele întregi cu metoda de prelucrare curentă.

\subsubsection{\mintinline{icl}{Numbers.process (a : double, b : double) : double}}

Prelucrează numerele decimale cu metoda de prelucrare curentă.

\subsubsection{\mintinline{icl}{Numbers.restoreProcess () : void}}

Șterge ultima înscriere sin stiva metodelor de prelucrare.

\subsubsection{\mintinline{icl}{Numbers.setProcess (proc : int) : void}}

Adaugă o nouă înscriere în stiva metodelor de prelucrare.

\subsection{{\color{orange} Math}}

Obiectul \mintinline{icl}{Math} are următoarele proprietăți:
\begin{icItems}
	\item \mintinline{icl}{[r/o] Math'1divPi : double};
	\item \mintinline{icl}{[r/o] Math'1divSqrt2 : double};
	\item \mintinline{icl}{[r/o] Math'2divPi : double};
	\item \mintinline{icl}{[r/o] Math'2divSqrtPi : double};
	\item \mintinline{icl}{[r/o] Math'e : double};
	\item \mintinline{icl}{[r/o] Math'ln2 : double};
	\item \mintinline{icl}{[r/o] Math'ln10 : double};
	\item \mintinline{icl}{[r/o] Math'log2e : double};
	\item \mintinline{icl}{[r/o] Math'log10e : double};
	\item \mintinline{icl}{[r/o] Math'pi : double};
	\item \mintinline{icl}{[r/o] Math'piDiv2 : double};
	\item \mintinline{icl}{[r/o] Math'piDiv4 : double};
	\item \mintinline{icl}{[r/o] Math'sqrt2 : double}.
\end{icItems}

Și următoarele metode:
\begin{icItems}
	\item \mintinline{icl}{Math.acos (v : double) : double};
	\item \mintinline{icl}{Math.asin (v : double) : double};
	\item \mintinline{icl}{Math.atan (v : double) : double};
	\item \mintinline{icl}{Math.ceil (v : double) : int};
	\item \mintinline{icl}{Math.cos (v : double) : double};
	\item \mintinline{icl}{Math.degreesToRadians (v : double) : double};
	\item \mintinline{icl}{Math.exp (v : double) : double};
	\item \mintinline{icl}{Math.floor (v : double) : int};
	\item \mintinline{icl}{Math.ln (v : double) : double};
	\item \mintinline{icl}{Math.min (arr : int...) : int};
	\item \mintinline{icl}{Math.min (arr : double...) : double};
	\item \mintinline{icl}{Math.max (arr : int...) : int};
	\item \mintinline{icl}{Math.max (arr : double...) : double};
	\item \mintinline{icl}{Math.radiansToDegrees (v : double) : double};
	\item \mintinline{icl}{Math.round (v : double) : int};
	\item \mintinline{icl}{Math.sin (v : double) : double};
	\item \mintinline{icl}{Math.tan (v : double) : double}.
\end{icItems}

\subsubsection{\mintinline{icl}{[r/o] Math'1divPi : double}}

1 împărțit la pi ($\frac{1}{\pi}$).

\subsubsection{\mintinline{icl}{[r/o] Math'1divSqrt2 : double}}

1 împărțit la radical din 2 ($\frac{1}{\sqrt{2}}$).

\subsubsection{\mintinline{icl}{[r/o] Math'2divPi : double}}

2 împărțit la pi ($\frac{2}{\pi}$).

\subsubsection{\mintinline{icl}{[r/o] Math'2divSqrtPi : double}}

2 împărțit la pi ($\frac{2}{\sqrt{\pi}}$).

\subsubsection{\mintinline{icl}{[r/o] Math'e : double}}

Numărul ($e$).

\subsubsection{\mintinline{icl}{[r/o] Math'ln2 : double}}

Logaritmul natural al numărului 2 ($\ln{2}$).

\subsubsection{\mintinline{icl}{[r/o] Math'ln10 : double}}

Logaritmul natural al numărului 10 ($\ln_{10}$).

\subsubsection{\mintinline{icl}{[r/o] Math'log2e : double}}

Logaritmul numărului $e$ cu baza 2 ($\log_{2}{e}$).

\subsubsection{\mintinline{icl}{[r/o] Math'log10e : double}}

Logaritmul numărului $e$ cu baza 10 ($\log_{10}{e}$).

\subsubsection{\mintinline{icl}{[r/o] Math'pi : double}}

Numărul pi ($\pi$).

\subsubsection{\mintinline{icl}{[r/o] Math'piDiv2 : double}}

Pi pe 2 ($\frac{\pi}{2}$).

\subsubsection{\mintinline{icl}{[r/o] Math'piDiv4 : double}}

Pi pe 4 ($\frac{\pi}{4}$).

\subsubsection{\mintinline{icl}{[r/o] Math'sqrt2 : double}}

Radical din 2 ($\sqrt{2}$).

\subsubsection{\mintinline{icl}{Math.acos (v : double) : double}}

Arccosinus ($\arccos{v}$).

\subsubsection{\mintinline{icl}{Math.asin (v : double) : double}}

Arcsinus ($\arcsin{v}$).

\subsubsection{\mintinline{icl}{Math.atan (v : double) : double}}

Arctangență ($\arctan{v}$).

\subsubsection{\mintinline{icl}{Math.ceil (v : double) : int}}

Cel mai mic număr întreg mai mare sau egal cu \mintinline{icl}{v}.

\subsubsection{\mintinline{icl}{Math.cos (v : double) : double}}

Cosinus ($\cos{v}$).

\subsubsection{\mintinline{icl}{Math.degreesToRadians (v : double) : double}}

Conversează grade în radiani.

\subsubsection{\mintinline{icl}{Math.exp (v : double) : double}}

Funcția exponent ($\exp{v}$).

\subsubsection{\mintinline{icl}{Math.floor (v : double) : int}}

Cel mai mare număr întreg mai mic sau egal cu \mintinline{icl}{v}.

\subsubsection{\mintinline{icl}{Math.ln (v : double) : double}}

Logaritm natural ($\ln{v}$).

\subsubsection{\mintinline{icl}{Math.min (arr : int...) : int}}

Returnează cel mai mic număr întreg.

\subsubsection{\mintinline{icl}{Math.min (arr : double...) : double}}

Returnează cel mai mic număr decimal.

\subsubsection{\mintinline{icl}{Math.max (arr : int...) : int}}

Returnează cel mai mare număr întreg.

\subsubsection{\mintinline{icl}{Math.max (arr : double...) : double}}

Returnează cel mai mare număr decimal.

\subsubsection{\mintinline{icl}{Math.radiansToDegrees (v : double) : double}}

Conversează radiani în grade.

\subsubsection{\mintinline{icl}{Math.round (v : double) : int}}

Returnează cel mai apropiat număr întreg.

\subsubsection{\mintinline{icl}{Math.sin (v : double) : double}}

Sinus ($\sin{v}$).

\subsubsection{\mintinline{icl}{Math.tan (v : double) : double}}

Cosinus ($\tan{v}$).

\subsection{{\color{orange} Import}}

Obiectul \mintinline{icl}{Import} deține următoarele metode:
\begin{icItems}
	\item \mintinline{icl}{Import.none (data : object, path : string) : void};
	\item \mintinline{icl}{Import.none (path : string) : void};
	\item \mintinline{icl}{Import.functions (data : object, path : string) : void};
	\item \mintinline{icl}{Import.functions (path : string) : void};
	\item \mintinline{icl}{Import.all (data : object, path : string) : void};
	\item \mintinline{icl}{Import.all (path : string) : void};
	\item \mintinline{icl}{Import.run (path : string) : void}.
\end{icItems}

Obiectul \mintinline{icl}{data} permite a transmite date în contextul izolat în care se execut fișierele externe, proprietățile obiectului \mintinline{icl}{data} vor fi disponibile ca variabile globale. Aceasta permite a deține într-un fișier mai multe versiuni ale bibliotecii de exemplu, și la folosire se poate de indicat care versiune e necesară.

\subsubsection{\mintinline{icl}{Import.none (data : object, path : string) : void}}

Creează context izolat în care se execută fișierul.

\subsubsection{\mintinline{icl}{Import.none (path : string) : void}}

Acronim pentru \mintinline{icl}{Import.none([=], @path)}.

\subsubsection{\mintinline{icl}{Import.functions (data : object, path : string) : void}}

Creează context izolat în care se execută fișierul, Apoi se importă funcțiile în contextul curent.

\subsubsection{\mintinline{icl}{Import.functions (path : string) : void}}

Acronim pentru \mintinline{icl}{Import.functions ([=], @path)}.

\subsubsection{\mintinline{icl}{Import.all (data : object, path : string) : void}}

Creează context izolat în care se execută fișierul, Apoi se importă funcțiile și variabilele globale în contextul curent.

\subsubsection{\mintinline{icl}{Import.all (path : string) : void}}

Acronim pentru \mintinline{icl}{Import.all ([=], @path)}.

\subsubsection{\mintinline{icl}{Import.run (path : string) : void}}

Execută fișierul în contextul curent.

%\newpage
