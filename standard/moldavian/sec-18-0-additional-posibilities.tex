% !TeX spellcheck = ro_RO
\section{Posibilități adiționale}

{\bf Posibilitățile adiționale} sunt prezente în icL pentru acoperirea completă a standardului W3C WebDriver și pentru rezolvarea unor probleme care pot apărea la practică.

Aceste posibilități nu trebuiesc numaidecît, se poate de se descurcat și fără ele. Pentru că la executarea scriptului automat se deschide o nouă sesie și un nou tab, iar la oprire sesia se închide automat.

\subsection{{\color{orange} \_sessions}}

Obiectul \sessions{} are următoarele proprietăți:
\begin{icItems}
	\item \lstinline|[r/o] _sessions'length : int|;
	\item \lstinline|[r/o] _sessions'(<int>i) : _session|.
\end{icItems}

Și următoarele metode:
\begin{icItems}
	\item \lstinline|_sessions.closeAll : void|;
	\item \lstinline|_sessions.get <int>i : _session|;
	\item \lstinline|_sessions.new : _session|.
\end{icItems}

\subsubsection{\lstinline|[r/o] _sessions'length : int|}

Returnează numărul de sesiuni deschise.

\subsubsection{\lstinline|[r/o] _sessions'(<int>i) : _session|}

Returnează a \code{i}-a sesiune.

Excepții posibile: \ferror{OutOfBounds}.

\subsubsection{\lstinline|_sessions.closeAll : void|}

Închide toate sesiunile.

\subsubsection{\lstinline|_sessions.get <int>i : _session|}

Returnează a \code{i}-a sesie.

Excepții posibile: \ferror{OutOfBounds}.

\subsubsection{\lstinline|_sessions.new : _session|}

Deschide o sesiune nouă.

Excepții posibile: \ferror{SessionNotCreated}.

\subsection{{\color{orange} \_session}}

Comanda \session{} returnează un link la sesiunea curentă.

Obiectul \session{} are următoarele proprietăți:
\begin{icItems}
	\item \lstinline|[r/w] _session'implicitTimeout : int|;
	\item \lstinline|[r/w] _session'pageLoadTimeout : int|;
	\item \lstinline|[r/w] _session'scriptTimeout : int|;
	\item \lstinline|[r/o] _session'source : string|;
	\item \lstinline|[r/o] _session'title : string|;
	\item \lstinline|[r/w] _session'url : string|.
\end{icItems}

Și următoarele metode:
\begin{icItems}
	\item \lstinline|_session.back : _session|;
	\item \lstinline|_session.close : void|;
	\item \lstinline|_session.forward : _session|;
	\item \lstinline|_session.refresh : _session|;
	\item \lstinline|_session.screenshot : string|;
	\item \lstinline|_session.switchTo : _session|.
\end{icItems}

\subsubsection{\lstinline|[r/w] _session'implicitTimeout : int|}

Tipul în milisecunde de așteptare a găsirii elementului sau al trecerii în stare interactivă. Implicit egal cu 0.

Excepții posibile: \ferror{NoSessions}, \ferror{InvalidArgument} (valoarea pentru instalare este mai mare sau mai mică decît intervalul sigur al numerelor întregi).

\subsubsection{\lstinline|[r/w] _session'pageLoadTimeout : int|}

Timpul de așteptare al încărcării paginii web. Implicit 300 000 de milisecunde.

Excepții posibile: \ferror{NoSessions}, \ferror{InvalidArgument}.

\subsubsection{\lstinline|[r/w] _session'scriptTimeout : int|}

Timpul de așteptare al executării codului Javascript. Implicit 30 000 de milisecunde.

Excepții posibile: \ferror{NoSessions}, \ferror{InvalidArgument}.

\subsubsection{\lstinline|[r/o] _session'source : string|}

Codul sursă al paginii.

Excepții posibile: \ferror{NoSessions}, \ferror{NoSuchWindow}.

\subsubsection{\lstinline|[r/o] _session'title : string|}

Titlu paginii.

Excepții posibile: \ferror{NoSessions}, \ferror{NoSuchWindow}.

\subsubsection{\lstinline|[r/w] _session'url : string|}

Adresa URL a pagini curente.

Excepții posibile: \ferror{NoSessions}, \ferror{NoSuchWindow}.

\subsubsection{\lstinline|_session.back : _session|}

Trece la pagina precedentă, cum ar fi să faceți clic pe butonul "Înapoi" în browserul Google Chrome.

Excepții posibile: \ferror{NoSessions}, \ferror{NoSuchWindow}, \ferror{Timeout}.

\subsubsection{\lstinline|_session.close : void|}

Închide sesia. Toate taburile vor fi închise.

Excepții posibile: \ferror{NoSessions}, \ferror{NoSessions}, \ferror{NoSuchSession}.

\subsubsection{\lstinline|_session.forward : _session|}

Trece la pagina precedentă, cum ar fi să faceți clic pe butonul "Înainte" în browserul Google Chrome.

Excepții posibile: \ferror{NoSessions}, \ferror{NoSuchWindow}, \ferror{Timeout}.

\subsubsection{\lstinline|_session.refresh : _session|}

Reîncarcă pagina.

Excepții posibile: \ferror{NoSessions}, \ferror{NoSuchWindow}, \ferror{Timeout}.

\subsubsection{\lstinline|_session.screenshot : string|}

Returnează un șir de caracter, care reprezintă captură de ecran codată în base64.

Excepții posibile: \ferror{NoSessions}, \ferror{NoSuchWindow}, \ferror{UnableToCaptureScreen}.

\subsubsection{\lstinline|_session.switchTo : _session|}

Schimbă focusul la altă sesie.

Excepții posibile: \ferror{NoSuchSession}.

%\subsubsection{}

\subsection{{\color{orange} \_windows}}

Obiectul \windows{} are următoarele proprietăți:
\begin{icItems}
	\item \lstinline|[r/o] _windows'length : int|;
	\item \lstinline|[r/o] _windows'(<int>i) : _window|.
\end{icItems}

Și următoarea metodă \lstinline|_windows.get <int>i : _window|.

\subsubsection{\lstinline|[r/o] _windows'length : int|}

Returnează numărul de ferestre în sesiunea curentă.

Excepții posibile: \ferror{NoSessions}.

\subsubsection{\lstinline|[r/o] _windows'(<int>i) : _window|}

Returnează a \code{i}-a fereastră.

Excepții posibile: \ferror{NoSessions}, \ferror{OutOfBounds}.

\subsubsection{\lstinline|_windows.get <int>i : _window|}

Returnează a \code{i}-a fereastră.

Excepții posibile: \ferror{NoSessions}, \ferror{OutOfBounds}.

\subsection{{\color{orange} \_window}}

Obiectul \window{} are următoarele proprietăți:
\begin{icItems}
	\item \lstinline|[r/w] _window'height : int|;
	\item \lstinline|[r/w] _window'width : int|;
	\item \lstinline|[r/w] _window'x : int|;
	\item \lstinline|[r/w] _window'y : int|.
\end{icItems}

Și următoarele metode:
\begin{icItems}
	\item \lstinline|_window.close : void|;
	\item \lstinline|_window.focus : _window|;
	\item \lstinline|_window.fullscreen : _window|;
	\item \lstinline|_window.maximize : _window|;
	\item \lstinline|_window.minimize : _window|;
	\item \lstinline|_window.restore : _window|;
	\item \lstinline|_window.switchToDefault : _window|;
	\item \lstinline|_window.switchToFrame <int>i : _window|;
	\item \lstinline|_window.switchToFrame <element>el : _window|;
	\item \lstinline|_window.switchToParent : _window|.
\end{icItems}

\subsubsection{\lstinline|[r/w] _window'height : int|}

Înălțimea ferestrei curente în pixeli.

Excepții posibile: \ferror{NoSessions}, \ferror{NoSuchWindow}, \ferror{InvalidArgument}, \ferror{UnsupportedOperation}.

\subsubsection{\lstinline|[r/w] _window'width : int|}

Lățimea ferestrei curente în pixeli.

Excepții posibile: \ferror{NoSessions}, \ferror{NoSuchWindow}, \ferror{InvalidArgument}, \ferror{UnsupportedOperation}.

\subsubsection{\lstinline|[r/w] _window'x : int|}

Coordonata $x$ a ferestrei curente în pixeli.

Excepții posibile: \ferror{NoSessions}, \ferror{NoSuchWindow}, \ferror{InvalidArgument}, \ferror{UnsupportedOperation}.

\subsubsection{\lstinline|[r/w] _window'y : int|}

Coordinata $y$ a ferestrei curente în pixeli.

Excepții posibile: \ferror{NoSessions}, \ferror{NoSuchWindow}, \ferror{InvalidArgument}, \ferror{UnsupportedOperation}.

\subsubsection{\lstinline|_window.close : void|}

Închide fereastra curentă, dacă ea este ultima se închide și sesia.

Excepții posibile: \ferror{NoSessions}, \ferror{NoSuchWindow}.

\subsubsection{\lstinline|_window.focus : _window|}

Schimbă focosul la fereastra aceasta.

Excepții posibile: \ferror{NoSessions}, \ferror{NoSuchWindow}.

\subsubsection{\lstinline|_window.fullscreen : _window|}

Schimbă regimul de afișare a ferestrei browserului la \textit{ecran complet}.

Excepții posibile: \ferror{NoSessions}, \ferror{NoSuchWindow}.

\subsubsection{\lstinline|_window.maximize : _window|}

Maximizează fereastra browserului.

Excepții posibile: \ferror{NoSessions}, \ferror{NoSuchWindow}.

\subsubsection{\lstinline|_window.minimize : _window|}

Minimizează fereastra browserului.

Excepții posibile: \ferror{NoSessions}, \ferror{NoSuchWindow}.

\subsubsection{\lstinline|_window.restore : _window|}

Restaurează fereastra în regim normal.

Excepții posibile: \ferror{NoSessions}, \ferror{NoSuchWindow}.

\subsubsection{\lstinline|_window.switchToDefault : _window|}

Schimbă focusul la frame-ul principal.

Excepții posibile: \ferror{NoSessions}, \ferror{NoSuchWindow}.

\subsubsection{\lstinline|_window.switchToFrame <int>i : _window|}

Schimbă focusul la frame-ul cu indexul \code{i}.

Excepții posibile: \ferror{NoSessions}, \ferror{NoSuchWindow}, \ferror{NoSuchFrame}.

\subsubsection{\lstinline|_window.switchToFrame <element>el : _window|}

Schimbă focusul la frame-ul elementului \code{el}. Elementul trebuie să fir numaidecît tag frame sau iframe.

Excepții posibile: \ferror{NoSessions}, \ferror{NoSuchWindow}, \ferror{NoSuchFrame}, \ferror{StaleElementReference}.

\subsubsection{\lstinline|_window.switchToParent : _window|}

Schimbă focusul la frame-ul ascensor.

Excepții posibile: \ferror{NoSessions}, \ferror{NoSuchWindow}.

\subsection{{\color{orange} \_cookies}}

Obiectul \cookies{} are următoarea proprietate: \lstinline|[r/o] _cookies'<string>name :|\\* \lstinline|_cookie|.

Și următoarele metode: 
\begin{icItems}
	\item \lstinline|_cookies.deleteAll : void|;
	\item \lstinline|_cookies.get <string>name : _cookie|.
\end{icItems}

\subsubsection{\lstinline|[r/o] _cookies'<string>name : _cookie|}

Returnează \cookie{} cu numele \code{name}.

Excepții posibile: \ferror{NoSessions}, \ferror{NoSuchWindow}, \ferror{NoSuchCookie}.

\subsubsection{\lstinline|_cookies.deleteAll : void|}

Șterge toate fișierele cookie.

\subsubsection{\lstinline|_cookies.get <string>name : _cookie|}

Returnează \cookie{} cu numele \code{name}.

Excepții posibile: \ferror{NoSessions}, \ferror{NoSuchWindow}, \ferror{NoSuchCookie}.

\subsection{{\color{orange} \_cookie}}

Obiectul \cookie{} are următoarele proprietăți:
\begin{icItems}
	\item \lstinline|[r/w] _cookie'domain : string|;
	\item \lstinline|[r/w] _cookie'expiry : int|;
	\item \lstinline|[r/w] _cookie'httpOnly : bool|;
	\item \lstinline|[r/w] _cookie'name : string|;
	\item \lstinline|[r/w] _cookie'path : string|;
	\item \lstinline|[r/w] _cookie'secure : bool|;
	\item \lstinline|[r/w] _cookie'value : string|.
\end{icItems}

Și următoarele metode:
\begin{icItems}
	\item \lstinline|_cookie.add <int>years <int>months <int>days <int>hours = 0 <int>minutes = 0|\\* \lstinline|<int>seconds = 0 : _cookie|;
	\item \lstinline|_cookie.load : _cookie|;
	\item \lstinline|_cookie.resetTime : _cookie|;
	\item \lstinline|_cookie.save : _cookie|.
\end{icItems}

\subsubsection{\lstinline|[r/w] _cookie'domain : string|}

Numele de domeniu, pe care \cookie{} este disponibil.

\subsubsection{\lstinline|[r/w] _cookie'expiry : int|}

Timpul de expirare al fișierului cookie. Implicit -1, indică că fișierul cookie va fi șters la sfîrșitul sesiei.

\subsubsection{\lstinline|[r/w] _cookie'httpOnly : bool|}

Doar pentru protocolul HTTP. Implicit \false.

\subsubsection{\lstinline|[r/w] _cookie'name : string|}

Denumire, obligatorie pentru completare.

\subsubsection{\lstinline|[r/w] _cookie'path : string|}

Calea pe care fișierul cookie este diponibil, implicit \lstinline|"/"|.

\subsubsection{\lstinline|[r/w] _cookie'secure : bool|}

Securizat, implicit \false.

\subsubsection{\lstinline|[r/w] _cookie'value : string|}

Valoarea fișierului cookie, obligatorie pentru completare.

\subsubsection{\lstinline|_cookie.add <int>years <int>months <int>days <int>hours = 0 <int>minutes = 0|\\*\noindent\lstinline|<int>seconds = 0 : _cookie|}

Adaugă la timpul de expirare numărul necesar de ani, luni, zile, ore, minute și secunde.

\subsubsection{\lstinline|_cookie.load : _cookie|}

Încarcă datele despre \cookie din browser.

Excepții posibile: \ferror{NoSessions}, \ferror{NoSuchWindow}, \ferror{NoSuchCookie}.

\subsubsection{\lstinline|_cookie.resetTime : _cookie|}

Setează timpul de expirare la timpul curent. De exemplu pentru ca fișierul să expire peste un an se folosește comanda următoare: \lstinline|_cookie.resetTime.add 1 0 0|.

\subsubsection{\lstinline|_cookie.save : _cookie|}

Transmite schimbările fișierului cookie în browser.

Excepții posibile: \ferror{NoSessions}, \ferror{NoSuchWindow}, \ferror{InvalidArgument}, \ferror{UnableToSetCookie}, \ferror{InvalidCookieDomain}.

\subsubsection{Crearea noilor fișiere cookie}

Pe foaia \ref{newcookies}, este prezentată metoda corectă de a crea fișiere cookie noi.

\begin{lstlisting}[caption=Crearea noilor fișiere cookie, label=newcookies]
for any _cookie {
	@'name = "age";
	@'domain = "example.org";
	@'value = 23 : string;
	@.resetTime.add 1 0 0;
	@.save;
};
\end{lstlisting}


\subsection{{\color{orange} \_alert}}

Obiectul \alert{} are următoarea proprietate: \lstinline|[r/o] _alert'text : string|.

Și următoarele metode:
\begin{icItems}
	\item \lstinline|_alert.accept : void|;
	\item \lstinline|_alert.dismiss : void|;
	\item \lstinline|_alert.sendKeys <string>keys : void|.
\end{icItems}

\subsubsection{\lstinline|[r/o] _alert'text : string|}

Returnează textul avertizării.

Excepții posibile: \ferror{NoSessions}, \ferror{NoSuchWindow}, \ferror{NoSuchAlert}.

\subsubsection{\lstinline|_alert.accept : void|}

Acceptă avertizarea.

Excepții posibile: \ferror{NoSessions}, \ferror{NoSuchWindow}, \ferror{NoSuchAlert}.

\subsubsection{\lstinline|_alert.dismiss : void|}

Refuză avertizarea.

Excepții posibile: \ferror{NoSessions}, \ferror{NoSuchWindow}, \ferror{NoSuchAlert}.

\subsubsection{\lstinline|_alert.sendKeys <string>keys : void|}

Completează formularul cu textul \code{keys} și confirmă inserarea.

Excepții posibile: \ferror{NoSessions}, \ferror{NoSuchWindow}, \ferror{NoSuchAlert}, \ferror{ElementNotIn\-teractable}.

\subsection{{\color{orange} \_tabs}}

Obiectul \tabs{} are următoarele proprietăți:
\begin{icItems}
	\item \lstinline|[r/o] _tabs'current : _tab|;
	\item \lstinline|[r/o] _tabs'first : _tab|;
	\item \lstinline|[r/o] _tabs'last : _tab|;
	\item \lstinline|[r/o] _tabs'length : int|;
	\item \lstinline|[r/o] _tabs'next : _tab|;
	\item \lstinline|[r/o] _tabs'previous : _tab|;
	\item \lstinline|[r/o] _tabs'(<int>i) : _tab|.
	% \item \lstinline|_tabs'|;
\end{icItems}

Și următoarele metode:
\begin{icItems}
	\item \lstinline|_tabs.close <string>template : int|;
	\item \lstinline|_tabs.close <regex>url : int|;
	\item \lstinline|_tabs.closeByTitle <string>template : int|;
	\item \lstinline|_tabs.closeByTitle <regex>title : int|;
	\item \lstinline|_tabs.closeOthers : int|;
	\item \lstinline|_tabs.closeToLeft : int|;
	\item \lstinline|_tabs.closeToRight : int|;
	\item \lstinline|_tabs.find <string>template : _tab|;
	\item \lstinline|_tabs.find <regex>url : _tab|;
	\item \lstinline|_tabs.findByTitle <string>template : _tab|;
	\item \lstinline|_tabs.findByTitle <regex>title : _tab|.
	% \item \lstinline|_tabs.|;
\end{icItems}

В режиме тестирования вкладки будут в произвольном порядке. В режиме автоматизации в строгом порядке.

\subsubsection{\lstinline|[r/o] _tabs'current : _tab|}

Текущая вкладка.

Excepții posibile: \ferror{NoSessions}.

\subsubsection{\lstinline|[r/o] _tabs'first : _tab|}

Первая вкладка.

Excepții posibile: \ferror{NoSessions}.

\subsubsection{\lstinline|[r/o] _tabs'last : _tab|}

Последняя вкладка.

Excepții posibile: \ferror{NoSessions}.

\subsubsection{\lstinline|[r/o] _tabs'length : int|}

Количество вкладок.

Excepții posibile: \ferror{NoSessions}.

\subsubsection{\lstinline|[r/o] _tabs'next : _tab|}

Следующая вкладка.

Excepții posibile: \ferror{NoSessions}, \ferror{NoSuchTab}.

\subsubsection{\lstinline|[r/o] _tabs'previous : _tab|}

Предыдущая вкладка.

Excepții posibile: \ferror{NoSessions}, \ferror{NoSuchTab}.

\subsubsection{\lstinline|[r/o] _tabs'(<int>i) : _tab|}

\code{i}-я вкладка.

Excepții posibile: \ferror{NoSessions}, \ferror{OutOfBounds}.

\subsubsection{\lstinline|_tabs.close <string>template : int|}

Закрывает все вкладки, у которых URL подходит по шаблону. Returnează количество закрытых вкладок.

Excepții posibile: \ferror{NoSessions}.

\subsubsection{\lstinline|_tabs.close <regex>url : int|}

Закрывает все вкладки, имеющее URL подходящий по регулярному выражению \code{url}. Returnează количество закрытых вкладок.

Excepții posibile: \ferror{NoSessions}.

\subsubsection{\lstinline|_tabs.closeByTitle <string>template : int|}

Закрывает все вкладки, у которых название подходит по шаблону. Returnează количество закрытых вкладок.

Excepții posibile: \ferror{NoSessions}.

\subsubsection{\lstinline|_tabs.closeByTitle <regex>title : int|}

Закрывает все вкладки, имеющее название подходяще по регулярному выражению \code{url}. Returnează количество закрытых вкладок.

Excepții posibile: \ferror{NoSessions}.

\subsubsection{\lstinline|_tabs.closeOthers : int|}

Закрывает все вкладки кроме текущей. Returnează количество закрытых вкладок.

Excepții posibile: \ferror{NoSessions}.

\subsubsection{\lstinline|_tabs.closeToLeft : int|}

Закрывает все вкладки которые находится с лева от текущей. Returnează количество закрытых вкладок.

Excepții posibile: \ferror{NoSessions}.

\subsubsection{\lstinline|_tabs.closeToRight : int|}

Закрывает все вкладки которые находится с права от текущей. Returnează количество закрытых вкладок.

Excepții posibile: \ferror{NoSessions}.

\subsubsection{\lstinline|_tabs.find <string>template : _tab|}

Returnează первая вкладка, у которой URL подходит по шаблону.

Excepții posibile: \ferror{NoSessions}.

\subsubsection{\lstinline|_tabs.find <regex>url : _tab|}

Returnează первая вкладка, у которой URL подходит по регулярному выражению.

Excepții posibile: \ferror{NoSessions}.

\subsubsection{\lstinline|_tabs.findByTitle <string>template : _tab|}

Returnează первая вкладка, у которой название подходит по шаблону.

Excepții posibile: \ferror{NoSessions}.

\subsubsection{\lstinline|_tabs.findByTitle <regex>title : _tab|}

Returnează первая вкладка, у которой название подходит по регулярному выражению.

Excepții posibile: \ferror{NoSessions}.

\subsection{{\color{orange} \_tab}}

Obiectul \tab{} are următoarele proprietăți:
\begin{icItems}
	\item \lstinline|[icL] [r/o] _tab'canGoBack : bool|;
	\item \lstinline|[icL] [r/o] _tab'canGoForward : bool|;
	\item \lstinline|[r/o] _tab'screenshot : string|;
	\item \lstinline|[r/o] _tab'source : string|;
	\item \lstinline|[r/*] _tab'title|;
	\item \lstinline|[r/w] _tab'url|.
\end{icItems}

Și următoarele metode:
\begin{icItems}
	\item \lstinline|_tab.back : void|;
	\item \lstinline|_tab.close : void|;
	\item \lstinline|_tab.focus : void|;
	\item \lstinline|_tab.forward : void|;
	\item \lstinline|_tab.get <string>url : bool|.
\end{icItems}

\subsubsection{\lstinline|[icL] [r/o] _tab'canGoBack : bool|}

Returnează \true, если можно вернутся на предыдущую страницу, иначе \false.

\subsubsection{\lstinline|[icL] [r/o] _tab'canGoForward : bool|}

Returnează \true, если можно вернутся на следующую страницу, иначе \false.

\subsubsection{\lstinline|[r/o] _tab'screenshot : string|}

Returnează скриншот вкладки, кодирован в base64.

Excepții posibile: \ferror{NoSessions}, \ferror{NoSuchWindow}, \ferror{UnableToCaptureScreen}.

\subsubsection{\lstinline|[r/o] _tab'source : string|}

Исходник страницы, открытой в указанной вкладке.

Excepții posibile: \ferror{NoSessions}, \ferror{NoSuchWindow}.

\subsubsection{\lstinline|[r/*] _tab'title|}

Название страницы, открытой в указанной вкладке.

Excepții posibile: \ferror{NoSessions}, \ferror{NoSuchWindow}.

\subsubsection{\lstinline|[r/w] _tab'url|}

URL страницы, открытой в указанной вкладке.

Excepții posibile: \ferror{NoSessions}, \ferror{NoSuchWindow}.

\subsubsection{\lstinline|_tab.back : void|}

Переходит на предыдущую страницу, как при нажатии на кнопке браузера "Назад".

Excepții posibile: \ferror{NoSessions}, \ferror{NoSuchWindow}, \ferror{Timeout}.

\subsubsection{\lstinline|_tab.close : void|}

Закрывает окну, если она последняя закрывается и сессию.

Excepții posibile: \ferror{NoSessions}, \ferror{NoSuchWindow}.

\subsubsection{\lstinline|_tab.focus : void|}

Переключает фокус на вкладку. Фокус можно менять только внутри активной сессии.

Excepții posibile: \ferror{NoSessions}.

\subsubsection{\lstinline|_tab.forward : void|}

Вернётся на следующую страницу, как при нажатии на кнопке браузера "Вперёд".

Excepții posibile: \ferror{NoSessions}, \ferror{NoSuchWindow}, \ferror{Timeout}.

\subsubsection{\lstinline|_tab.get <string>url : bool|}

Загружает страницу, URL должен быть абсолютным.

Excepții posibile: \ferror{NoSessions}, \ferror{NoSuchWindow}, \ferror{InvalidArgument}, \ferror{Timeout}, \ferror{InsecureCertificate}.

\subsection{{\color{orange} \_dom}}

Obiectul \dom{} имеет следующие методы:
\begin{icItems}
	\item \lstinline|_dom.query <int>by <string>selector : element|;
	\item \lstinline|_dom.query <string>cssSelector : element|;
	\item \lstinline|_dom.queryAll <int>by <string>selecor : element|;
	\item \lstinline|_dom.queryAll <string>cssSelector : lement|;
	\item \lstinline|_dom.queryAllByXPath <string>xpath : element|;
	\item \lstinline|_dom.queryByXPath <string>xpath : element|;
	\item \lstinline|_dom.queryLink <string>name <bool>isFragment = false : element|;
	\item \lstinline|_dom.queryLinks <string>name <bool>isFragment = false : element|;
	\item \lstinline|_dom.queryTag <string>name : element|;
	\item \lstinline|_dom.queryTags <string>name : element|.
\end{icItems}

\subsubsection{\lstinline|_dom.query <int>by <string>selector : element|}

Принимает то же самые параметры, как и \lstinline|element.query|, только поиск ведётся по всему документу.

Excepții posibile: \ferror{NoSessions}, \ferror{Timeout}.

\subsubsection{\lstinline|_dom.query <string>cssSelector : element|}

Акроним для \lstinline|_dom.query _by'cssSelector @cssSelector|.

\subsubsection{\lstinline|_dom.queryAll <int>by <string>selecor : element|}

Принимает то же самые параметры, как и \lstinline|element.query|, только поиск ведётся по всему документу.

Excepții posibile: \ferror{NoSessions}.

\subsubsection{\lstinline|_dom.queryAll <string>cssSelector : lement|}

Акроним для \lstinline|_dom.queryAll _by'cssSelector @cssSelector|.

\subsubsection{\lstinline|_dom.queryAllByXPath <string>xpath : element|}

Акроним для \lstinline|_dom.queryAll _by'xPath @xpath|.

\subsubsection{\lstinline|_dom.queryByXPath <string>xpath : element|}

Акроним для \lstinline|_dom.query _by'xPath @xpath|.

\subsubsection{\lstinline|_dom.queryLink <string>name <bool>isFragment = false : element|}

Акроним для:
\begin{icItems}
	\item \lstinline|_dom.query _by'linkText @name|;
	\item \lstinline|_dom.query _by'partialLinkText @name|;
\end{icItems}

\subsubsection{\lstinline|_dom.queryLinks <string>name <bool>isFragment = false : element|}

Акроним для:
\begin{icItems}
	\item \lstinline|_dom.queryAll _by'linkText @name|;
	\item \lstinline|_dom.queryAll _by'partialLinkText @name|;
\end{icItems}

\subsubsection{\lstinline|_dom.queryTag <string>name : element|}

Акроним для \lstinline|_dom.query _by'tagName @name|.

\subsubsection{\lstinline|_dom.queryTags <string>name : element|}

Акроним для \lstinline|_dom.queryAll _by'tagName @name|.

\subsection{{\color{orange} \_files}}

Obiectul \files{} имеет следующие методы:
\begin{icItems}
	\item \lstinline|_files.open <string>path : _file|;
	\item \lstinline|_files.create <string>path : _file|;
	\item \lstinline|_files.createDir <string>path : void|;
	\item \lstinline|_files.createPath <string>path : void|.
\end{icItems}

\subsubsection{\lstinline|_files.open <string>path : _file|}

Открывает файл.

Excepții posibile: \ferror{FileNotFound}.

\subsubsection{\lstinline|_files.create <string>path : _file|}

Открывает файл, если он не существует, он создаётся.

Excepții posibile: \ferror{FolderNotFound}.

\subsubsection{\lstinline|_files.createDir <string>path : void|}

Создаёт папку, папка в которым она создаётся должна уже существовать.

Excepții posibile: \ferror{FolderNotFound}.

\subsubsection{\lstinline|_files.createPath <string>path : void|}

Создаёт все несуществующие папки пути.

\subsection{{\color{orange} \_file}}

Obiectul \file{} are următoarele proprietăți:
\begin{icItems}
	\item \lstinline|[r/o] _file'csv : 1|;
	\item \lstinline|[r/w] _file'format : int|;
	\item \lstinline|[r/o] _file'none : 0|;
	\item \lstinline|[r/o] _file'tsv : 2|;
	\item \lstinline|[r/o] _file'valid : bool|.
\end{icItems}

Și următoarele metode:
\begin{icItems}
	\item \lstinline|_file.close : void|;
	\item \lstinline|_file.delete : void|.
\end{icItems}

\subsubsection{\lstinline|[r/o] _file'csv : 1|}

Формат CSV.

\subsubsection{\lstinline|[r/w] _file'format : int|}

Returnează формат файла.

\subsubsection{\lstinline|[r/o] _file'none : 0|}

Не инициализированный файл.

\subsubsection{\lstinline|[r/o] _file'tsv : 2|}

Формат TSV.

\subsubsection{\lstinline|[r/o] _file'valid : bool|}

Returnează \true, если файл инициализированный, иначе \false.

\subsubsection{\lstinline|_file.close : void|}

Закрывает файл.

\subsubsection{\lstinline|_file.delete : void|}

Удаляет файл.

\subsection{{\color{orange} \_make}}

Obiectul \make{} имеет следующие методы:
\begin{icItems}
	\item \lstinline|_make.image <string>base64 <string>path : void|;
	\item \lstinline|_make.int <string>str <int>base : int|;
	\item \lstinline|_make.list <set>s : list|;
	\item \lstinline|_make.object <string>json : object|;
	\item \lstinline|_make.regex <string>pattern : regex|;
	\item \lstinline|_make.string <bool>b : string|;
	\item \lstinline|_make.string <double>d : string|;
	\item \lstinline|_make.string <int>number <int>base : string|;
	\item \lstinline|_make.string <object>obj : string|;
	\item \lstinline|_make.string <set>s : string|.
\end{icItems}

\subsubsection{\lstinline|_make.image <string>base64 <string>path : void|}

Сохраняет скриншот на жёстком диске.

\subsubsection{\lstinline|_make.int <string>str <int>base : int|}

Парсит целое число в виде строки в нужном базисе.

Excepții posibile: \ferror{ParsingFailed}.

\subsubsection{\lstinline|_make.list <set>s : list|}

Акроним для \lstinline|set : list|.

\subsubsection{\lstinline|_make.object <string>json : object|}

Акроним для \lstinline|object : string|.

\subsubsection{\lstinline|_make.regex <string>pattern : regex|}

Создаёт регулярное выражение на основе паттерна.

\subsubsection{\lstinline|_make.string <bool>b : string|}

Returnează \lstinline|"true"|, если \lstinline|b == true|, иначе \lstinline|"true"|.

\subsubsection{\lstinline|_make.string <double>d : string|}

Акроним для \lstinline|double : string|.

\subsubsection{\lstinline|_make.string <int>number <int>base : string|}

Returnează представления числа в заданном базисе.

\subsubsection{\lstinline|_make.string <object>obj : string|}

Акроним для \lstinline|object : string|.

\subsubsection{\lstinline|_make.string <set>s : string|}

Акроним для \lstinline|set : string|.

\subsection{{\color{orange} \_log}}

Obiectul \logtype{} имеет следующие методы:
\begin{icItems}
	\item \lstinline|_log.error <string>message : void|;
	\item \lstinline|_log.info <string>message : void|;
	\item \lstinline|_log.out <any ...>args : void|;
	\item \lstinline|_log.stack <any>var : void|;
	\item \lstinline|_log.state <any>var : void|.
\end{icItems}

\subsubsection{\lstinline|_log.error <string>message : void|}

Выводит сообщение об ошибке.

\subsubsection{\lstinline|_log.info <string>message : void|}

Выводит информационное сообщение.

\subsubsection{\lstinline|_log.out <any ...>args : void|}

Выводит отладочное сообщение, принимает несколько параметров любого типа. При получении переменных выводит исходник, название перемены, тип данных и значение. При получении константных только значение. При получении результата выполнения функции - тип и значение. 

\subsubsection{\lstinline|_log.stack <any>var : void|}

Выводит список всех стеков, указывая в каких из них встречается переменная с указанным именем, и какие значения имеются.

\subsubsection{\lstinline|_log.state <any>var : void|}

Выводит список всех состояний, указывая в каких из них встречается переменная с указанным именем, и какие значения имеются.

\subsection{{\color{orange} \_numbers}}

Obiectul \numbers{} are următoarele proprietăți:
\begin{icItems}
	\item \lstinline|[r/o] _numbers'max : 4|;
	\item \lstinline|[r/o] _numbers'min : 3|;
	\item \lstinline|[r/o] _numbers'product : 2|;
	\item \lstinline|[r/o] _numbers'process : int|;
	\item \lstinline|[r/o] _numbers'sum : 1|.
\end{icItems}

Și următoarele metode:
\begin{icItems}
	\item \lstinline|_numbers.process <int>a <int>b : int|;
	\item \lstinline|_numbers.process <double>a <double>b : double|;
	\item \lstinline|_numbers.restoreProcess : void|;
	\item \lstinline|_numbers.setProcess <int>proc : void|.
\end{icItems}

\subsubsection{\lstinline|[r/o] _numbers'max : 4|}

Выбирать максимум.

\subsubsection{\lstinline|[r/o] _numbers'min : 3|}

Выбирать минимум.

\subsubsection{\lstinline|[r/o] _numbers'product : 2|}

Умножить числа.

\subsubsection{\lstinline|[r/o] _numbers'process : int|}

Текущий способ обработки чисел.

\subsubsection{\lstinline|[r/o] _numbers'sum : 1|}

Складывать числа.

\subsubsection{\lstinline|_numbers.process <int>a <int>b : int|}

Обработать целых чисел текущим методом.

\subsubsection{\lstinline|_numbers.process <double>a <double>b : double|}

Обработать дробных чисел текучим методом.

\subsubsection{\lstinline|_numbers.restoreProcess : void|}

Удаляет последняя запись стека способов обработки.

\subsubsection{\lstinline|_numbers.setProcess <int>proc : void|}

Добавляет новая запись в стеке способов обработки.

\subsection{{\color{orange} \_math}}

Obiectul \lstinline|_math| are următoarele proprietăți:
\begin{icItems}
	\item \lstinline|[r/o] _math'1divPi : double|;
	\item \lstinline|[r/o] _math'1divSqrt2 : double|;
	\item \lstinline|[r/o] _math'2divPi : double|;
	\item \lstinline|[r/o] _math'2divSqrtPi : double|;
	\item \lstinline|[r/o] _math'e : double|;
	\item \lstinline|[r/o] _math'ln2 : double|;
	\item \lstinline|[r/o] _math'ln10 : double|;
	\item \lstinline|[r/o] _math'log2e : double|;
	\item \lstinline|[r/o] _math'log10e : double|;
	\item \lstinline|[r/o] _math'pi : double|;
	\item \lstinline|[r/o] _math'piDiv2 : double|;
	\item \lstinline|[r/o] _math'piDiv4 : double|;
	\item \lstinline|[r/o] _math'sqrt2 : double|.
\end{icItems}

Și următoarele metode:
\begin{icItems}
	\item \lstinline|_math.acos <double>v : double|;
	\item \lstinline|_math.asin <double>v : double|;
	\item \lstinline|_math.atan <double>v : double|;
	\item \lstinline|_math.ceil <double>v : int|;
	\item \lstinline|_math.cos <double>v : double|;
	\item \lstinline|_math.degreesToRadians <double>v : double|;
	\item \lstinline|_math.exp <double>v : double|;
	\item \lstinline|_math.floor <double>v : int|;
	\item \lstinline|_math.ln <double>v : double|;
	\item \lstinline|_math.min <int ...>arr : int|;
	\item \lstinline|_math.min <double ...>arr : double|;
	\item \lstinline|_math.max <int ...>arr : int|;
	\item \lstinline|_math.max <double ...>arr : double|;
	\item \lstinline|_math.radiansToDegrees <double>v : double|;
	\item \lstinline|_math.round <double> : int|;
	\item \lstinline|_math.sin <double>v : double|;
	\item \lstinline|_math.tan <double>v : double|.
\end{icItems}

\subsubsection{\lstinline|[r/o] _math'1divPi : double|}

1 делить на пи ($\frac{1}{\pi}$).

\subsubsection{\lstinline|[r/o] _math'1divSqrt2 : double|}

1 делить на корень из числа 2 ($\frac{1}{\sqrt{2}}$).

\subsubsection{\lstinline|[r/o] _math'2divPi : double|}

2 делить на пи ($\frac{2}{\pi}$).

\subsubsection{\lstinline|[r/o] _math'2divSqrtPi : double|}

2 делить на корень из числа пи ($\frac{2}{\sqrt{\pi}}$).

\subsubsection{\lstinline|[r/o] _math'e : double|}

Число ($e$).

\subsubsection{\lstinline|[r/o] _math'ln2 : double|}

Натуральный логарифм числа 2 ($\ln{2}$).

\subsubsection{\lstinline|[r/o] _math'ln10 : double|}

Натуральный логарифм числа 10 ($\ln_{10}$).

\subsubsection{\lstinline|[r/o] _math'log2e : double|}

Логарифм числа е по основанию 2 ($\log_{2}{e}$).

\subsubsection{\lstinline|[r/o] _math'log10e : double|}

Логарифм числа е по основанию 10 ($\log_{10}{e}$).

\subsubsection{\lstinline|[r/o] _math'pi : double|}

Число пи ($\pi$).

\subsubsection{\lstinline|[r/o] _math'piDiv2 : double|}

Пи по полам ($\frac{\pi}{2}$).

\subsubsection{\lstinline|[r/o] _math'piDiv4 : double|}

Пи на 4 ($\frac{\pi}{4}$).

\subsubsection{\lstinline|[r/o] _math'sqrt2 : double|}

Корень из числа 2 ($\sqrt{2}$).

\subsubsection{\lstinline|_math.acos <double>v : double|}

Арккосинус ($\arccos{v}$).

\subsubsection{\lstinline|_math.asin <double>v : double|}

Арксинус ($\arcsin{v}$).

\subsubsection{\lstinline|_math.atan <double>v : double|}

Арктангенс ($\arctan{v}$).

\subsubsection{\lstinline|_math.ceil <double>v : int|}

Наименьшее целое число больше или равна \code{v}.

\subsubsection{\lstinline|_math.cos <double>v : double|}

Косинус ($\cos{v}$).

\subsubsection{\lstinline|_math.degreesToRadians <double>v : double|}

Преобразует градусы в радианы.

\subsubsection{\lstinline|_math.exp <double>v : double|}

Функция экспонент ($\exp{v}$).

\subsubsection{\lstinline|_math.floor <double>v : int|}

Наибольшее целое число меньше или равна \code{v}.

\subsubsection{\lstinline|_math.ln <double>v : double|}

Натуральный логарифм ($\ln{v}$).

\subsubsection{\lstinline|_math.min <int ...>arr : int|}

Returnează наименьшее целое число.

\subsubsection{\lstinline|_math.min <double ...>arr : double|}

Returnează наименьшее дробное число.

\subsubsection{\lstinline|_math.max <int ...>arr : int|}

Returnează наибольшее целое число.

\subsubsection{\lstinline|_math.max <double ...>arr : double|}

Returnează наибольшее дробное число.

\subsubsection{\lstinline|_math.radiansToDegrees <double>v : double|}

Преобразует радианы в градусы.

\subsubsection{\lstinline|_math.round <double> : int|}

Returnează ближайшее целое число.

\subsubsection{\lstinline|_math.sin <double>v : double|}

Синус ($\sin{v}$).

\subsubsection{\lstinline|_math.tan <double>v : double|}

Тангенс ($\tan{v}$).

\subsection{{\color{orange} \_import}}

Obiectul \lstinline|_import| имеет следующие методы:
\begin{icItems}
	\item \lstinline|_import.none <object>data <string>path : void|;
	\item \lstinline|_import.none <string>path : void|;
	\item \lstinline|_import.functions <object>data <string>path : void|;
	\item \lstinline|_import.functions <string>path : void|;
	\item \lstinline|_import.all <object>data <string>path : void|;
	\item \lstinline|_import.all <string>path : void|;
	\item \lstinline|_import.run <string>path : void|.
\end{icItems}

Obiectul \code{data} позволяет передать данные в изолированном контексте, они там будут доступны как глобальные переменные. Это позволяет в одном файле хранить несколько версий библиотеки например, и при использовании указать какую версию загрузить.

\subsubsection{\lstinline|_import.none <object>data <string>path : void|}

Создаёт изолируемый контекст, в нём выполняет файл.

\subsubsection{\lstinline|_import.none <string>path : void|}

Акроним для \lstinline|_import.none [<>] @path|.

\subsubsection{\lstinline|_import.functions <object>data <string>path : void|}

Создаёт изолируемый контекст, в нём выполняет файл, потом импортирует все функции в текущем контексте.

\subsubsection{\lstinline|_import.functions <string>path : void|}

Акроним для \lstinline|_import.functions [<>] @path|.

\subsubsection{\lstinline|_import.all <object>data <string>path : void|}

Создаёт изолируемый контекст, в нём выполняет файл, потом импортирует все функции и глобальные переменный в текущем контексте.

\subsubsection{\lstinline|_import.all <string>path : void|}

Акроним для \lstinline|_import.all [<>] @path|.

\subsubsection{\lstinline|_import.run <string>path : void|}

Выполняет файл в текущем контексте.

\subsection{Модификатор {\color{blue2} reverse}}

Ключевое слово \code{reverse} позволяет изменить ход выполнения некоторых конструкций языка.

\subsubsection{\lstinline|reverse if|}

\lstinline|reverse if (@var == true)| эквивалентно \lstinline|if (!(@var == true))|, в этом случае уменьшения количестве скобок упрощает чтения кода. 

Внимание: \lstinline|reverse if| не принимает \lstinline|else|.

\subsubsection{\lstinline|reverse if exists|}

\lstinline|if exists| хорош тем что позволяет выполнить код только когда есть нужные данные. А что если нам нужно генерировать исключение когда нужные данные отсутствуют? Эту проблему решает \lstinline|reverse if exists|. В сравнении с \lstinline|if exists|, никакие данные не будут переданы в блоке команд.

\subsubsection{\lstinline|reverse for| - универсальный цикл}

В универсальном цикле меняется порядок выполнения действий. \lstinline|reverse for| гарантирует выполнение блока команд в минимум один раз.

Порядок действий конструкций \lstinline|for|:
\begin{icEnum}
	\item инициализация;
	\item проверка условий;
	\item выполнения блока команд;
	\item выполнения команды перехода на следующую итерацию.
\end{icEnum}

Порядок действий конструкций \lstinline|reverse for|:
\begin{icEnum}
	\item инициализация;
	\item выполнения блока команд;
	\item выполнения команды перехода на следующую итерацию;
	\item проверка условий.
\end{icEnum}

\subsubsection{\lstinline|reverse while| - условное повторение кода}

\lstinline|reverse while| будет выполнить блок команд, пока условие остаётся ложной.

\subsubsection{\lstinline|do reverse while| - цикл с постусловий}

\lstinline|reverse while| будет выполнить блок команд, пока условие остаётся ложной. Так же как и \lstinline|do while| гарантирует что тело цикла будет выполнена минимум 1 раз.

\subsubsection{\lstinline|reverse for| - прохождение коллекций}

\lstinline|reverse for| будет пройти коллекцию в обратном порядке.

\subsubsection{\lstinline|reverse filter| - выборочное прохождение коллекций}

\lstinline|reverse filter| будет пройти коллекцию в обратном порядке.

\subsubsection{\lstinline|reverse range| - частичное прохождение коллекций}

\lstinline|reverse filter| будет пройти интервал в обратном порядке.

%\newpage
