% !TeX spellcheck = ro_RO
\section{Introducere}

\indent \textbf{icL} este un \textbf{limbaj de script-are}, optimizat pentru descrierea scenariilor de testare al aplicaților web. Acest document (la fel ca și programul icL) este distribuit sub licența GNU GLPv3.

Limbajul icL are calități speciale și este creat be bază de filozofia icL. În acord cu ea codul trebuie să fie simplu, scurt, să asigure funcționalul necesar și să fie rezistent la erori (privește capitolul \ref{errorless-sec}).

\subsection{Cititorilor}

Acest document este desemnat celor, ce vor să înceapă să învețe limbajul icL. De asemenea acest document v-a fi folosit în procesul de dezvoltare al procesorului de comande, comportarea lui în situațiile care lipsesc se declară nedefinită.

Simțiți-vă liberi să indicați la erori și prezentarea ideilor și punctelor de vedere noi, aștept scrisorile voastre pe adresa {\bf icl@vivaldi.net}.

\subsection{Ce capabilități trebuiesc}

Înainte de a învăța icL, ar trebuie să aveți cunoștințe generale în programare.

\subsection{Recenzie}

icL este un limbaj de descriere al scenariilor de aplicațiilor web. Desolvatarea a început în anul 2017, lansarea primei versiuni se plănuiește în anul 2020. În timpul de față se află în dezvoltare activă.

icL are \textbf{sintaxă bazată pe sintaxa limbajului C}, care folosește tipizare statică. În icL nu se poate de definit tipuri de date proprii, pentru că el nu este destinat pentru programare, cunoștințele primite în timpul cursului de programare în școală ar trebui să fie destule. Limbajul icL suportă o singură paradigmă de programare - procedurală. În caz de necesitate de a prelucra date externe, se poate de folosit export/import în CSV și baze de date.

\subsection{Exemplu de cod}

În icL puntul de intrare în script este începutul fișierului, programul \textit{Hello world!} este demonstrat pe foia \ref{example0}.

\begin{listing}
    \captionof{listing}{Exemplu}
    \label{example0}
    \inputminted[linenos]{icl}{../sources/helloworld.icL}
\end{listing}

\subsection{Învățarea limbajului icL}

Cel mai important în învățarea limbajului icL este de a se concentra la idei și a nu atrage atenția la detaliile realizării.

\subsection{Unde se folosește icL}

Limbajul icL este o parte a \textbf{programei icL}, cu ajutorul ei se poate de controlat un browser, anume:
\begin{icItems}
\item
	a deschide un tab;
\item
	a închide un tab;
\item
	a trece la altă pagină;
\item
	a simula evenimentele mouse-ului și claviaturii;
\item
	a interacționa cu pagina web;
\item
	a executa cod javascript;
\item
	a controla pagina web;
\item
	a face schimb de informație cu pagina web;
\item
	a captura ecranul;
\item
	a controla memoria;
\item
	a exporta date in fișiere CSV;
\item
	a importa date din fișier CSV;
\item
	a executa comenzi SQL.
\end{icItems}

\subsection{Începutul lucrului}

Pentru a începe a lucra folosind limbajul icL e destul de a instala programul icL.
