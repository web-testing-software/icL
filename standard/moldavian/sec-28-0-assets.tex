% !TeX spellcheck = ro_RO
\section{Confirmații}

{\bf Confirmațiile} permit a aprecia cît de bine au se execută testele. Confirmațiile în icL sunt de 2 tipuri: \mintinline{icl}{assert (condition)} și \mintinline{icl}{assert (condition, message)}. În primul este prezentă doar condiție \mintinline{icl}{condition}, dacă ea este egală cu \false{} sau \void, atunci icL atunci icL va confirma că este o problemă în aplicația testată, trimițînd mesajul \mintinline{icl}{message}, dacă el lipsește se va trimite condiția în calitate de mesaj. Cum să trimiți corect același mesaj pentru cîteva condiții este demonstrat pe foaia \ref{assertexample}. Rezultatul confirmației va fi salvat numai dacă confirmația se execută în pas din test. Dacă afirmația e falsă, programa se oprește imediat.

\begin{sourcecode}
    \captionof{listing}{Exemple de folosire a confirmațiilor}
    \label{assertexample}
    \inputminted[linenos]{icl}{../sources/assertexample.icL}
\end{sourcecode}
