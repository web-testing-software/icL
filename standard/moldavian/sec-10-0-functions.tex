% !TeX spellcheck = ro_RO
\section{Funcții}

{\bf Funcțiile} în icL permit a folosi repetat fragmente de cod și a structura codul. Toate funcțiile în icL sunt globale.

\subsubsection{Definirea funcțiilor}

{\bf Definirea funcției} constă din header și corp.

\noindent Sintaxă -
\begin{lstlisting}[numbers=none]
!name = parameters : return_type {
	commands
}
\end{lstlisting}

Descrierea parților funcției:
\begin{icItems}
\item
	\code{name} - {\bf numele funcției};
\item
	\code{parameters} - {\bf lista parametrilor}. La chemarea funcției fiecare parametru primește o valoare. Lista parametrilor definește tipul, cantitatea și ordinea parametrilor. Lista nu este obligatorie (vezi foile \ref{fullfunc} și \ref{noargsfunc});
\item
	\code{return_type} - {\bf tipul de datelor} returnate de funcție. Dacă funcția nimic nu întoarce se tipul poate fi omis sau indicat clar - \void{}. Pentru a întoarce date se folosește comanda \code{@ = value} (vezi foile \ref{fullfunc}, \ref{notypefunc} și \ref{minfunc});
\item
	\code{commands} - {\bf corpul} conține comenzi, care definesc acțiunile executate de funcție.
\end{icItems}

\begin{lstlisting}[caption=Funcție completă, label=fullfunc]
!sum = <int>a <int>b : int {
	@ = @a + @b;
};
\end{lstlisting}

\begin{lstlisting}[caption=Funcție fără argumente, label=noargsfunc]
!pi = double {
	@ = 3.14;
};
\end{lstlisting}

\begin{lstlisting}[caption=Funcție făra tip de date, label=notypefunc]
!out = <int>a <int>b {
	_log.out @a @b;
};
\end{lstlisting}

\subsubsection{Apelul funcției}

Puteți să apelați funcția în felul următor -
\begin{lstlisting}[numbers=none]
!name arguments;
\end{lstlisting}
unde \code{name} este {\bf numele funcției}, \code{arguments} este {\bf lista argumentelor} (ea trebuie să fie compatibilă cu lista parametrilor). Cum se apelează funcțiile, definite pe foile \ref{fullfunc} - \ref{minfunc}, este demonstrat pe foaia \ref{callfunc}.

\begin{lstlisting}[caption=Funcție fără argumente și tip de date, label=minfunc]
!do = {
	_log.out "It's work!";
};
\end{lstlisting}

\begin{lstlisting}[caption=Apeluri de funcții, label=callfunc]
!sum 2 3; 	`` returns 5
!pi; 		`` returns 3.14
!out 4 5; 	`` returns void
!do; 		`` returns void

!sum (!sum 1 2) 3;	`` returns 6
!out 1 (!sum 4 5);	`` returns void
\end{lstlisting}

%\newpage
