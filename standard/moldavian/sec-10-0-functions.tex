% !TeX spellcheck = ro_RO
\section{Funcții}

{\bf Funcțiile} în icL permit a folosi repetat fragmente de cod și a structura codul. Toate funcțiile în icL sunt globale. Funcțiile se deosebesc de metodă prin ceea că ele sunt desemnate de utilizator, iar metodele sunt predefinite.

\subsubsection{Definirea funcțiilor}

{\bf Definirea funcției} constă din header și corp.

\noindent Sintaxă -
\begin{minted}{icl}
nume = (parametri) : tipDeDate {
	comenzi
}
\end{minted}

Descrierea parților funcției:
\begin{icItems}
\item
	\mintinline{icl}{nume} - {\bf numele funcției};
\item
	\mintinline{icl}{parametri} - {\bf lista parametrilor}. La chemarea funcției fiecare parametru primește o valoare. Lista parametrilor definește tipul, cantitatea și ordinea parametrilor. Lista nu este obligatorie (vezi foile \ref{fullfunc} și \ref{noargsfunc});
\item
	\mintinline{icl}{tipDeDate} - {\bf tipul de datelor} returnate de funcție. Dacă funcția nimic nu întoarce se tipul poate fi omis sau indicat clar - \void{}. Pentru a întoarce date se folosește comanda \mintinline{icl}{@ = valoare} sau \mintinline{icl}{Stack.return valoare} (vezi foile \ref{fullfunc}, \ref{notypefunc} și \ref{minfunc});
\item
	\mintinline{icl}{comenzi} - {\bf corpul} conține comenzi, care definesc acțiunile executate de funcție.
\end{icItems}

Parametrii se despart prin virgulă. Chiar dacă funcția deține doar on parametru, parantezele rotunde nu pot fi omise.

Parametrul se descriere prin următoarea sintaxă \mintinline{icl}{@name : type}, unde \mintinline{icl}{@name} este numele, iar \mintinline{icl}{type} e un tip de date. Se poate de descris numai tipul (fără doua puncte), în acest caz parametrul ca avea același nume ca și tipul de date.

Parametrul cu valoare implicită se descrie cu următoarea sintaxă \mintinline{icl}{@name = value}, unde \mintinline{icl}{@name} este numele, iar \mintinline{icl}{value} e valoarea implicită, ea pate fi o constantă, o variabilă sau o expresie. Valoarea expresiei va fi calculată numai o dată, la definirea funcției. Valoare predefinită pot avea orice parametri, nu numai ultimii.

\begin{sourcecode}
	\captionof{listing}{Funcție completă}
	\label{fullfunc}
	\inputminted[linenos]{icl}{../sources/fullfunc.icL}
\end{sourcecode}

\begin{listing}[p]
	\captionof{listing}{Funcție fără argumente}
	\label{noargsfunc}
	\inputminted[linenos]{icl}{../sources/noargsfunc.icL}
\end{listing}

\begin{listing}[p]
	\captionof{listing}{Funcție făra tip de date}
	\label{notypefunc}
	\inputminted[linenos]{icl}{../sources/notypefunc.icL}
\end{listing}

\begin{listing}[p]
	\captionof{listing}{Funcție fără argumente și tip de date}
	\label{minfunc}
	\inputminted[linenos]{icl}{../sources/minfunc.icL}
\end{listing}

\subsubsection{Apelul funcției}

Puteți să apelați funcția în felul următor -
\begin{minted}{icl}
name (arguments);
\end{minted}
unde \mintinline{icl}{name} este {\bf numele funcției}, \mintinline{icl}{arguments} este {\bf lista argumentelor} (ea trebuie să fie compatibilă cu lista parametrilor). Cum se apelează funcțiile, definite pe foile \ref{fullfunc} - \ref{minfunc}, este demonstrat pe foaia \ref{callfunc}.

\begin{listing}[h]
	\captionof{listing}{Apeluri de funcții}
	\label{callfunc}
	\inputminted[linenos]{icl}{../sources/callfunc.icL}
\end{listing}

Ultimii parametri cu valoare implicită pot fi omiși, restul parametrilor cu valoare implicită pot fi transmisa valoare \void{} (pentru a folosi valoare implicită). Valoarea \void{} poate fi descrisă cu un singur simbol - tilda (\mintinline{icl}{~}). De exemplu avem funcția \mintinline{icl}{func = (@var1 = 3, @var2 : int) {}}, ea poate fi apelată în felul următor \mintinline{icl}{fun (~, 3)}. Valorile implicite pot scurta codul, privește foaia \ref{defaultparametrs}.

\begin{listing}
	\captionof{listing}{Valori predefinite}
	\label{defaultparametrs}
	\inputminted[linenos]{icl}{../sources/defaultparametrs.icL}
\end{listing}

Funcțiile care dețin doar un parametru fără valoare implicită, pot fi apelate folosind acronimul \mintinline{icl}{name arg} sau \mintinline{icl}{name (arg)}. Poziția argumentului nu are importanță (la început, la mijloc sau la sfîrșit).
