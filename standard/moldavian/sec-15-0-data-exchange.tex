% !TeX spellcheck = ro_RO
\section{Schimbul de date cu pagina web}
\label{dataexchange}

\subsubsection{Titlul paginii}

Titlu pagini poate fi obținut cu ajutorul \lstinline|[r/*] Tab'title : string|.

\subsubsection{Codul sursă}

Codul sursă al paginii poate \lstinline|[r/o] Tab'source : string|.

\subsubsection{Captură de ecran}

Captura de ecran, codată în base64, poate fi obținută cu ajutorul \lstinline|[r/o] Tab'screenshot : string|.

\subsubsection{URL}

Adresa URL este disponibilă prin proprietatea \lstinline|[r/w] Tab'url : string|.

\subsubsection{Navigare}

\lstinline|Tab.back () : void| permite a se întoarce la pagina precedentă.

\lstinline|Tab.forward () : void| permite a se întoarce la pagina următoare.

\lstinline|Tab.refresh () : void|  permite a reîncărca pagina.

\lstinline|[icL] [r/o] Tab'canGoBack : bool| informează dacă se poate de se întors la pagina precedentă.

\lstinline|[icL] [r/o] Tab'canGoForward : bool| informează dacă se poate de se întors la pagina următoare.


\subsubsection{Avertizări}

\lstinline|[r/o] Alert'text : string| este textul avertizării

\lstinline|Alert.accept : void| închide avertizarea cu răspuns pozitiv.

\lstinline|Alert.dismiss : void| închide avertizarea cu răspuns negativ.

\lstinline|Alert.sendKeys <string>str : void| răspunde cu textul \code{str}.

%\newpage
