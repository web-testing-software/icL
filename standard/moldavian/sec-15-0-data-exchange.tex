% !TeX spellcheck = ro_RO
\section{Schimbul de date cu pagina web}
\label{dataexchange}

\subsubsection{Titlul paginii}

Titlu pagini poate fi obținut cu ajutorul \mintinline{icl}{[r/*] Tab'title : string}.

\subsubsection{Codul sursă}

Codul sursă al paginii poate \mintinline{icl}{[r/o] Tab'source : string}.

\subsubsection{Captură de ecran}

Captura de ecran, codată în base64, poate fi obținută cu ajutorul \\*\mintinline{icl}{[r/o] Tab'screenshot: string}.

\subsubsection{URL}

Adresa URL este disponibilă prin proprietatea \mintinline{icl}{[r/w] Tab'url : string}.

\subsubsection{Navigare}

\mintinline{icl}{Tab.back () : void} permite a se întoarce la pagina precedentă.

\mintinline{icl}{Tab.forward () : void} permite a se întoarce la pagina următoare.

\mintinline{icl}{Tab.refresh () : void}  permite a reîncărca pagina.

\mintinline{icl}{[icL] [r/o] Tab'canGoBack : bool} informează dacă se poate de se întors la pagina precedentă.

\mintinline{icl}{[icL] [r/o] Tab'canGoForward : bool} informează dacă se poate de se întors la pagina următoare.


\subsubsection{Avertizări}

\mintinline{icl}{[r/o] Alert'text : string} este textul avertizării

\mintinline{icl}{Alert.accept : void} închide avertizarea cu răspuns pozitiv.

\mintinline{icl}{Alert.dismiss : void} închide avertizarea cu răspuns negativ.

\mintinline{icl}{Alert.sendKeys (str : string) : void} răspunde cu textul \mintinline{icl}{str}.

%\newpage
