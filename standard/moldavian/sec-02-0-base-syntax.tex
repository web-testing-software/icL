% !TeX spellcheck = ro_RO
\section{Sintaxă de bază}

icL este destul de simplu în învățare, inserați și executați codul de pe foia \ref{first} în icL. Primul program se poate de salvat in fișier cu extensia \textit{icl}.

\begin{lstlisting}[caption=Prima programă, label=first]
Log.info "Test!";
\end{lstlisting}

În consolă ar trebui să apară următorul \textbf{text}:

\begin{lstlisting}[numbers=none]
Test!
\end{lstlisting}

\subsection{Import în icL}

Toate \textbf{bibliotecile standarde} sunt include in limbaj, dar se poate de inclus biblioteci \textbf{personalizate} folosind:

\begin{icItems}
\item
	\lstinline|Import.none "path/to/file.iclib"| - execută codul care se află în fișier, nu importă nimic;
\item
	\lstinline|Import.functions "path/to/file.iclib"| - execută codul și importă funcțiile; {\color{red}Important:} funcțiile importante nu ar trebui să folosească variabile globale;
\item
	\lstinline|Import.all "path/to/file.iclib"| -  execută codul, importa funcțiile și variabilele globale;
\item
	\lstinline|Import.run "path/to/file.iclib"| - execută codul în contextul curent, toate funcțiile si variabilele globale se importă și exportă.
\end{icItems}

\subsection{Semne în icL}

Programa în icL conține diferite \textbf{semne} (literale, construcții semantice). Semnul poate fi cuvînt cheie, identificator, constantă ori caracter. De exemple comanda următoare conține 4 semne: \lstinline|Log.info "Hello world!";|.

Semnele prezente:
\begin{icItems}
\item
	\lstinline`Log` - identificator de obiect;
\item
	\lstinline`.info` - identificator de metodă;
\item
	\lstinline`"Hello world!"` - literal, sir de caracter;
\item
	\lstinline`;` - delimitator, sfîrșit de comandă.
\end{icItems}

\subsection{Comentarii}

\textbf{Comentariu} este un text adjunct, care ajută să înțelegi scenariile descrise. Comentariile sunt ignorate de procesorul de comande.

\textbf{Comentariu în linie} (\textit{inline}) este un fragment de text limitat cu caractere speciale \texttt{`}, cum este arătat pe foaia \ref{inlinecomment}.

\begin{lstlisting}[caption=Comentariu în linie,label=inlinecomment]
No comment `comment` no comment
\end{lstlisting}

\textbf{Comentariu al liniei} este desemnat de începutul \texttt{``}, cum este prezentat pe foaia \ref{linecomment}.

\begin{lstlisting}[caption=Comentariu al liniei,label=linecomment]
No comment `` comment
\end{lstlisting}

\textbf{Comentariu în cîteva linii} se începe și se termină cu \texttt{```}, exemplu de acest comentariu este prezent pe foaia \ref{multilinecomment}.

\begin{lstlisting}[caption=Comentariu în cîteva linii,label=multilinecomment]
No comment
``` comment 1
	comment 2
	comment 3
``` No comment
\end{lstlisting}

\subsection{Identificatori}

\textbf{Identificatorul} în icL este un nume, folosit pentru a identifica o variabilă, o funcție, o metodă sau o proprietate. Identificatorul se începe cu un caracter special care descrie funcția lui (\lstinline`@`, \lstinline`#`, \lstinline`!`, \lstinline`_`, \lstinline`.` или \lstinline`'`), după care merg de la 1 la 32 de caracter: litere latine, diacritice și cifre.

icL diferențiază majuscule și minuscule. Așa că \textit{@var} și \textit{@Var} sunt identificatori diferiți. Cîteva exemple de identificatori:

\begin{lstlisting}[numbers=none]
#loop		Tab		.Append		'Length	DOM	@i	 	@VAR
@variable	!sumPoints	#global		.Merge	.Get	#01		!SIN
\end{lstlisting}

\subsection{Cuvinte cheie}

În icL \textbf{cuvintele cheie} nu sunt rezervate. Ele sunt doar 15: \lstinline`if`, \lstinline|else| \lstinline`for`, \lstinline`filter`, \lstinline`range`, \lstinline`exists`, \lstinline`while`, \lstinline`do`, \lstinline`any`, \lstinline`emit`, \lstinline`emiter`, \lstinline`slot`, \lstinline|jammer|, \lstinline|reverse| и \lstinline|assert|. În acest document ele sunt accentuate folosind culoarea albastră.

\subsection{Spații si delimitatori}

\textbf{Spațiu} (\textit{whitespace}) - este un termen, folosit în icL, pentru a însemna caracterul "spațiu", caracterul de tabulare, caracterul "rînd nou" și comentariu. Spațiile nu sunt obligatorii, ele sunt folosite pentru a face codul mai citabil. Pe foia \ref{unreadable} este prezent un exemplu de cod fără spații, dar pe foia \ref{readable} cu spații.

\begin{lstlisting}[caption=Cod fără spații,label=unreadable]
if(Tab.get"mai.ru"){(DOM.query"button").click;}else{Log.error"The site mai.ru is unaviable";};
\end{lstlisting}

În icL este prezent doar un singur delimitator - \textbf{delimitatorul de comenzi} \lstinline`;`. Comanda este o succesiune de semne, ordonate într-o ordine strictă și însemnînd anumite acțiuni. Exemple de comenzi: deschide site-ul - \lstinline`Tab.get "URL"`, închide tabul - \lstinline`Tab.close`.

În descrierea succesiunii de acțiuni, acțiunile trebuie despărțite. De exemplu succesiunea din comenzile precedente se descrie în felul următor -
\begin{lstlisting}[numbers=none]
Tab.get "URL"; Tab.close
\end{lstlisting}

\begin{lstlisting}[caption=Cod cu spații,label=readable]
`` the begin of program

`` try to go to mai.ru
if (Tab.get "mai.ru") {
	`` site loaded successfull
	`` click the button
	(DOM.query "button").click;
}
else {
	`` try again later
	`` now log the error
	Log.error "The site mai.ru is unaviable";
};

`` end of the program
\end{lstlisting}

În acest fel se descriu scenariile. Înainte de acolada închisă delimitatorul de comenzi poate lipsi.

\subsubsection{Informație adițională}

Dacă nu aveți cunoștințe în programare, vă rog să săriți imediat la următorul capitol.

În icL lipsesc delimitatorii dintre valori în listă. Exemple:

\begin{icItems}
	\item Inițializarea listei în С++:
\begin{lstlisting}[numbers=none, language=C++]
std::list<std::string> list = {"one", "two", "three"};
\end{lstlisting}
	Inițializarea listei în icL:
\begin{lstlisting}[numbers=none]
@list = ["one" "two" "three"];
\end{lstlisting}
	\item Funcție în С++:
\begin{lstlisting}[numbers=none, language=C++]
int sum (int number1, int number2) { return number1 + number2; };
\end{lstlisting}
	Funcție în icL:
\begin{lstlisting}[numbers=none]
!sum = <int>number1 <int>number2 : int { @ = number1 + number2 };
\end{lstlisting}
	\item Apelul funcției în С++:
\begin{lstlisting}[numbers=none, language=C++]
int s = sum (100, 200);
\end{lstlisting}
	Apelul funcției în icL:
\begin{lstlisting}[numbers=none]
@sum = !sum 100 200;
\end{lstlisting}
\end{icItems}
