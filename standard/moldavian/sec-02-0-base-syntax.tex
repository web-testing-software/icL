% !TeX spellcheck = ro_RO
\section{Sintaxă de bază}

icL este destul de simplu în învățare, inserați și executați codul de pe foia \ref{first} în icL. Primul program se poate de salvat in fișier cu extensia \textit{icL}.

\begin{sourcecode}
\captionof{listing}{Prima programă}
\label{first}
\begin{minted}[linenos]{icl}
Log.info "Test!";
\end{minted}
\end{sourcecode}

În consolă ar trebui să apară următorul \textbf{text}:

\begin{minted}{icl}
Test!
\end{minted}

\subsection{Import în icL}

Toate \textbf{bibliotecile standarde} sunt include în limbaj, dar se poate de inclus biblioteci \textbf{personalizate} folosind:

\begin{icItems}
\item
	\mintinline{icl}{Import.none "project/lib.icL"} - execută codul care se află în fișier, nu importă nimic;
\item
	\mintinline{icl}{Import.functions "project/lib.icL"} - execută codul și importă funcțiile; {\color{red}Important:} funcțiile importante nu ar trebui să folosească variabile globale;
\item
	\mintinline{icl}{Import.all "project/lib.icL"} -  execută codul, importa funcțiile și variabilele globale;
\item
	\mintinline{icl}{Import.run "project/lib.icL"} - execută codul în contextul curent, toate funcțiile și variabilele globale se importă și exportă.
\end{icItems}

\subsection{Semne în icL}

Programa în icL conține diferite \textbf{semne} (literale, construcții semantice, operatori, delimitatori, cuvinte cheie). Semnul poate fi cuvînt cheie, identificator, constantă ori caracter. De exemple comanda următoare conține 4 semne: \mintinline{icl}{Log.info "Hello world!";}.

Semnele prezente:
\begin{icItems}
\item
	\mintinline{icl}{Log} - identificator de obiect;
\item
	\mintinline{icl}{.info} - identificator de metodă;
\item
	\mintinline{icl}{"Hello world!"} - literal, șir de caracter;
\item
	\mintinline{icl}{;} - delimitator, sfîrșit de comandă.
\end{icItems}

\subsection{Comentarii}

\textbf{Comentariu} este un text adjunct, care ajută să înțelegi scenariile descrise. Comentariile sunt ignorate de procesorul de comande.

\textbf{Comentariu în linie} (\textit{inline}) este un fragment de text limitat cu caractere speciale \texttt{`}, cum este arătat pe foaia \ref{inlinecomment}.

\begin{sourcecode}
\captionof{listing}{Comentariu în linie}
\label{inlinecomment}
\begin{minted}[linenos]{icl}
No comment `comment` no comment
\end{minted}
\end{sourcecode}

\textbf{Comentariu al liniei} este desemnat de \texttt{``}, cum este prezentat pe foaia \ref{linecomment}.

\begin{sourcecode}
\captionof{listing}{Comentariu al liniei}
\label{linecomment}
\begin{minted}[linenos]{icl}
No comment `` comment
\end{minted}
\end{sourcecode}

\textbf{Comentariu în cîteva linii} se începe și se termină cu \texttt{```}, exemplu de acest comentariu este prezent pe foaia \ref{multilinecomment}.

\begin{sourcecode}
\captionof{listing}{Comentariu în cîteva linii}
\label{multilinecomment}
\begin{minted}[linenos]{icl}
No comment
``` comment 1
	comment 2
	comment 3
``` No comment
\end{minted}
\end{sourcecode}

\subsection{Identificatori}

\textbf{Identificatorul} în icL este un nume, folosit pentru a identifica o variabilă, o funcție, o metodă sau o proprietate. Identificatorul se începe cu un caracter special care descrie funcția lui (\mintinline{icl}{@}, \mintinline{icl}{#}, \mintinline{icl}{.}, \mintinline{icl}{'} sau literă, după care merg de la 1 la 32 de caractere: litere latine, diacritice, sublinia și cifre.

icL diferențiază majuscule și minuscule. Așa că \textit{@var} și \textit{@Var} sunt identificatori diferiți. Cîteva exemple de identificatori:

\begin{minted}{icl}
#loop		Tab	    	.append		'length	Doc 	@i	 	@VAR
@variable	sumPoints	#global		.merge	.get	#01		sin
\end{minted}

\subsection{Cuvinte cheie}

În icL \textbf{cuvintele cheie} sunt rezervate. Ele sunt doar 18: \mintinline{icl}{if}, \mintinline{icl}{else} \mintinline{icl}{for}, \mintinline{icl}{filter}, \mintinline{icl}{range}, \mintinline{icl}{exists}, \mintinline{icl}{while}, \mintinline{icl}{do}, \mintinline{icl}{any}, \mintinline{icl}{emit}, \mintinline{icl}{emiter}, \mintinline{icl}{slot}, \mintinline{icl}{jammer}, \mintinline{icl}{listen}, \mintinline{icl}{wait}, \mintinline{icl}{switch}, \mintinline{icl}{case} și \mintinline{icl}{assert}. În acest document ele sunt accentuate folosind culoarea albastră.

\subsection{Spații si delimitatori}

\textbf{Spațiu} (\textit{whitespace}) - este un termen, folosit în icL, pentru a însemna caracterul "spațiu", caracterul de tabulare, caracterul "rînd nou" și comentariu. Spațiile nu sunt obligatorii, ele sunt folosite pentru a face codul mai citabil. Pe foia \ref{unreadable} este prezent un exemplu de cod fără spații, dar pe foia \ref{readable} cu spații.

\begin{sourcecode}
\captionof{listing}{Cod fără spații}
\label{unreadable}
\begin{minted}[linenos]{icl}
if(Tab.get"mai.ru"){Doc.query("button").click();}else{Log.error"[..]";};
\end{minted}
\end{sourcecode}

În icL este prezent \textbf{delimitatorul de comenzi} \mintinline{icl}{;}, el poate fi ignorat, dar nu se recomandă. Comanda este o succesiune de semne, ordonate într-o ordine strictă și însemnînd anumite acțiuni. Exemple de comenzi: deschide site-ul - \mintinline{icl}{Tab.get "URL"}, închide tabul - \mintinline{icl}{Tab.close()}.

În descrierea succesiunii de acțiuni, acțiunile trebuie despărțite. De exemplu succesiunea din comenzile precedente se descrie în felul următor -
\begin{minted}{icl}
Tab.get "URL"; Tab.close()
\end{minted}

\begin{sourcecode}
\captionof{listing}{Cod cu spații}
\label{readable}
\begin{minted}[linenos]{icl}
`` the begin of program

`` try to go to mai.ru
if (Tab.get "mai.ru") {
	`` site loaded successfull
	`` click the button
	Doc.query("button").click();
}
else {
	`` try again later
	`` now log the error
	Log.error "The site mai.ru is unaviable";
};

`` end of the program
\end{minted}
\end{sourcecode}
