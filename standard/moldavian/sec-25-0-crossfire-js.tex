% !TeX spellcheck = ro_RO
\section{crossfire.js}

Tehnologia {\bf crossfire.js} permite a apela funcții icL din pagina web. Funcția va fi executată asincron.

A insera funcții icL în JavaScript se poate cu următorul înlocuitor \mintinline{icl}{!{name}}. La apelul funcției icL, parametri trebuie să fie compatibili, în caz contrar va fi generat un semnal care va opri executarea funcției, în urma căruia executarea scriptului va fi oprită. Pentru a garanta înțelegerea corectă a datelor se recomadă de folosit următoarele funcții JavaScript:
\begin{icItems}
	\item \mintinline{icl}{crossfire.bool(arg)};
	\item \mintinline{icl}{crossfire.int(arg)};
	\item \mintinline{icl}{crossfire.double(arg)};
	\item \mintinline{icl}{crossfire.string(arg)};
	\item \mintinline{icl}{crossfire.list(arg)};
	\item \mintinline{icl}{crossfire.object(arg)};
	\item \mintinline{icl}{crossfire.set(arg)};
	\item \mintinline{icl}{crossfire.element(arg)};
\end{icItems}

Exemplu elementar de apelare corectă a funcției icL este prezentat pe foaia \ref{crossfireexample}. Funcția \mintinline{icl}{onclick} primește un parametru, dar \mintinline{icl}{func} - nu. În cauză trebuie interface, cel mai simplu mod este de a defini o funcție anonimă.

\begin{sourcecode}
    \captionof{listing}{Exemplu de apelare a funcției icL}
    \label{crossfireexample}
    \inputminted[linenos]{icl}{../sources/crossfireexample.icL}
\end{sourcecode}

