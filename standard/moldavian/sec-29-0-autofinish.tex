% !TeX spellcheck = ro_RO
\section{Auto-finisarea comenzilor}

\textbf{Delimitatorul de comenzi} în icL poate fi omis, dar nu se recomandă. Dar pentru a face codul mai clar e mai bine de omis delimitatorul de comenzi după \code{if}, \code{else}, \code{while}, \code{do while}, \code{for}, \code{filter}, \code{range}. Diferența o puteți vedea comparînd foaia \ref{nodelimiterskipping} cu  \ref{delimiterskipping}.

\begin{lstlisting}[caption=Făra delimitatori omiși, label=nodelimiterskipping]
fun = () : int { @ = 23; };

if (fun() == 12) { Log.out 12345678; } 
else             { Log.out 87654321; };

emiter { func(@x); }
slot:Error1 { @x = 45; };
\end{lstlisting}

\begin{lstlisting}[caption=Cu delimitatori omiși, label=delimiterskipping]
fun = () : int { @ = 23 }

if (fun() == 12) { Log.out 12345678 } 
else             { Log.out 87654321 }

emiter { func(@x) }
slot:Error1 { @x = 45 }
\end{lstlisting}

Comanda poate fi finisată automat după următoarele construcții semantice:
\begin{icItems}
	\item literal;
	\item valoare;
	\item proprietate;
	\item pachet de valori;
	\item grup de comenzi.
\end{icItems}