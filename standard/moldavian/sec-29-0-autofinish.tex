% !TeX spellcheck = ro_RO
\section{Auto-finisarea comenzilor}

\textbf{Delimitatorul de comenzi} în icL poate fi omis, dar nu se recomandă. Dar pentru a face codul mai clar e mai bine de omis delimitatorul de comenzi după \mintinline{icl}{if}, \mintinline{icl}{else}, \mintinline{icl}{while}, \mintinline{icl}{do while}, \mintinline{icl}{for}, \mintinline{icl}{filter}, \mintinline{icl}{range}. Diferența o puteți vedea comparînd foaia \ref{nodelimiterskipping} cu  \ref{delimiterskipping}.

\begin{sourcecode}
    \captionof{listing}{Făra delimitatori omiși}
    \label{nodelimiterskipping}
    \inputminted[linenos]{icl}{../sources/nodelimiterskipping.icL}
\end{sourcecode}

\begin{sourcecode}
    \captionof{listing}{Cu delimitatori omiși}
    \label{delimiterskipping}
    \inputminted[linenos]{icl}{../sources/delimiterskipping.icL}
\end{sourcecode}

Comanda poate fi finisată automat după următoarele construcții semantice:
\begin{icItems}
	\item literal;
	\item valoare;
	\item proprietate;
	\item pachet de valori;
	\item grup de comenzi.
\end{icItems}
