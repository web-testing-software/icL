% !TeX spellcheck = ro_RO
\section{Interacționări adiționale}

Aceste posibilități sunt disponibile numai în regim de automatizare.

\subsection{Așteptare}

Pentru așteptare se propun următoarele funcții:
\begin{icItems}
	\item \lstinline|wait { condition }|;
	\item \lstinline|wait:Xs { condition }|;
	\item \lstinline|wait:Xms { condition }|;
	\item \lstinline|wait:ajax { condition }|.
\end{icItems}

\subsubsection{\lstinline|wait|}

Așteaptă pînă cînd condiția va returna \true. Timpul de așteptare: \code{Session'im\- plicitTimeout} milisecunde. 

\subsubsection{\lstinline|wait:Xs|}

Așteaptă pînă cînd condiția va returna \true. Timpul de așteptare: \code{X} secunde. 

\subsubsection{\lstinline|wait:Xms|}

Așteaptă pînă cînd condiția va returna \true. Timpul de așteptare: \code{X} milisecunde. 

\subsubsection{\lstinline|wait:ajax|}

Așteaptă pînă cînd condiția va returna \true. Timpul de așteptare: \code{Session'implicitTimeout} milisecunde. În comparație cu \code{wait}-ul obișnuit, aici vor fi transmise următoarele date: \code{@response} - datele răspunsului serverului și \code{@count} - numărul de request-uri active.

\subsection{Claviatură}

Pentru a lucra cu claviatura se propun următoarele funcții:
\begin{icItems}
	\item \lstinline|element.keyDown (modifiers : int, key : string) : element|;
	\item \lstinline|element.keyUp (modifiers : int, key : string) : element|;
	\item \lstinline|element.keyPress (modifiers : int, delay : int, keys : string) : element|;
	\item \lstinline|element.fastType (text : string) : element|;
	\item \lstinline|element.paste (text : string) : element|.
	\item \lstinline|element.forceType (text : string) : element|.
	\item \lstinline|Doc.type (text : string) : Doc|.
\end{icItems}

\subsubsection{\lstinline|element.keyDown (modifiers : int, keys : text) : element|}

Simulează apăsarea butonului.

\subsubsection{\lstinline|element.keyUp (modifiers : int, <text>) : element|}

Simulează eliberarea butonului.

\subsubsection{\lstinline|element.keyPress (modifiers : int, delay : int, keys : string) : element|}

Simulează procesul de culegere a textului cu intervalul \code{delay}.

\subsubsection{\lstinline|element.fastType (text : string) : element|}

Simulează procesul de culegere a textului pe claviatură cu viteză maximală.

\subsubsection{\lstinline|element.paste (text : string) : element|}

Copiază textul în clipboard, 

\subsubsection{\lstinline|element.forceType (text : string) : element|}

Adăugă valoare prin JavaScript.

\subsubsection{\lstinline|Doc.type (text : string) : element|}

Simulează procesul de culegere al textului fără a schimba focosul.

\subsection{Mouse}
\label{mouse}

Pentru a lucra cu mouse-ul se propun următoarele funcții:
\begin{icItems}
	\item \lstinline|element.click (data : object) : element|;
	\item \lstinline|element.forceClick (data : object) : element|;
	\item \lstinline|element.mouseDown (data : object) : element|;
	\item \lstinline|element.mouseUp (data : object) : element|;
	\item \lstinline|element.hover (data : object) : element|.
	\item \lstinline|Doc.click (data : object) : Doc|.
	\item \lstinline|Doc.mouseDown (data : object) : Doc|;
	\item \lstinline|Doc.mouseUp (data : object) : Doc|;
	\item \lstinline|Doc.hover (data : object) : Doc|.
\end{icItems}

În aceste funcții obiectul \code{data} poate avea următoarele cîmpuri:
\begin{icItems}
	\item \lstinline|[r/w] @data'button : int| e butonul mouse-lui (una din următoarele valori):
	\begin{icItems}
		\item \lstinline|[r/o] Mouse'left : 1| e butonul sțing;
		\item \lstinline|[r/o] Mouse'middle : 2| e butonul mijlociu;
		\item \lstinline|[r/o] Mouse'right : 3| e butonul drept.
	\end{icItems}
	Implicit \lstinline|Mouse'left|;
	\item \lstinline|[r/w] @data'rx : double| - \lstinline|ax = rx * width|;
	\item \lstinline|[r/w] @data'ry : double| - \lstinline|ay = rx * height|;
	\item \lstinline|[r/w] @data'ax : int| - coordonata x relativă la marginea stîngă a elementului;
	\item \lstinline|[r/w] @data'ay : int| - coordonata y relativă la marginea de sus a elementului.
\end{icItems}

\subsubsection{\lstinline|element.click (data : object) : element|}

Simulează un clic. Obiectul \code{data} deține următoarele proprietăți adiționale
\begin{icItems}
	\item \lstinline|[r/w] @data'delay : int| e intervalul între apăsarea și eliberarea butonului;
	\item \lstinline|[r/w] @data'count : int| e numărul de clicuri consecutive simulate.
\end{icItems}

\subsubsection{\lstinline|element.forceClick (data : object) : element|}

Simulează un click fără a controla dacă elementul este disponibil.

\subsubsection{\lstinline|element.mouseDown (data : object) : element|}

Simulează apăsarea butonului mouse-ului.

\subsubsection{\lstinline|element.mouseUp (data : object) : element|}

Simulează eliberarea butonului mouse-ului.

\subsubsection{\lstinline|element.hover (data : object) : element|}

Mută mouse-ul deasupra elementului. \code{data} are următoarele proprietăți adiționale:
\begin{icItems}
	\item \lstinline|[r/w] @data'time : int| e timpul mediu în care cursorul se mută la distanța de 100 de pixeli;
	\item \lstinline|[r/w] @data'func : int| e funcția care descrie mișcarea cursorului. Primește una din următoarele valori:
	\begin{icItems}
		\item \lstinline|[r/o] Move'teleport : 1| e mișcare simultană;
		\item \lstinline|[r/o] Move'linear : 2| e mișcare pe linie dreaptă;
		\item \lstinline|[r/o] Move'quadratic : 3| e mișcare pe curbă de ordinul 2;
		\item \lstinline|[r/o] Move'cubic : 4| e mișcare pe curbă de ordinul 3;
		\item \lstinline|[r/o] Move'bezier : 5| e mișcare pe curba Bezie de ordinul 3. Folosește următoarele proprietăți adiționale \code{p1x}, \code{p1y}, \code{p2x} și \code{p2y}. Dacă ele nu sunt indicate vor primi valori aleatorii din intervalul [0,0; 1,0].
	\end{icItems}
	Implicit \lstinline|Move'teleport|.
	\item \lstinline|[r/w] @data'p1x : double| - Coordinata $x$ primului punct.
	\item \lstinline|[r/w] @data'p1y : double| - Coordinata $y$ primului punct.
	\item \lstinline|[r/w] @data'p2x : double| - Coordinata $x$ punctului secundar.
	\item \lstinline|[r/w] @data'p2y : double| - Coordinata $y$ punctului secundar.
\end{icItems}

\subsubsection{\lstinline|Doc.click (data : object) : Doc|}

Simulează un click pe document.

\subsubsection{\lstinline|Doc.mouseDown (data : object) : Doc|}

Simulează apăsarea butonului mouse-ului pe document.

\subsubsection{\lstinline|Doc.mouseUp (data : object) : Doc|}

Simulează eliberarea butonului mouse-ului pe document.

\subsubsection{\lstinline|Doc.hover (data : object) : Doc|}

Simulează mutarea mouse-ului pe document.

\subsection{Setarea detaliilor}

Pot fi setați următorii parametri:
\begin{icItems}
	\item \lstinline|[r/w] icL'clickTime : int|;
	\item \lstinline|[r/w] icL'pressTime : int|;
	\item \lstinline|[r/w] icL'moveTime : int|;
	\item \lstinline|[r/w] icL'flashMode : bool|;
	\item \lstinline|[r/w] icL'humanMode : bool|.
\end{icItems}

\subsubsection{\lstinline|icL'clickTime : int|}

Intervalul clicului, implicit 300.

\subsubsection{\lstinline|icL'pressTime : int|}

Intervalul apăsării butonului pe claviatură, implicit 60.

\subsubsection{\lstinline|icL'moveTime : int|}

Timpul mediu în milisecunde în care cursorul se mută la distanța de 100 de pixeli, implicit 32.

\subsubsection{\lstinline|icL'flashMode : bool|}

Regim "Flash", toate operațiile vor fi executate maximum de rapid. Anulează \lstinline|icL'humanMode|. Implicit închis. \lstinline|element.sendKeys| și \lstinline|element.fastType| vor lucra la fel de rapid ca \lstinline|element.paste|, schimbarea conținutului clipboard-ului poate încurca la experiența de utilizare a calculatorului.

\subsubsection{\lstinline|icL'humanMode : bool|}

Regim "Om", toate operațiile for fi executate încet, cît mai asemănător cu comportarea omului. Anulează \lstinline|icL'flashMode|. Pe ecran va fi adăugat cursor, care va arăta poziția curentă a cursorului virtual. Implicit acest regim este închis.

%\newpage
