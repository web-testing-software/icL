\section{Утверждения}

{\bf Утверждения} позволяет оценить на сколько успешно прошел тест. Утверждения в icL имеют 2 синтаксисы: \lstinline|assert (condition)| и \lstinline|assert (condition, message)|. В первом присутствует только условие \code{condition}, если оно равно \false{} или \void, то icL будет утверждать что есть проблема в тестированным ПО, отправляя сообщение \code{message}, если оно отсутствует - условие будет отправлено в качестве сообщений. Как правильно отправлять одно и то же сообщение нескольких утверждений показано на листинге \ref{assertexample}.

\begin{lstlisting}[caption=Пример использования утверждениях, label=assertexample]
`` wrong
assert (@val != 2 || @val != 4 || @val != 127, "Message");

`` correct
@message = "Message";
assert (@val != 2,   @message);
assert (@val != 4,   @message);
assert (@val != 127, @message);
\end{lstlisting}
