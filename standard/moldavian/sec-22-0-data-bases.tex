% !TeX spellcheck = ro_RO
\section{Baze de date}

Modulul definește următoarele obiecte globale: \lstinline|_dbManager| este managerul de baze de date, \lstinline|_db| este ultima bază de date deschisă, \lstinline|_query| este ultimul query.

Modulul \textit{Baze de date} definește următoarele posibilități:
\begin{icItems}
	\item \lstinline|_dbManager.openSQLite <string>path : _db|;
	\item \lstinline|_db.query <_code>q : _query|;
	\item \lstinline|_db.close : void|;
	\item \lstinline|[w/o] _query'(<string>name) : any|;
	\item \lstinline|_query.set <string>field <any>value : void|;
	\item \lstinline|_query.exec : bool|;
	\item Disponibile după executarea query-ului:
	\begin{icItems}
		\item \lstinline|[r/o] _query'(<string>name) : any|;
		\item \lstinline|_query.getRowsAffected : int|;
		\item \lstinline|_query.getError : bool|;
		\item \lstinline|_query.getLength : int|;
		\item \lstinline|_query.get <string>field : any|;
		\item \lstinline|_query.next : bool|;
		\item \lstinline|_query.previous : bool|;
		\item \lstinline|_query.first : bool|;
		\item \lstinline|_query.last : bool|;
		\item \lstinline|_query.seek <int>i <bool>relative = false : bool|;
	\end{icItems}
\end{icItems}

\subsubsection{\lstinline|_dbManager.openSQLite <string>path : _db|}

Deschide o nouă conexiune. \code{path} e calea spre fișierul bazei de date.

\subsubsection{\lstinline|_db.query <_code>q : _query|}

Creează un query, pe bază de comandă SQL izolată în acolade.

\subsubsection{\lstinline|_db.close : void|}

Închide conexiunea cu baza de date.

\subsubsection{\lstinline|[w/o] _query'(<string>name) : any|}

Întoarce o valoare, care permite a schimba valoarea înlocuitorului prin atribuire. Înlocuitorul are următoarea sintaxă \lstinline|:name|.

\subsubsection{\lstinline|_query.set <string>field <any>value : void|}

Atribuie valoarea înlocuitorului \code{field}.

\subsubsection{\lstinline|_query.exec : bool|}

Returnează \true, dacă query-ului a fost executată cu succes, în caz contrar \false.

\subsubsection{\lstinline|_query.getError : string|}

Returnează textul erorii, dacă la executarea query-ului a apărut o eroare.

\subsubsection{\lstinline|[r/o] _query'(<string>name) : any|}

După executarea \lstinline|_query.exec| proprietățile vor returna valoarea cîmpurilor necesare. Dacă așa cîmp nu există, va fi returnat \void.

\subsubsection{\lstinline|_query.getRowsAffected : int|}

Numărul de rînduri afectate/adăugate în baza de date.

\subsubsection{\lstinline|_query.getLength : int|}

Numărul de rezultate primit în rezultatul executării comenzii \code{SELECT}.

\subsubsection{\lstinline|_query.get <string>field : any|}

Returnează valoarea cîmpului \code{field} sau \void{} dacă așa cîmp nu există.

\subsubsection{\lstinline|_query.next : bool|}

Returnează \true, dacă s-a primit să treacă la următorul rînd, în caz contrar \false.

\subsubsection{\lstinline|_query.previous : bool|}

Returnează \true, dacă s-a primit să treacă la rîndul precedent, în caz contrar \false.

\subsubsection{\lstinline|_query.first : bool|}

Returnează \true, dacă s-a primit să treacă la primul rînd, în caz contrar \false.

\subsubsection{\lstinline|_query.last : bool|}

Returnează \true, dacă s-a primit să treacă la ultimul rînd, în caz contrar \false.

\subsubsection{\lstinline|_query.seek <int>i <bool>relative = false : bool|}

Returnează \true, dacă s-a primit să treacă la al \code{i}-lea rînd sau să mute cursorul cu \code{i} rînduri (cu condiția că \code{@relative == true}), în caz contrar \false.

\subsubsection{Exemplu}

Exemplu de cod, care folosește baze de date este prezentat pe foaia \ref{dbexample}. La fel ca și la executarea codului Javascript, în cod pot să fie prezente variabile icL, variabilele locale au următoarea sintaxă \code{@:name}, dar variabilele globale - \code{\#:name}.

\newpage
\begin{lstlisting}[caption=Exemplu de cod care folosește baze de date, label=dbexample]
_dbManager.openSQLite "db.sqlite";

`` simple query
_db.query {
	SELECT country
	FROM artists
};
_query.exec;

while (_query.next) {
	!doSomething _query'country;
};

`` query with parameters
_db.query {
	SELECT name
	FROM programmers
	WHERE country = :country
	LIMIT 1
};
_query'country = "Moldova";
`` is equivalent to
@md = "Moldova";
_db.query {
	SELECT name
	FROM programmers
	WHERE country = @:md
	LIMIT 1
};

`` insert query
_db.query {
	INSERT INTO person (id, forname, name)
	VALUE (:id, :forname, :name)
};
_query'id = 1001;
_query'forname = "Bart";
_query'name = "Simpson";
_query.exec;
\end{lstlisting}

%\newpage
