% !TeX spellcheck = ro_RO
\section{Mecanismul de stări}

Toate variabilele icL se păstrează în containere de 2 tipuri: stivă și stare.

Stiva reprezintă containere grupate în stivă, fiecare container este strict legat de fragment de cod icL, la intrarea în fragment containerul se creează, iar la ieșire se distruge.

Starea reprezintă o listă dublă de containere, ele nu sunt legate de cod și sunt disponibile global. Implicit se creează un singur container, dar containere noi sunt create de utilizator.

\subsection{Controlul stivelor}

Pentru a controla stivele, se folosesc următoarele funcții și proprietăți:
\begin{icItems}
	\item \lstinline|[r/o] Stacks'(<int>i) : Stack|;
	\item \lstinline|[r/o] Stacks'(<string>name) : Stack|;
	\item \lstinline|Stacks.if <int>i = 0 : Stack|;
	\item \lstinline|Stacks.ifexists <int>i = 0 : Stack|;
	\item \lstinline|Stacks.loop <int>i = 0 : Stack|;
	\item \lstinline|Stacks.while <int>i = 0 : Stack|;
	\item \lstinline|Stacks.dowhile <int>i = 0 : Stack|;
	\item \lstinline|Stacks.for <int>i = 0 : Stack|;
	\item \lstinline|Stacks.forany <int>i = 0 : |;
	\item \lstinline|Stacks.range <int>i = 0 : Stack|;
	\item \lstinline|Stacks.filter <int>i = 0 : Stack|;
	\item \lstinline|Stack.clear : void|;
	\item \lstinline|Stack.continue : void|;
	\item \lstinline|Stack.break : void|.
\end{icItems}

\subsubsection{[r/o] \lstinline|Stacks'(<int>i) : Stack|}

Returnează al \code{i}-lea container. Containerele sunt repartizate pe principiul FILO, adică containerul cu numărul 0  (\lstinline|Stacks'0|) este containerul curent. El poate fi obținut printr-un mod mai simplu \lstinline|Stack|.

Excepții posibile: \ferror{OutOfBounds}.

\subsubsection{[r/o] \lstinline|Stacks'(<string>name) : Stack|}

Returnează containerul cu numele \code{name}. A numi containerul se poate în felul următor -
\begin{lstlisting}[numbers=none]
{:name
	`` content
}
\end{lstlisting}
Pot avea nume numai grupurile de comenzi (cod în acolade). Kodul în paranteze rotunde nu poate avea nume. Apelul funcții creează un container numele funcției.

\subsubsection{\lstinline|Stacks.if <int>i = 0 : Stack|}

Returnează al \code{i}-lea container \code{if}.

Excepții posibile: \ferror{OutOfBounds}.

\subsubsection{\lstinline|Stacks.ifexists <int>i = 0 : Stack|}

Returnează al \code{i}-lea container \code{if exists}.

Excepții posibile: \ferror{OutOfBounds}.

\subsubsection{\lstinline|Stacks.loop <int>i = 0 : Stack|}

Returnează al \code{i}-lea container \code{while}, \code{dowhile}, \code{for}, \code{range} sau \code{filter}.

Excepții posibile: \ferror{OutOfBounds}.

\subsubsection{\lstinline|Stacks.while <int>i = 0 : Stack|}

Returnează al \code{i}-lea container \code{while}.

Excepții posibile: \ferror{OutOfBounds}.

\subsubsection{\lstinline|Stacks.dowhile <int>i = 0 : Stack|}

Returnează al \code{i}-lea container \code{do while}.

Excepții posibile: \ferror{OutOfBounds}.

\subsubsection{\lstinline|Stacks.for <int>i = 0 : Stack|}

Returnează al \code{i}-lea container \code{for}.

Excepții posibile: \ferror{OutOfBounds}.

\subsubsection{\lstinline|Stacks.forany <int>i = 0 : |}

Returnează al \code{i}-lea container \code{for any}.

Excepții posibile: \ferror{OutOfBounds}.

\subsubsection{\lstinline|Stacks.range <int>i = 0 : Stack|}

Returnează al \code{i}-lea container \code{range}.

Excepții posibile: \ferror{OutOfBounds}.

\subsubsection{\lstinline|Stacks.filter <int>i = 0 : Stack|}

Returnează al \code{i}-lea container \code{filter}.

Excepții posibile: \ferror{OutOfBounds}.

\subsubsection{\lstinline|Stack'(<string>name) : any|}

Returnează variabila, care se află anume în acest container. De exemplu: deja există o variabilă cu numele \code{i} în containerul ascendent, trebui de definit o variabilă cu același nume în containerul curent. Problema se rezolvă în felul următor: \lstinline|Stack'i = 0| în loc de \lstinline|@i = 0|.

\subsubsection{\lstinline|Stack.clear : void|}

Șterge toate variabilele din container.

\subsubsection{\lstinline|Stack.continue : void|}

Trece la următoarea iterație, disponibil doar pentru cicluri.

Excepții posibile: \ferror{InvalidArgument}.

\subsubsection{\lstinline|Stack.break : void|}

Întrerupe imediat ciclul sau executarea funcției. Disponibil doar pentru cicluri și funcții.

Excepții posibile: \ferror{InvalidArgument}.

\subsection{Controlul stărilor}

Pentru a controla stările, se folosesc următoarele metode și proprietăți:
\begin{icItems}
	\item \lstinline|States.new : void|;
	\item \lstinline|States.newAtEnd : void|;
	\item \lstinline|States.toNext : void|;
	\item \lstinline|States.toPrev : void|;
	\item \lstinline|States.clear : void|;
	\item \lstinline|States.delete : void|;
	\item \lstinline|States.toFirst : void|;
	\item \lstinline|States.toLast : void|.
\end{icItems}

Obiectul \lstinline|States| este o colecție, înseamnă că ea poate fi răsfoită prin \lstinline|for (States) {}|. Suportul pentru alte cicluri lipsește.

\subsubsection{\lstinline|States.new <object>data = [<>] : void|}

Creează o nouă stare după starea curentă și schimbă focusul la ea. Toate datele obiectului \code{data} se adaugă în containerul nou.

\subsubsection{\lstinline|States.newAtEnd <object>data = [<>] : void|}

Creează o nouă stare la sfîrșit, care nu va fi primi focus.  Toate datele obiectului \code{data} se adaugă în containerul nou.

\subsubsection{\lstinline|States.toNext : bool|}

Returnează \true, dacă există o stare următoare și focusul a fost schimbat la ea, în caz contrar \false.

\subsubsection{\lstinline|States.toPrev : bool|}

Returnează \true, dacă există o stare precedentă și focusul a fost schimbat la ea, în caz contrar \false.

\subsubsection{\lstinline|States.clear : void|}

Șterge toate variabilele din stare.

\subsubsection{\lstinline|States.delete : void|}

Șterge starea, dacă există o stare următoare focusul trece la ea, în caz contrar dacă există o stare precedentă focusul trece la ea, iar dacă această este ultima, ea este doar curățită.

\subsubsection{\lstinline|States.toFirst : void|}

Schimbă focusul la prima stare.

\subsubsection{\lstinline|States.toLast : void|}

Schimbă focusul la ultima stare.

\subsubsection{\lstinline|for (States) \{\}|}

Îndeplinește ciclul pentru fiecare stare în parte.

%\newpage
