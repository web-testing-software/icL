% !TeX spellcheck = ro_RO
\section{Operatori condiționali}

Operatorii condiționali dețin condiții (valoarea cărora trebuie socotită) și grupuri de comenzi, execuția cărora depinde de condiție.

În icL sunt prezenți următorii operatori logici:
\begin{icItems}
	\item \code{if};
	\item \code{if-else};
	\item \code{if-else} în cascadă;
	\item \code{switch-case};
\end{icItems}

\subsubsection{\lstinline|if|}

\code{if} permite să condiționezi executarea grupei de comenzi. Sintaxa lui -
\begin{lstlisting}[numbers=none]
if (condition) {
	commands
};
\end{lstlisting}
unde \code{condition} orice expresie care returnează valoare \bool{} și \code{commands} orice succesiune de comenzi.

De asemenea se permite de exclus parantezele rotunde dacă expresia este simplă, sintaxă -
\begin{lstlisting}[numbers=none]
if true  { commands };
if !true { commands };
if @var  { commands };
if !@var { commands };
\end{lstlisting}

\subsubsection{\lstinline|if-else|}

\code{if-else} permite a alege dintre 2 grupe de comenzi, în cazul în care condiția este adevărată se execută prima, în caz contrar a doua.

Sintaxa \code{if-else}-ului -
\begin{lstlisting}[numbers=none]
if (condition) {
	commands1
}
else {
	commands2
};
\end{lstlisting}

\subsubsection{\lstinline|if-else| în cascadă}

\code{if-else} în cascadă permite a alege dintre n grupe de comenzi, numai că necesită n-1 condiții.

Sintaxa \code{if-else}-ului în cascadă -
\begin{lstlisting}[numbers=none]
if (condition1) {
	commands1
} else if (condition2) {
	commands2
} else {
	commands3
};
\end{lstlisting}

\subsubsection{\lstinline|switch-case|}

\code{switch-case} permite a alege între n blocuri de comenzi, pe bază de n condiții.

Синтаксис -
\begin{lstlisting}[numbers=none]
switch (value) {
	case (caseValue) { ```code``` }
	`` more cases
}
\end{lstlisting}

\code{switch} primește valoarea \code{value}, ea poate avea orice tip în arară de \bool. Mai departe în \code{switch} se definesc cîteva \code{case}-uri. Dacă valoarea \code{case}-ului - \code{caseValue} coincide cu \code{value}, arunci \code{case} va fi executat.

\code{case} poate deține mai multe valori despărțite prin virgulă.

El poate primi valoare \void, în felul următor \lstinline|case (~) { ```code``` }|, acest tip de \code{case} va fi executat numai dacă nici unul din \code{case}-urile precedente nu a fost executat.

Este a valoare specială - \code{#}, prin ea se desemnează ca \code{case}-ul trebuie executat dacă precedentul a fost executat.

Descurcăm prin exemplu toate posibilitățile \code{switch-case}-ului, exemplu este prezent pe foaia \ref{switchcaseex}.

\begin{lstlisting}[caption=Folosirea switch-case-ului, label=switchcaseex]
@v = 0 | 1 | 2 | 3 | 4 | 5;

switch (@v) {
	case (0) { ```code1``` }
	case (1, 2) { ```code2``` }
	case (#, 3) { ```code3``` }
	case (4) { ```code4``` }
	case (~) { ```code5``` }
}
\end{lstlisting}

Variabila \code{@v} primește valori in intervalul [0, 6].

Descurcăm care grupuri de comenzi vor fi executate pentru fiecare valoare posibilă:
\begin{icItems}
	\item 0 - va fi executat grupul \code{code1};
	\item 1 - vor fi executate grupurile \code{code2} și \code{code3};
	\item 2 - vor fi executate grupurile \code{code2} și \code{code3};
	\item 3 - va fi executat grupul \code{code3};
	\item 4 - va fi executat grupul \code{code4};
	\item 5 - va fi executat grupul \code{code5};
	\item 6 - va fi executat grupul \code{code5}.
\end{icItems}


