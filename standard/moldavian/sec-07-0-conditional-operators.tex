% !TeX spellcheck = ro_RO
\section{Operatorii condiționali}

Construcțiile condiționale dețin condiții, valoarea cărora trebuie socotită, și grupuri de comenzi, execuția cărora depinde de condiție.

În icL sunt prezenți următorii operatori logici:
\begin{icItems}
	\item \code{if};
	\item \code{if-else};
	\item \code{if-else} în cascadă;
	\item \code{exists};
	\item \code{if-exists};
	\item \code{for-any}.
\end{icItems}

\subsubsection{\lstinline|if|}

\code{if} permite să condiționezi executarea grupei de comenzi. Sintaxa lui -
\begin{lstlisting}[numbers=none]
if (condition) {
	commands
};
\end{lstlisting}
unde \code{condition} orice expresie care returnează valoare \bool{} și \code{commands} orice succesiune de comenzi.

De asemenea se permite de exclus parantezele rotunde dacă expresia este simplă, sintaxă -
\begin{lstlisting}[numbers=none]
if true  { commands };
if @var  { commands };
\end{lstlisting}

\subsubsection{\lstinline|if-else|}

\code{if-else} permite a alege dintre 2 grupe de comenzi, în cazul în care condiția este adevărată se execută prima, în caz contrar a doua.

Sintaxa \code{if-else}-ului -
\begin{lstlisting}[numbers=none]
if (condition) {
	commands1
}
else {
	commands2
};
\end{lstlisting}

\subsubsection{\lstinline|if-else| în cascadă}

\code{if-else} în cascadă permite a alege dintre n grupe de comenzi, numai că necesită n-1 condiții.

Sintaxa \code{if-else}-ului în cascadă -
\begin{lstlisting}[numbers=none]
if (condition1) {
	commands1
} else if (condition2) {
	commands2
} else {
	commands3
};
\end{lstlisting}

\subsubsection{\lstinline|exists|}

\code{exists} permite a returna date condițional, filtrînd informația după criteriile dorite.

Condiții implicite:
\begin{icItems}
\item
	pentru \bool{} - \code{# == true};
\item
	pentru \integer{} - \code{# != 0};
\item
	pentru \double{} - \code{# != 0.0};
\item
	pentru \str{} - \code{!#'empty};
\item
	pentru \listtype{} - \code{!#'empty};
\item
	pentru \set{} - \code{!#'empty};
\item
	pentru \element{} - \code{#'empty}.
\end{icItems}

În cazul in care se folosește condiția implicită, se folosește următoarea sintaxă - \lstinline|exists(expression)|.
Dacă e necesar a instala o condiție proprie, sintaxa oleacă se schimbă -\lstinline|exists(expression, condition)|.

\subsubsection{\lstinline|if-exists|}

Construcția \code{if-exists} permite a execută o grupă de comenzi în condiție de rezultatul \code{exists}-ului și de a folosi valoarea repetat. În cazul în care condiția \code{exists}-ului e adevărată, se execută grupa de comenzi și se transmitwe valoare returnată de \code{exists} sub numele \code{@}.
Exemplu simplu și elementar este demonstrat pe foaia \ref{ifexistsex}.

\subsubsection{\lstinline|for-any|}

\code{for-any} permite a folosi repetat orice valoare. Exemplu de folosire este demonstrat pe foaia \ref{foranyex}.

\begin{lstlisting}[caption=Folosirea if-exist-ului, label=ifexistsex]
if exists(23 + 3, # > 20) {
_log.out "@ = " @; `` @ = 26
};
\end{lstlisting}

\begin{lstlisting}[caption=Folosirea for-any-ului, label=foranyex]
for any(23 + 3) {
	_log.out "@ = " @; `` @ = 26
};
\end{lstlisting}
