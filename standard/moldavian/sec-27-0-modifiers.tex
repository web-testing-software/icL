% !TeX spellcheck = ro_RO

\section{Modificatori de control}

\label{sec-modifiers}

Modificatorii de control permit a schimba calea de îndeplinire a unor construcții.

\subsubsection{\lstinline|if:not|}

\lstinline|if:not (@var == true)| este echivalent cu \lstinline|if (!(@var == true))|, doar în acest caz se micșorează numărul de paranteze și codul arată mai simplu.

\subsubsection{\lstinline|for:alt| - ciclu universal alternativ}

În ciclul universal se schimbă ordinea de îndeplinire al operațiilor. \lstinline|for:alt| garantează ca grupa de comenzi va fi executat cel puțin o dată.

Ordinea acțiunilor pentru \lstinline|for|:
\begin{icEnum}
    \item inițializare;
	\item controlul condiției;
	\item executarea grupei de comenzi;
	\item trecerea la următoarea iterație.
\end{icEnum}

Ordinea acțiunilor pentru \lstinline|for:alt|:
\begin{icEnum}
    \item inițializare;
	\item executarea grupei de comenzi;
	\item trecerea la următoarea iterație;
	\item controlul condiției.
\end{icEnum}

\subsubsection{\lstinline|while:not| - executare condiționată repetată}

\lstinline|while:not| va executa drupa de comenzi atît timp, cît condiția va rămîne falsă.

\subsubsection{\lstinline|do while:not| - ciclu cu condiție posterioară}

\lstinline|do-while:not| va executa drupa de comenzi atît timp, cît condiția va rămîne falsă. La fel ca \lstinline|do-while| garantează că grupa de comenzi va fi executată minimum o dată.

\subsubsection{\lstinline|for:reverse| - răsfoirea colecție}

\lstinline|for:reverse| va răsfoi colecția de la sfîrșit la început.

\subsubsection{\lstinline|filter:reverse| - răsfoirea selectivă a colecției}

\lstinline|filter:reverse| va răsfoi colecția de la sfîrșit la început.

\subsubsection{\lstinline|range:reverse| - răsfoirea parțială a colecției}

\lstinline|filter:reverse| va răsfoi colecția de la sfîrșit la început.

\subsubsection{\lstinline|emiter:alive| - a doua răsuflare}

În lista tehnologiilor \textit{icL} este prezentă tehnologia \textit{icL ALive}. Ea permite a restaura abilitatea de lucra a procesorului de comenzi după eșec, dar dacă excepția este prinsă de construcția \code{emiter-slot}, excepția va fi prelucrată de ea. Dacă este necesar ca excepția să fie prelucrată de \textit{icL ALive} atunci cînd este posibil, folosiți modificatorul \lstinline|emiter:alive|.

Lista tehnologiilor \textit{icL} este disponibilă pe linkul: \ferror{https://gitlab.com/lixcode/icL/tree/standard/technologies\#technologies}.

\subsubsection{\lstinline|listen:ignore| - regim "offline"}

Construcția \lstinline|listen| este o parte a tehnologiei \textit{icL-Sync}, \lstinline|listen:ignore| permite a crea spion, care nu va spiona de la inițializare, dar va aștepta apelul metodei \lstinline|Stack.listen()|. Mai multă informație poate fi găsită in capitolul \ref{sync}.