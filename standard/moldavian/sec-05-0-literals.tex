% !TeX spellcheck = ro_RO
\section{Literale}

Valorile constante, care sunt prezente în cod ca o parte a lui, se numesc {\bf literale}.

Literalul poate fi o valoare din tipurile următoare:

\begin{icItems}
	\item
		valoare booleană - \bool{};
	\item
		număr întreg - \integer{};
	\item
		număr decimal - \double{};
	\item
		șir de caractere - \str{};
	\item
		listă - \listtype{};
	\item
		obiect - \object{};
	\item
		mulțime - \set{}.
\end{icItems}

\subsubsection{Valori booleene}

Pentru definirea {\bf valorilor booleene} se folosesc următoarele literale:
\begin{icItems}
	\item \true{} - adevărat;
	\item \false{} - fals.
\end{icItems}

\subsubsection{Numere întregi}

Pentru definirea {\bf numerelor întregi} se folosesc succesiuni de cifre, înaintea cărora poate fi prezent minus. Între minus și cifre nu ar trebui să fie spațiu, în caz contrar minusul va fi interpretat ca operator.

\noindent Exemple:
\begin{lstlisting}[numbers=none]
23; -23; - 23; +23 + 3; 12 + -34; 15 - 24; 89--56; 2-3; `` ok
23-; 23+; -2А; 3f5; 23f; 23l; 12u; 89i; 2w1; 1q1; rt2;  `` error
\end{lstlisting}

\subsubsection{Numere decimale}

{\bf Literalul decimal} constă din 2 părți: întreagă și fracțională. Ele se despart prin minus. Fiecare parte este un literal întreg. Partea fracțională nu poate fi negativă.

\noindent Exemple -
\begin{lstlisting}[numbers=none]
23.233452; 29229992.2391; 100.0; -23.29199; -0.23; -0.45 - 1000.5;  `` ok
23.-4; 3а.34; 23-.44; 34.+23; -25.f; -23.5f; -w.45; -2.4e10; -2.E2; `` error
\end{lstlisting}

\subsubsection{Șiruri de caractere}

{\bf Șirul de caractere} este o succesiune de caractere, limitată din ambele părți cu ghilimele \lstinline|"|. Pentru a adăuga ghilimele \lstinline|"| în șir se folosește succesiunea \lstinline|\"|. Alte simboluri de asemenea se scriu ca succesiuni din altele două: simbolul de tabulare - \lstinline|\t|, simbolul {\it linie nouă} - \lstinline|\n|, simbolul {\it pas în urmă} - \lstinline|\b|, \textbackslash \ - \lstinline|\\|.

\noindent Exemple -
\begin{lstlisting}[numbers=none]
"Hello \"to\" you!"; "Line1\nLine2"; "Tag1\n\bTag2\n\b"; "text"; "\\ \\ \n \\ \\";
\end{lstlisting}

\subsubsection{Liste}

{\bf Literalul listei} este o succesiune de șiruri de caractere, limitată în paranteze pătrate.

\noindent Exemple -
\begin{lstlisting}[numbers=none]
@fruits = ["Apple" "Mango" "Banana" "Lime" "Lemon" "Olive"];
@vegetables = ["Cress" "Mustard" "Guar" "Soybean" "Leek" "Ahipa"];
\end{lstlisting}

\subsubsection{Obiecte}

{\bf Obiectul} este o uniune din cîteva variabile sub un nume, variabila definită în obiect se numește {\it cîmp}. Literalul {\it cîmpului} are sintaxa următoare \lstinline|<value>name|, unde  {\it value} e o valoare și {\it name} e numele variabilei. În icL cîmpurile neinițializate, ca și variabilele neinițializate, nu se poate de definit.

\noindent Exemple -
\begin{lstlisting}[numbers=none]
@quotation = [<"author">author <"text">text];
@child = [<4>age <true>hasBrothers <true>hasParents];
@file = [<false>isEmpty <25220>size <true>readOnly];
\end{lstlisting}

\subsubsection{Mulțimi}

Numai {\bf mulțimile vide} pot fi definite cu literal, literalul este foarte asemănător cu literalul obiectelor, numai că în loc de date se indică tipuri de date.

\noindent Exemple -
\begin{lstlisting}[numbers=none]
@quotations = [<string>author <string>text];
@children = [<int>age <bool>hasBrothers <bool>hasParents];
@files = [<bool>isEmpty <int>size <bool>readOnly];
\end{lstlisting}

%\newpage
