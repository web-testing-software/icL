% !TeX spellcheck = ro_RO
\section{Baze de date}

Modulul definește următoarele obiecte globale: \mintinline{icl}{DBManager} este managerul de baze de date, \mintinline{icl}{DB} este ultima bază de date deschisă, \mintinline{icl}{Query} este ultimul query.

Modulul \textit{Baze de date} definește următoarele posibilități:
\begin{icItems}
	\item \mintinline{icl}{DBManager.openSQLite (path : string) : DB};
	\item \mintinline{icl}{DBManager.connect (server : string, user : string, password : string) : DB};
	\item \mintinline{icl}{DBManager.connect (server : string) : DB};
	\item \mintinline{icl}{DB.query (q : Code) : Query};
	\item \mintinline{icl}{DB.close () : void};
	\item \mintinline{icl}{[w/o] Query'(name : string) : any};
	\item \mintinline{icl}{Query.set (field : string, value : any) : void};
	\item \mintinline{icl}{Query.exec () : bool};
	\item Disponibile după executare:
	\begin{icItems}
		\item \mintinline{icl}{[r/o] Query'(name : string) : any};
		\item \mintinline{icl}{Query.getRowsAffected () : int};
		\item \mintinline{icl}{Query.getError () : bool};
		\item \mintinline{icl}{Query.getLength () : int};
		\item \mintinline{icl}{Query.get (field : string) : any};
		\item \mintinline{icl}{Query.next () : bool};
		\item \mintinline{icl}{Query.previous () : bool};
		\item \mintinline{icl}{Query.first () : bool};
		\item \mintinline{icl}{Query.last () : bool};
		\item \mintinline{icl}{Query.seek (i : int, relative = false) : bool};
	\end{icItems}
\end{icItems}

\subsubsection{\mintinline{icl}{DBManager.openSQLite (path : string) : DB}}

Deschide o nouă conexiune. \mintinline{icl}{path} e calea spre fișierul bazei de date.

\subsubsection{\mintinline{icl}{DBManager.connect (server : string, user : string, password : string) : DB}}

Se conectează la icL DB Client prin WebSockets, clienții pot fi scriși in orice limbaj de programare, pentru a crea clienți proprii faceți cunoștință cu \textit{icL DataBase Share over WebSockets}, descris în \textit{icL Wire Protocol}.

Excepții posibile: \ferror{NoSuchServer} și \ferror{WrongUserPassword} (pr. tab. \ref{errors}).

\subsubsection{\mintinline{icl}{DBManager.connect (server : string) : DB}}

Se conectează la icL DB Client prin HTTP, clienții pot fi scriși in orice limbaj de programare, pentru a crea clienți proprii faceți cunoștință cu \textit{icL DataBase Share over HTTP}, descris în \textit{icL Wire Protocol}.

Excepții posibile: \ferror{NoSuchServer} (pr. tab. \ref{errors}).

\subsubsection{\mintinline{icl}{DB.query (q : Code) : Query}}

Creează un query, pe bază de comandă SQL izolată în acolade.

\subsubsection{\mintinline{icl}{DB.close () : void}}

Închide conexiunea cu baza de date.

\subsubsection{\mintinline{icl}{[w/o] Query'(name : string) : any}}

Întoarce o valoare, care permite a schimba valoarea înlocuitorului prin atribuire. Înlocuitorul are următoarea sintaxă \mintinline{icl}{:name}.

\subsubsection{\mintinline{icl}{Query.set (field : string, value : any) : void}}

Atribuie valoarea înlocuitorului \mintinline{icl}{field}.

\subsubsection{\mintinline{icl}{Query.exec () : bool}}

Returnează \true, dacă query-ului a fost executată cu succes, în caz contrar \false.

\subsubsection{\mintinline{icl}{Query.getError () : string}}

Returnează textul erorii, dacă la executarea query-ului a apărut o eroare.

\subsubsection{\mintinline{icl}{[r/o] Query'(name : string) : any}}

După executarea \mintinline{icl}{Query.exec} proprietățile vor returna valoarea cîmpurilor necesare. Dacă așa cîmp nu există, va fi returnat \void.

\subsubsection{\mintinline{icl}{Query.getRowsAffected () : int}}

Numărul de rînduri afectate/adăugate în baza de date.

Excepții posibile: \ferror{QueryNotExecutedYet} (pr. tab. \ref{errors}).

\subsubsection{\mintinline{icl}{Query.getLength () : int}}

Numărul de rezultate primit în rezultatul executării comenzii \mintinline{icl}{SELECT}.

Excepții posibile: \ferror{QueryNotExecutedYet} (pr. tab. \ref{errors}).

\subsubsection{\mintinline{icl}{Query.get (field : string) : any}}

Returnează valoarea cîmpului \mintinline{icl}{field} sau \void{} dacă așa cîmp nu există.

Excepții posibile: \ferror{QueryNotExecutedYet} (pr. tab. \ref{errors}).

\subsubsection{\mintinline{icl}{Query.next () : bool}}

Returnează \true, dacă s-a primit să treacă la următorul rînd, în caz contrar \false.

Excepții posibile: \ferror{QueryNotExecutedYet} (pr. tab. \ref{errors}).

\subsubsection{\mintinline{icl}{Query.previous () : bool}}

Returnează \true, dacă s-a primit să treacă la rîndul precedent, în caz contrar \false.

Excepții posibile: \ferror{QueryNotExecutedYet} (pr. tab. \ref{errors}).

\subsubsection{\mintinline{icl}{Query.first () : bool}}

Returnează \true, dacă s-a primit să treacă la primul rînd, în caz contrar \false.

Excepții posibile: \ferror{QueryNotExecutedYet} (pr. tab. \ref{errors}).

\subsubsection{\mintinline{icl}{Query.last () : bool}}

Returnează \true, dacă s-a primit să treacă la ultimul rînd, în caz contrar \false.

Excepții posibile: \ferror{QueryNotExecutedYet} (pr. tab. \ref{errors}).

\subsubsection{\mintinline{icl}{Query.seek (i : int, relative = false) : bool}}

Returnează \true, dacă s-a primit să treacă la al \mintinline{icl}{i}-lea rînd sau să mute cursorul cu \mintinline{icl}{i} rînduri (cu condiția că \mintinline{icl}{@relative == true}), în caz contrar \false.

Excepții posibile: \ferror{QueryNotExecutedYet} (pr. tab. \ref{errors}).

\subsubsection{Exemplu}

Exemplu de cod, care folosește baze de date este prezentat pe foaia \ref{dbexample}. La fel ca și la executarea codului Javascript, în cod pot să fie prezente variabile icL, variabilele locale au următoarea sintaxă \mintinline{icl}{@:nume}, dar variabilele globale - \mintinline{icl}{#:nume}.

\newpage
\begin{sourcecode}
    \captionof{listing}{Exemplu de cod care folosește baze de date}
    \label{dbexample}
    \inputminted[linenos]{icl}{../sources/dbexample.icL}
\end{sourcecode}

%\newpage
