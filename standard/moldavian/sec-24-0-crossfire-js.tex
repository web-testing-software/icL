% !TeX spellcheck = ro_RO
\section{crossfire.js}

Tehnologia {\bf crossfire.js} permite a apela funcții icL din pagina web. Funcția va fi executată asincron.

A insera funcții icL în JavaScript se poate cu următorul înlocuitor \lstinline|!{name}|. La apelul funcției icL, parametri trebuie să fie compatibili, în caz contrar va fi generat un semnal care va opri executarea funcției, în urma căruia executarea scriptului va fi oprită. Pentru a garanta înțelegerea corectă a datelor se recomadă de folosit următoarele funcții JavaScript:
\begin{icItems}
	\item \lstinline|crossfire.bool(arg)|;
	\item \lstinline|crossfire.int(arg)|;
	\item \lstinline|crossfire.double(arg)|;
	\item \lstinline|crossfire.string(arg)|;
	\item \lstinline|crossfire.list(arg)|;
	\item \lstinline|crossfire.object(arg)|;
	\item \lstinline|crossfire.set(arg)|;
	\item \lstinline|crossfire.element(arg)|;
\end{icItems}

Exemplu elementar de apelare corectă a funcției icL este prezentat pe foaia \ref{crossfireexample}. Funcția \code{onclick} primește un parametru, dar \code{func} - nu. În cauză trebuie interface, cel mai simplu mod este de a defini o funcție anonimă.

\begin{lstlisting}[caption=Exemplu de apelare a funcției icL, label=crossfireexample]
@el = Doc.query "button";
!func = { Log.info "It's working!" };

$run {
	@{el}[0].onlick = function (ev) { !{func}() }
};
\end{lstlisting}

