% !TeX spellcheck = ro_RO
\section{Liste}

{\bf Listele} (tipul \code{list}) permit a salva cîteva șiruri de caractere într-o variabilă. Accesul la șiruri se realizează prin index - numărul de ordine al șirului în listă.

\subsection{Proprietăți}

Listele dețin următoarele proprietăți adiționale:
\begin{icItems}
\item
	\lstinline|[r/o] list'empty : bool|;
\item
	\lstinline|[r/o] list'length : int|;
\item
	\lstinline|[r/o] list'last : string|;
\item
	\lstinline|[r/o] list'(n : int) : string|.
\end{icItems}

Pe foaia \ref{listprop} este prezentat cod, care folosește proprietățile enumerate mai sus.

\begin{lstlisting}[caption=Proprietățile clasei list, label=listprop]
@empty = [];
@fonts = ["Arial" "Helvetica" "Times" "Courier"];

@empty'empty; `` true
@fonts'empty; `` false

@empty'length; `` 0
@fonts'length; `` 4

@empty'last; `` error
@fonts'last; `` "Courier"

@empty'0; `` error
@fonts'0; `` "Arial"
\end{lstlisting}

\subsubsection{\lstinline|[r/o] list'empty : bool|}

Listă se consideră deșartă dacă nu conține nici un șir.

\subsubsection{\lstinline|[r/o] list'length : int|}

Lungimea listei este egală cu numărul de șiruri în ea.

\subsubsection{\lstinline|[r/o] list'last : string|}

Ultimul șir din listă, echivalent \lstinline|list'at (list'length - 1)|.

Excepții posibile: \ferror{EmptyList}.

\subsubsection{\lstinline|[r/o] list'(n : int) : string|}

Al n-lea șir, n trebuie să fie un literal întreg.

Excepții posibile: \ferror{OutOfBounds}.

\subsection{Metode}

Listele dețin următoarele metode adiționale:
\begin{icItems}
\item \lstinline|list.append (str : string) : list|;
\item \lstinline|list.at (i : int) : string|;
\item \lstinline|list.contains (str : string, caseSensitive = true) : bool|;
\item \lstinline|list.clear () : list|;
\item \lstinline|list.count (what : string) : int|;
\item \lstinline|list.filter (str : string, caseSensitive = true) : bool|;
\item \lstinline|list.indexOf (str : string, start = 0) : int|;
\item \lstinline|list.insert (index : int, str : string) : list|;
\item \lstinline|list.join (separator : string) : string|;
\item \lstinline|list.lastIndexOf (str : string, start = -1) : int|;
\item \lstinline|list.mid (pos : int, n = -1) : list|;
\item \lstinline|list.prepend (str : string) : list|;
\item \lstinline|list.move (from : int, to : int) : list|;
\item \lstinline|list.removeAll (str : string) : list|;
\item \lstinline|list.removeAt (i : int) : list|;
\item \lstinline|list.removeDuplicates () : list|;
\item \lstinline|list.removeFirst () : list|;
\item \lstinline|list.removeLast () : list|;
\item \lstinline|list.removeOne (str : string) : bool|;
\item \lstinline|list.replaceInStrings (before : string, after : string) : list|;
\item \lstinline|list.sort (caseSensitive = true) : list|.
\end{icItems}

Unele metode au fost omise, ele for fi enumerate în capitulul \ref{regex}. Exemplu de cod, care folosește metodele enumerate mai sus, este prezentat pe foaia \ref{listmethods}. 

\begin{lstlisting}[caption=Metodele clasei list, label=listmethods]
@empty = [];
@fonts = ["Arial", "Helvetica", "Times", "Courier"];

@empty.append "";                 `` [""]
@fonts.at 2;                      `` "Times"
@fonts.contains "arial";          `` false
@fonts.contains ("arial", false); `` true
@empty.clear;                     `` []
@fonts.count "Arial";             `` 1

@fonts.filter "e";           `` ["Helvetica", "Times", "Courier"]
@fonts.indexOf "Times";      `` 2
@fonts.insert (1, "DejaVu"); `` ["Arial", "DejaVu", "Helvetica", "Times", "Courier"]
@fonts.join ", ";            `` "Arial, DejaVu, Helvetica, Times, Courier"
@fonts.lastIndexOf "Arial";  `` 0

@fonts.mid (2, 2);       `` ["Times", "Courier"]
@fonts.prepend "DejaVu"; `` ["DejaVu", "Arial", "DejaVu" .. "Courier"]
@fonts.move (1, 2);      `` ["DejaVu", "DejaVu", "Arial", "Helvetica" ..]

@fonts.removeAll "Helvetica"; `` ["DejaVu", "DejaVu", "Arial", "Times", "Courier"]
@fonts.removeAt 3;            `` ["DejaVu", "DejaVu", "Arial", "Courier"]
@fonts.removeDuplicates;      `` ["DejaVu", "Arial", "Courier"]
@fonts.removeFirst;           `` ["Arial", "Courier"]
@fonts.removeLast;            `` ["Arial"]
@fonts.removeOne "Arial";     `` []

@fonts = ["Arial", "Helvetica", "Times", "Courier"];

@fonts.replaceInStrings ("e", "-"); `` ["Arial", "H-lv-tica", "Tim-s", "Couri-r"];
@fonts.sort;                        `` ["Arial", "Couri-r", "H-lv-tica", "Tim-s"];
\end{lstlisting}

Parametrul \code{caseSensitive} în toate funcțiile răspunde de diferențierea majusculelor și minusculelor. Dacă el va primi valoarea \false, diferența dintre majuscule și minuscule va fi ignorată.

\subsubsection{\lstinline|list.append (str : string) : list|}

Inserează șirul \code{str} la sfîrșitul listei.

\subsubsection{\lstinline|list.at (i : int) : string|}

Returnează al \code{i}-lea șir.

Excepții posibile: \ferror{OutOfBounds}.

\subsubsection{\lstinline|list.contains (str : string, caseSensitive = true) : bool|}

Returnează \true, dacă lista conține șir egal cu \code{str}, în caz contrar \false.

\subsubsection{\lstinline|list.clear () : list|}

Curăță lista.

\subsubsection{\lstinline|list.count (what : string) : int|}

Returnează de cîte ori șirul \code{what} se repetă în listă.

\subsubsection{\lstinline|list.filter (str : string, caseSensitive = true) : bool|}

Returnează o nouă listă de șiruri, care conține numai șirurile, care conțin subșirul \code{str}. 

\subsubsection{\lstinline|list.indexOf (str : string, start = 0) : int|}

Returnează indexul primii includeri al șirului \code{str} în listă, căutînd înainte de pe poziția \code{start}, dacă lista nu conține așa șir se returnează -1.

\subsubsection{\lstinline|list.insert (index : int, str : string) : list|}

Inserează șirul \code{str} pe poziția \code{index}, dacă \code{index <= 0} șirul se înserează la începutul listei, dacă \code{index >= list'length} șirul se inserează la sfîrșitul listei.

\subsubsection{\lstinline|list.join (separator : string) : string|}

Creează un șir nou, din șirurile listei, prin concatenarea lor, între șiruri se inserează șirul \code{separator}, el poate fi un sir deșert.

\subsubsection{\lstinline|list.lastIndexOf (str : string, start = -1) : int|}

Returnează indexul primii includeri al șirului \code{str} în listă, căutînd înapoi de pe poziția \code{start}, dacă lista nu conține așa șir se returnează -1.

\subsubsection{\lstinline|list.mid (pos : int, n = -1) : list|}

Returnează o listă nouă, care conține \code{n} șiruri din listă, începînd cu poziția \code{pos}. Dacă \code{n} are valoarea \code{-1} vor fi adăugate toate șirurile pînă la sfîrșitul listei.

\subsubsection{\lstinline|list.prepend (str : string) : list|}

Inserează șirul \code{str} la începutul listei.

\subsubsection{\lstinline|list.move (from : int, to : int) : list|}

Mută șirul de pe poziția \code{from} pe poziția \code{to}.

Excepții posibile: \ferror{OutOfBounds}.

\subsubsection{\lstinline|list.removeAll (str : string) : list|}

Șterge toate șirurile egale cu \code{str}.

\subsubsection{\lstinline|list.removeAt (i : int) : list|}

Șterge al \code{i}-lea șir.

Excepții posibile: \ferror{OutOfBounds}.

\subsubsection{\lstinline|list.removeDuplicates () : list|}

Șterge șirurile care se repetă.

\subsubsection{\lstinline|list.removeFirst () : list|}

Șterge primul șir.

Excepții posibile: \ferror{EmptyList}.

\subsubsection{\lstinline|list.removeLast () : list|}

Șterge ultimul șir.

Excepții posibile: \ferror{EmptyList}.

\subsubsection{\lstinline|list.removeOne (str : string) : bool|}

Șterge primul șir egal cu \code{str}. Returnează \true, dacă așa șir a fost găsit, în caz contrar \false.

\subsubsection{\lstinline|list.replaceInStrings (before : string, after : string) : list|}

Înlocuiește subșirul \code{before} cu subșirul \code{after} în fiecare șir din listă.

\subsubsection{\lstinline|list.sort (caseSensitive = true) : list|}

Sortează șirurile în ordine alfabetică.

%\newpage
