% !TeX spellcheck = ro_RO
\section{Liste}

{\bf Listele} (tipul \mintinline{icl}{list}) permit a salva cîteva șiruri de caractere într-o variabilă. Accesul la șiruri se realizează prin index - numărul de ordine al șirului în listă.

\subsection{Proprietăți}

Listele dețin următoarele proprietăți adiționale:
\begin{icItems}
\item
	\mintinline{icl}{[r/o] list'empty : bool};
\item
	\mintinline{icl}{[r/o] list'length : int};
\item
	\mintinline{icl}{[r/o] list'last : string};
\item
	\mintinline{icl}{[r/o] list'(n : int) : string}.
\end{icItems}

Pe foaia \ref{listprop} este prezentat cod, care folosește proprietățile enumerate mai sus.

\begin{sourcecode}
    \captionof{listing}{Proprietățile clasei list}
    \label{listprop}
    \inputminted[linenos]{icl}{../sources/listprop.icL}
\end{sourcecode}

\subsubsection{\mintinline{icl}{[r/o] list'empty : bool}}

Listă se consideră deșartă dacă nu conține nici un șir.

\subsubsection{\mintinline{icl}{[r/o] list'length : int}}

Lungimea listei este egală cu numărul de șiruri în ea.

\subsubsection{\mintinline{icl}{[r/o] list'last : string}}

Ultimul șir din listă, echivalent \mintinline{icl}{list'at (list'length - 1)}.

Excepții posibile: \ferror{EmptyList} (pr. tab. \ref{errors}).

\subsubsection{\mintinline{icl}{[r/o] list'(n : int) : string}}

Al n-lea șir, n trebuie să fie un literal întreg.

Excepții posibile: \ferror{OutOfBounds} (pr. tab. \ref{errors}).

\subsection{Metode}

Listele dețin următoarele metode adiționale:
\begin{icItems}
\item \mintinline{icl}{list.append (str : string) : list};
\item \mintinline{icl}{list.at (i : int) : string};
\item \mintinline{icl}{list.contains (str : string, caseSensitive = true) : bool};
\item \mintinline{icl}{list.clear () : list};
\item \mintinline{icl}{list.count (what : string) : int};
\item \mintinline{icl}{list.filter (str : string, caseSensitive = true) : bool};
\item \mintinline{icl}{list.indexOf (str : string, start = 0) : int};
\item \mintinline{icl}{list.insert (index : int, str : string) : list};
\item \mintinline{icl}{list.join (separator : string) : string};
\item \mintinline{icl}{list.lastIndexOf (str : string, start = -1) : int};
\item \mintinline{icl}{list.mid (pos : int, n = -1) : list};
\item \mintinline{icl}{list.prepend (str : string) : list};
\item \mintinline{icl}{list.move (from : int, to : int) : list};
\item \mintinline{icl}{list.removeAll (str : string) : list};
\item \mintinline{icl}{list.removeAt (i : int) : list};
\item \mintinline{icl}{list.removeDuplicates () : list};
\item \mintinline{icl}{list.removeFirst () : list};
\item \mintinline{icl}{list.removeLast () : list};
\item \mintinline{icl}{list.removeOne (str : string) : bool};
\item \mintinline{icl}{list.replaceInStrings (before : string, after : string) : list};
\item \mintinline{icl}{list.sort (caseSensitive = true) : list}.
\end{icItems}

Unele metode au fost omise, ele for fi enumerate în capitulul \ref{regex}. Exemplu de cod, care folosește metodele enumerate mai sus, este prezentat pe foaia \ref{listmethods}. 

Parametrul \mintinline{icl}{caseSensitive} în toate funcțiile răspunde de diferențierea majusculelor și minusculelor. Dacă el va primi valoarea \false, diferența dintre majuscule și minuscule va fi ignorată.

\subsubsection{\mintinline{icl}{list.append (str : string) : list}}

Inserează șirul \mintinline{icl}{str} la sfîrșitul listei.

\newpage
\begin{sourcecode}
    \captionof{listing}{Metodele clasei list}
    \label{listmethods}
    \inputminted[linenos]{icl}{../sources/listmethods.icL}
\end{sourcecode}

\subsubsection{\mintinline{icl}{list.at (i : int) : string}}

Returnează al \mintinline{icl}{i}-lea șir.

Excepții posibile: \ferror{OutOfBounds} (pr. tab. \ref{errors}).

\subsubsection{\mintinline{icl}{list.contains (str : string, caseSensitive = true) : bool}}

Returnează \true, dacă lista conține șir egal cu \mintinline{icl}{str}, în caz contrar \false.

\subsubsection{\mintinline{icl}{list.clear () : list}}

Curăță lista.

\subsubsection{\mintinline{icl}{list.count (what : string) : int}}

Returnează de cîte ori șirul \mintinline{icl}{what} se repetă în listă.

\subsubsection{\mintinline{icl}{list.filter (str : string, caseSensitive = true) : bool}}

Returnează o nouă listă de șiruri, care conține numai șirurile, care conțin subșirul \mintinline{icl}{str}. 

\subsubsection{\mintinline{icl}{list.indexOf (str : string, start = 0) : int}}

Returnează indexul primii includeri al șirului \mintinline{icl}{str} în listă, căutînd înainte de pe poziția \mintinline{icl}{start}, dacă lista nu conține așa șir se returnează -1.

\subsubsection{\mintinline{icl}{list.insert (index : int, str : string) : list}}

Inserează șirul \mintinline{icl}{str} pe poziția \mintinline{icl}{index}, dacă \mintinline{icl}{index <= 0} șirul se înserează la începutul listei, dacă \mintinline{icl}{index >= list'length} șirul se inserează la sfîrșitul listei.

\subsubsection{\mintinline{icl}{list.join (separator : string) : string}}

Creează un șir nou, din șirurile listei, prin concatenarea lor, între șiruri se inserează șirul \mintinline{icl}{separator}, el poate fi un sir deșert.

\subsubsection{\mintinline{icl}{list.lastIndexOf (str : string, start = -1) : int}}

Returnează indexul primii includeri al șirului \mintinline{icl}{str} în listă, căutînd înapoi de pe poziția \mintinline{icl}{start}, dacă lista nu conține așa șir se returnează -1.

\subsubsection{\mintinline{icl}{list.mid (pos : int, n = -1) : list}}

Returnează o listă nouă, care conține \mintinline{icl}{n} șiruri din listă, începînd cu poziția \mintinline{icl}{pos}. Dacă \mintinline{icl}{n} are valoarea \mintinline{icl}{-1} vor fi adăugate toate șirurile pînă la sfîrșitul listei.

\subsubsection{\mintinline{icl}{list.prepend (str : string) : list}}

Inserează șirul \mintinline{icl}{str} la începutul listei.

\subsubsection{\mintinline{icl}{list.move (from : int, to : int) : list}}

Mută șirul de pe poziția \mintinline{icl}{from} pe poziția \mintinline{icl}{to}.

Excepții posibile: \ferror{OutOfBounds} (pr. tab. \ref{errors}).

\subsubsection{\mintinline{icl}{list.removeAll (str : string) : list}}

Șterge toate șirurile egale cu \mintinline{icl}{str}.

\subsubsection{\mintinline{icl}{list.removeAt (i : int) : list}}

Șterge al \mintinline{icl}{i}-lea șir.

Excepții posibile: \ferror{OutOfBounds} (pr. tab. \ref{errors}).

\subsubsection{\mintinline{icl}{list.removeDuplicates () : list}}

Șterge șirurile care se repetă.

\subsubsection{\mintinline{icl}{list.removeFirst () : list}}

Șterge primul șir.

Excepții posibile: \ferror{EmptyList} (pr. tab. \ref{errors}).

\subsubsection{\mintinline{icl}{list.removeLast () : list}}

Șterge ultimul șir.

Excepții posibile: \ferror{EmptyList} (pr. tab. \ref{errors}).

\subsubsection{\mintinline{icl}{list.removeOne (str : string) : bool}}

Șterge primul șir egal cu \mintinline{icl}{str}. Returnează \true, dacă așa șir a fost găsit, în caz contrar \false.

\subsubsection{\mintinline{icl}{list.replaceInStrings (before : string, after : string) : list}}

Înlocuiește subșirul \mintinline{icl}{before} cu subșirul \mintinline{icl}{after} în fiecare șir din listă.

\subsubsection{\mintinline{icl}{list.sort (caseSensitive = true) : list}}

Sortează șirurile în ordine alfabetică.

%\newpage
