% !TeX spellcheck = ro_RO

\section{Sigleton-uri adiționali}

Singleton-urile conțin valori constante și funcții statice.

\subsection{Files}

Obiectul \files{} are următoarele metode:
\begin{icItems}
	\item \mintinline{icl}{Files.open (path : string) : File};
	\item \mintinline{icl}{Files.create (path : string) : File};
	\item \mintinline{icl}{Files.createDir (path : string) : void};
	\item \mintinline{icl}{Files.createPath (path : string) : void}.
\end{icItems}

\subsubsection{\mintinline{icl}{Files.open (path : string) : File}}

Deschide fișierul.

Excepții posibile: \ferror{FileNotFound} (pr. tab. \ref{errors}).

\subsubsection{\mintinline{icl}{Files.create (path : string) : File}}

Deschide fișierul, dacă el nu există se creează.

Excepții posibile: \ferror{FolderNotFound} (pr. tab. \ref{errors}).

\subsubsection{\mintinline{icl}{Files.createDir (path : string) : void}}

Creează folder, folderul ascendent trebuie să existe deja.

Excepții posibile: \ferror{FolderNotFound} (pr. tab. \ref{errors}).

\subsubsection{\mintinline{icl}{Files.createPath (path : string) : void}}

Creează toate folderele care nu există în calea indicată.

\subsection{File}

Obiectul \file{} are următoarele proprietăți:
\begin{icItems}
	\item \mintinline{icl}{[r/o] File'csv : 1};
	\item \mintinline{icl}{[r/o] File'none : 0};
	\item \mintinline{icl}{[r/o] File'tsv : 2};
\end{icItems}

\subsubsection{\mintinline{icl}{[r/o] File'csv : 1}}

Format CSV.

\subsubsection{\mintinline{icl}{[r/o] File'none : 0}}

File neinițializat.

\subsubsection{\mintinline{icl}{[r/o] File'tsv : 2}}

Format TSV.

\subsection{Make}

Obiectul \make{} deține următoarea metodă: \mintinline{icl}{Make.image (base64 : string, path :}\\*\mintinline{icl}{string) : void}.

\subsubsection{\mintinline{icl}{Make.image (base64 : string, path : string) : void}}

Salvează captura de ecran pe disc.

\subsection{Log}

Obiectul \logtype{} deține următoarele metode:
\begin{icItems}
	\item \mintinline{icl}{Log.error (message : string) : void};
	\item \mintinline{icl}{Log.info (message : string) : void};
	\item \mintinline{icl}{Log.out (args : any ...) : void};
	\item \mintinline{icl}{Log.stack (var : any) : void};
	\item \mintinline{icl}{Log.state (var : any) : void}.
\end{icItems}

\subsubsection{\mintinline{icl}{Log.error (message : string) : void}}

Printează un mesaj de eroare.

\subsubsection{\mintinline{icl}{Log.info (message : string) : void}}

Printează un mesaj informațional.

\subsubsection{\mintinline{icl}{Log.out (args : any ...) : void}}

Printează informație pentru debug, primește cîțeva argumente de orice tip. La transmiterea variabilelor se printează containerul, numele variabilei, tipul de date și valoarea. La transmiterea constantelor numai valoarea. La transmiterea rezultatului funcției se printează tipul și valoarea.

\subsubsection{\mintinline{icl}{Log.stack (var : any) : void}}

Printează lista de stive, arătînd în care din ele se întilnește așa variabilă și ce valori are.

\subsubsection{\mintinline{icl}{Log.state (var : any) : void}}

Printează lista tuturor stărilor, arătînd în care din ele se întilnește așa variabilă și ce valori are.

\subsection{Numbers}

Obiectul \numbers{} are următoarele proprietăți:
\begin{icItems}
	\item \mintinline{icl}{[r/o] Numbers'max : 4};
	\item \mintinline{icl}{[r/o] Numbers'min : 3};
	\item \mintinline{icl}{[r/o] Numbers'product : 2};
	\item \mintinline{icl}{[r/o] Numbers'process : int};
	\item \mintinline{icl}{[r/o] Numbers'sum : 1}.
\end{icItems}

Și următoarele metode:
\begin{icItems}
	\item \mintinline{icl}{Numbers.process (a : int, b : int) : int};
	\item \mintinline{icl}{Numbers.process (a : double, b : double) : double};
	\item \mintinline{icl}{Numbers.restoreProcess () : void};
	\item \mintinline{icl}{Numbers.setProcess (proc : int) : void}.
\end{icItems}

\subsubsection{\mintinline{icl}{[r/o] Numbers'max : 4}}

A alege maximum.

\subsubsection{\mintinline{icl}{[r/o] Numbers'min : 3}}

A alege minimum.

\subsubsection{\mintinline{icl}{[r/o] Numbers'product : 2}}

A înmulți numerele.

\subsubsection{\mintinline{icl}{[r/o] Numbers'process : int}}

Modul curent de a prelucra numerele.

\subsubsection{\mintinline{icl}{[r/o] Numbers'sum : 1}}

A aduna numerele.

\subsubsection{\mintinline{icl}{Numbers.process (a : int, b : int) : int}}

Prelucrează numerele întregi cu metoda de prelucrare curentă.

\subsubsection{\mintinline{icl}{Numbers.process (a : double, b : double) : double}}

Prelucrează numerele decimale cu metoda de prelucrare curentă.

\subsubsection{\mintinline{icl}{Numbers.restoreProcess () : void}}

Șterge ultima înscriere sin stiva metodelor de prelucrare.

\subsubsection{\mintinline{icl}{Numbers.setProcess (proc : int) : void}}

Adaugă o nouă înscriere în stiva metodelor de prelucrare.

\subsection{Math}

Obiectul \mintinline{icl}{Math} are următoarele proprietăți:
\begin{icItems}
	\item \mintinline{icl}{[r/o] Math'1divPi : double};
	\item \mintinline{icl}{[r/o] Math'1divSqrt2 : double};
	\item \mintinline{icl}{[r/o] Math'2divPi : double};
	\item \mintinline{icl}{[r/o] Math'2divSqrtPi : double};
	\item \mintinline{icl}{[r/o] Math'e : double};
	\item \mintinline{icl}{[r/o] Math'ln2 : double};
	\item \mintinline{icl}{[r/o] Math'ln10 : double};
	\item \mintinline{icl}{[r/o] Math'log2e : double};
	\item \mintinline{icl}{[r/o] Math'log10e : double};
	\item \mintinline{icl}{[r/o] Math'pi : double};
	\item \mintinline{icl}{[r/o] Math'piDiv2 : double};
	\item \mintinline{icl}{[r/o] Math'piDiv4 : double};
	\item \mintinline{icl}{[r/o] Math'sqrt2 : double}.
\end{icItems}

Și următoarele metode:
\begin{icItems}
	\item \mintinline{icl}{Math.acos (v : double) : double};
	\item \mintinline{icl}{Math.asin (v : double) : double};
	\item \mintinline{icl}{Math.atan (v : double) : double};
	\item \mintinline{icl}{Math.ceil (v : double) : int};
	\item \mintinline{icl}{Math.cos (v : double) : double};
	\item \mintinline{icl}{Math.degreesToRadians (v : double) : double};
	\item \mintinline{icl}{Math.exp (v : double) : double};
	\item \mintinline{icl}{Math.floor (v : double) : int};
	\item \mintinline{icl}{Math.ln (v : double) : double};
	\item \mintinline{icl}{Math.min (arr : int...) : int};
	\item \mintinline{icl}{Math.min (arr : double...) : double};
	\item \mintinline{icl}{Math.max (arr : int...) : int};
	\item \mintinline{icl}{Math.max (arr : double...) : double};
	\item \mintinline{icl}{Math.radiansToDegrees (v : double) : double};
	\item \mintinline{icl}{Math.round (v : double) : int};
	\item \mintinline{icl}{Math.sin (v : double) : double};
	\item \mintinline{icl}{Math.tan (v : double) : double}.
\end{icItems}

\subsubsection{\mintinline{icl}{[r/o] Math'1divPi : double}}

1 împărțit la pi ($\frac{1}{\pi}$).

\subsubsection{\mintinline{icl}{[r/o] Math'1divSqrt2 : double}}

1 împărțit la radical din 2 ($\frac{1}{\sqrt{2}}$).

\subsubsection{\mintinline{icl}{[r/o] Math'2divPi : double}}

2 împărțit la pi ($\frac{2}{\pi}$).

\subsubsection{\mintinline{icl}{[r/o] Math'2divSqrtPi : double}}

2 împărțit la pi ($\frac{2}{\sqrt{\pi}}$).

\subsubsection{\mintinline{icl}{[r/o] Math'e : double}}

Numărul ($e$).

\subsubsection{\mintinline{icl}{[r/o] Math'ln2 : double}}

Logaritmul natural al numărului 2 ($\ln{2}$).

\subsubsection{\mintinline{icl}{[r/o] Math'ln10 : double}}

Logaritmul natural al numărului 10 ($\ln_{10}$).

\subsubsection{\mintinline{icl}{[r/o] Math'log2e : double}}

Logaritmul numărului $e$ cu baza 2 ($\log_{2}{e}$).

\subsubsection{\mintinline{icl}{[r/o] Math'log10e : double}}

Logaritmul numărului $e$ cu baza 10 ($\log_{10}{e}$).

\subsubsection{\mintinline{icl}{[r/o] Math'pi : double}}

Numărul pi ($\pi$).

\subsubsection{\mintinline{icl}{[r/o] Math'piDiv2 : double}}

Pi pe 2 ($\frac{\pi}{2}$).

\subsubsection{\mintinline{icl}{[r/o] Math'piDiv4 : double}}

Pi pe 4 ($\frac{\pi}{4}$).

\subsubsection{\mintinline{icl}{[r/o] Math'sqrt2 : double}}

Radical din 2 ($\sqrt{2}$).

\subsubsection{\mintinline{icl}{Math.acos (v : double) : double}}

Arccosinus ($\arccos{v}$).

\subsubsection{\mintinline{icl}{Math.asin (v : double) : double}}

Arcsinus ($\arcsin{v}$).

\subsubsection{\mintinline{icl}{Math.atan (v : double) : double}}

Arctangență ($\arctan{v}$).

\subsubsection{\mintinline{icl}{Math.ceil (v : double) : int}}

Cel mai mic număr întreg mai mare sau egal cu \mintinline{icl}{v}.

\subsubsection{\mintinline{icl}{Math.cos (v : double) : double}}

Cosinus ($\cos{v}$).

\subsubsection{\mintinline{icl}{Math.degreesToRadians (v : double) : double}}

Conversează grade în radiani.

\subsubsection{\mintinline{icl}{Math.exp (v : double) : double}}

Funcția exponent ($\exp{v}$).

\subsubsection{\mintinline{icl}{Math.floor (v : double) : int}}

Cel mai mare număr întreg mai mic sau egal cu \mintinline{icl}{v}.

\subsubsection{\mintinline{icl}{Math.ln (v : double) : double}}

Logaritm natural ($\ln{v}$).

\subsubsection{\mintinline{icl}{Math.min (arr : int...) : int}}

Returnează cel mai mic număr întreg.

\subsubsection{\mintinline{icl}{Math.min (arr : double...) : double}}

Returnează cel mai mic număr decimal.

\subsubsection{\mintinline{icl}{Math.max (arr : int...) : int}}

Returnează cel mai mare număr întreg.

\subsubsection{\mintinline{icl}{Math.max (arr : double...) : double}}

Returnează cel mai mare număr decimal.

\subsubsection{\mintinline{icl}{Math.radiansToDegrees (v : double) : double}}

Conversează radiani în grade.

\subsubsection{\mintinline{icl}{Math.round (v : double) : int}}

Returnează cel mai apropiat număr întreg.

\subsubsection{\mintinline{icl}{Math.sin (v : double) : double}}

Sinus ($\sin{v}$).

\subsubsection{\mintinline{icl}{Math.tan (v : double) : double}}

Cosinus ($\tan{v}$).

\subsection{Import}

Obiectul \mintinline{icl}{Import} deține următoarele metode:
\begin{icItems}
	\item \mintinline{icl}{Import.none (data : object, path : string) : void};
	\item \mintinline{icl}{Import.none (path : string) : void};
	\item \mintinline{icl}{Import.functions (data : object, path : string) : void};
	\item \mintinline{icl}{Import.functions (path : string) : void};
	\item \mintinline{icl}{Import.all (data : object, path : string) : void};
	\item \mintinline{icl}{Import.all (path : string) : void};
	\item \mintinline{icl}{Import.run (path : string) : void}.
\end{icItems}

Obiectul \mintinline{icl}{data} permite a transmite date în contextul izolat în care se execut fișierele externe, proprietățile obiectului \mintinline{icl}{data} vor fi disponibile ca variabile globale. Aceasta permite a deține într-un fișier mai multe versiuni ale bibliotecii de exemplu, și la folosire se poate de indicat care versiune e necesară.

\subsubsection{\mintinline{icl}{Import.none (data : object, path : string) : void}}

Creează context izolat în care se execută fișierul.

\subsubsection{\mintinline{icl}{Import.none (path : string) : void}}

Acronim pentru \mintinline{icl}{Import.none([=], @path)}.

\subsubsection{\mintinline{icl}{Import.functions (data : object, path : string) : void}}

Creează context izolat în care se execută fișierul, Apoi se importă funcțiile în contextul curent.

\subsubsection{\mintinline{icl}{Import.functions (path : string) : void}}

Acronim pentru \mintinline{icl}{Import.functions ([=], @path)}.

\subsubsection{\mintinline{icl}{Import.all (data : object, path : string) : void}}

Creează context izolat în care se execută fișierul, Apoi se importă funcțiile și variabilele globale în contextul curent.

\subsubsection{\mintinline{icl}{Import.all (path : string) : void}}

Acronim pentru \mintinline{icl}{Import.all ([=], @path)}.

\subsubsection{\mintinline{icl}{Import.run (path : string) : void}}

Execută fișierul în contextul curent.

%\newpage
