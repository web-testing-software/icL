% !TeX spellcheck = ro_RO
\section{DSV: CSV și TSV}

{\bf DSV (Delimiter Separated Values, Valori Delimitate de Delimitator)} sunt fișiere text, care conțin valori sub formă de text.

{\bf CSV (Comma Separated Values, Valori Delimitate de Virgulă)} folosește virgula în calitate de delimitator, formatul este forate popular și folosit des de programe office. Mai multă informație despre el puteți găsi în internet.

{\bf TSV (Tab Separated Values, Valori Delimitate de Tabulator)} este mult mai simplu, folosește tabulatorul în calitate de delimitator și se folosește des în aplicații web.

icL permite a efectua următoarele operații cu fișiere de acest tip:
\begin{icItems}
	\item a încărca;
	\item a scrie;
	\item a insera date;
	\item a sincroniza.
\end{icItems}

\subsection{Încărcare}

\subsubsection{\lstinline|DSV.load (delimiter : string, f : File, base : set) : set|;}

Returnează o mulțime care conține toate datele din fișierul \code{f}. Șirul de caractere \code{delimiter} trebuie să conțină un singur caracter - caracterul delimitator. Obiect de tipul \code{File} poate fi creat cu ajutorul funcției \lstinline|Files.create (path : string) : File|, dacă fișierul nu există el va fi creat, în caz contrat deschis.

Parametrul \code{base} primește mulțimea pe baza cărei va fi creată noua mulțime, header-ul ei trebui să fie compatibil cu datele din fișier. De asemenea poate conține date care vor fi prezente în mulțimea returnată.

Excepții posibile: \ferror{ParsingFailed}, \ferror{WrongDelimiter} (pr. tab. \ref{errors}).

\subsubsection{\lstinline|DSV.loadCSV (f : File, base : set) : File|;}

Acronim pentru \lstinline|DSV.load (",", @f, @base)|.

\subsubsection{\lstinline|DSV.loadTSV (f : File, base : set) : File|;}

TSV se deosebește de restul, fișierele se prelucrează după un alt algoritm, de aceea această funcție nu are alternativă și nu este un acronim.

\subsection{Scriere}

\subsubsection{\lstinline|DSV.write (f : File, s : set) : File|;}

Tot conținutul fișierului va fi rescris.

\subsection{Inserare}

Conținutul fișierului nu va fi atins, datele noi vor fi înscrise la sfârșitul fișierului. De compatibilitate răspunde utilizatorul, icL nu controlează datele existente.

\subsubsection{\lstinline|DSV.append (f : File, s : set) : File|;}

Inserează toate obiectele din mulțime la sfîrșitul fișierului \code{s}.

\subsubsection{\lstinline|DSV.append (f : File, obj : object) : File|;}

Inserează un obiect la sfîrșitul fișierului \code{obj}.

\subsection{Sincronizare}

\subsubsection{\lstinline|DSV.sync (f : File, s : set) : set|;}

La instalarea legăturii de sincronizare fișierul este rescris complet. La adăugarea datelor ele vor fi inserate la sfîrșitul fișierului. La ștergerea datelor fișierul va fi rescris complet.

%\newpage
