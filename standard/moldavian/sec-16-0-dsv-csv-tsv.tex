% !TeX spellcheck = ro_RO
\section{DSV: CSV și TSV}

{\bf DSV (Delimiter Separated Values, Valori Delimitate de Delimitator)} sunt fișiere text, care conțin valori sub formă de text.

{\bf CSV (Comma Separated Values, Valori Delimitate de Virgulă)} folosește virgula în calitate de delimitator, formatul este forate popular și folosit des de programe office. Mai multă informație despre el puteți găsi în internet.

{\bf TSV (Tab Separated Values, Valori Delimitate de Tabulator)} este mult mai simplu, folosește tabulatorul în calitate de delimitator și se folosește des în aplicații web.

icL permite a efectua următoarele operații cu fișiere de acest tip:
\begin{icItems}
	\item a încărca;
	\item a scrie;
	\item a insera date;
	\item a sincroniza.
\end{icItems}

\subsection{Încărcarea}

\subsubsection{\lstinline|_dsv.load <string>delimiter <_file>f <set>base : set|;}

Returnează o mulțime care conține toate datele din fișierul \code{f}. Șirul de caractere \code{delimiter} trebuie să conțină un singur caracter - caracterul delimitator. Obiect de tipul \code{\_file} poate fi creat cu ajutorul funcției \lstinline|_files.create <string>path : _file|, dacă fișierul nu există el va fi creat, în caz contrat deschis.

Parametrul \code{base} primește mulțimea pe baza cărei va fi creată noua mulțime, header-ul ei trebui să fie compatibil cu datele din fișier. De asemenea poate conține date care vor fi prezente în mulțimea returnată.

Excepții posibile: \ferror{ParsingFailed}, \ferror{WrongDelimiter}.

\subsubsection{\lstinline|_dsv.loadCSV <_file>f <set>base : _file|;}

Acronim pentru \lstinline|_dsv.load "," f base|.

\subsubsection{\lstinline|_dsv.loadTSV <_file>f <set>base : _file|;}

TSV se deosebește de restul, fișierele se prelucrează după un alt algoritm, de aceea această funcție nu are alternativă și nu este un acronim.

\subsection{Scrierea}

\subsubsection{\lstinline|_dsv.write <_file>f <set>s: _file|;}

Tot conținutul fișierului va fi rescris.

\subsection{Inserarea}

Conținutul fișierului nu va fi atins, datele noi vor fi înscrise la sfârșitul fișierului. De compatibilitate răspunde utilizatorul, icL nu controlează datele existente.

\subsubsection{\lstinline|_dsv.append <_file>f <set>s : _file|;}

Inserează toate obiectele din mulțime la sfîrșitul fișierului \code{s}.

\subsubsection{\lstinline|_dsv.append <_file>f <object>obj : _file|;}

Inserează un obiect la sfîrșitul fișierului \code{obj}.

\subsection{Sincronizarea}

\subsubsection{\lstinline|_dsv.sync <_file>f <set>s : set|;}

La instalarea legăturii de sincronizare fișierul este rescris complet. La adăugarea datelor ele vor fi inserate la sfîrșitul fișierului. La ștergerea datelor fișierul va fi rescris complet.

%\newpage
