% !TeX spellcheck = ro_RO
\section{Elemente web}
\label{webelments}

{\bf Elementele web} (tipul de date \element) sunt linkuri la taguri HTML din pagina web.

Elementul se numește unic dacă este creat de funcția \mintinline{icl}{document.query} sau \mintinline{icl}{element}\\*\mintinline{icl}{.query}. Elementul se numește colecție dacă este creat de funcția \mintinline{icl}{document.queryAll} sau \mintinline{icl}{element.queryAll}. Extragerea elementului după index returnează element unic.

\subsection{Literal}

Elementele web pot fi declarate prin literalul \mintinline{icl}{metodă@scop[selector]}, unde \mintinline{icl}{metodă} e unul din următoarele cuvinte: \mintinline{icl}{css}, \mintinline{icl}{xpath}, \mintinline{icl}{link}, \mintinline{icl}{links}, \mintinline{icl}{tag}, \mintinline{icl}{tags}, \mintinline{icl}{button}, \mintinline{icl}{input}, \mintinline{icl}{field}, \mintinline{icl}{web}, \mintinline{icl}{h1}, \mintinline{icl}{h2}, \mintinline{icl}{h3}, \mintinline{icl}{h4}, \mintinline{icl}{h5}, \mintinline{icl}{h6}, \mintinline{icl}{legend}, \mintinline{icl}{span}. \mintinline{icl}{@scop} e o variabilă locală, care conține un element web. \mintinline{icl}{selector} e un selector, un șir de caractere, după care se definește, care elemente vor fi capturate, fiecare metodă trebuiește un selector aparte.

Metoda \mintinline{icl}{css} primește în calitate de selector un selector CSS. Acestă metodă va returna primul element compatibil. Pentru a obține toate elemente folosiți \mintinline{icl}{css:all}.

Metoda \mintinline{icl}{xpath} primește în calitate de selector o cale XPath. Acestă metodă va returna primul element compatibil. Pentru a obține toate elemente folosiți \mintinline{icl}{xpath:all}.

Metoda \mintinline{icl}{link} primește în calitate de selector textul linkului și returnează linkul corespunzător. Pentru a indica un fragment de text folosiți \mintinline{icl}{link:fragment}.

Metoda \mintinline{icl}{links} primește în calitate de selector un text și returnează toate linkurile, care conțin exact acest text. Pentru a indica un fragment de text folosiți \mintinline{icl}{links:fragment}.

Metoda \mintinline{icl}{tag} primește în calitate de selector un nume de tag și returnează primul element al tagului corespunzător.

Metoda \mintinline{icl}{tags} primește în calitate de selector un nume de tag și returnează toate elemente tag-ului corespunzător.

Metoda \mintinline{icl}{button} primește în calitate de selector textul butonului, returnează primul buton corespunzător.

Metoda \mintinline{icl}{input} primește în calitate de selector valoarea atributului \mintinline{icl}{name}, returnează primul input cu numele indicat.

Metoda \mintinline{icl}{field} primește în calitate de selector textul etichetei, returnează cîmpul cu numele indicat.

Metodele \mintinline{icl}{h1}, \mintinline{icl}{h2}, \mintinline{icl}{h3}, \mintinline{icl}{h4}, \mintinline{icl}{h5}, \mintinline{icl}{h6} și \mintinline{icl}{legend} primesc în calitate de selector textul antetului. Returnează primul antet cu textul indicat. 

Metoda \mintinline{icl}{span} primește în calitate de selector textul span-ului, returnează primul span cu textul indicat.

Metoda \mintinline{icl}{web} nu primește nici un selector, returează un element nou si deșert.

\newpage
Exemple de literale -
\begin{minted}{icl}
@div = tag[div];
@a = tag@div[a];
`` @a == css[div a]
`` @a != css[div > a]

css[#content > div:nth-of-type(3)]
css[header ul > li:nth-child(even)]
\end{minted}

\

\subsection{Proprietăți}

Elementele web au următoarele proprietăți:
\begin{icItems}
\item \mintinline{icl}{[r/o] element'attr-* : string};
\item \mintinline{icl}{[r/o] element'attrs-* : list};
\item \mintinline{icl}{[r/o] element'css-* : string};
\item \mintinline{icl}{[r/o] element'document : document};
\item \mintinline{icl}{[r/o] element'empty : bool};
\item \mintinline{icl}{[r/o] element'enabled : bool};
\item \mintinline{icl}{[r/o] element'length : int};
\item \mintinline{icl}{[r/o] element'prop-* : any};
\item \mintinline{icl}{[r/o] element'props-* : list};
\item \mintinline{icl}{[r/o] element'rect : object};
\item \mintinline{icl}{[r/o] element'rects : set};
\item \mintinline{icl}{[r/o] element'selected : bool};
\item \mintinline{icl}{[r/o] element'tag : string};
\item \mintinline{icl}{[r/o] element'tags : list};
\item \mintinline{icl}{[r/o] element'text : string};
\item \mintinline{icl}{[r/o] element'texts : list};
\item \mintinline{icl}{[r/o] element'(<int> n) : element};
\end{icItems} 

\subsubsection{\mintinline{icl}{[r/o] element'attr-* : string}}

Returnează valoarea atributului \mintinline{icl}{*}.

Excepții posibile: \ferror{NoSessions}, \ferror{EmptyElement}, \ferror{MultiElement}, \ferror{NoSuchWindow} și \ferror{StaleElementReference} (pr. tab. \ref{errors}).

\subsubsection{\mintinline{icl}{[r/o] element'attrs-* : list}}

Returnează valoarea atributului \mintinline{icl}{*} fiecărui element. 

Excepții posibile: \ferror{NoSessions}, \ferror{NoSuchWindow} și \ferror{StaleElementReference} (pr. tab. \ref{errors}).

\subsubsection{\mintinline{icl}{[r/o] element'css-* : string}}

Returnează valoarea proprietății CSS \mintinline{icl}{*}.

Excepții posibile: \ferror{NoSessions}, \ferror{EmptyElement}, \ferror{MultiElement}, \ferror{NoSuchWindow} și \ferror{StaleElementReference} (pr. tab. \ref{errors}).

\subsubsection{\mintinline{icl}{[r/o] element'document : document}}

Returnează documentul în care se afla acest web element.

\subsubsection{\mintinline{icl}{[r/o] element'empty : bool}}

Returnează \true, dacă colecția nu conține nici un element, în caz contrar \false.

\subsubsection{\mintinline{icl}{[r/o] element'enabled : bool}}

Returnează \false, dacă elementul este un formular și el este închis, în caz contrar \true.

Excepții posibile: \ferror{NoSessions}, \ferror{EmptyElement}, \ferror{MultiElement}, \ferror{NoSuchWindow} și \ferror{StaleElementReference} (pr. tab. \ref{errors}).

\subsubsection{\mintinline{icl}{[r/o] element'length : int}}

Returnează numărul de elemente în colecție.

\subsubsection{\mintinline{icl}{[r/o] element'prop-* : string}}

Returnează valoare proprietății \mintinline{icl}{*}.

Excepții posibile: \ferror{NoSessions}, \ferror{EmptyElement}, \ferror{MultiElement}, \ferror{NoSuchWindow} și \ferror{StaleElementReference} (pr. tab. \ref{errors}).

\subsubsection{\mintinline{icl}{[*/*] element'props-* : list}}

Returnează valorile proprietăților \mintinline{icl}{*}.

Excepții posibile: \ferror{NoSessions}, \ferror{NoSuchWindow} și \ferror{StaleElementReference} (pr. tab. \ref{errors}).

\subsubsection{\mintinline{icl}{[r/o] element'rect : obj}}

Returnează poziția și mărimea elementului pe ecran in pixeli CSS, dar mai exact un obiect cu următoarele cîmpuri:
\begin{icItems}
	\item \mintinline{icl}{x : double} - coordinata x relativ la marginea stîngă a ecranului;
	\item \mintinline{icl}{y : double} - coordinata y relativ la marginea de sus a ecranului;
	\item \mintinline{icl}{width : double} - lungimea;
	\item \mintinline{icl}{height : double} - înălțimea;
\end{icItems}

Excepții posibile: \ferror{NoSessions}, \ferror{EmptyElement}, \ferror{MultiElement}, \ferror{NoSuchWindow} și \ferror{StaleElementReference} (pr. tab. \ref{errors}).

\subsubsection{\mintinline{icl}{[r/o] element'rects : set}}

Returnează pozițiile și mărimile obiectele tuturor elementele web.

Excepții posibile: \ferror{NoSessions}, \ferror{NoSuchWindow} și \ferror{StaleElementReference} (pr. tab. \ref{errors}).

\subsubsection{\mintinline{icl}{[r/o] element'selected : bool}}

Returnează \true, dacă elementul este o casetă bifată, radio buton selectat sau o opțiune selectată, în caz contrar \false.

Excepții posibile: \ferror{NoSessions}, \ferror{EmptyElement}, \ferror{MultiElement}, \ferror{NoSuchWindow} și \ferror{StaleElementReference} (pr. tab. \ref{errors}).

\subsubsection{\mintinline{icl}{[r/o] element'tag : string}}

Returnează numele tagului.

Excepții posibile: \ferror{NoSessions}, \ferror{EmptyElement}, \ferror{MultiElement}, \ferror{NoSuchWindow} și \ferror{StaleElementReference} (pr. tab. \ref{errors}).

\subsubsection{\mintinline{icl}{[r/o] element'tags : list}}

Returnează denumirile tag-urilor in listă.

Excepții posibile: \ferror{NoSessions}, \ferror{NoSuchWindow} și \ferror{StaleElementReference} (pr. tab. \ref{errors}).

\subsubsection{\mintinline{icl}{[r/o] element'text : string}}

Returnează textul elementului vizibil pe ecran. 

Excepții posibile: \ferror{NoSessions}, \ferror{EmptyElement}, \ferror{MultiElement}, \ferror{NoSuchWindow} și \ferror{StaleElementReference} (pr. tab. \ref{errors}).

\subsubsection{\mintinline{icl}{[r/o] element'texts : list}}

Returnează textul elementului vizibil pe ecran al fiecăruia element.

Excepții posibile: \ferror{NoSessions}, \ferror{NoSuchWindow} și \ferror{StaleElementReference} (pr. tab. \ref{errors}).

\subsubsection{\mintinline{icl}{[r/o] element'(<int> n) : element}}

Returnează element unic care conține al n-lea element.

Excepții posibile: \ferror{OutOfBounds}, \ferror{StaleElementReference} (pr. tab. \ref{errors}).

\subsection{Operatori}

În contextul elementelor web apare un operator nou: \mintinline{icl}{[element ...] : element};

\subsubsection{\mintinline{icl}{[element ...] : element}}

Returnează o colecție nouă, care conține toate elementele indicate.

\subsection{Metode}

{\bf Metodele} clasei \mintinline{icl}{element} se împart în 2 categorii: de bază și adiționale. Metodele de bază sunt definite pe baza de standardul W3C WebDriver. Metodele adiționale sunt definite de standardul icL și pot fi modificate în versiunile succesive ale limbajului.

\subsection{Metode de bază}

Lista metodelor de bază:
\begin{icItems}
\item \mintinline{icl}{element.clear () : element};
\item \mintinline{icl}{element.click () : element};
\item \mintinline{icl}{element.query (cssSelector : string) : element};
\item \mintinline{icl}{element.query (by : int, selector : string) : element};
\item \mintinline{icl}{element.queryAll (cssSelector : string) : element};
\item \mintinline{icl}{element.queryAll (by : int, selector : string) : element};
\item \mintinline{icl}{element.queryAllByXPath (xpath : string) : element};
\item \mintinline{icl}{element.queryByXPath (xpath : string) : element};
\item \mintinline{icl}{element.queryLink (name : string, isFragment = false) : element};
\item \mintinline{icl}{element.queryLinks (name : string, isFragment = false) : element};
\item \mintinline{icl}{element.queryTag (name : string) : element};
\item \mintinline{icl}{element.queryTags (name : string) : element};
\item \mintinline{icl}{element.screenshot () : string};
\item \mintinline{icl}{element.sendKeys (modifiers : int, text : string) : element};
\end{icItems}

\subsubsection{\mintinline{icl}{element.clear () : element}}

Dacă elementul este un cîmp de inserarea a datelor, atunci valoarea lui se anulează. Dacă este un element editabil, atunci proprietății lui \mintinline{icl}{innerHtml} se atribuie un șir deșert.

Excepții posibile: \ferror{NoSessions}, \ferror{InvalidArrgument}, \ferror{EmptyElement}, \ferror{MultiElement}, \ferror{NoSuchWindow}, \ferror{StaleElementReference}, \ferror{InvalidElementState} (pr. tab. \ref{errors}).

\subsubsection{\mintinline{icl}{element.click () : element}}

Simulează un clic pe centrul elementului, și așteaptă încărcarea paginii.

Excepții posibile: \ferror{NoSessions}, \ferror{EmptyElement}, \ferror{MultiElement}, \ferror{NoSuchWindow}, \ferror{InvalidElement}, \ferror{ElementNotInteractable}, \ferror{ElementClickIntercepted} și \ferror{StaleElementReference} (pr. tab. \ref{errors}).

\subsubsection{\mintinline{icl}{element.query (by = By'cssSelector, selector : string) : element}}

Parametrul \mintinline{icl}{by} primește una din următoarele valori:
\begin{icItems}
    \item \mintinline{icl}{[r/o] By'cssSelector : 1} - selector CSS;
	\item \mintinline{icl}{[r/o] By'linkText : 2} - textul linkului;
	\item \mintinline{icl}{[r/o] By'partialLinkText : 3} - fragment al textului linkului;
	\item \mintinline{icl}{[r/o] By'tagName : 4} - denumirea tagului;
	\item \mintinline{icl}{[r/o] By'xPath : 5} - XPath.
\end{icItems}

Parametru \mintinline{icl}{selector} primește un selector CSS, un text al linkului, un fragment de text al linkului, un nume de tag sau un XPath, în dependență de valoarea primului argument.

Metoda returnează un element nou care conține primul element, găsit după criteriul necesar în elementul current.

Excepții posibile: \ferror{NoSessions}, \ferror{EmptyElement}, \ferror{ElementNotFound}, \ferror{MultiElement}, \ferror{NoSuchWindow}, \ferror{NoSuchElement}, \ferror{InvalidSelector}, \ferror{StaleElementReference} (pr. tab. \ref{errors}).

\subsubsection{\mintinline{icl}{element.query (by = By'cssSelector, selector : string) : element}}

Primește aceiași parametri ca și \mintinline{icl}{element.query}, numai că metoda dată returnează o colecție din elementele găsite, sau o colecție deșartă dacă nimic nu o fost găsit.

Excepții posibile: \ferror{NoSessions}, \ferror{EmptyElement}, \ferror{MultiElement}, \ferror{NoSuchWindow} și \ferror{StaleElementReference} (pr. tab. \ref{errors}).

\subsubsection{\mintinline{icl}{element.queryAllByXPath (xpath : string) : element}}

Acronim pentru \mintinline{icl}{element.queryAll (By'xPath, @xpath)};

\subsubsection{\mintinline{icl}{element.queryByXPath (xpath : string) : element}}

Acronim pentru \mintinline{icl}{element.query (By'xPath, @xpath)};

\subsubsection{\mintinline{icl}{element.queryLink (name : string, isFragment = false) : element}}

Acronim pentru:
\begin{icItems}
	\item \mintinline{icl}{element.query (By'linkText, @name)};
	\item \mintinline{icl}{element.query (By'partialLinkText, @name)}.
\end{icItems}

\subsubsection{\mintinline{icl}{element.queryLinks (name : string, isFragment = false) : element}}

Acronim pentru:
\begin{icItems}
	\item \mintinline{icl}{element.queryAll (By'linkText, @name)};
	\item \mintinline{icl}{element.queryAll (By'partialLinkText, @name)}.
\end{icItems}

\subsubsection{\mintinline{icl}{element.queryTag (name : string) : element}}

Acronim pentru \mintinline{icl}{element.query (By'tagName, @name)};

\subsubsection{\mintinline{icl}{element.queryTags (name : string) : element}}

Acronim pentru \mintinline{icl}{element.queryAll (By'tagName, @name)};

\subsubsection{\mintinline{icl}{element.screenshot () : string}}

Returnează un șir de caractere, care conține captura elementului, codată în base64. Captura poate fi salvată ca imagine cu \mintinline{icl}{Make.image (base64 : string, path : string) :}\\*\mintinline{icl}{void}.

Excepții posibile: \ferror{NoSessions}, \ferror{EmptyElement}, \ferror{MultiElement}, \ferror{NoSuchWindow} și \ferror{StaleElementReference} (pr. tab. \ref{errors}).

\subsubsection{\mintinline{icl}{element.sendKeys (modifiers : int, text : string) : element}}

Parametrul \mintinline{icl}{modifiers} primește una din valorile următoare (sau suma din cîteva dintre ele):
\begin{icItems}
	\item \mintinline{icl}{[r/o] Key'ctrl : 1} - Control;
	\item \mintinline{icl}{[r/o] Key'shift : 2} - Shift;
	\item \mintinline{icl}{[r/o] Key'alt : 4} - Alt;
\end{icItems}

Parametrul \mintinline{icl}{text} primește textul, care va fi tastat tastatură.

Excepții posibile: \ferror{NoSessions}, \ferror{EmptyElement}, \ferror{MultiElement}, \ferror{ElementNotIntractable} \ferror{NoSuchWindow} și \ferror{StaleElementReference} (pr. tab. \ref{errors}).

\subsection{Metode adiționale}

Metodele adiționale pot lucra încet în regimul de testare la folosirea browserului extern. În cazul colecției, operațiile vor fi aplicate pentru fiecare element, și vor fi returnate colecții, nu elemente unice. 

Lista metodelor adiționale:
\begin{icItems}
	\item \mintinline{icl}{element.add (other : element) : element};
	\item \mintinline{icl}{element.child (index : int) : element};
	\item \mintinline{icl}{element.closest (cssSelector : string) : element};
	\item \mintinline{icl}{element.contains (template : string) : element};
	\item \mintinline{icl}{element.copy () : element};
	\item \mintinline{icl}{element.get (i : int) : element};
	\item \mintinline{icl}{element.filter (cssSelector : string) : element};
	\item \mintinline{icl}{element.next () : element};
	\item \mintinline{icl}{element.prev () : element};
	\item \mintinline{icl}{element.parent () : element};
\end{icItems}

\subsubsection{\mintinline{icl}{element.add (other : element) : element}}

Adăugă toate elementele colecției \mintinline{icl}{other}.

Excepții posibile: \ferror{NoSessions}, \ferror{NoSuchWindow}, \ferror{StaleElementReference} (pr. tab. \ref{errors}).

\subsubsection{\mintinline{icl}{element.child (i : int) : element}}

Returnează un element nou, care conține al \mintinline{icl}{i}-lea succesor.

Excepții posibile: \ferror{NoSessions}, \ferror{NoSuchWindow}, \ferror{StaleElementReference}, \ferror{OutOfBounds} (pr. tab. \ref{errors}).

\subsubsection{\mintinline{icl}{element.closest (cssSelector : string) : element}}

Returnează un element nou, care conține cel mai apropiat predecesor, care este compatibil cu selectorul CSS \mintinline{icl}{cssSelector}.

Excepții posibile: \ferror{NoSessions}, \ferror{NoSuchWindow}, \ferror{StaleElementReference} (pr. tab. \ref{errors}).

\subsubsection{\mintinline{icl}{element.contains (text : string) : element}}

Returnează o colecție nouă, care conține toate elementele, care conțin un fragment de text care convine șablonului \mintinline{icl}{text}.

Excepții posibile: \ferror{NoSessions}, \ferror{NoSuchWindow}, \ferror{StaleElementReference} (pr. tab. \ref{errors}).

\subsubsection{\mintinline{icl}{element.copy () : element}}

Returnează o colecție nouă, care conține toate elementele colecției curente.

Excepții posibile: \ferror{NoSessions}, \ferror{NoSuchWindow}, \ferror{StaleElementReference} (pr. tab. \ref{errors}).

\subsubsection{\mintinline{icl}{element.get (i : int) : element}}

Returnează al i-lea element al colecției.

Excepții posibile: \ferror{NoSessions}, \ferror{NoSuchWindow}, \ferror{StaleElementReference}, \ferror{OutOfBounds} (pr. tab. \ref{errors}).

\subsubsection{\mintinline{icl}{element.filter (cssSelector : string) : element}}

Returnează o colecție nouă, care conține doar elementele care sunt compatibile cu selectorul CSS \mintinline{icl}{cssSelector}.

Excepții posibile: \ferror{NoSessions}, \ferror{NoSuchWindow}, \ferror{StaleElementReference} (pr. tab. \ref{errors}).
\subsubsection{\mintinline{icl}{element.next () : element}}

Returnează un element nou, care conține următorul nod.

Excepții posibile: \ferror{NoSessions}, \ferror{NoSuchWindow}, \ferror{StaleElementReference} (pr. tab. \ref{errors}).

\subsubsection{\mintinline{icl}{element.prev () : element}}

Returnează un element nou, care conține nodul precedent.

Excepții posibile: \ferror{NoSessions}, \ferror{NoSuchWindow}, \ferror{StaleElementReference} (pr. tab. \ref{errors}).

\subsubsection{\mintinline{icl}{element.parent () : element}}

Returnează ul element nou, care conține nodul ascensor.

Excepții posibile: \ferror{NoSessions}, \ferror{NoSuchWindow}, \ferror{StaleElementReference} (pr. tab. \ref{errors}).

\subsection{Lista proprietăților prestabilite}
\label{elements:predefined:properties}

În icL sunt prestabilite doar proprietățile ce au tipuri de date compatibile cu tipurile de date prezente în icL:
\begin{icItems}
	\item Node:	
	\begin{icItems}
		\item \mintinline{icl}{[r/o] element'prop-childNodes : element [c]};
		\item \mintinline{icl}{[r/o] element'prop-firstChild : element [i]};
		\item \mintinline{icl}{[r/o] element'prop-innerText : string};
		\item \mintinline{icl}{[r/o] element'prop-isConnected : bool};
		\item \mintinline{icl}{[r/o] element'prop-lastChild : element [i]};
		\item \mintinline{icl}{[r/o] element'prop-nodeName : string};
		\item \mintinline{icl}{[r/o] element'prop-nodeType : int};
		\item \mintinline{icl}{[r/o] element'prop-nodeValue : string};
		\item \mintinline{icl}{[r/o] element'prop-parentElement : element [i]};
		\item \mintinline{icl}{[r/o] element'prop-textContent : string};
	\end{icItems}
	
	\item Element:
	\begin{icItems}
		\item \mintinline{icl}{[r/o] element'prop-className : string};
		\item \mintinline{icl}{[r/o] element'prop-clientHeight : double};
		\item \mintinline{icl}{[r/o] element'prop-clientLeft : double};
		\item \mintinline{icl}{[r/o] element'prop-clientTop : double};
		\item \mintinline{icl}{[r/o] element'prop-clientWidth : double};
		\item \mintinline{icl}{[r/o] element'prop-computedName : string};
		\item \mintinline{icl}{[r/o] element'prop-computedRole : string};
		\item \mintinline{icl}{[r/o] element'prop-id : string};
		\item \mintinline{icl}{[r/o] element'prop-innerHTML : string};
		\item \mintinline{icl}{[r/o] element'prop-localName : string};
		\item \mintinline{icl}{[r/o] element'prop-nextElementSibling : element [i]};
		\item \mintinline{icl}{[r/o] element'prop-outerHTML : string};
		\item \mintinline{icl}{[r/o] element'prop-prefix : string};
		\item \mintinline{icl}{[r/o] element'prop-previousElementSibling : element [i]};
		\item \mintinline{icl}{[r/o] element'prop-scrollHeight : double};
		\item \mintinline{icl}{[r/o] element'prop-scrollLeft : double};
		\item \mintinline{icl}{[r/o] element'prop-scrollTop : double};
		\item \mintinline{icl}{[r/o] element'prop-scrollWidth : double};
		\item \mintinline{icl}{[r/o] element'prop-tagName : string};
		\item \mintinline{icl}{[r/o] element'prop-baseURI : string};
	\end{icItems}
	
	\item HTMLElement:
	\begin{icItems}
		\item \mintinline{icl}{[r/o] element'prop-accessKey : string};
		\item \mintinline{icl}{[r/o] element'prop-accessKeyLabel : string};
		\item \mintinline{icl}{[r/o] element'prop-contentEditable : string};
		\item \mintinline{icl}{[r/o] element'prop-isContentEditable : bool};
		\item \mintinline{icl}{[r/o] element'prop-dataset : object};
		\item \mintinline{icl}{[r/o] element'prop-dir : string};
		\item \mintinline{icl}{[r/o] element'prop-draggable : bool};
		\item \mintinline{icl}{[r/o] element'prop-hidden : bool};
		\item \mintinline{icl}{[r/o] element'prop-inert : bool};
		\item \mintinline{icl}{[r/o] element'prop-lang : string};
		\item \mintinline{icl}{[r/o] element'prop-offsetHeight : double};
		\item \mintinline{icl}{[r/o] element'prop-offsetLeft : double};
		\item \mintinline{icl}{[r/o] element'prop-offsetParent : element [i]};
		\item \mintinline{icl}{[r/o] element'prop-offsetTop : double};
		\item \mintinline{icl}{[r/o] element'prop-offsetWidth : double};
		\item \mintinline{icl}{[r/o] element'prop-spellcheck : double};
		\item \mintinline{icl}{[r/o] element'prop-title : string};
	\end{icItems}
	
	\item HTMLAnchorElement:
	\begin{icItems}
		\item \mintinline{icl}{[r/o] element'prop-download : string};
		\item \mintinline{icl}{[r/o] element'prop-hash : string};
		\item \mintinline{icl}{[r/o] element'prop-host : string};
		\item \mintinline{icl}{[r/o] element'prop-hostname : string};
		\item \mintinline{icl}{[r/o] element'prop-href : string};
		\item \mintinline{icl}{[r/o] element'prop-hreflang : string};
		\item \mintinline{icl}{[r/o] element'prop-media : string};
		\item \mintinline{icl}{[r/o] element'prop-password : string};
		\item \mintinline{icl}{[r/o] element'prop-origin : string};
		\item \mintinline{icl}{[r/o] element'prop-pathname : string};
		\item \mintinline{icl}{[r/o] element'prop-port : string};
		\item \mintinline{icl}{[r/o] element'prop-protocol : string};
		\item \mintinline{icl}{[r/o] element'prop-rel : string};
		\item \mintinline{icl}{[r/o] element'prop-search : string};
		\item \mintinline{icl}{[r/o] element'prop-target : string};
		\item \mintinline{icl}{[r/o] element'prop-text : string};
		\item \mintinline{icl}{[r/o] element'prop-type : string};
		\item \mintinline{icl}{[r/o] element'prop-username : string};
	\end{icItems}
	
	\item HTMLAreaElement:
	\begin{icItems}
		\item \mintinline{icl}{[r/o] element'prop-alt : string};
		\item \mintinline{icl}{[r/o] element'prop-coords : string};
	\end{icItems}
	
	\item HTMLButtonElement:
	\begin{icItems}
		\item \mintinline{icl}{[r/o] element'prop-autofocus : bool};
		\item \mintinline{icl}{[r/o] element'prop-disabled : bool};
		\item \mintinline{icl}{[r/o] element'prop-form : element [i]};
		\item \mintinline{icl}{[r/o] element'prop-formAction : string};
		\item \mintinline{icl}{[r/o] element'prop-formEnctype : string};
		\item \mintinline{icl}{[r/o] element'prop-formMethod : string};
		\item \mintinline{icl}{[r/o] element'prop-formNoValidate : bool};
		\item \mintinline{icl}{[r/o] element'prop-formTarget : string};
		\item \mintinline{icl}{[r/o] element'prop-labels : element [c]};
		\item \mintinline{icl}{[r/o] element'prop-name : string};
		\item \mintinline{icl}{[r/o] element'prop-value : string};
		\item \mintinline{icl}{[r/o] element'prop-willValidate  : bool};
	\end{icItems}
	
	\item HTMLCanvasElement:
	\begin{icItems}
		\item \mintinline{icl}{[r/o] element'prop-height : int};
		\item \mintinline{icl}{[r/o] element'prop-width : int};
	\end{icItems}
	
	\item HTMLDataListElement: \mintinline{icl}{[r/o] element'prop-options : element [c]};
	
	\item HTMLFormElement:
	\begin{icItems}
		\item \mintinline{icl}{[r/o] element'prop-elements : element [c]};
		\item \mintinline{icl}{[r/o] element'prop-length : int};
		\item \mintinline{icl}{[r/o] element'prop-action : string};
		\item \mintinline{icl}{[r/o] element'prop-encoding : string};
		\item \mintinline{icl}{[r/o] element'prop-enctype : string};
		\item \mintinline{icl}{[r/o] element'prop-acceptCharset : string};
		\item \mintinline{icl}{[r/o] element'prop-autocomplete : string};
		\item \mintinline{icl}{[r/o] element'prop-noValidate : string};
	\end{icItems}
	
	\item HTMLIFrameElement: \mintinline{icl}{[r/o] element'prop-allowPaymentRequest};
	
	\item HTMLImageElement:
	\begin{icItems}
		\item \mintinline{icl}{[r/o] element'prop-complete : bool};
		\item \mintinline{icl}{[r/o] element'prop-crossOrigin : string};
		\item \mintinline{icl}{[r/o] element'prop-isMap : bool};
		\item \mintinline{icl}{[r/o] element'prop-naturalHeight : int};
		\item \mintinline{icl}{[r/o] element'prop-naturalWidth : int};
		\item \mintinline{icl}{[r/o] element'prop-src : string};
		\item \mintinline{icl}{[r/o] element'prop-useMap : string};
	\end{icItems}
	
	\item HTMLInputElement:
	\begin{icItems}
		\item \mintinline{icl}{[r/o] element'prop-accept : string};
		\item \mintinline{icl}{[r/o] element'prop-checked : bool};
		\item \mintinline{icl}{[r/o] element'prop-defaultChecked : bool};
		\item \mintinline{icl}{[r/o] element'prop-defaultValue : string};
		\item \mintinline{icl}{[r/o] element'prop-dirName : string};
		\item \mintinline{icl}{[r/o] element'prop-indeterminate : bool};
		\item \mintinline{icl}{[r/o] element'prop-list : element [i]};
		\item \mintinline{icl}{[r/o] element'prop-min : string};
		\item \mintinline{icl}{[r/o] element'prop-max : string};
		\item \mintinline{icl}{[r/o] element'prop-maxLength : int};
		\item \mintinline{icl}{[r/o] element'prop-multiple : bool};
		\item \mintinline{icl}{[r/o] element'prop-pattern : string};
		\item \mintinline{icl}{[r/o] element'prop-placeholder : string};
		\item \mintinline{icl}{[r/o] element'prop-readOnly : bool};
		\item \mintinline{icl}{[r/o] element'prop-required : bool};
		\item \mintinline{icl}{[r/o] element'prop-selectionStart : int};
		\item \mintinline{icl}{[r/o] element'prop-selectionEnd : int};
		\item \mintinline{icl}{[r/o] element'prop-selectionDirection : string};
		\item \mintinline{icl}{[r/o] element'prop-size : int};
		\item \mintinline{icl}{[r/o] element'prop-step : string};
		\item \mintinline{icl}{[r/o] element'prop-validity : bool};
		\item \mintinline{icl}{[r/o] element'prop-validationMessage : string};
		\item \mintinline{icl}{[r/o] element'prop-valueAsNumber : double};
	\end{icItems}
	
	\item HTMLLabelElement:
	\begin{icItems}
		\item \mintinline{icl}{[r/o] element'prop-control : element [i]};
		\item \mintinline{icl}{[r/o] element'prop-htmlFor : string};
	\end{icItems}
	
	\item HTMLLinkElement: \mintinline{icl}{[r/o] element'prop-as : string};
	\item HTMLMapElement: \mintinline{icl}{[r/o] element'prop-areas : element [c]};
	
	\item HTMLMediaElement:
	\begin{icItems}
		\item \mintinline{icl}{[r/o] element'prop-autoplay : bool};
		\item \mintinline{icl}{[r/o] element'prop-controls : bool};
		\item \mintinline{icl}{[r/o] element'prop-currentSrc : string};
		\item \mintinline{icl}{[r/o] element'prop-currentTime : double};
		\item \mintinline{icl}{[r/o] element'prop-defaultMuted : bool};
		\item \mintinline{icl}{[r/o] element'prop-defaultPlaybackRate : bool};
		\item \mintinline{icl}{[r/o] element'prop-disableRemotePlayback : bool};
		\item \mintinline{icl}{[r/o] element'prop-duration : double};
		\item \mintinline{icl}{[r/o] element'prop-ended : bool};
		\item \mintinline{icl}{[r/o] element'prop-loop : bool};
		\item \mintinline{icl}{[r/o] element'prop-mediaGroup : string};
		\item \mintinline{icl}{[r/o] element'prop-muted : bool};
		\item \mintinline{icl}{[r/o] element'prop-networkState : int};
		\item \mintinline{icl}{[r/o] element'prop-paused : bool};
		\item \mintinline{icl}{[r/o] element'prop-playbackRate : double};
		\item \mintinline{icl}{[r/o] element'prop-preload : string};
		\item \mintinline{icl}{[r/o] element'prop-readyState : int};
		\item \mintinline{icl}{[r/o] element'prop-seeking : bool};
		\item \mintinline{icl}{[r/o] element'prop-volume : double};
	\end{icItems}
	
	\item HTMLMetaElement:
	\begin{icItems}
		\item \mintinline{icl}{[r/o] element'prop-content : string};
		\item \mintinline{icl}{[r/o] element'prop-httpEquiv : string};
	\end{icItems}
	
	\item HTMLMeterElement:
	\begin{icItems}
		\item \mintinline{icl}{[r/o] element'prop-high : double};
		\item \mintinline{icl}{[r/o] element'prop-low : double};
	\end{icItems}
	
	\item HTMLModElement: \mintinline{icl}{[r/o] element'prop-cite : string};
	
	\item HTMLOListElement:
	\begin{icItems}
		\item \mintinline{icl}{[r/o] element'prop-reversed : bool};
		\item \mintinline{icl}{[r/o] element'prop-start : int};
	\end{icItems}
	
	\item HTMLOptionElement:
	\begin{icItems}
		\item \mintinline{icl}{[r/o] element'prop-defaultSelected : bool};
		\item \mintinline{icl}{[r/o] element'prop-index : int};
		\item \mintinline{icl}{[r/o] element'prop-label : string};
		\item \mintinline{icl}{[r/o] element'prop-selected : bool};
	\end{icItems}
	
	\item HTMLProgressElement: \mintinline{icl}{[r/o] element'prop-position : double};
	
	\item HTMLScriptElement:
	\begin{icItems}
		\item \mintinline{icl}{[r/o] element'prop-charset : string};
		\item \mintinline{icl}{[r/o] element'prop-async : bool};
		\item \mintinline{icl}{[r/o] element'prop-defer : bool};
		\item \mintinline{icl}{[r/o] element'prop-noModule : bool};
	\end{icItems}
	
	\item HTMLSelectElement:
	\begin{icItems}
		\item \mintinline{icl}{[r/o] element'prop-selectedIndex : int};
		\item \mintinline{icl}{[r/o] element'prop-selectedOptions : element [c]};
	\end{icItems}
	
	\item HTMLTableCellElement:
	\begin{icItems}
		\item \mintinline{icl}{[r/o] element'prop-abbr : string};
		\item \mintinline{icl}{[r/o] element'prop-cellIndex : int};
		\item \mintinline{icl}{[r/o] element'prop-colSpan : int};
		\item \mintinline{icl}{[r/o] element'prop-rowSpan : int};
		\item \mintinline{icl}{[r/o] element'prop-scope : string};
	\end{icItems}
	
	\item HTMLTableColElement: \mintinline{icl}{[r/o] element'prop-span : int};
	
	\item HTMLTableElement:
	\begin{icItems}
		\item \mintinline{icl}{[r/o] element'prop-caption : element [i]};
		\item \mintinline{icl}{[r/o] element'prop-tBodies : element [c]};
		\item \mintinline{icl}{[r/o] element'prop-tHead : element [i]};
		\item \mintinline{icl}{[r/o] element'prop-tFoot : element [i]};
	\end{icItems}
	
	\item HTMLTableRowElement:
	\begin{icItems}
		\item \mintinline{icl}{[r/o] element'prop-cells : element [c]};
		\item \mintinline{icl}{[r/o] element'prop-rowIndex : int};
	\end{icItems}
	
	\item HTMLTextAreaElement:
	\begin{icItems}
		\item \mintinline{icl}{[r/o] element'prop-cols : int};
		\item \mintinline{icl}{[r/o] element'prop-textLength : int};
		\item \mintinline{icl}{[r/o] element'prop-wrap : string};
	\end{icItems}
	
	\item HTMLTimeElement: \mintinline{icl}{[r/o] element'prop-dateTime : string};
	
	\item HTMLTrackElement:
	\begin{icItems}
		\item \mintinline{icl}{[r/o] element'prop-kind : string};
		\item \mintinline{icl}{[r/o] element'prop-srclang : string};
		\item \mintinline{icl}{[r/o] element'prop-label : string};
		\item \mintinline{icl}{[r/o] element'prop-default : bool};
	\end{icItems}
	
	\item HTMLVideoElement:
	\begin{icItems}
		\item \mintinline{icl}{[r/o] element'prop-poster : string};
		\item \mintinline{icl}{[r/o] element'prop-videoHeight : int};
		\item \mintinline{icl}{[r/o] element'prop-videoWidth : int};
	\end{icItems}
	
	% \item \mintinline{icl}{[r/w] element'prop-};
\end{icItems}

\mintinline{icl}{element [c]} întotdeauna va fi o colecție, \mintinline{icl}{element [i]} - variază în dependența de subiect, dacă subiectul este un element unic, atunci rezultatul va vi un element unic, dacă subiectul este o colecție, atunci rezultatul va fi de asemenea colecție. 

%\newpage
