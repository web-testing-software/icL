\documentclass[a4paper, 14pt]{extarticle}

\usepackage[top=2.5cm, bottom=1.5cm, left=2cm, right=1cm]{geometry}
\usepackage{fancyvrb}
\fvset{tabsize=2}
\usepackage{polyglossia}
\setmainlanguage{russian} 
\setotherlanguage{english}
  
\setmainfont{Liberation Serif}
\newfontfamily\cyrillicfont{Liberation Serif}
\setmonofont{Liberation Mono}
\newfontfamily\cyrillicfontmono{Liberation Mono}

\usepackage{indentfirst}
\setlength{\parindent}{1.25cm}

\usepackage{caption}
\usepackage{listings, lstautogobble}
\usepackage{color}

\usepackage{caption}
\DeclareCaptionFont{black}{\color{black}}
\DeclareCaptionFormat{listing}{\colorbox{codeheaderbg}
	{\parbox{\textwidth}{#1#2#3}}}
\captionsetup[lstlisting]
	{format=listing,labelfont=black,textfont=black}

\usepackage{enumitem}

\lstdefinestyle{framed}
{
     frame=none,         
     belowcaptionskip=2pt,
     xleftmargin=8pt,
     framexleftmargin=7pt,
     framexrightmargin=5pt,
     framextopmargin=10pt,
     framexbottommargin=5pt,
     framesep=0pt,
     rulesep=0pt,
 }
 
\definecolor{lightgray}{rgb}{.9,.9,.9}
\definecolor{darkgray}{rgb}{.4,.4,.4}
\definecolor{purple}{rgb}{0.65, 0.12, 0.82}

\definecolor{codebg}{rgb}{0.97, 0.97, 0.97}
\definecolor{codeheaderbg}{rgb}{0.90, 0.90, 0.90}
\definecolor{orange}{rgb}{0.8, 0.4, 0.0}
\definecolor{bluemarin}{RGB}{12, 134, 145}
\definecolor{grey}{rgb}{0.5, 0.5, 0.5}
\definecolor{function}{RGB}{11, 145, 89}
\definecolor{blue2}{RGB}{15, 92, 198}

\lstdefinelanguage{icL}{
	keywords={if, for, filter, range, exists, while, do, any,
			  emit, emiter, slot},
	keywordstyle=\color{blue2},
	keywords=[2]{boolean, int, double, string, list, element,
				 set, item, object, function},
	keywordstyle=[2]\color{bluemarin},
	classoffset=3,
	morekeywords={_log, _define, _tab, _dom, _define, _import},
	keywordstyle=\color{orange},
	classoffset=4,
	morekeywords={info, warm, error, signal, get, query,
				  queryAll, null, Prepend, Append, Insert,
				  Merge, PopFront, PopBack, Remove, RemoveOnce,
				  RemoveAll, Get, IndexOf, LastIndexOf,
				  Join, SumUp, Max, Min, LogicAnd, LogicOr,
				  Length, Text, HTML, Width, Height, Click,
				  ScrollTo, SendKeys, IsValid, Copy, Add, Filter,
				  Query, QueryAll, Visible, Clickable, Next,
				  Prev, Parent, Child, Closest, AddClass,
				  HasClass, RemoveClass, all, functions, none,
				  run},
	keywordstyle=\color{function},
	classoffset=5,
	morekeywords={UnrealCast, StringParse, EmptyList,
				  MultipleValues, NullElement, NotVisible,
				  WrongName, OutOfBounds, OutOfScreen,
				  FileNotFound},
	classoffset=0,
	identifierstyle=\color{black},
	sensitive=true,
	comment=[l]{``},
	morecomment=[s]{```}{```},
	morecomment=[s]{`c}{t`},
	commentstyle=\color{grey},
	stringstyle=\color{purple},
	morestring=[b]",
	autogobble=true
}

\newcommand\digitstyle{\color{red}}
\makeatletter
\newcommand{\ProcessDigit}[1]
{%
  \ifnum\lst@mode=\lst@Pmode\relax%
   {\digitstyle #1}%
  \else
    #1%
  \fi
}
\makeatother

  
\begin{document}	

	\lstset{
		language=icL,
		literate=
	  		{0}{{{\ProcessDigit{0}}}}1
	  		{1}{{{\ProcessDigit{1}}}}1
	  		{2}{{{\ProcessDigit{2}}}}1
	  		{3}{{{\ProcessDigit{3}}}}1
	  		{4}{{{\ProcessDigit{4}}}}1
	  		{5}{{{\ProcessDigit{5}}}}1
	  		{6}{{{\ProcessDigit{6}}}}1
	  		{7}{{{\ProcessDigit{7}}}}1
	  		{8}{{{\ProcessDigit{8}}}}1
	  		{9}{{{\ProcessDigit{9}}}}1
	  		{-}{{{\ProcessDigit{-}}}}1
	  		{<=}{{\(\leq\)}}1,
		extendedchars=true,
		basicstyle=\footnotesize\ttfamily,
		showstringspaces=false,
		showspaces=false,
		showtabs=false,
		numbers=left,
		stepnumber=1,
		showstringspaces=false,
		tabsize=4,
		breaklines=true,
		breakatwhitespace=true,
		backgroundcolor=\color{codebg},
		style=framed
	}

	\title{Стандарт языка описания сценариях icL v1.0.0 черновик}
	\author{Лелицак Василе}

	\maketitle
	
	\newpage
	\tableofcontents
	
	\newpage

	\section{Введение}	
	
	\indent icL - язык описания сценариев, оптимизирован под описания сценариев тестирования веб-приложений.
	
	\subsection{Читатели}
	
	Этот документ предназначен для всех тех людей, которые ищут отравную точку, откуда можно начать изучать язык icL. Также данный документ используется при разработке интерпретатора, поведения интерпретатора во всех ситуациях не описанных в данном документе считается не определённой.
	
	\subsection{Что вы должны уметь}
	
	Прежде чем выступать к изучения этого языка, вам желательно иметь базовое представление о компьютерном программировании.
	
	\subsection{Обзор языка icL}
	
	icL - язык сценариев тестирования веб-приложений. Его разработка началась в 2017 году и первый выпуск планируется к 2020 году. В настоящий момент он находится в активном разработке.
	
	icL - язык с С-подобным синтаксисом, который использует статическую типизацию. В icL нельзя определить собственные типы данных, так как он разработан не для программистов, знания получены в школе на уроках информатики должны быть достаточными. Язык icL поддерживает только одну парадигму программирования - процедурная. При необходимости обработать данные, можно использовать экспорт/импорт в/из cvs и базы данных.
	
	\subsection{Пример кода}

	В icL точка входа в программе - начало файла, программа \textit{Hello world!} иллюстрирована на листинге \ref{example0}.
	
	\begin{lstlisting}[caption=Пример, label=example0]
`` comment example
_log.info "Hello world";
	\end{lstlisting}
  
	\subsection{Изучение icL}
	
	Самое важное при изучении icL - это сосредоточиться на идеях и не потеряться в технических деталях его реализации.
	
	\subsection{Области применения icL}
	
	Язык icL является частью программы icL, с его помощью
	можно управлять браузером, а именно:
	\begin{list}{•}{}
	\item открыть вкладку;
	\item закрыть вкладку;
	\item перейти на веб-страницу;
	\item симулировать события клавиатуры и мышке;
	\item взаимодействовать с веб-страницей;
	\item выполнить код на языке javascript;
	\item управлять веб-страницей;
	\item обменять информацию с веб-страничкой;
	\item сделать screenshot;
	\item сохранить страницу в формате pdf;
	\item управлять памятью;
	\item экспортировать данные в csv файле;
	\item импортировать данные из csv файл;
	\item выполнить запросы на языке SQL.
	\end{list}
	
	\subsection{Начало работы}
	
	Чтобы начать работать достаточно установить и запустить программу icL.
	
	\newpage
	{\color{red}Материал для продвинутых пользователей.}
  	
  	\begin{enumerate}
  	\item - Введение
  	\item - Базовый синтаксис
  	\item - Переменные
  	\item - Типы данных
  	\item - Литералы
  	\item - Операторы
  	\item - Циклы
  	\item - Условные операторы
  	\item - Интеграция с Javascript
  	\item - Функции
  	\item - Строки
  	\item - Списки
  	\item - Многоженства
  	\item - Объекты
  	\item - Веб-элементы (симуляция мышки и клавиатуры)
  	\item - Обмен данных в веб-странницей (конкретные примеры)
  	\item - DSV / CSV / TSV
  	\end{enumerate}
  	
  	{\color{red}далее материал для программистов, дополнительные возможности, просьба не читать далее материал если нету желание или необходимость.}
	
	\begin{enumerate}[resume]
  	\item - Базы данных (только SQLite в первой версии)
  	\item - Обработка ошибок
  	\item - Программирование errorless
  	\item - Программирование на лету
  	\end{enumerate}
	
\end{document}
