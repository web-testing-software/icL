% !TeX spellcheck = en_US
\section{Control operators}

The control operators have condition and command group. The execution of a command group depends on value of condition.
 
 List of control operators in icL:
\begin{icItems}
	\item \mintinline{icl}{if};
	\item \mintinline{icl}{if-else};
	\item cascade \mintinline{icl}{if-else};
	\item \mintinline{icl}{switch-case};
\end{icItems}

\subsubsection{\mintinline{icl}{if}}

\mintinline{icl}{if} switches executing of a group of commands by a condition —
\begin{minted}{icl}
if (condition) {
	commands
};
\end{minted}
Here \mintinline{icl}{condtition} is an expression which returns \bool{} and \mintinline{icl}{commands} is a set of statements.

The round brackets can be omitted if the expression is simple, syntax —
\begin{minted}{icl}
if true  { commands };
if !true { commands };
if @var  { commands };
if !@var { commands };
\end{minted}

\subsubsection{\mintinline{icl}{if-else}}

\mintinline{icl}{if-else} selects one of two sets of statements, if condition is \true{} the first set of statements gets executed, otherwise — the second.

Syntax of \mintinline{icl}{if-else} —
\begin{minted}{icl}
if (condition) {
	commands1
}
else {
	commands2
};
\end{minted}

\subsubsection{Cascade \mintinline{icl}{if-else}}

Cascade \mintinline{icl}{if-else} selects one of n sets of statements using n-1 conditions.

Syntax of cascade \mintinline{icl}{if-else} —
\begin{minted}{icl}
if (condition1) {
	commands1
} else if (condition2) {
	commands2
} else {
	commands3
};
\end{minted}

\subsubsection{\mintinline{icl}{switch-case}}

\mintinline{icl}{switch-case} selects m (one or more) sets of statements form n sets of statements, using n or more values, where m <= n.

Syntax —
\begin{minted}{icl}
switch (value) {
	case (caseValue) { `code` }
	`` more cases
}
\end{minted}

\mintinline{icl}{switch} gets a \mintinline{icl}{value}, which can have any type excluding \bool. Next, in \mintinline{icl}{switch} shall be defined some cases (\mintinline{icl}{case}), if \mintinline{icl}{value == caseValue} then the set of statements gets executed.

A \mintinline{icl}{case} can contains some values delimited by comma.

A \mintinline{icl}{case} can get a \void{} value — \mintinline{icl}{case (~) { `code` }}, this type of \mintinline{icl}{case} will be executed if no one of previous cases was executed.

There is a special value — \mintinline{icl}{#}, it means that this \mintinline{icl}{case} with the condition that the previous \mintinline{icl}{case} was executed.

Example of using the \mintinline{icl}{switch-case} —
\inputminted[linenos]{icl}{../sources/switchcaseex.icL}

Variable \mintinline{icl}{@v} gets values of interval [0, 6].

\newpage
Let's see which sets of statements will be executed for each value of \mintinline{icl}{@v} —
\begin{icItems}
	\item 0 — will be executed \mintinline{icl}{code1};
	\item 1 — will be executed \mintinline{icl}{code2} and \mintinline{icl}{code3};
	\item 2 — will be executed \mintinline{icl}{code2} and \mintinline{icl}{code3};
	\item 3 — will be executed \mintinline{icl}{code3};
	\item 4 — will be executed \mintinline{icl}{code4};
	\item 5 — will be executed \mintinline{icl}{code5};
	\item 6 — will be executed \mintinline{icl}{code5}.
\end{icItems}


