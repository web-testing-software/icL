% !TeX spellcheck = en_US
\section{Synchronization}
\label{sync}

icL is not limited to synchronization with the load of web pages and execution of JavaScript code in page. It can be synchronized with the back-end of web application using technology \textit{icL-Sync}. List of icL technologies is available on \ferror{https://gitlab.com/lixcode/icL/tree/standard/technologies\#technologies}.

\subsubsection{Connection}

To receive messages form server must be created a spy, syntax —
\begin{iclcode}
listen addressOfServer : (parameters) {
	`` code
}
\end{iclcode}

Here \mintinline{icl}{addressOfServer} is the URL of synchronization service, \mintinline{icl}{parameters} is a list of parameters, messages with incompatible messages will be ignored. To one synchronization service can be connected some spies. If a parameter has an implicit value, the spy will catch just these messages, which will have the same value of argument. 

Example of spy which will catch just errors messages —
\inputminted[linenos]{icl}{../sources/errorcatch.icL}

\subsubsection{Disconnection} \mintinline{icl}{Stack.ignore()} or \mintinline{icl}{Stack.destroy()}. Read §\ref{stack:control} for more information.
