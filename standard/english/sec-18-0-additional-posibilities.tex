% !TeX spellcheck = en_US
% !TeX spellcheck = ro_RO
\part{Material pentru programatori}

Pentru a înțelege acest material trebuiesc cunoștințe în programare. În ea nu vor fi explicați termini tehnici. Necătînd la aceea că limbajul icL este simplu, în el se poate de realizat scripturi destul de complicate, de lucrat cu baze de date, expresii regulare și multe altele, se așteaptă dezvoltare intensă a acestei părți în următoarele versiuni ale limbajului icL.


\section{Additional Possibilities}

There are {\bf additional possibilities} for complete cover of W3C WebDriver standard and add some practices.

These possibilities are required for complex scripts. On start icL automatically start a session and close it on finish.

\subsection{{\color{orange} Sessions}}

The object \sessions{} has following properties —
\begin{icItems}
	\item \mintinline{icl}{[r/o] Sessions'length : int};
	\item \mintinline{icl}{[r/o] Sessions'(i : int) : Session}.
\end{icItems}

And following methods —
\begin{icItems}
	\item \mintinline{icl}{Sessions.closeAll () : void};
	\item \mintinline{icl}{Sessions.get (i : int) : Session};
	\item \mintinline{icl}{Sessions.new () : Session}.
\end{icItems}

\subsubsection{\mintinline{icl}{[r/o] Sessions'length : int}}

The number od opened sessions.

\subsubsection{\mintinline{icl}{[r/o] Sessions'(i : int) : Session}}

The \mintinline{icl}{i}th session.

Possible exceptions: \ferror{OutOfBounds} (see table \ref{errors}).

\subsubsection{\mintinline{icl}{Sessions.closeAll () : void}}

Close all sessions.

\subsubsection{\mintinline{icl}{Sessions.get (i : int) : Session}}

Returns the \mintinline{icl}{i}th session.

Possible exceptions: \ferror{OutOfBounds} (see table \ref{errors}).

\subsubsection{\mintinline{icl}{Sessions.new () : Session}}

Opens and returns a new session.

Possible exceptions: \ferror{SessionNotCreated} (see table \ref{errors}).

\subsection{{\color{orange} Session}}

\session{} returns a reference to current session.

The object \session{} has following properties —
\begin{icItems}
	\item \mintinline{icl}{[r/w] Session'implicitTimeout : int};
	\item \mintinline{icl}{[r/w] Session'pageLoadTimeout : int};
	\item \mintinline{icl}{[r/w] Session'scriptTimeout : int};
	\item \mintinline{icl}{[r/o] Session'source : string};
	\item \mintinline{icl}{[r/o] Session'title : string};
	\item \mintinline{icl}{[r/w] Session'url : string}.
\end{icItems}

And following methods —
\begin{icItems}
	\item \mintinline{icl}{Session.back () : Session};
	\item \mintinline{icl}{Session.close () : void};
	\item \mintinline{icl}{Session.forward () : Session};
	\item \mintinline{icl}{Session.refresh () : Session};
	\item \mintinline{icl}{Session.screenshot () : string};
	\item \mintinline{icl}{Session.switchTo () : Session}.
\end{icItems}

\subsubsection{\mintinline{icl}{[r/w] Session'implicitTimeout : int}}

Implicit timeout. By default, is 0.

Possible exceptions: \ferror{NoSessions}, \ferror{InvalidArgument} (see table \ref{errors}).

\subsubsection{\mintinline{icl}{[r/w] Session'pageLoadTimeout : int}}

Page load timeout. By default, 300000ms.

Possible exceptions: \ferror{NoSessions}, \ferror{InvalidArgument} (see table \ref{errors}).

\subsubsection{\mintinline{icl}{[r/w] Session'scriptTimeout : int}}

JavaScript code execution timeout. By default, 30000ms.

Possible exceptions: \ferror{NoSessions}, \ferror{InvalidArgument} (see table \ref{errors}).

\subsubsection{\mintinline{icl}{[r/o] Session'source : string}}

The source code of page.

Possible exceptions: \ferror{NoSessions}, \ferror{NoSuchWindow} (see table \ref{errors}).

\subsubsection{\mintinline{icl}{[r/o] Session'title : string}}

the title of page.

Possible exceptions: \ferror{NoSessions}, \ferror{NoSuchWindow} (see table \ref{errors}).

\subsubsection{\mintinline{icl}{[r/w] Session'url : string}}

URL of current page.

Possible exceptions: \ferror{NoSessions}, \ferror{NoSuchWindow} (see table \ref{errors}).

\subsubsection{\mintinline{icl}{Session.back () : Session}}

Is equivalent to pressing the back button in the browser.

Possible exceptions: \ferror{NoSessions}, \ferror{NoSuchWindow}, \ferror{Timeout} (see table \ref{errors}).

\subsubsection{\mintinline{icl}{Session.close () : void}}

Close session, all its tabs will be closed.

Possible exceptions: \ferror{NoSessions}, \ferror{NoSessions}, \ferror{NoSuchSession} (see table \ref{errors}).

\subsubsection{\mintinline{icl}{Session.forward () : Session}}

Is equivalent to pressing the forward button in the browser.

Possible exceptions: \ferror{NoSessions}, \ferror{NoSuchWindow}, \ferror{Timeout} (see table \ref{errors}).

\subsubsection{\mintinline{icl}{Session.refresh () : Session}}

Reloads page.

Possible exceptions: \ferror{NoSessions}, \ferror{NoSuchWindow}, \ferror{Timeout} (see table \ref{errors}).

\subsubsection{\mintinline{icl}{Session.screenshot () : string}}

Returns a base64 encoded screenshot.

Possible exceptions: \ferror{NoSessions}, \ferror{NoSuchWindow}, \ferror{UnableToCaptureScreen} (see table \ref{errors}).

\subsubsection{\mintinline{icl}{Session.switchTo () : Session}}

Focus this session.

Possible exceptions: \ferror{NoSuchSession} (see table \ref{errors}).

\subsection{{\color{orange} Windows}}

The object \windows{} has following properties —
\begin{icItems}
	\item \mintinline{icl}{[r/o] Windows'length : int};
	\item \mintinline{icl}{[r/o] Windows'(i : int) : Window}.
\end{icItems}

And a method \mintinline{icl}{Windows.get (i : int) : Window}.

\subsubsection{\mintinline{icl}{[r/o] Windows'length : int}}

The number of windows in current session.

Possible exceptions: \ferror{NoSessions} (see table \ref{errors}).

\subsubsection{\mintinline{icl}{[r/o] Windows'(i : int) : Window}}

The \mintinline{icl}{i}th window.

Possible exceptions: \ferror{NoSessions}, \ferror{OutOfBounds} (see table \ref{errors}).

\subsubsection{\mintinline{icl}{Windows.get (i : int) : Window}}

Returns the \mintinline{icl}{i}th window.

Possible exceptions: \ferror{NoSessions}, \ferror{OutOfBounds} (see table \ref{errors}).

\subsection{{\color{orange} Window}}

The object \window{} has following properties —
\begin{icItems}
	\item \mintinline{icl}{[r/w] Window'height : int};
	\item \mintinline{icl}{[r/w] Window'width : int};
	\item \mintinline{icl}{[r/w] Window'x : int};
	\item \mintinline{icl}{[r/w] Window'y : int}.
\end{icItems}

And following methods —
\begin{icItems}
	\item \mintinline{icl}{Window.close () : void};
	\item \mintinline{icl}{Window.focus () : Window};
	\item \mintinline{icl}{Window.fullscreen () : Window};
	\item \mintinline{icl}{Window.maximize () : Window};
	\item \mintinline{icl}{Window.minimize () : Window};
	\item \mintinline{icl}{Window.restore () : Window};
	\item \mintinline{icl}{Window.switchToDefault () : Window};
	\item \mintinline{icl}{Window.switchToFrame (i : int) : Window};
	\item \mintinline{icl}{Window.switchToFrame (el : element) : Window};
	\item \mintinline{icl}{Window.switchToParent () : Window}.
\end{icItems}

\subsubsection{\mintinline{icl}{[r/w] Window'height : int}}

The height of window in pixels.

Possible exceptions: \ferror{NoSessions}, \ferror{NoSuchWindow}, \ferror{InvalidArgument}, \ferror{UnsupportedOperation} (see table \ref{errors}).

\subsubsection{\mintinline{icl}{[r/w] Window'width : int}}

The width of window in pixels.

Possible exceptions: \ferror{NoSessions}, \ferror{NoSuchWindow}, \ferror{InvalidArgument}, \ferror{UnsupportedOperation} (see table \ref{errors}).

\subsubsection{\mintinline{icl}{[r/w] Window'x : int}}

The $x$ coordinate of window in pixels.

Possible exceptions: \ferror{NoSessions}, \ferror{NoSuchWindow}, \ferror{InvalidArgument}, \ferror{UnsupportedOperation} (see table \ref{errors}).

\subsubsection{\mintinline{icl}{[r/w] Window'y : int}}

The $y$ a coordinate of window in pixels.

Possible exceptions: \ferror{NoSessions}, \ferror{NoSuchWindow}, \ferror{InvalidArgument}, \ferror{UnsupportedOperation} (see table \ref{errors}).

\subsubsection{\mintinline{icl}{Window.close () : void}}

Close current window, if last will close the session.

Possible exceptions: \ferror{NoSessions}, \ferror{NoSuchWindow} (see table \ref{errors}).

\subsubsection{\mintinline{icl}{Window.focus () : Window}}

Focus this window.

Possible exceptions: \ferror{NoSessions}, \ferror{NoSuchWindow} (see table \ref{errors}).

\subsubsection{\mintinline{icl}{Window.fullscreen () : Window}}

Enter full-screen mode.

Possible exceptions: \ferror{NoSessions}, \ferror{NoSuchWindow} (see table \ref{errors}).

\subsubsection{\mintinline{icl}{Window.maximize () : Window}}

Maximize the browser window.

Possible exceptions: \ferror{NoSessions}, \ferror{NoSuchWindow} (see table \ref{errors}).

\subsubsection{\mintinline{icl}{Window.minimize () : Window}}

Minimize the browser window.

Possible exceptions: \ferror{NoSessions}, \ferror{NoSuchWindow} (see table \ref{errors}).

\subsubsection{\mintinline{icl}{Window.restore () : Window}}

Restore browser window.

Possible exceptions: \ferror{NoSessions}, \ferror{NoSuchWindow} (see table \ref{errors}).

\subsubsection{\mintinline{icl}{Window.switchToDefault () : Window}}

Restore focus to main frame.

Possible exceptions: \ferror{NoSessions}, \ferror{NoSuchWindow} (see table \ref{errors}).

\subsubsection{\mintinline{icl}{Window.switchToFrame (i : int) : Window}}

Switch focus to \mintinline{icl}{i}th frame.

Possible exceptions: \ferror{NoSessions}, \ferror{NoSuchWindow}, \ferror{NoSuchFrame} (see table \ref{errors}).

\subsubsection{\mintinline{icl}{Window.switchToFrame (el : element) : Window}}

Switch focus to element \mintinline{icl}{el}, the element must be a frame or iframe tag.

Possible exceptions: \ferror{NoSessions}, \ferror{NoSuchWindow}, \ferror{NoSuchFrame}, \ferror{StaleElementReference} (see table \ref{errors}).

\subsubsection{\mintinline{icl}{Window.switchToParent () : Window}}

Switch focut to parent frame.

Possible exceptions: \ferror{NoSessions}, \ferror{NoSuchWindow} (see table \ref{errors}).

\subsection{{\color{orange} Cookies}}

The object \cookies{} are următoarea proprietate: \mintinline{icl}{[r/o] Cookies'(name : string) :} \\*\mintinline{icl}{Cookie}.

And following methods — 
\begin{icItems}
	\item \mintinline{icl}{Cookies.deleteAll () : void};
	\item \mintinline{icl}{Cookies.get (name : string) : Cookie}.
\end{icItems}

\subsubsection{\mintinline{icl}{[r/o] Cookies'(name : string) : Cookie}}

Returnează \cookie{} cu numele \mintinline{icl}{name}.

Possible exceptions: \ferror{NoSessions}, \ferror{NoSuchWindow}, \ferror{NoSuchCookie} (see table \ref{errors}).

\subsubsection{\mintinline{icl}{Cookies.deleteAll : void}}

Șterge toate fișierele cookie.

\subsubsection{\mintinline{icl}{Cookies.get (name : string) : Cookie}}

Returnează \cookie{} cu numele \mintinline{icl}{name}.

Possible exceptions: \ferror{NoSessions}, \ferror{NoSuchWindow}, \ferror{NoSuchCookie} (see table \ref{errors}).

\subsection{{\color{orange} Cookie}}

The object \cookie{} has following properties —
\begin{icItems}
	\item \mintinline{icl}{[r/w] Cookie'domain : string};
	\item \mintinline{icl}{[r/w] Cookie'expiry : int};
	\item \mintinline{icl}{[r/w] Cookie'httpOnly : bool};
	\item \mintinline{icl}{[r/w] Cookie'name : string};
	\item \mintinline{icl}{[r/w] Cookie'path : string};
	\item \mintinline{icl}{[r/w] Cookie'secure : bool};
	\item \mintinline{icl}{[r/w] Cookie'value : string}.
\end{icItems}

And following methods —
\begin{icItems}
	\item \mintinline{icl}{Cookie.add (years : int, months : int, days : int, hours = 0, minutes = 0,}\\* \mintinline{icl}{seconds = 0) : Cookie};
	\item \mintinline{icl}{Cookie.load () : Cookie};
	\item \mintinline{icl}{Cookie.resetTime () : Cookie};
	\item \mintinline{icl}{Cookie.save () : Cookie}.
\end{icItems}

\subsubsection{\mintinline{icl}{[r/w] Cookie'domain : string}}

Numele de domeniu, pe care \cookie{} este disponibil.

\subsubsection{\mintinline{icl}{[r/w] Cookie'expiry : int}}

Timpul de expirare al fișierului cookie. Implicit -1, indică că fișierul cookie va fi șters la sfîrșitul sesiei.

\subsubsection{\mintinline{icl}{[r/w] Cookie'httpOnly : bool}}

Doar pentru protocolul HTTP. Implicit \false.

\subsubsection{\mintinline{icl}{[r/w] Cookie'name : string}}

Denumire, obligatorie pentru completare.

\subsubsection{\mintinline{icl}{[r/w] Cookie'path : string}}

Calea pe care fișierul cookie este diponibil, implicit \mintinline{icl}{"/"}.

\subsubsection{\mintinline{icl}{[r/w] Cookie'secure : bool}}

Securizat, implicit \false.

\subsubsection{\mintinline{icl}{[r/w] Cookie'value : string}}

Valoarea fișierului cookie, obligatorie pentru completare.

\subsubsection{\mintinline{icl}{Cookie.add (years : int, months : int, days : int, hours = 0, minutes = 0}\\*\noindent\mintinline{icl}{seconds = 0) : Cookie}}

Adaugă la timpul de expirare numărul necesar de ani, luni, zile, ore, minute și secunde.

\subsubsection{\mintinline{icl}{Cookie.load () : Cookie}}

Încarcă datele despre \cookie{} din browser.

Possible exceptions: \ferror{NoSessions}, \ferror{NoSuchWindow}, \ferror{NoSuchCookie} (see table \ref{errors}).

\subsubsection{\mintinline{icl}{Cookie.resetTime () : Cookie}}

Setează timpul de expirare la timpul curent. De exemplu pentru ca fișierul să expire peste un an se folosește comanda următoare: \mintinline{icl}{Cookie.resetTime().add(1, 0, 0)}.

\subsubsection{\mintinline{icl}{Cookie.save () : Cookie}}

Transmite schimbările fișierului cookie în browser.

Possible exceptions: \ferror{NoSessions}, \ferror{NoSuchWindow}, \ferror{InvalidArgument}, \ferror{UnableToSetCookie}, \ferror{InvalidCookieDomain} (see table \ref{errors}).

\subsubsection{Crearea noilor fișiere cookie}

Pe foaia \ref{newcookies}, este prezentată metoda corectă de a crea fișiere cookie noi.

\begin{sourcecode}
    \captionof{listing}{Crearea noilor fișiere cookie}
    \label{newcookies}
    \inputminted[linenos]{icl}{../sources/newcookies.icL}
\end{sourcecode}


\subsection{{\color{orange} Alert}}

The object \alert{} are următoarea proprietate: \mintinline{icl}{[r/o] Alert'text : string}.

And following methods —
\begin{icItems}
	\item \mintinline{icl}{Alert.accept () : void};
	\item \mintinline{icl}{Alert.dismiss () : void};
	\item \mintinline{icl}{Alert.sendKeys (keys : string) : void}.
\end{icItems}

\subsubsection{\mintinline{icl}{[r/o] Alert'text : string}}

Returnează textul avertizării.

Possible exceptions: \ferror{NoSessions}, \ferror{NoSuchWindow}, \ferror{NoSuchAlert} (see table \ref{errors}).

\subsubsection{\mintinline{icl}{Alert.accept () : void}}

Acceptă avertizarea.

Possible exceptions: \ferror{NoSessions}, \ferror{NoSuchWindow}, \ferror{NoSuchAlert} (see table \ref{errors}).

\subsubsection{\mintinline{icl}{Alert.dismiss () : void}}

Refuză avertizarea.

Possible exceptions: \ferror{NoSessions}, \ferror{NoSuchWindow}, \ferror{NoSuchAlert} (see table \ref{errors}).

\subsubsection{\mintinline{icl}{Alert.sendKeys (keys : string) : void}}

Completează formularul cu textul \mintinline{icl}{keys} și confirmă inserarea.

Possible exceptions: \ferror{NoSessions}, \ferror{NoSuchWindow}, \ferror{NoSuchAlert}, \ferror{ElementNotInteractable} (see table \ref{errors}).

\subsection{{\color{orange} Tabs}}

The object \tabs{} has following properties —
\begin{icItems}
	\item \mintinline{icl}{[r/o] Tabs'current : Tab};
	\item \mintinline{icl}{[r/o] Tabs'first : Tab};
	\item \mintinline{icl}{[r/o] Tabs'last : Tab};
	\item \mintinline{icl}{[r/o] Tabs'length : int};
	\item \mintinline{icl}{[r/o] Tabs'next : Tab};
	\item \mintinline{icl}{[r/o] Tabs'previous : Tab};
	\item \mintinline{icl}{[r/o] Tabs'(i : int) : Tab}.
	% \item \mintinline{icl}{Tabs'};
\end{icItems}

And following methods —
\begin{icItems}
	\item \mintinline{icl}{Tabs.close (template : string) : int};
	\item \mintinline{icl}{Tabs.close (url : regex) : int};
	\item \mintinline{icl}{Tabs.closeByTitle (template : string) : int};
	\item \mintinline{icl}{Tabs.closeByTitle (title : regex) : int};
	\item \mintinline{icl}{Tabs.closeOthers () : int};
	\item \mintinline{icl}{Tabs.closeToLeft () : int};
	\item \mintinline{icl}{Tabs.closeToRight () : int};
	\item \mintinline{icl}{Tabs.find (template : string) : Tab};
	\item \mintinline{icl}{Tabs.find (url : regex) : Tab};
	\item \mintinline{icl}{Tabs.findByTitle (template : string) : Tab};
	\item \mintinline{icl}{Tabs.findByTitle (title : regex) : Tab}.
	% \item \mintinline{icl}{Tabs.};
\end{icItems}

În regimul de testare taburile vor fi în ordine aleatorie. În regim de automatizare în ordine strictă.

\subsubsection{\mintinline{icl}{[r/o] Tabs'current : Tab}}

Tabul curent.

Possible exceptions: \ferror{NoSessions} (see table \ref{errors}).

\subsubsection{\mintinline{icl}{[r/o] Tabs'first : Tab}}

Primul tab.

Possible exceptions: \ferror{NoSessions} (see table \ref{errors}).

\subsubsection{\mintinline{icl}{[r/o] Tabs'last : Tab}}

Ultimul tab.

Possible exceptions: \ferror{NoSessions} (see table \ref{errors}).

\subsubsection{\mintinline{icl}{[r/o] Tabs'length : int}}

Numărul de taburi în sesie.

Possible exceptions: \ferror{NoSessions} (see table \ref{errors}).

\subsubsection{\mintinline{icl}{[r/o] Tabs'next : Tab}}

Următorul tab.

Possible exceptions: \ferror{NoSessions}, \ferror{NoSuchTab} (see table \ref{errors}).

\subsubsection{\mintinline{icl}{[r/o] Tabs'previous : Tab}}

Tabul precedent.

Possible exceptions: \ferror{NoSessions}, \ferror{NoSuchTab} (see table \ref{errors}).

\subsubsection{\mintinline{icl}{[r/o] Tabs'(i : int) : Tab}}

Al \mintinline{icl}{i}-lea tab.

Possible exceptions: \ferror{NoSessions}, \ferror{OutOfBounds} (see table \ref{errors}).

\subsubsection{\mintinline{icl}{Tabs.close (url : regex) : int}}

Închide toate taburile, la care adresa URL convine șablonului. Returnează numărul de taburi închise.

Possible exceptions: \ferror{NoSessions} (see table \ref{errors}).

\subsubsection{\mintinline{icl}{Tabs.close (url : regex) : int}}

Închide toate taburile, la care adresa URL convine expresii regulare \mintinline{icl}{url}. Returnează numărul de taburi închise.

Possible exceptions: \ferror{NoSessions} (see table \ref{errors}).

\subsubsection{\mintinline{icl}{Tabs.closeByTitle (template : string) : int}}

Închide toate taburile, la care titlul convine șablonului. Returnează numărul de taburi închise.

Possible exceptions: \ferror{NoSessions} (see table \ref{errors}).

\subsubsection{\mintinline{icl}{Tabs.closeByTitle (title : regex) : int}}

Închide toate taburile, la care titlul convine expresii regulare \mintinline{icl}{url}. Returnează numărul de taburi închise.

Possible exceptions: \ferror{NoSessions} (see table \ref{errors}).

\subsubsection{\mintinline{icl}{Tabs.closeOthers () : int}}

Închide toate taburile în afară de tabul curent. Returnează numărul de taburi închise.

Possible exceptions: \ferror{NoSessions} (see table \ref{errors}).

\subsubsection{\mintinline{icl}{Tabs.closeToLeft () : int}}

Închide toate taburile care se află la stînga de tabul curent. Returnează numărul de taburi închise.

Possible exceptions: \ferror{NoSessions} (see table \ref{errors}).

\subsubsection{\mintinline{icl}{Tabs.closeToRight () : int}}

Închide toate taburile care se află la dreapta de tabul curent. Returnează numărul de taburi închise.

Possible exceptions: \ferror{NoSessions} (see table \ref{errors}).

\subsubsection{\mintinline{icl}{Tabs.find (template : string) : Tab}}

Returnează primul tab, la care adresa URL convine șablonului.

Possible exceptions: \ferror{NoSessions} (see table \ref{errors}).

\subsubsection{\mintinline{icl}{Tabs.find (url : regex) : Tab}}

Returnează primul tab, la care adresa URL convine expresii regulare.

Possible exceptions: \ferror{NoSessions} (see table \ref{errors}).

\subsubsection{\mintinline{icl}{Tabs.findByTitle (template : string) : Tab}}

Returnează primul tab, la care titlul convine șablonului.

Possible exceptions: \ferror{NoSessions} (see table \ref{errors}).

\subsubsection{\mintinline{icl}{Tabs.findByTitle (title : regex) : Tab}}

Returnează primul tab, la care titlul convine expresii regulare.

Possible exceptions: \ferror{NoSessions} (see table \ref{errors}).

\subsection{{\color{orange} Tab}}

The object \tab{} has following properties —
\begin{icItems}
	\item \mintinline{icl}{[icL] [r/o] Tab'canGoBack : bool};
	\item \mintinline{icl}{[icL] [r/o] Tab'canGoForward : bool};
	\item \mintinline{icl}{[r/o] Tab'screenshot : string};
	\item \mintinline{icl}{[r/o] Tab'source : string};
	\item \mintinline{icl}{[r/*] Tab'title};
	\item \mintinline{icl}{[r/w] Tab'url}.
\end{icItems}

And following methods —
\begin{icItems}
	\item \mintinline{icl}{Tab.back () : void};
	\item \mintinline{icl}{Tab.close () : void};
	\item \mintinline{icl}{Tab.focus () : void};
	\item \mintinline{icl}{Tab.forward () : void};
	\item \mintinline{icl}{Tab.get (url : string) : bool};
	\item \mintinline{icl}{Tab.load (url : string) : bool}.
\end{icItems}

\subsubsection{\mintinline{icl}{[icL] [r/o] Tab'canGoBack : bool}}

Returnează \true, dacă butonul \textit{Înapoi} e disponibil, în caz contrar \false.

\subsubsection{\mintinline{icl}{[icL] [r/o] Tab'canGoForward : bool}}

Returnează \true, dacă butonul \textit{Înainte} e disponibil, în caz contrar \false.

\subsubsection{\mintinline{icl}{[r/o] Tab'screenshot : string}}

Returnează captura tabului, codată în base64.

Possible exceptions: \ferror{NoSessions}, \ferror{NoSuchWindow}, \ferror{UnableToCaptureScreen} (see table \ref{errors}).

\subsubsection{\mintinline{icl}{[r/o] Tab'source : string}}

Codul sursă al paginii deschise în tab.

Possible exceptions: \ferror{NoSessions}, \ferror{NoSuchWindow} (see table \ref{errors}).

\subsubsection{\mintinline{icl}{[r/*] Tab'title}}

Titlu paginii deschise în tab.

Possible exceptions: \ferror{NoSessions}, \ferror{NoSuchWindow} (see table \ref{errors}).

\subsubsection{\mintinline{icl}{[r/w] Tab'url}}

Adresa URL a paginii deschise în tab.

Possible exceptions: \ferror{NoSessions}, \ferror{NoSuchWindow} (see table \ref{errors}).

\subsubsection{\mintinline{icl}{Tab.back () : void}}

Trece la pagina precedentă, cum ar fi cînd apăsați pe butonul \textit{Înapoi} în browser.

Possible exceptions: \ferror{NoSessions}, \ferror{NoSuchWindow}, \ferror{Timeout} (see table \ref{errors}).

\subsubsection{\mintinline{icl}{Tab.close () : void}}

Închide tabul, dacă e ultimul închide și sesia.

Possible exceptions: \ferror{NoSessions}, \ferror{NoSuchWindow} (see table \ref{errors}).

\subsubsection{\mintinline{icl}{Tab.focus () : void}}

Schimbă focusul la tab. Focusul se schimbă între taburile din sesiune.

Possible exceptions: \ferror{NoSessions} (see table \ref{errors}).

\subsubsection{\mintinline{icl}{Tab.forward () : void}}

Trece la pagina precedentă, cum ar fi cînd apăsați pe butonul \textit{Înainte} în browser.


Possible exceptions: \ferror{NoSessions}, \ferror{NoSuchWindow}, \ferror{Timeout} (see table \ref{errors}).

\subsubsection{\mintinline{icl}{Tab.get (url : string) : bool}}

Trece la pagină, URL trebuie să fie absolut. Returnează \true{} dacă pagina a fost descărcată cu succes, în caz contrar \false.

\subsubsection{\mintinline{icl}{Tab.load (url : string) : void}}

Trece la pagină, URL trebuie să fie absolut. În caz de eroare generează excepție.

Possible exceptions: \ferror{NoSessions}, \ferror{NoSuchWindow}, \ferror{InvalidArgument}, \ferror{Timeout}, \ferror{InsecureCertificate} (see table \ref{errors}).

\subsection{{\color{orange} Doc}}

The object \dom{} are următoarele metode:
\begin{icItems}
	\item \mintinline{icl}{Doc.query (by = By'cssSelector, selector : string) : element};
	\item \mintinline{icl}{Doc.queryAll (by = By'cssSelector, selector : string) : element};
	\item \mintinline{icl}{Doc.queryAllByXPath (xpath : string) : element};
	\item \mintinline{icl}{Doc.queryByXPath (xpath : string) : element};
	\item \mintinline{icl}{Doc.queryLink (name : string, isFragment = false) : element};
	\item \mintinline{icl}{Doc.queryLinks (name : string, isFragment = false) : element};
	\item \mintinline{icl}{Doc.queryTag (name : string) : element};
	\item \mintinline{icl}{Doc.queryTags (name : string) : element}.
\end{icItems}

\subsubsection{\mintinline{icl}{Doc.query (by = By'cssSelector, selector : string) : element}}

Primește aceiași parametri ca și  \mintinline{icl}{element.query}, numai că această funcție va căuta în tot documentul.

Possible exceptions: \ferror{NoSessions}, \ferror{Timeout} (see table \ref{errors}).

\subsubsection{\mintinline{icl}{Doc.queryAll (by = By'cssSelector, selecor : string) : element}}

Primește aceiași parametri ca și \mintinline{icl}{element.queryAll}, numai că această funcție va căuta în tot documentul.

Possible exceptions: \ferror{NoSessions} (see table \ref{errors}).

\subsubsection{\mintinline{icl}{Doc.queryAllByXPath (xpath : string) : element}}

Acronim pentru \mintinline{icl}{Doc.queryAll (By'xPath, @xpath)}.

\subsubsection{\mintinline{icl}{Doc.queryByXPath (xpath : string) : element}}

Acronim pentru \mintinline{icl}{Doc.query (By'xPath, @xpath)}.

\subsubsection{\mintinline{icl}{Doc.queryLink (name : string, isFragment = false) : element}}

Acronim pentru:
\begin{icItems}
	\item \mintinline{icl}{Doc.query (By'linkText, @name)};
	\item \mintinline{icl}{Doc.query (By'partialLinkText, @name)};
\end{icItems}

\subsubsection{\mintinline{icl}{Doc.queryLinks (name : string, isFragment = false) : element}}

Acronim pentru:
\begin{icItems}
	\item \mintinline{icl}{Doc.queryAll (By'linkText, @name)};
	\item \mintinline{icl}{Doc.queryAll (By'partialLinkText, @name)};
\end{icItems}

\subsubsection{\mintinline{icl}{Doc.queryTag (name : string) : element}}

Acronim pentru \mintinline{icl}{Doc.query (By'tagName, @name)}.

\subsubsection{\mintinline{icl}{Doc.queryTags (name : string) : element}}

Acronim pentru \mintinline{icl}{Doc.queryAll (By'tagName, @name)}.

\subsection{{\color{orange} Files}}

The object \files{} are următoarele metode:
\begin{icItems}
	\item \mintinline{icl}{Files.open (path : string) : File};
	\item \mintinline{icl}{Files.create (path : string) : File};
	\item \mintinline{icl}{Files.createDir (path : string) : void};
	\item \mintinline{icl}{Files.createPath (path : string) : void}.
\end{icItems}

\subsubsection{\mintinline{icl}{Files.open (path : string) : File}}

Deschide fișierul.

Possible exceptions: \ferror{FileNotFound} (see table \ref{errors}).

\subsubsection{\mintinline{icl}{Files.create (path : string) : File}}

Deschide fișierul, dacă el nu există se creează.

Possible exceptions: \ferror{FolderNotFound} (see table \ref{errors}).

\subsubsection{\mintinline{icl}{Files.createDir (path : string) : void}}

Creează folder, folderul ascendent trebuie să existe deja.

Possible exceptions: \ferror{FolderNotFound} (see table \ref{errors}).

\subsubsection{\mintinline{icl}{Files.createPath (path : string) : void}}

Creează toate folderele care nu există în calea indicată.

\subsection{{\color{orange} File}}

The object \file{} has following properties —
\begin{icItems}
	\item \mintinline{icl}{[r/o] File'csv : 1};
	\item \mintinline{icl}{[r/w] File'format : int};
	\item \mintinline{icl}{[r/o] File'none : 0};
	\item \mintinline{icl}{[r/o] File'tsv : 2};
	\item \mintinline{icl}{[r/o] File'valid : bool}.
\end{icItems}

And following methods —
\begin{icItems}
	\item \mintinline{icl}{File.close () : void};
	\item \mintinline{icl}{File.delete () : void}.
\end{icItems}

\subsubsection{\mintinline{icl}{[r/o] File'csv : 1}}

Format CSV.

\subsubsection{\mintinline{icl}{[r/w] File'format : int}}

Returnează formatul fișierului.

\subsubsection{\mintinline{icl}{[r/o] File'none : 0}}

File neinițializat.

\subsubsection{\mintinline{icl}{[r/o] File'tsv : 2}}

Format TSV.

\subsubsection{\mintinline{icl}{[r/o] File'valid : bool}}

Returnează \true, dacă fișierul este inițializat, în caz contrar \false.

\subsubsection{\mintinline{icl}{File.close () : void}}

Închide fișierul.

\subsubsection{\mintinline{icl}{File.delete () : void}}

Șterge fișierul.

\subsection{{\color{orange} Make}}

The object \make{} deține următoarea metodă: \mintinline{icl}{Make.image (base64 : string, path :}\\*\mintinline{icl}{string) : void}.

\subsubsection{\mintinline{icl}{Make.image (base64 : string, path : string) : void}}

Salvează captura de ecran pe disc.

\subsection{{\color{orange} Log}}

The object \logtype{} deține următoarele metode:
\begin{icItems}
	\item \mintinline{icl}{Log.error (message : string) : void};
	\item \mintinline{icl}{Log.info (message : string) : void};
	\item \mintinline{icl}{Log.out (args : any ...) : void};
	\item \mintinline{icl}{Log.stack (var : any) : void};
	\item \mintinline{icl}{Log.state (var : any) : void}.
\end{icItems}

\subsubsection{\mintinline{icl}{Log.error (message : string) : void}}

Printează un mesaj de eroare.

\subsubsection{\mintinline{icl}{Log.info (message : string) : void}}

Printează un mesaj informațional.

\subsubsection{\mintinline{icl}{Log.out (args : any ...) : void}}

Printează informație pentru debug, primește cîțeva argumente de orice tip. La transmiterea variabilelor se printează containerul, numele variabilei, tipul de date și valoarea. La transmiterea constantelor numai valoarea. La transmiterea rezultatului funcției se printează tipul și valoarea.

\subsubsection{\mintinline{icl}{Log.stack (var : any) : void}}

Printează lista de stive, arătînd în care din ele se întilnește așa variabilă și ce valori are.

\subsubsection{\mintinline{icl}{Log.state (var : any) : void}}

Printează lista tuturor stărilor, arătînd în care din ele se întilnește așa variabilă și ce valori are.

\subsection{{\color{orange} Numbers}}

The object \numbers{} has following properties —
\begin{icItems}
	\item \mintinline{icl}{[r/o] Numbers'max : 4};
	\item \mintinline{icl}{[r/o] Numbers'min : 3};
	\item \mintinline{icl}{[r/o] Numbers'product : 2};
	\item \mintinline{icl}{[r/o] Numbers'process : int};
	\item \mintinline{icl}{[r/o] Numbers'sum : 1}.
\end{icItems}

And following methods —
\begin{icItems}
	\item \mintinline{icl}{Numbers.process (a : int, b : int) : int};
	\item \mintinline{icl}{Numbers.process (a : double, b : double) : double};
	\item \mintinline{icl}{Numbers.restoreProcess () : void};
	\item \mintinline{icl}{Numbers.setProcess (proc : int) : void}.
\end{icItems}

\subsubsection{\mintinline{icl}{[r/o] Numbers'max : 4}}

A alege maximum.

\subsubsection{\mintinline{icl}{[r/o] Numbers'min : 3}}

A alege minimum.

\subsubsection{\mintinline{icl}{[r/o] Numbers'product : 2}}

A înmulți numerele.

\subsubsection{\mintinline{icl}{[r/o] Numbers'process : int}}

Modul curent de a prelucra numerele.

\subsubsection{\mintinline{icl}{[r/o] Numbers'sum : 1}}

A aduna numerele.

\subsubsection{\mintinline{icl}{Numbers.process (a : int, b : int) : int}}

Prelucrează numerele întregi cu metoda de prelucrare curentă.

\subsubsection{\mintinline{icl}{Numbers.process (a : double, b : double) : double}}

Prelucrează numerele decimale cu metoda de prelucrare curentă.

\subsubsection{\mintinline{icl}{Numbers.restoreProcess () : void}}

Șterge ultima înscriere sin stiva metodelor de prelucrare.

\subsubsection{\mintinline{icl}{Numbers.setProcess (proc : int) : void}}

Adaugă o nouă înscriere în stiva metodelor de prelucrare.

\subsection{{\color{orange} Math}}

The object \mintinline{icl}{Math} has following properties —
\begin{icItems}
	\item \mintinline{icl}{[r/o] Math'1divPi : double};
	\item \mintinline{icl}{[r/o] Math'1divSqrt2 : double};
	\item \mintinline{icl}{[r/o] Math'2divPi : double};
	\item \mintinline{icl}{[r/o] Math'2divSqrtPi : double};
	\item \mintinline{icl}{[r/o] Math'e : double};
	\item \mintinline{icl}{[r/o] Math'ln2 : double};
	\item \mintinline{icl}{[r/o] Math'ln10 : double};
	\item \mintinline{icl}{[r/o] Math'log2e : double};
	\item \mintinline{icl}{[r/o] Math'log10e : double};
	\item \mintinline{icl}{[r/o] Math'pi : double};
	\item \mintinline{icl}{[r/o] Math'piDiv2 : double};
	\item \mintinline{icl}{[r/o] Math'piDiv4 : double};
	\item \mintinline{icl}{[r/o] Math'sqrt2 : double}.
\end{icItems}

And following methods —
\begin{icItems}
	\item \mintinline{icl}{Math.acos (v : double) : double};
	\item \mintinline{icl}{Math.asin (v : double) : double};
	\item \mintinline{icl}{Math.atan (v : double) : double};
	\item \mintinline{icl}{Math.ceil (v : double) : int};
	\item \mintinline{icl}{Math.cos (v : double) : double};
	\item \mintinline{icl}{Math.degreesToRadians (v : double) : double};
	\item \mintinline{icl}{Math.exp (v : double) : double};
	\item \mintinline{icl}{Math.floor (v : double) : int};
	\item \mintinline{icl}{Math.ln (v : double) : double};
	\item \mintinline{icl}{Math.min (arr : int...) : int};
	\item \mintinline{icl}{Math.min (arr : double...) : double};
	\item \mintinline{icl}{Math.max (arr : int...) : int};
	\item \mintinline{icl}{Math.max (arr : double...) : double};
	\item \mintinline{icl}{Math.radiansToDegrees (v : double) : double};
	\item \mintinline{icl}{Math.round (v : double) : int};
	\item \mintinline{icl}{Math.sin (v : double) : double};
	\item \mintinline{icl}{Math.tan (v : double) : double}.
\end{icItems}

\subsubsection{\mintinline{icl}{[r/o] Math'1divPi : double}}

1 împărțit la pi ($\frac{1}{\pi}$).

\subsubsection{\mintinline{icl}{[r/o] Math'1divSqrt2 : double}}

1 împărțit la radical din 2 ($\frac{1}{\sqrt{2}}$).

\subsubsection{\mintinline{icl}{[r/o] Math'2divPi : double}}

2 împărțit la pi ($\frac{2}{\pi}$).

\subsubsection{\mintinline{icl}{[r/o] Math'2divSqrtPi : double}}

2 împărțit la pi ($\frac{2}{\sqrt{\pi}}$).

\subsubsection{\mintinline{icl}{[r/o] Math'e : double}}

Numărul ($e$).

\subsubsection{\mintinline{icl}{[r/o] Math'ln2 : double}}

Logaritmul natural al numărului 2 ($\ln{2}$).

\subsubsection{\mintinline{icl}{[r/o] Math'ln10 : double}}

Logaritmul natural al numărului 10 ($\ln_{10}$).

\subsubsection{\mintinline{icl}{[r/o] Math'log2e : double}}

Logaritmul numărului $e$ cu baza 2 ($\log_{2}{e}$).

\subsubsection{\mintinline{icl}{[r/o] Math'log10e : double}}

Logaritmul numărului $e$ cu baza 10 ($\log_{10}{e}$).

\subsubsection{\mintinline{icl}{[r/o] Math'pi : double}}

Numărul pi ($\pi$).

\subsubsection{\mintinline{icl}{[r/o] Math'piDiv2 : double}}

Pi pe 2 ($\frac{\pi}{2}$).

\subsubsection{\mintinline{icl}{[r/o] Math'piDiv4 : double}}

Pi pe 4 ($\frac{\pi}{4}$).

\subsubsection{\mintinline{icl}{[r/o] Math'sqrt2 : double}}

Radical din 2 ($\sqrt{2}$).

\subsubsection{\mintinline{icl}{Math.acos (v : double) : double}}

Arccosinus ($\arccos{v}$).

\subsubsection{\mintinline{icl}{Math.asin (v : double) : double}}

Arcsinus ($\arcsin{v}$).

\subsubsection{\mintinline{icl}{Math.atan (v : double) : double}}

Arctangență ($\arctan{v}$).

\subsubsection{\mintinline{icl}{Math.ceil (v : double) : int}}

Cel mai mic număr întreg mai mare sau egal cu \mintinline{icl}{v}.

\subsubsection{\mintinline{icl}{Math.cos (v : double) : double}}

Cosinus ($\cos{v}$).

\subsubsection{\mintinline{icl}{Math.degreesToRadians (v : double) : double}}

Conversează grade în radiani.

\subsubsection{\mintinline{icl}{Math.exp (v : double) : double}}

Funcția exponent ($\exp{v}$).

\subsubsection{\mintinline{icl}{Math.floor (v : double) : int}}

Cel mai mare număr întreg mai mic sau egal cu \mintinline{icl}{v}.

\subsubsection{\mintinline{icl}{Math.ln (v : double) : double}}

Logaritm natural ($\ln{v}$).

\subsubsection{\mintinline{icl}{Math.min (arr : int...) : int}}

Returnează cel mai mic număr întreg.

\subsubsection{\mintinline{icl}{Math.min (arr : double...) : double}}

Returnează cel mai mic număr decimal.

\subsubsection{\mintinline{icl}{Math.max (arr : int...) : int}}

Returnează cel mai mare număr întreg.

\subsubsection{\mintinline{icl}{Math.max (arr : double...) : double}}

Returnează cel mai mare număr decimal.

\subsubsection{\mintinline{icl}{Math.radiansToDegrees (v : double) : double}}

Conversează radiani în grade.

\subsubsection{\mintinline{icl}{Math.round (v : double) : int}}

Returnează cel mai apropiat număr întreg.

\subsubsection{\mintinline{icl}{Math.sin (v : double) : double}}

Sinus ($\sin{v}$).

\subsubsection{\mintinline{icl}{Math.tan (v : double) : double}}

Cosinus ($\tan{v}$).

\subsection{{\color{orange} Import}}

The object \mintinline{icl}{Import} deține următoarele metode:
\begin{icItems}
	\item \mintinline{icl}{Import.none (data : object, path : string) : void};
	\item \mintinline{icl}{Import.none (path : string) : void};
	\item \mintinline{icl}{Import.functions (data : object, path : string) : void};
	\item \mintinline{icl}{Import.functions (path : string) : void};
	\item \mintinline{icl}{Import.all (data : object, path : string) : void};
	\item \mintinline{icl}{Import.all (path : string) : void};
	\item \mintinline{icl}{Import.run (path : string) : void}.
\end{icItems}

The object \mintinline{icl}{data} permite a transmite date în contextul izolat în care se execut fișierele externe, proprietățile obiectului \mintinline{icl}{data} vor fi disponibile ca variabile globale. Aceasta permite a deține într-un fișier mai multe versiuni ale bibliotecii de exemplu, și la folosire se poate de indicat care versiune e necesară.

\subsubsection{\mintinline{icl}{Import.none (data : object, path : string) : void}}

Creează context izolat în care se execută fișierul.

\subsubsection{\mintinline{icl}{Import.none (path : string) : void}}

Acronim pentru \mintinline{icl}{Import.none([=], @path)}.

\subsubsection{\mintinline{icl}{Import.functions (data : object, path : string) : void}}

Creează context izolat în care se execută fișierul, Apoi se importă funcțiile în contextul curent.

\subsubsection{\mintinline{icl}{Import.functions (path : string) : void}}

Acronim pentru \mintinline{icl}{Import.functions ([=], @path)}.

\subsubsection{\mintinline{icl}{Import.all (data : object, path : string) : void}}

Creează context izolat în care se execută fișierul, Apoi se importă funcțiile și variabilele globale în contextul curent.

\subsubsection{\mintinline{icl}{Import.all (path : string) : void}}

Acronim pentru \mintinline{icl}{Import.all ([=], @path)}.

\subsubsection{\mintinline{icl}{Import.run (path : string) : void}}

Execută fișierul în contextul curent.

%\newpage
