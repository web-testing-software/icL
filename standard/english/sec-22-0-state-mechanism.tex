% !TeX spellcheck = en_US
\section{State mechanism}

All variables in icL are contained by 2 types of containers: stack and state.

The stack contains come containers, each container is liked to a fragment of code. On fragment touched the container is created, on exited the container is destroyed.

The state is double-linked list, containers of it has global visibility. By default, there is a single container, new containers can be created by user.

\subsection{Stack Control}
\label{stack:control}

There are following properties and methods to manipulate stack content —
\begin{icItems}
	\item \mintinline{icl}{[r/o] Stacks'(name : string) : Stack};
	\item \mintinline{icl}{[r/w] Stack'(name : string) : any};
	\item \mintinline{icl}{Stack.clear () : void};
	\item \mintinline{icl}{Stack.continue () : void};
	\item \mintinline{icl}{Stack.break () : void};
	\item \mintinline{icl}{Stack.return (value : any) : void};
	\item \mintinline{icl}{Stack.markTest (name : string) : Stack};
	\item \mintinline{icl}{Stack.markStep (name : string) : Stack};
	\item \mintinline{icl}{Stack.addDescription (description : string) : void};
	\item \mintinline{icl}{Stack.destroy () : void};
	\item \mintinline{icl}{Stack.ignore () : Stack};
	\item \mintinline{icl}{Stack.listen () : Stack}.
\end{icItems}

\subsubsection{[r/o] \mintinline{icl}{Stacks'(name : string) : Stack}}

Returns the container with \mintinline{icl}{name}. Use following syntax to name a container —
\begin{minted}{icl}
{:nume
	`` conținut
}
\end{minted}
A function call creates a container with name of function.


\subsubsection{\mintinline{icl}{Stack'(name : string) : any}}

Return a local variable located in this container. For example, if already exists a variable named \mintinline{icl}{i} in parent container, use \mintinline{icl}{Stack'i = 0} to create a variable named \mintinline{icl}{i} in current content.

\subsubsection{\mintinline{icl}{Stack.clear () : void}}

Removes all variables of container.

\subsubsection{\mintinline{icl}{Stack.continue () : void}}

Skip current iteration, available just for loops.

Possible exceptions: \ferror{InvalidArgument} (see table \ref{errors}).

\subsubsection{\mintinline{icl}{Stack.break () : void}}

Break the loop. Available just for loops.

Possible exceptions: \ferror{InvalidArgument} (see table \ref{errors}).

\subsubsection{\mintinline{icl}{Stack.return (value : any) : void}}

Returns the function result.

Possible exceptions: \ferror{InvalidArgument} (see table \ref{errors}).

\subsubsection{\mintinline{icl}{Stack.markTest (name : string) : Stack}}

Marks container as test named \mintinline{icl}{name}. Definition of a test inner another test is restricted.

Possible exceptions: \ferror{InvalidArgument} (see table \ref{errors}).

\subsubsection{\mintinline{icl}{Stack.markStep (name : string) : Stack}}

Marks container as step named \mintinline{icl}{name}. A step can be defined in a test or step.

Possible exceptions: \ferror{InvalidArgument} (see table \ref{errors}).

\subsubsection{\mintinline{icl}{Stack.addDescription (description : string) : void}}

Adds a description for test or step.

Possible exceptions: \ferror{InvalidArgument} (see table \ref{errors}).

\subsubsection{\mintinline{icl}{Stack.destroy () : void}}

Removes \mintinline{icl}{listen} container.

Possible exceptions: \ferror{InvalidArgument} (see table \ref{errors}).

\subsubsection{\mintinline{icl}{Stack.ignore () : Stack}}

Switch off \mintinline{icl}{listen}.

Possible exceptions: \ferror{InvalidArgument} (see table \ref{errors}).

\subsubsection{\mintinline{icl}{Stack.listen () : Stack}}

Switch on \mintinline{icl}{listen}.

Possible exceptions: \ferror{InvalidArgument} (see table \ref{errors}).

\subsection{State Control}

There are following properties and methods to manipulate state content —
\begin{icItems}
	\item \mintinline{icl}{State'first : object};
	\item \mintinline{icl}{State'prev : object};
	\item \mintinline{icl}{State'next : object};
	\item \mintinline{icl}{State'last : object};
	\item \mintinline{icl}{State.new (data = [=]) : void};
	\item \mintinline{icl}{State.newAtEnd (data = [=]) : void};
	\item \mintinline{icl}{State.toNext () : void};
	\item \mintinline{icl}{State.toPrev () : void};
	\item \mintinline{icl}{State.clear () : void};
	\item \mintinline{icl}{State.delete () : void};
	\item \mintinline{icl}{State.toFirst () : void};
	\item \mintinline{icl}{State.toLast () : void}.
\end{icItems}

\mintinline{icl}{State} is a collection, that means that it ccan be iterated by \mintinline{icl}{for (State) {}}. Other loops are not supported.

\subsubsection{\mintinline{icl}{State'first : object}}

Returns the first container as object.

\subsubsection{\mintinline{icl}{State'prev : object}}

Returns previous container as object.

\subsubsection{\mintinline{icl}{State'next : object}}

Returns next container as object.

\subsubsection{\mintinline{icl}{State'last : object}}

Returns the last container as object.

\subsubsection{\mintinline{icl}{State.new (data = [=]) : void}}

Creates a new container after the current and focuses it. All fields of data object will be copied to container.

\subsubsection{\mintinline{icl}{State.newAtEnd (data = [=]) : void}}

Creates a new container on end of state. All fields of data object will be copied to container.

\subsubsection{\mintinline{icl}{State.toNext () : bool}}

Returns \true{} if there is a next container and the focus was switched to it, otherwise returns \false.

\subsubsection{\mintinline{icl}{State.toPrev () : bool}}

Returns \true{} if there is a previous container and the focus was switched to it, otherwise returns \false.

\subsubsection{\mintinline{icl}{State.clear () : void}}

Clears the state.

\subsubsection{\mintinline{icl}{State.delete () : void}}

Removes the current container, if there is a next container — will be focused, if there is a previous container — will be focused, if the current is the last — will be cleared.

\subsubsection{\mintinline{icl}{State.toFirst () : void}}

Switch focus to first state.

\subsubsection{\mintinline{icl}{State.toLast () : void}}

Switch focus to last state.

\subsubsection{\mintinline{icl}{for (State) {}}}

Run commands for each container of state.
