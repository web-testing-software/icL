% !TeX spellcheck = en_US

\section{Error-less Methodology}

\label{errorless-sec}

Methodology of programming, named {\it error-less} is a part of icL philosophy. According to it, code must be resistant to errors.

\subsection{Variables}

icL variables in their own right are not generating exceptions. The exceptions are generated by their wrong using. The global variables can be set from project file, always defines default values for its, for example \mintinline{icl}{#server = #server | "192.168.7.12"} (see §\ref{void-operators}).

Reading property and calling methods of \void, always returns \void.

Sending a \void{} value to short form of function call, will return \void{} (the function will not be called). 

\subsection{Void Operators}

\label{void-operators}

\void{} operators have the rank 2. The type, named \mintinline{icl}{any} i this section, can be one of following types — \integer{}, \double{}, \str{}, \listtype{}, \object{}, \set{}, \element{}.

List of \void{} operators —
\stablethree{1.0cm}{7.0cm}{6.0cm}
{}{}{Rank}{Operation}{Operator}
{
    2 & alternative select   & \mintinline{icl}{void | void : void}        \\ \hline
    2 & alternative select   & \mintinline{icl}{void | any : any}          \\ \hline
    2 & alternative select   & \mintinline{icl}{any | void : any}          \\ \hline
    2 & alternative select   & \mintinline{icl}{any | any : any}           \\ \hline
    2 & primary select       & \mintinline{icl}{void & void : void}        \\ \hline
    2 & primary select       & \mintinline{icl}{any & void : void}         \\ \hline
    2 & primary select       & \mintinline{icl}{void & any : void}         \\ \hline
    2 & primary select       & \mintinline{icl}{any & any : any}           \\ \hline
    2 & secondary select     & \mintinline{icl}{void ~ void : void}        \\ \hline
    2 & secondary select     & \mintinline{icl}{void ~ any : void}         \\ \hline
    2 & secondary select     & \mintinline{icl}{any ~ void : void}         \\ \hline
    2 & secondary select     & \mintinline{icl}{any ~ any : any}           \\ \hline
    2 & exclusive select     & \mintinline{icl}{void ^ void : void}        \\ \hline
    2 & exclusive select     & \mintinline{icl}{void ^ any : any}          \\ \hline
    2 & exclusive select     & \mintinline{icl}{any ^ void : any}          \\ \hline
    2 & exclusive select     & \mintinline{icl}{any ^ any : void}          \\ \hline
    2 & selective select     & \mintinline{icl}{void \% void : void}       \\ \hline
    2 & selective select     & \mintinline{icl}{void \% any : any}         \\ \hline
    2 & selective select     & \mintinline{icl}{any \% void : any}         \\ \hline
    2 & selective select     & \mintinline{icl}{int \% int : int}          \\ \hline
    2 & selective select     & \mintinline{icl}{double \% double : double} \\ \hline
    2 & selective select     & \mintinline{icl}{string \% string : list}   \\ \hline
    2 & selective select     & \mintinline{icl}{list \% string : list}     \\ \hline
    2 & selective select     & \mintinline{icl}{list \% list : list}       \\ \hline
    2 & selective select     & \mintinline{icl}{object \% object : set}    \\ \hline
    2 & selective select     & \mintinline{icl}{set \% object : set}       \\ \hline
    2 & selective select     & \mintinline{icl}{set \% set : set}          \\ \hline
}

\subsubsection{\mintinline{icl}{(arg1 : void or any) | (arg2 : void or any) : void or any}}

\begin{table}[H]
	\caption{Alternative select}
	\label{orhacktable}
	\begin{tabular}{|l|l|l|}
		\hline
		\mintinline{icl}{arg1} & \mintinline{icl}{arg2} & \mintinline{icl}{arg1 | arg2} \\ \hline
		\void{}     & \void{}     & \void{}  			\\ \hline
		\void{}     & \mintinline{icl}{any}  & \mintinline{icl}{arg2}  		\\ \hline
		\mintinline{icl}{any}  & \void{}     & \mintinline{icl}{arg1}  		\\ \hline
		\mintinline{icl}{any}  & \mintinline{icl}{any}  & \mintinline{icl}{arg1}  		\\ \hline
	\end{tabular}
\end{table}

\subsubsection{\mintinline{icl}{(arg1 : void or any) & (arg2 : void or any) : void or any}}

\begin{table}[H]
	\caption{Primary select}
	\label{andhacktable}
	\begin{tabular}{|l|l|l|}
		\hline
		\mintinline{icl}{arg1} & \mintinline{icl}{arg2} & \mintinline{icl}{arg1 & arg2} \\ \hline
		\void{}     & \void{}     & \void{}   			\\ \hline
		\void{}     & \mintinline{icl}{any}  & \void{}   			\\ \hline
		\mintinline{icl}{any}  & \void{}     & \void{}   			\\ \hline
		\mintinline{icl}{any}  & \mintinline{icl}{any}  & \mintinline{icl}{arg1}   		\\ \hline
	\end{tabular}
\end{table}

\subsubsection{\mintinline{icl}{(arg1 : void or any) ~ (arg2 : void or any) : void or any}}

\begin{table}[H]
	\caption{Secondary select}
	\label{eqhacktable}
	\begin{tabular}{|l|l|l|}
		\hline
		\mintinline{icl}{arg1} & \mintinline{icl}{arg2} & \mintinline{icl}{arg1 ~ arg2} \\ \hline
		\void{}     & \void{}     & \void{}   			\\ \hline
		\void{}     & \mintinline{icl}{any}  & \void{}   			\\ \hline
		\mintinline{icl}{any}  & \void{}     & \void{}   			\\ \hline
		\mintinline{icl}{any}  & \mintinline{icl}{any}  & \mintinline{icl}{arg2}   		\\ \hline
	\end{tabular}
\end{table}

\subsubsection{\mintinline{icl}{(arg1 : void or any) ^ (arg2 : void or any) : void or any}}

\begin{table}[H]
	\caption{Exclusive select}
	\label{xorhacktable}
	\begin{tabular}{|l|l|l|}
		\hline
		\mintinline{icl}{arg1} & \mintinline{icl}{arg2} & \mintinline{icl}{arg1 ^ arg2} \\ \hline
		\void{}     & \void{}     & \void{}   			\\ \hline
		\void{}     & \mintinline{icl}{any}  & \mintinline{icl}{arg2}   		\\ \hline
		\mintinline{icl}{any}  & \void{}     & \mintinline{icl}{arg1}   		\\ \hline
		\mintinline{icl}{any}  & \mintinline{icl}{any}  & \void{}   			\\ \hline
	\end{tabular}
\end{table}

\subsubsection{\mintinline{icl}{(arg1 : void or any) \% (arg2 : void or any) : void or any}}

\begin{table}[H]
	\caption{Selective select}
	\label{centhacktable}
	\begin{tabular}{|l|l|l|}
		\hline
		\mintinline{icl}{arg1} & \mintinline{icl}{arg2} & \mintinline{icl}{arg1 \% arg2}			\\ \hline
		\void{}     & \void{}     & \void{}						\\ \hline
		\void{}     & \mintinline{icl}{any}  & \mintinline{icl}{arg2}					\\ \hline
		\mintinline{icl}{any}  & \void{}     & \mintinline{icl}{arg1}					\\ \hline
		\integer{}  & \integer{}  & \mintinline{icl}{Numbers.process(arg1, arg2)}		\\ \hline
		\double{}   & \double{}   & \mintinline{icl}{Numbers.process(arg1, arg2)}		\\ \hline
		\str{}      & \str{}      & \mintinline{icl}{list - [arg1, arg2]}	\\ \hline
		\listtype{} & \str{}      & \mintinline{icl}{list - [arg1, arg2]}	\\ \hline
		\listtype{} & \listtype{} & \mintinline{icl}{list - [arg1, arg2]}	\\ \hline
		\object{}   & \object{}   & \mintinline{icl}{set - [arg1, arg2]}	\\ \hline
		\set{}      & \object{}   & \mintinline{icl}{set - [arg1, arg2]}	\\ \hline
		\set{}      & \set{}      & \mintinline{icl}{set - [arg1, arg2]}	\\ \hline
		\element{}  & \element{}  & \mintinline{icl}{element - [arg1, arg2]}\\ \hline
	\end{tabular}
\end{table}

\subsection{Control Operators}

There are following control operators —
\begin{icItems}
	\item \mintinline{icl}{exists};
	\item \mintinline{icl}{if-exists};
	\item \mintinline{icl}{for-any}.
\end{icItems}

\subsubsection{\mintinline{icl}{exists}}

\mintinline{icl}{exists} returns valid information or \void. It integrates a jammer. 

Default validation condition —
\begin{icItems}
	\item
	for \bool{} — \mintinline{icl}{# == true};
	\item
	for \integer{} — \mintinline{icl}{# != 0};
	\item
	for \double{} — \mintinline{icl}{# != 0.0};
	\item
	for \str{} — \mintinline{icl}{!#'empty};
	\item
	for \listtype{} — \mintinline{icl}{!#'empty};
	\item
	for \set{} — \mintinline{icl}{!#'empty};
	\item
	for \element{} — \mintinline{icl}{!#'empty}.
\end{icItems}

Syntax for default condition —
\mintinline{icl}{exists(expression)}.

Syntax for customized condition —
\mintinline{icl}{exists(expression, condition)}.

\subsubsection{\mintinline{icl}{if-exists}}

\mintinline{icl}{if-exists} executes a set of statements just for valid data. If \mintinline{icl}{exists} returns a value different from \void, the set of statements gets executed and the value of \mintinline{icl}{exists} will be available as \mintinline{icl}{@}.

Example —
\inputminted[linenos]{icl}{../sources/ifexistsex.icL}

\subsubsection{\mintinline{icl}{for-any}}

\mintinline{icl}{for-any} is used to sometimes use a value without creating a variable.

Example —
\inputminted[linenos]{icl}{../sources/foranyex.icL}

\subsection{Data Casting}

There is a single operator to cast data safe — \mintinline{icl}{data :? type : bool}. It will return the cast result on success and \void{} on fail.

\subsection{Control Modifiers}

Modifiers have syntax — \mintinline{icl}{control:modifier}. Also there is a possibility to use some modifiers, syntax — \mintinline{icl}{control:[mod1,mod2]}. Using of some incompatible modifiers will cause semantic exception.

List of control modifier —
\begin{icItems}
	\item \mintinline{icl}{if:not exists};
	\item \mintinline{icl}{for:Xtimes};
	\item \mintinline{icl}{for:ever};
	\item \mintinline{icl}{while:minX};
	\item \mintinline{icl}{while:maxX};
\end{icItems}

\subsubsection{\mintinline{icl}{if:not exists}}

\mintinline{icl}{if:not-exists} will execute the set of statements when the \mintinline{icl}{exists} returns \void.

\subsubsection{\mintinline{icl}{for:Xtimes}}

Executes the set of statements \mintinline{icl}{X} times.

Example —
\inputminted[linenos]{icl}{../sources/forxtimes.icL}

\subsubsection{\mintinline{icl}{for:ever}}

Executes a loop forever. Use it to run a loop dynamical number of times. Example —
\inputminted[linenos]{icl}{../sources/foreverloop.icL}

\subsubsection{\mintinline{icl}{while:minX}}

Sets the minimal number of loop execution to \mintinline{icl}{X}. In loop \mintinline{icl}{@} will be \true{} if condition was checked; that is the minimal number of loop is completed.

\subsubsection{\mintinline{icl}{while:maxX}}

Sets the maximal number of loop execution to \mintinline{icl}{X}. Also, it is available for \mintinline{icl}{for-each}, \mintinline{icl}{filter} and \mintinline{icl}{range}.

Example -
\inputminted[linenos]{icl}{../sources/maxlimitsheet.icL}

\subsection{Web Elements}

All errors of web elements are defined by W3C WebDriver Standard. In automation mode the error \ferror{NoSuchElement} gets omitted and the request returns an empty collection.

Examples —
\inputminted[linenos]{icl}{../sources/webelementsheet.icL}

