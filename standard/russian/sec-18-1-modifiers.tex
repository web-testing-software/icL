% !TeX spellcheck = ru_RU

\section{Модификаторы управления}

Модификаторы управления позволяют изменить ход выполнения некоторых конструкций языка.

\subsubsection{\lstinline|if:not|}

\lstinline|if:not (@var == true)| эквивалентно \lstinline|if (!(@var == true))|, в этом случае уменьшения количестве скобок упрощает чтения кода.

\subsubsection{\lstinline|for:alt| - альтернативный параметрический цикл}

В параметрическом цикле меняется порядок выполнения действий. \lstinline|for:alt| гарантирует выполнение блока команд в минимум один раз.

Порядок действий конструкций \lstinline|for|:
\begin{icEnum}
    \item инициализация;
	\item проверка условий;
	\item выполнения блока команд;
	\item выполнения команды перехода на следующую итерацию.
\end{icEnum}

Порядок действий конструкций \lstinline|for:alt|:
\begin{icEnum}
    \item инициализация;
	\item выполнения блока команд;
	\item выполнения команды перехода на следующую итерацию;
	\item проверка условий.
\end{icEnum}

\subsubsection{\lstinline|while:not| - условное повторение кода}

\lstinline|while:not| будет выполнить блок команд, пока условие остаётся ложной.

\subsubsection{\lstinline|do while:not| - цикл с постусловий}

\lstinline|do while:not| будет выполнить блок команд, пока условие остаётся ложной. Так же как и \lstinline|do while| гарантирует что тело цикла будет выполнена минимум 1 раз.

\subsubsection{\lstinline|for:reverse| - прохождение коллекций}

\lstinline|for::reverse| будет пройти коллекцию в обратном порядке.

\subsubsection{\lstinline|filter:reverse| - выборочное прохождение коллекций}

\lstinline|filter:reverse| будет пройти коллекцию в обратном порядке.

\subsubsection{\lstinline|range::reverse| - частичное прохождение коллекций}

\lstinline|filter:reverse| будет пройти интервал в обратном порядке.
