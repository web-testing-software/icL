\section{Базы данных}

Модуль баз данных определяет следующее глобальные объекты: \lstinline|DBManager| - менеджер баз данных, \lstinline|DB| - последняя открытая база данных, \lstinline|Query| - последний запрос.

Модуль баз данных обеспечивает следующий набор возможности:
\begin{icItems}
	\item \lstinline|DBManager.openSQLite <string>path : DB|;
	\item \lstinline|DB.query <Code>q : Query|;
	\item \lstinline|DB.close : void|;
	\item \lstinline|[w/o] Query'(<string>name) : any|;
	\item \lstinline|Query.set <string>field <any>value : void|;
	\item \lstinline|Query.exec : bool|;
	\item Доступны после выполнение:
	\begin{icItems}
		\item \lstinline|[r/o] Query'(<string>name) : any|;
		\item \lstinline|Query.getRowsAffected : int|;
		\item \lstinline|Query.getError : bool|;
		\item \lstinline|Query.getLength : int|;
		\item \lstinline|Query.get <string>field : any|;
		\item \lstinline|Query.next : bool|;
		\item \lstinline|Query.previous : bool|;
		\item \lstinline|Query.first : bool|;
		\item \lstinline|Query.last : bool|;
		\item \lstinline|Query.seek <int>i <bool>relative = false : bool|;
	\end{icItems}
\end{icItems}

\subsubsection{\lstinline|DBManager.openSQLite <string>path : DB|}

Открывает новое подключение. \code{path} - путь к файлу базы данных.

\subsubsection{\lstinline|DB.query <Code>q : Query|}

Создаёт запрос, на основе кода SQL, изолированный в фигурных скобках.

\subsubsection{\lstinline|DB.close : void|}

Закрывает подключение к базу данных.

\subsubsection{\lstinline|[w/o] Query'(<string>name) : any|}

Возвращает объект, позволяющий заменить заменитель на значение через присваивание. Заменители в коде имеют следующий синтаксис \lstinline|:name|.

\subsubsection{\lstinline|Query.set <string>field <any>value : void|}

Установит значения заменителя с именем \code{field}.

\subsubsection{\lstinline|Query.exec : bool|}

Возвращает \true, если выполнения запроса была успешной, иначе \false.

\subsubsection{\lstinline|Query.getError : string|}

Возвращает текст ошибки, если при выполнении кода произошла ошибка.

\subsubsection{\lstinline|[r/o] Query'(<string>name) : any|}

После вызова функции \lstinline|Query.exec| свойства будут возвращать значения запрошенных полей. Если такое поле отсутствует, будет возвращать \void.

\subsubsection{\lstinline|Query.getRowsAffected : int|}

Количество обновлённых/добавленных строк в базе данных.

\subsubsection{\lstinline|Query.getLength : int|}

Количество результатов полученных при выполнении команды \code{SELECT}.

\subsubsection{\lstinline|Query.get <string>field : any|}

Возвращает значение поле или \void{} если столбец \code{field} отсутствует.

\subsubsection{\lstinline|Query.next : bool|}

Возвращает \true, если получилось перейти на следующую запись, иначе \false.

\subsubsection{\lstinline|Query.previous : bool|}

Возвращает \true, если получилось перейти на предыдущую запись, иначе \false.

\subsubsection{\lstinline|Query.first : bool|}

Возвращает \true, если получилось перейти на первую запись, иначе \false.

\subsubsection{\lstinline|Query.last : bool|}

Возвращает \true, если получилось перейти на последнюю запись, иначе \false.

\subsubsection{\lstinline|Query.seek <int>i <bool>relative = false : bool|}

Возвращает \true, если получилось перейти на \code{i}-ю запись или сдвинуть курсор на \code{i}-e количество шагов (при условии \code{@relative == true}), иначе \false.

\subsubsection{Пример}

Пример кода, использующий базы данных представлен на листинге \ref{dbexample}. Так же как и при использовании кода на языке Javascript, в коде можно встроить переменные icL, локальные переменный имеет синтаксис \code{@:name}, а глобальные - \code{\#:name}.

\begin{lstlisting}[caption=Пример кода использующий базу данных, label=dbexample]
DBManager.openSQLite "db.sqlite";

`` simple query
DB.query {
	SELECT country
	FROM artists
};
Query.exec;

while (Query.next) {
	!doSomething Query'country;
};

`` query with parameters
DB.query {
	SELECT name
	FROM programmers
	WHERE country = :country
	LIMIT 1
};
Query'country = "Moldova";
`` is equivalent to
@md = "Moldova";
DB.query {
	SELECT name
	FROM programmers
	WHERE country = @:md
	LIMIT 1
};

`` insert query
DB.query {
	INSERT INTO person (id, forname, name)
	VALUE (:id, :forname, :name)
};
Query'id = 1001;
Query'forname = "Bart";
Query'name = "Simpson";
Query.exec;
\end{lstlisting}

%\newpage
