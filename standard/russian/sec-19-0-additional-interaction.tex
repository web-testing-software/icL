\section{Дополнительное взаимодействие}

Данные возможности доступны только в режиме автоматизации.

\subsection{Ожидание}

Для ожиданий предусмотрены следующие функций:
\begin{icItems}
	\item \lstinline|_wait.title <object>data : bool|;
	\item \lstinline|_wait.url <object>data : bool|;
	\item \lstinline|_wait.for <element>el <object>data : bool|;
	\item \lstinline|_wait.ajax <object>data : bool|.
\end{icItems}

\subsubsection{\lstinline|_wait.title <object>data : bool|}

Объект \code{data} может иметь следующие поля:
\begin{icItems}
	\item \lstinline|[r/w] @data'template : string| - шаблон нужней имени;
	\item \lstinline|[r/w] @data'expression : regex| - регулярное выражение описывающая ножное имя (имеет приоритет над шаблоном);
	\item \lstinline|[r/w] @data'timeout : int| - сколько миллисекунд ждать, по умолчанию 300 000.
\end{icItems}

\subsubsection{\lstinline|_wait.url <object>data : bool|}

Объект \code{data} может иметь следующие поля:
\begin{icItems}
	\item \lstinline|[r/w] @data'template : string| - шаблон нужного URL;
	\item \lstinline|[r/w] @data'expression : regex| - регулярное выражение описывающая нужный URL;
	\item \lstinline|[r/w] @data'timeout : int| - сколько миллисекунд ждать, по умолчанию 300 000.
\end{icItems}

\subsubsection{\lstinline|_wait.for <element>el <object>data : bool|}

Объект \code{data} может иметь следующие поля:
\begin{icItems}
	\item \lstinline|[r/w] @data'count : int| - количество нужных элементов в коллекции. По умолчанию 1;
	\item \lstinline|[r/w] @data'toBe : int| - нужное состояние для достижений. Предлагается одно из следующих значения:
	\begin{icItems}
		\item \lstinline|[r/o] _wait'present : 1|;
		\item \lstinline|[r/o] _wait'visible : 2|;
		\item \lstinline|[r/o] _wait'hidden : 3|;
		\item \lstinline|[r/o] _wait'interactable : 4|;
		\item \lstinline|[r/o] _wait'selected : 5|.
	\end{icItems}
	По умолчанию \lstinline|_wait'present|;
	\item \lstinline|[r/w] @data'timeout : int| - сколько миллисекунд ждать, по умолчанию 300 000.
\end{icItems}

\subsubsection{\lstinline|_wait.ajax <object>data : bool|}

Объект \code{data} может иметь следующие поля:
\begin{icItems}
	\item \lstinline|[r/w] @data'count : int| - количество максимальных допустимых активных запросов, по умолчанию 0;
	\item \lstinline|[r/w] @data'template : string| - шаблон нужного URL;
	\item \lstinline|[r/w] @data'expression : regex| - регулярное выражение описывающая нужный URL;
	\item \lstinline|[r/w] @data'timeout : int| - сколько миллисекунд ждать, по умолчанию 300 000.
\end{icItems}

\subsection{Клавиатура}

Для работы с клавиатуры предусмотрены следующие функций:
\begin{icItems}
	\item \lstinline|element.keyDown <int>modifiers <string>key : element|;
	\item \lstinline|element.keyUp <int>modifiers <string>key : element|;
	\item \lstinline|element.keyPress <int>modifiers <int>delay <string>keys : element|;
	\item \lstinline|element.fastType <string>text : element|;
	\item \lstinline|element.paste <string>text : element|.
\end{icItems}

\subsubsection{\lstinline|element.keyDown <int>modifiers <text>keys : element|}

Симулирует нажатия кнопки.

\subsubsection{\lstinline|element.keyUp <int>modifiers <text> : element|}

Симулирует отпускания кнопки.

\subsubsection{\lstinline|element.keyPress <int>modifiers <int>delay <string>keys : element|}

Симулирует процесс печати на клавиатуре с установленной задержкой.

\subsubsection{\lstinline|element.fastType <string>text : element|}

Симулирует процесс нажатия на клавиатуре с максимальной скоростью.

\subsubsection{\lstinline|element.paste <string>text : element|}

Копирует текст в буфер обмена, фокусирует элемент и вставит текст.

\subsection{Мышка}
\label{mouse}

Для работы с мышкой предусмотрены следующие функций:
\begin{icItems}
	\item \lstinline|element.click <object>data : element|;
	\item \lstinline|element.mouseDown <object>data : element|;
	\item \lstinline|element.mouseUp <object>data : element|;
	\item \lstinline|element.hover <object>data : element|.
\end{icItems}

В следующих функциях объект \code{data} может иметь следующие поля:
\begin{icItems}
	\item \lstinline|[r/w] @data'button : int| - кнопка мышки, принимает одно из следующих значений:
	\begin{icItems}
		\item \lstinline|[r/o] _mouse'left : 1| - левая кнопка;
		\item \lstinline|[r/o] _mouse'middle : 2| - средняя кнопка;
		\item \lstinline|[r/o] _mouse'right : 3| - правая кнопка.
	\end{icItems}
	По умолчанию \lstinline|_mouse'left|;
	\item \lstinline|[r/w] @data'rx : double| - \lstinline|ax = rx * width|;
	\item \lstinline|[r/w] @data'ry : double| - \lstinline|ay = rx * height|;
	\item \lstinline|[r/w] @data'ax : int| - относительная координата от левого края элемента;
	\item \lstinline|[r/w] @data'ay : int| - относительная координата от верхнего края элемента.
\end{icItems}

\subsubsection{\lstinline|element.click <object>data : element|}

Симулирует клик. Объект \code{data} имеет следующие дополнительные свойства:
\begin{icItems}
	\item \lstinline|[r/w] @data'delay : int| - задержка между нажатием и отпускании кнопке;
	\item \lstinline|[r/w] @data'count : int| - количество симулированных кликов.
\end{icItems}

\subsubsection{\lstinline|element.mouseDown <object>data : element|}

Симулирует нажатие кнопки мышки.

\subsubsection{\lstinline|element.mouseUp <object>data : element|}

Симулирует отпускание кнопки мышки.

\subsubsection{\lstinline|element.hover <object>data : element|}

Перемещает мышку над элементом. \code{data} имеет следующие дополнительные свойства:
\begin{icItems}
	\item \lstinline|[r/w] @data'time : int| - среднее время за которой курсор перемещается на 100 пикселей;
	\item \lstinline|[r/w] @data'func : int| - функция описывающая движение курсора. Принимает одно из следующих значений:
	\begin{icItems}
		\item \lstinline|[r/o] _move'teleport : 1| - перемещает курсор немедленно;
		\item \lstinline|[r/o] _move'linear : 2| - линейная интерполяция;
		\item \lstinline|[r/o] _move'quadratic : 3| - квадратичная интерполяция;
		\item \lstinline|[r/o] _move'cubic : 4| - кубическая интерполяция;
		\item \lstinline|[r/o] _move'bezier : 5| - использовает кубическую кривую Безье. Использует дополнительные параметры \code{p1x}, \code{p1y}, \code{p2x} и \code{p2y}. Если они не установлены то получат случайное значение от 0,0 до 1,0.
	\end{icItems}
	По умолчанию \lstinline|_move'teleport|.
	\item \lstinline|[r/w] @data'p1x : double| - Координата $x$ первой точке.
	\item \lstinline|[r/w] @data'p1y : double| - Координата $y$ первой точке.
	\item \lstinline|[r/w] @data'p2x : double| - Координата $x$ второй точке.
	\item \lstinline|[r/w] @data'p2y : double| - Координата $y$ второй точке.
\end{icItems}

\subsection{Тонкая настройка}

Настроить можно следующее параметры:
\begin{icItems}
	\item \lstinline|[r/w] _icL'clickTime : int|;
	\item \lstinline|[r/w] _icL'pressTime : int|;
	\item \lstinline|[r/w] _icL'moveTime : int|;
	\item \lstinline|[r/w] _icL'flashMode : bool|;
	\item \lstinline|[r/w] _icL'humanMode : bool|.
\end{icItems}

\subsubsection{\lstinline|_icL'clickTime : int|}

Время задержки клика, по умолчанию 300.

\subsubsection{\lstinline|_icL'pressTime : int|}

Время задержки нажатий кнопки, по умолчанию 60.

\subsubsection{\lstinline|_icL'moveTime : int|}

Среднее время в миллисекундах за которой курсор перемещается на 100 пикселей, по умолчанию 32.

\subsubsection{\lstinline|_icL'flashMode : bool|}

Режим "Вспышки", все операции будут выполнены максимально быстро. Отменяет \lstinline|_icL'humanMode|. По умолчанию выключен. \lstinline|element.sendKeys| и \lstinline|element.fastType| будут работать также быстро как и \lstinline|element.paste|, изменение содержимого буфера обмены может влиять отрицательно на опыт использования компьютера.

\subsubsection{\lstinline|_icL'humanMode : bool|}

Режим "Человека", все операции будут сделаны медленно и максимально человеко подобно. Отменяет \lstinline|_icL'flashMode|. Также будет нарисован дополнительный курсор, указывающий положение симулировано курсора. По умолчанию данный режим выключен.

%\newpage
