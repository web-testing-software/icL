% !TeX spellcheck = ru_RU
\section{Дополнительное взаимодействие}

Данные возможности доступны только в режиме автоматизации.

\subsection{Ожидание}

Для ожиданий предусмотрены следующие конструкции:
\begin{icItems}
	\item \lstinline|wait { condition }|;
	\item \lstinline|wait:Xs { condition }|;
	\item \lstinline|wait:Xms { condition }|;
	\item \lstinline|wait:ajax { condition }|.
\end{icItems}

\subsubsection{\lstinline|wait|}

Ожидает пока условие не станет истинной. Время ожидании: \code{Session'im\- plicitTimeout} миллисекунды. 

\subsubsection{\lstinline|wait:Xs|}

Ожидает пока условие не станет истинной. Время ожидании: \code{X} секунды. 

\subsubsection{\lstinline|wait:Xms|}

Ожидает пока условие не станет истинной. Время ожидании: \code{X} миллисекунды. 

\subsubsection{\lstinline|wait:ajax|}

Ожидает пока условие не станет истинной. Время ожидания: \code{Session'implicit\- Timeout} миллисекунды. В сравнении с обычным \code{wait}, тут будет передано следующие данные: \code{@response} - объект с данными о завершённом запросе и \code{@count} - количество активных запросов.

\subsection{Клавиатура}

Для работы с клавиатурой предусмотрены следующие функций:
\begin{icItems}
	\item \lstinline|element.keyDown (modifiers : int, key : string) : element|;
	\item \lstinline|element.keyUp (modifiers : int, key : string) : element|;
	\item \lstinline|element.keyPress (modifiers : int, delay : int, keys : string) : element|;
	\item \lstinline|element.fastType (text : string) : element|;
	\item \lstinline|element.paste (text : string) : element|.
	\item \lstinline|element.forceType (text : string) : element|.
	\item \lstinline|Doc.type (text : string) : Doc|.
\end{icItems}

\subsubsection{\lstinline|element.keyDown (modifiers : int, keys : text) : element|}

Симулирует нажатие кнопки.

\subsubsection{\lstinline|element.keyUp (modifiers : int, <text>) : element|}

Симулирует отпускание кнопки.

\subsubsection{\lstinline|element.keyPress (modifiers : int, delay : int, keys : string) : element|}

Симулирует процесс печати на клавиатуре с установленной задержкой.

\subsubsection{\lstinline|element.fastType (text : string) : element|}

Симулирует процесс нажатие на клавиатуре с максимальной скоростью.

\subsubsection{\lstinline|element.paste (text : string) : element|}

Копирует текст в буфер обмена, фокусирует элемент и вставит текст.

\subsubsection{\lstinline|element.forceType (text : string) : element|}

Вставит значения в текстовом поле с помощью фрагмента кода на языке JavaScript.

\subsubsection{\lstinline|Doc.type (text : string) : element|}

Симулирует процесс печати без фокусировки какого-то либо элемента.

\subsection{Мышка}
\label{mouse}

Для работы с мышкой предусмотрены следующие функций:
\begin{icItems}
	\item \lstinline|element.click (data : object) : element|;
	\item \lstinline|element.forceClick (data : object) : element|;
	\item \lstinline|element.mouseDown (data : object) : element|;
	\item \lstinline|element.mouseUp (data : object) : element|;
	\item \lstinline|element.hover (data : object) : element|.
	\item \lstinline|Doc.click (data : object) : Doc|.
	\item \lstinline|Doc.mouseDown (data : object) : Doc|;
	\item \lstinline|Doc.mouseUp (data : object) : Doc|;
	\item \lstinline|Doc.hover (data : object) : Doc|.
\end{icItems}

В следующих функциях объект \code{data} может иметь следующие поля:
\begin{icItems}
	\item \lstinline|[r/w] @data'button : int| - кнопка мышки, принимает одно из следующих значений:
	\begin{icItems}
		\item \lstinline|[r/o] Mouse'left : 1| - левая кнопка;
		\item \lstinline|[r/o] Mouse'middle : 2| - средняя кнопка;
		\item \lstinline|[r/o] Mouse'right : 3| - правая кнопка.
	\end{icItems}
	По умолчанию \lstinline|Mouse'left|;
	\item \lstinline|[r/w] @data'rx : double| - \lstinline|ax = rx * width|;
	\item \lstinline|[r/w] @data'ry : double| - \lstinline|ay = rx * height|;
	\item \lstinline|[r/w] @data'ax : int| - относительная координата от левого края элемента;
	\item \lstinline|[r/w] @data'ay : int| - относительная координата от верхнего края элемента.
\end{icItems}

\subsubsection{\lstinline|element.click (data : object) : element|}

Симулирует клик. Объект \code{data} имеет следующие дополнительные свойства:
\begin{icItems}
	\item \lstinline|[r/w] @data'delay : int| - задержка между нажатием и отпускании кнопке;
	\item \lstinline|[r/w] @data'count : int| - количество симулированных кликов.
\end{icItems}

\subsubsection{\lstinline|element.forceClick (data : object) : element|}

Симулирует клик на элемент без проверки его доступности.

\subsubsection{\lstinline|element.mouseDown (data : object) : element|}

Симулирует нажатие кнопки мышки.

\subsubsection{\lstinline|element.mouseUp (data : object) : element|}

Симулирует отпускание кнопки мышки.

\subsubsection{\lstinline|element.hover (data : object) : element|}

Перемещает мышку над элементом. \code{data} имеет следующие дополнительные свойства:
\begin{icItems}
	\item \lstinline|[r/w] @data'time : int| - среднее время за которой курсор перемещается на 100 пикселей;
	\item \lstinline|[r/w] @data'func : int| - функция описывающая движение курсора. Принимает одно из следующих значений:
	\begin{icItems}
		\item \lstinline|[r/o] Move'teleport : 1| - перемещает курсор немедленно;
		\item \lstinline|[r/o] Move'linear : 2| - линейная интерполяция;
		\item \lstinline|[r/o] Move'quadratic : 3| - квадратичная интерполяция;
		\item \lstinline|[r/o] Move'cubic : 4| - кубическая интерполяция;
		\item \lstinline|[r/o] Move'bezier : 5| - использовает кубическую кривую Безье. Использует дополнительные параметры \code{p1x}, \code{p1y}, \code{p2x} и \code{p2y}. Если они не установлены то получат случайное значение от 0,0 до 1,0.
	\end{icItems}
	По умолчанию \lstinline|Move'teleport|.
	\item \lstinline|[r/w] @data'p1x : double| - Координата $x$ первой точке.
	\item \lstinline|[r/w] @data'p1y : double| - Координата $y$ первой точке.
	\item \lstinline|[r/w] @data'p2x : double| - Координата $x$ второй точке.
	\item \lstinline|[r/w] @data'p2y : double| - Координата $y$ второй точке.
\end{icItems}

\subsubsection{\lstinline|Doc.click (data : object) : Doc|}

Симулирует клик без привязанности к элементу.

\subsubsection{\lstinline|Doc.mouseDown (data : object) : Doc|}

Симулирует нажатие клавиши мышки без привязанности к элементу.

\subsubsection{\lstinline|Doc.mouseUp (data : object) : Doc|}

Симулирует отпускание клавиши мышки без привязанности к элементу.

\subsubsection{\lstinline|Doc.hover (data : object) : Doc|}

Симулирует движение курсора без привязанности к элементу.

\subsection{Тонкая настройка}

Настроить можно следующее параметры:
\begin{icItems}
	\item \lstinline|[r/w] icL'clickTime : int|;
	\item \lstinline|[r/w] icL'pressTime : int|;
	\item \lstinline|[r/w] icL'moveTime : int|;
	\item \lstinline|[r/w] icL'flashMode : bool|;
	\item \lstinline|[r/w] icL'humanMode : bool|.
\end{icItems}

\subsubsection{\lstinline|icL'clickTime : int|}

Время задержки клика, по умолчанию 300.

\subsubsection{\lstinline|icL'pressTime : int|}

Время задержки нажатий кнопки, по умолчанию 60.

\subsubsection{\lstinline|icL'moveTime : int|}

Среднее время в миллисекундах за которой курсор перемещается на 100 пикселей, по умолчанию 32.

\subsubsection{\lstinline|icL'flashMode : bool|}

Режим "Вспышки", все операции будут выполнены максимально быстро. Отменяет \lstinline|icL'humanMode|. По умолчанию выключен. \lstinline|element.sendKeys| и \lstinline|element.fastType| будут работать также быстро как и \lstinline|element.paste|, изменение содержимого буфера обмены может влиять отрицательно на опыт использования компьютера.

\subsubsection{\lstinline|icL'humanMode : bool|}

Режим "Человека", все операции будут сделаны медленно и максимально человеко подобно. Отменяет \lstinline|icL'flashMode|. Также будет нарисован дополнительный курсор, указывающий положение симулировано курсора. По умолчанию данный режим выключен.

%\newpage
