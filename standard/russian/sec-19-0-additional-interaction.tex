% !TeX spellcheck = ru_RU
\section{Дополнительное взаимодействие}

Данные возможности доступны только в режиме автоматизации.

\subsection{Ожидание}

Для ожиданий предусмотрены следующие конструкции:
\begin{icItems}
	\item \mintinline{icl}{wait { condition }};
	\item \mintinline{icl}{wait:Xs { condition }};
	\item \mintinline{icl}{wait:Xms { condition }};
	\item \mintinline{icl}{wait:ajax { condition }}.
\end{icItems}

\subsubsection{\mintinline{icl}{wait}}

Ожидает пока условие не станет истинной. Время ожидании: \mintinline{icl}{Session}\\*\mintinline{icl}{'implicitTimeout} миллисекунды. 

\subsubsection{\mintinline{icl}{wait:Xs}}

Ожидает пока условие не станет истинной. Время ожидании: \mintinline{icl}{X} секунды. 

\subsubsection{\mintinline{icl}{wait:Xms}}

Ожидает пока условие не станет истинной. Время ожидании: \mintinline{icl}{X} миллисекунды. 

\subsubsection{\mintinline{icl}{wait:ajax}}

Ожидает пока условие не станет истинной. Время ожидания: \mintinline{icl}{Session}\\*\mintinline{icl}{'implicitTimeout} миллисекунды. В сравнении с обычным \mintinline{icl}{wait}, тут будет передано следующие данные: \mintinline{icl}{@response} — объект с данными о завершённом запросе и \mintinline{icl}{@count} — количество активных запросов.

\subsection{Клавиатура}

Для работы с клавиатурой предусмотрены следующие функций:
\begin{icItems}
	\item \mintinline{icl}{element.keyDown (modifiers : int, key : string) : element};
	\item \mintinline{icl}{element.keyUp (modifiers : int, key : string) : element};
	\item \mintinline{icl}{element.keyPress (modifiers : int, delay : int, keys : string) : element};
	\item \mintinline{icl}{element.fastType (text : string) : element};
	\item \mintinline{icl}{element.paste (text : string) : element}.
	\item \mintinline{icl}{element.forceType (text : string) : element}.
	\item \mintinline{icl}{Doc.type (text : string) : Doc}.
\end{icItems}

\subsubsection{\mintinline{icl}{element.keyDown (modifiers : int, keys : text) : element}}

Симулирует нажатие кнопки.

\subsubsection{\mintinline{icl}{element.keyUp (modifiers : int, <text>) : element}}

Симулирует отпускание кнопки.

\subsubsection{\mintinline{icl}{element.keyPress (modifiers : int, delay : int, keys : string) : element}}

Симулирует процесс печати на клавиатуре с установленной задержкой.

\subsubsection{\mintinline{icl}{element.fastType (text : string) : element}}

Симулирует процесс нажатие на клавиатуре с максимальной скоростью.

\subsubsection{\mintinline{icl}{element.paste (text : string) : element}}

Копирует текст в буфер обмена, фокусирует элемент и вставит текст.

\subsubsection{\mintinline{icl}{element.forceType (text : string) : element}}

Вставит значения в текстовом поле с помощью фрагмента кода на языке JavaScript.

\subsubsection{\mintinline{icl}{Doc.type (text : string) : element}}

Симулирует процесс печати без фокусировки какого-то либо элемента.

\subsection{Мышка}
\label{mouse}

Для работы с мышкой предусмотрены следующие функций:
\begin{icItems}
	\item \mintinline{icl}{element.click (data : object) : element};
	\item \mintinline{icl}{element.forceClick (data : object) : element};
	\item \mintinline{icl}{element.mouseDown (data : object) : element};
	\item \mintinline{icl}{element.mouseUp (data : object) : element};
	\item \mintinline{icl}{element.hover (data : object) : element}.
	\item \mintinline{icl}{Doc.click (data : object) : Doc}.
	\item \mintinline{icl}{Doc.mouseDown (data : object) : Doc};
	\item \mintinline{icl}{Doc.mouseUp (data : object) : Doc};
	\item \mintinline{icl}{Doc.hover (data : object) : Doc}.
\end{icItems}

В следующих функциях объект \mintinline{icl}{data} может иметь следующие поля:
\begin{icItems}
	\item \mintinline{icl}{[r/w] @data'button : int} — кнопка мышки, принимает одно из следующих значений:
	\begin{icItems}
		\item \mintinline{icl}{[r/o] Mouse'left : 1} — левая кнопка;
		\item \mintinline{icl}{[r/o] Mouse'middle : 2} — средняя кнопка;
		\item \mintinline{icl}{[r/o] Mouse'right : 3} — правая кнопка.
	\end{icItems}
	По умолчанию \mintinline{icl}{Mouse'left};
	\item \mintinline{icl}{[r/w] @data'rx : double} — \mintinline{icl}{ax = rx * width};
	\item \mintinline{icl}{[r/w] @data'ry : double} — \mintinline{icl}{ay = rx * height};
	\item \mintinline{icl}{[r/w] @data'ax : int} — относительная координата от левого края элемента;
	\item \mintinline{icl}{[r/w] @data'ay : int} — относительная координата от верхнего края элемента.
\end{icItems}

\subsubsection{\mintinline{icl}{element.click (data : object) : element}}

Симулирует клик. Объект \mintinline{icl}{data} имеет следующие дополнительные свойства:
\begin{icItems}
	\item \mintinline{icl}{[r/w] @data'delay : int} — задержка между нажатием и отпускании кнопке;
	\item \mintinline{icl}{[r/w] @data'count : int} — количество симулированных кликов.
\end{icItems}

\subsubsection{\mintinline{icl}{element.forceClick (data : object) : element}}

Симулирует клик на элемент без проверки его доступности.

\subsubsection{\mintinline{icl}{element.mouseDown (data : object) : element}}

Симулирует нажатие кнопки мышки.

\subsubsection{\mintinline{icl}{element.mouseUp (data : object) : element}}

Симулирует отпускание кнопки мышки.

\subsubsection{\mintinline{icl}{element.hover (data : object) : element}}

Перемещает мышку над элементом. \mintinline{icl}{data} имеет следующие дополнительные свойства:
\begin{icItems}
	\item \mintinline{icl}{[r/w] @data'time : int} — среднее время за которой курсор перемещается на 100 пикселей;
	\item \mintinline{icl}{[r/w] @data'func : int} — функция описывающая движение курсора. Принимает одно из следующих значений:
	\begin{icItems}
		\item \mintinline{icl}{[r/o] Move'teleport : 1} — перемещает курсор немедленно;
		\item \mintinline{icl}{[r/o] Move'linear : 2} — линейная интерполяция;
		\item \mintinline{icl}{[r/o] Move'quadratic : 3} — квадратичная интерполяция;
		\item \mintinline{icl}{[r/o] Move'cubic : 4} — кубическая интерполяция;
		\item \mintinline{icl}{[r/o] Move'bezier : 5} — использовает кубическую кривую Безье. Использует дополнительные параметры \mintinline{icl}{p1x}, \mintinline{icl}{p1y}, \mintinline{icl}{p2x} и \mintinline{icl}{p2y}. Если они не установлены то получат случайное значение от 0,0 до 1,0.
	\end{icItems}
	По умолчанию \mintinline{icl}{Move'teleport}.
	\item \mintinline{icl}{[r/w] @data'p1x : double} — Координата $x$ первой точке.
	\item \mintinline{icl}{[r/w] @data'p1y : double} — Координата $y$ первой точке.
	\item \mintinline{icl}{[r/w] @data'p2x : double} — Координата $x$ второй точке.
	\item \mintinline{icl}{[r/w] @data'p2y : double} — Координата $y$ второй точке.
\end{icItems}

\subsubsection{\mintinline{icl}{Doc.click (data : object) : Doc}}

Симулирует клик без привязанности к элементу.

\subsubsection{\mintinline{icl}{Doc.mouseDown (data : object) : Doc}}

Симулирует нажатие клавиши мышки без привязанности к элементу.

\subsubsection{\mintinline{icl}{Doc.mouseUp (data : object) : Doc}}

Симулирует отпускание клавиши мышки без привязанности к элементу.

\subsubsection{\mintinline{icl}{Doc.hover (data : object) : Doc}}

Симулирует движение курсора без привязанности к элементу.

\subsection{Тонкая настройка}

Настроить можно следующее параметры:
\begin{icItems}
	\item \mintinline{icl}{[r/w] icL'clickTime : int};
	\item \mintinline{icl}{[r/w] icL'pressTime : int};
	\item \mintinline{icl}{[r/w] icL'moveTime : int};
	\item \mintinline{icl}{[r/w] icL'flashMode : bool};
	\item \mintinline{icl}{[r/w] icL'humanMode : bool}.
\end{icItems}

\subsubsection{\mintinline{icl}{icL'clickTime : int}}

Время задержки клика, по умолчанию 300.

\subsubsection{\mintinline{icl}{icL'pressTime : int}}

Время задержки нажатий кнопки, по умолчанию 60.

\subsubsection{\mintinline{icl}{icL'moveTime : int}}

Среднее время в миллисекундах за которой курсор перемещается на 100 пикселей, по умолчанию 32.

\subsubsection{\mintinline{icl}{icL'flashMode : bool}}

Режим "Вспышки", все операции будут выполнены максимально быстро. Отменяет \mintinline{icl}{icL'humanMode}. По умолчанию выключен. \mintinline{icl}{element.sendKeys} и \mintinline{icl}{element.fastType} будут работать также быстро как и \mintinline{icl}{element.paste}, изменение содержимого буфера обмены может влиять отрицательно на опыт использования компьютера.

\subsubsection{\mintinline{icl}{icL'humanMode : bool}}

Режим "Человека", все операции будут сделаны медленно и максимально человеко подобно. Отменяет \mintinline{icl}{icL'flashMode}. Также будет нарисован дополнительный курсор, указывающий положение симулировано курсора. По умолчанию данный режим выключен.

%\newpage
