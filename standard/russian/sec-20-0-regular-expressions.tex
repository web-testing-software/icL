\section{Регулярные выражения}
\label{regex}

Для объявления {\bf регулярных выражения} используется литерал \lstinline|//pattern//ms|, где \code{pattern} - шаблон регулярней выражений, \code{ms} - модификаторы. В шаблоне группы можно наименовать, анонимная группа имеет следующий синтаксис \lstinline|(content)|, именованная группа - \lstinline|(?<name>content)|.

Внутренний символ может быть не только \code{/}, но и любой другой на свой вкус, кроме апострофа (\lstinline|'|) он конфликтует с оператором {\it корень} (\lstinline|/'|), \lstinline|@|, \lstinline|#|, \lstinline|-|, \lstinline|+| и \lstinline|!|. Например  \lstinline|/"pattern"/mods|, \lstinline|/:pattern:/mods|, \lstinline|/$pattern$/mods|.

Для того чтобы захватить начало строки можно использовать \lstinline|^| и \lstinline|\A|, а для конца строки - \lstinline|$| и \lstinline|\z|. Первый вариант несмотря на его популярность может срабатывать неправильно, так как им захватывается начало и конец строки в контексте многострочного текста.

\subsection{Модификаторы}

В языке icL присутствуют следующее модификаторы: \code{/i}, \code{/s}, \code{/m}, \code{/x}, \code{/u} и \code{/f}. Модификаторы можно скомбинировать, например \code{/xi}.

\subsubsection{Модификатор \code{/i}}

Регистр букв будет проигнорирован. 

\subsubsection{Модификатор \code{/s}}

Точка будет означать любой символ, включая символа {\it новая строка}.

\subsubsection{Модификатор \code{/m}}

Метасимволы \lstinline|^| и \lstinline|$| будут означать начало и конец строки не только в общем смысле, но и в контексте многострочного текста.

\subsubsection{Модификатор \code{/x}}

Все пробельные символы будут проигнорированы, и символ \lstinline|#| будет означать начало комментария, который заканчивается символом {\it новая строка}.

\subsubsection{Модификатор \code{/u}}

Классы символов \lstinline|\w|, \lstinline|\d| и т.д. и их отрицаний \lstinline|\W|, \lstinline|\D| и т.д. будут соответствовать не только английских символов, но и национальных.

\subsubsection{Модификатор \code{/f}}

Инвертирует жадность квантификаторов \lstinline|*|, \lstinline|+|, \lstinline|?| и т.д. и они становятся ленивыми. А их ленивые версии \lstinline|*?|, \lstinline|+?|, \lstinline|??| и т.д. становятся жадными.

{\bf Жадный квантификатор} захватывает максимальное количество символов. {\bf Ленивый квантификатор} - минимальное.

\subsection{Операторы}

Для регулярных выражений доступны следующие операторы:
\begin{icItems}
\item
	3-ого ранга:
	\begin{icItems}
	\item
		\lstinline|string << regex : bool|;
	\item
		\lstinline|string <* regex : list|.
	\end{icItems}

\item 
	6$^+$-ого ранга: \lstinline|string ** regex : object|.
\end{icItems}

Примеры использования операторов приведены на листинге \ref{regexpopsex}.

\begin{lstlisting}[caption=Операторы регулярных выражений, label=regexpopsex]
"string" << //s.r//;            `` true
"string" <* //s(.)r//;          `` ["str", "t"]
"string" ** //s(?<letter>.)r//; `` [letter = "t"]
\end{lstlisting}

\subsubsection{\lstinline|string << regex : bool|}

Возвращает \true{}, если строка содержит подстроку подходящая по шаблону регулярней выражений, иначе \false{}.

\subsubsection{\lstinline|string <* regex : list|}

Возвращает список захваченных подстрок если строка содержит подстроку подходящая по шаблону регулярней выражений, иначе пустой список.

\subsubsection{\lstinline|string ** regex : object|}

Возвращает объект, содержащий значения всех именованных групп, если строка содержит подстроку подходящая по шаблону регулярней выражений и объявлены именованные группы, иначе пустой объект.

% \subsubsection{}

\subsection{Методы}

Методы классы \str{} и \listtype, имеющие параметры типа \regex{}:
\begin{icItems}
\item
	\lstinline|string.count (re : regex) : int|;
\item
	\lstinline|string.indexOf (re : regex, startPos = -1) : int|;
\item
	\lstinline|string.lastIndexOf (re : regex, startPos = -1) : int|;
\item
	\lstinline|string.remove (re : regex) : string|;
\item 
	\lstinline|string.replace (re : regex, after : string) : string|;
\item
	\lstinline|string.split (re : regex, keepEmptyParts = true) : list|.
\item
	\lstinline|list.filter (re : regex) : list|;
\item
	\lstinline|list.indexOf (re : regex, start : int) : int|;
\item
	\lstinline|list.lastIndexOf (re : regex, start : int) : int|;
\end{icItems}

Примеры использования выше перечисленных методов, представлены на листинге \ref{regexplikearg}.

\subsubsection{\lstinline|string.count (re : regex) : int|}

Считает количество подстрок подходящие по шаблону \code{re}.

\subsubsection{\lstinline|string.indexOf (re : regex, startPos = -1) : int|}

Возвращает индекс первой подстроки подходящей по шаблону \code{re}, ища вперёд с позицией \code{startPos}; если такая подстрока не нашлось, возвращает -1.

\begin{lstlisting}[caption=Регулярные выражения в качестве аргумента, label=regexplikearg]
@str = "Hello world, collaborators!";

@str.count //\wo//;       `` 5
@str.indexOf //\w{6}//;   `` 13
@str.lastIndexOf //l+o//; `` 2
@str.remove //\w{2}o//;   `` "He world, collrrs!"
@str.split //,?\s//;      `` ["He", "world", "collrrs!"]

@fonts = ["Arial", "Helvetica", "Times", "Courier"];

@fonts.filter //^[ac]//i;    `` ["Arial", "Courier"]
@fonts.indexOf //e.//;       `` 1
@fonts.indexOf //e.// 2;     `` 2
@fonts.lastIndexOf //e.//;   `` 3
@fonts.lastIndexOf //e.// 2; `` 2
\end{lstlisting}

\subsubsection{\lstinline|string.lastIndexOf (re : regex, startPos = -1) : int|}

Возвращает индекс первой подстроки подходящей по шаблону \code{re}, ища назад с позицией \code{startPos}; если такая подстрока не нашлось, возвращает -1.

\subsubsection{\lstinline|string.remove (re : regex) : string|}

Удаляет из строки все фрагменты подходящие по шаблону \code{re}.

\subsubsection{\lstinline|string.replace (re : regex, after : string) : string|}

Заменяет каждый фрагмент, подходящий по шаблону \code{re}, строкой \code{after}.

\subsubsection{\lstinline|string.split (re : regex, keepEmptyParts = true) : list|}

Разрывает строку на подстроки при каждой встречи подстроки, подходящей по шаблону \code{re} и собирает список этих строк; если \code{keepEmptyParts == false} то пустые строки пропускаются.

\subsubsection{\lstinline|list.filter (re : regex) : list|}

Возвращает новый список строк, содержащий только строки этого списка, подходящих по регулярней выражении \code{re}.

\subsubsection{\lstinline|list.indexOf (re : regex, start = 0) : int|}

Возвращает индекс первой строки, подходящей по регулярней выражении \code{re}, ища вперёд с позицией \code{start}, если такой строке не найдено возвращает -1.

\subsubsection{\lstinline|list.lastIndexOf (re : regex, start = -1) : int|}

Возвращает индекс первой строки, подходящей по регулярней выражении \code{re}, ища назад с позицией \code{start}, если такой строке не найдено возвращает -1.

% \subsubsection{}

%\newpage
