\section{Обмен данных с веб-странице}
\label{dataexchange}

\subsubsection{Название странице}

Название страницы можно получить вызывая \lstinline|[r/*] _tab'title : string|.

\subsubsection{Исходный код}

Исходный код страницы можно получить вызывая \lstinline|[r/o] _tab'source : string|.

\subsubsection{Screenshot}

Screenshot страницы в base64 можно получить вызывая \lstinline|[r/o] _tab'screenshot : string|.

\subsubsection{URL}

Адрес доступен через свойство \lstinline|[r/w] _tab'url : string|.

\subsubsection{Навигация}

\lstinline|_tab.back : void| - позволяет возвращаться на предыдущую страницу.

\lstinline|_tab.forward : void| - позволяет возвращаться на следующую страницу.

\lstinline|_tab.refresh : void| -  позволяет перезагрузить страницу.

\lstinline|[icL] [r/o] _tab'canGoBack : bool| - сообщает можно ли перейти на предыдущую страницу.

\lstinline|[icL] [r/o] _tab'canGoForward : bool| - сообщает можно ли перейти на следующую страницу.


\subsubsection{Предупреждения}

\lstinline|[r/o] _alert'text : string| - текст предупреждений.

\lstinline|_alert.accept : void| - закрыть окно отвечая положительным ответом.

\lstinline|_alert.dismiss : void| - закрыть окно отвечая отрицательным ответом.

\lstinline|_alert.sendKeys <string>str : void| - ответить текстом \code{str}.

% Возможные исключения: \ferror{NoSuchWindow}, \ferror{NoSuchAlert}, \ferror{ElementNotInteractable}.

%\newpage
