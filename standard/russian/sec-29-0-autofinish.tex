% !TeX spellcheck = ru_RU
\section{Автозавершение команд}

\textbf{Разделитель команд} в icL может быть пропущен, но пропускать его не рекомендуется. Для улучшения читабельности кода разделитель лучше пропустить после конструкции \mintinline{icl}{if}, \mintinline{icl}{else}, \mintinline{icl}{while}, \mintinline{icl}{do while}, \mintinline{icl}{for}, \mintinline{icl}{filter}, \mintinline{icl}{range}. Для наглядности можете сравнить листинг \ref{nodelimiterskipping} с листингом \ref{delimiterskipping}.

\begin{sourcecode}
	\captionof{listing}{Без пропущенных разделители}
	\label{nodelimiterskipping}
    \inputminted[linenos]{icl}{../sources/nodelimiterskipping.icL}
\end{sourcecode}

\begin{sourcecode}
	\captionof{listing}{С пропущенными разделителями}
	\label{delimiterskipping}
    \inputminted[linenos]{icl}{../sources/delimiterskipping.icL}
\end{sourcecode}

Команда может быть завершена автоматически после следующих семантических конструкции:
\begin{icItems}
	\item литерал;
	\item значение;
	\item свойство;
	\item запакованные данные;
	\item блок кода.
\end{icItems}