\section{Множества}

{\bf Множества} (тип \set) позволяют хранить несколько объектов (тип \object) без повторений, для множеств доступны операции как пересечение, объединение.

\subsection{{\color{lightblue} object}}

Объект позволяет хранить несколько переменных, каждая хранимая переменная имеет тип и название. Доступ к переменным осуществляется через свойства объекта, где название свойств и есть название переменных.

Поле объектов множества описаны в заголовке множества.

\subsection{Свойства}

Множества имеют следующие свойства:
\begin{icItems}
\item \lstinline|[r/o] set'capacity : int|;
\item \lstinline|[r/o] set'empty : bool|;
\end{icItems}

\subsubsection{\lstinline|[r/o] set'capacity : int|}

Возвращает количество объектов которых можно хранить в память без дополнительных выделений.

\subsubsection{\lstinline|[r/o] set'empty : bool|}

Множество считается пустой если оно не содержит ни один объект.

\subsection{Методы}

Множество имеет следующие методы:
\begin{icItems}
\item \lstinline|set.applicate <list ...>data : set|;
\item \lstinline|set.clear : set|;
\item \lstinline|set.getField <string>name : list|;
\item \lstinline|set.hasField <string>name : bool|;
\item \lstinline|set.insert <any ...>data : set|;
\item \lstinline|set.insert <object>obj : set|;
\item \lstinline|set.insertBoolField <string>name <list>values : set|;
\item \lstinline|set.insertDoubleField <string>name <list>values : set|;
\item \lstinline|set.insertIntField <string>name <list>values : set|;
\item \lstinline|set.insertListField <string>name <list>value : set|;
\item \lstinline|set.insertStringField <string>name <list>values : set|;
\item \lstinline|set.remove <any ...>data : set|;
\item \lstinline|set.remove <object>obj : set|;
\item \lstinline|set.removeField <string>name : set|.
\end{icItems}

Код использующий выше перечисленных методов представлен на листинге \ref{setmethods};

\begin{lstlisting}[caption=Методы класса set, label=setmethods]
@set = [<int>age <string>name <string>sex]

@set.applicate 
	["23"    "45"     "67"]
	["Basil" "Nataly" "Siri"]
	["m"     "f"      "f"];
``` set = [
	[<23>age <"Basil">name  <"m">sex]
	[<45>age <"Nataly">name <"f">sex]
	[<67>age <"Siri">name   <"f">sex]
	] ```

@set.getField "name"; `` ["Basil" "Nataly" "Siri"]

@set.insert 34 "Critiano" "m";
``` set = [
	[<23>age <"Basil">name    <"m">sex]
	[<45>age <"Nataly">name   <"f">sex]
	[<67>age <"Siri">name     <"f">sex]
	[<34>age <"Cristiano"name <"m">sex]
	] ```

@set.insert [<"f">sex <"Sofia">name <19>age];
``` set = [
	[<23>age <"Basil">name    <"m">sex]
	[<45>age <"Nataly">name   <"f">sex]
	[<67>age <"Siri">name     <"f">sex]
	[<34>age <"Cristiano"name <"m">sex]
	[<19>age <"Sofia">name    <"f">sex]
	] ```

@set.insertStringField "parity" ["odd" "even"];
``` set = [
	[<23>age <"Basil">name    <"m">sex <"odd">parity]
	[<45>age <"Nataly">name   <"f">sex <"even">parity]
	[<67>age <"Siri">name     <"f">sex <"odd">parity]
	[<34>age <"Cristiano"name <"m">sex <"even">parity]
	[<19>age <"Sofia">name    <"f">sex <"odd">parity]
	] ```

@set.remove 45 "Nataly" "f" "even";
``` set = [
	[<23>age <"Basil">name    <"m">sex <"odd">parity]
	[<67>age <"Siri">name     <"f">sex <"odd">parity]
	[<34>age <"Cristiano"name <"m">sex <"even">parity]
	[<19>age <"Sofia">name    <"f">sex <"odd">parity]
	] ```

@set.remove [<34>age <"Cristiano"name <"m">sex <"even">parity];
``` set = [
	[<23>age <"Basil">name <"m">sex <"odd">parity]
	[<67>age <"Siri">name  <"f">sex <"odd">parity]
	[<19>age <"Sofia">name <"f">sex <"odd">parity]
	] ```

@set.removeField "parity";
``` set = [
	[<23>age <"Basil">name <"m">sex]
	[<67>age <"Siri">name  <"f">sex]
	[<19>age <"Sofia">name <"f">sex]
	] ```

@set.clear; `` set = []
\end{lstlisting}

\subsubsection{\lstinline|set.applicate <list ...>data : set|}

Количество полученных списков должна быть равна количеству полей объектов множества. Количество добавленных объектов будет ровна длине самого длинного списка из \code{data}. Из списков в которых не хватает строк, элементы будут добавлены повторно.

Возможные исключения: \ferror{IncompatibleData}.

\subsubsection{\lstinline|set.clear : set|}

Очищает множество.

\subsubsection{\lstinline|set.getField <string>name : list|}

Возвращает список значений поле \code{name} всех объектов.

Возможные исключения: \ferror{FieldNotFound}.

\subsubsection{\lstinline|set.hasField <string>name : bool|}

Возвращает \true, если заголовок содержит поле \code{name}, иначе \false.

\subsubsection{\lstinline|set.insert <any ...>data : set|}

Вставит новый объект, значения полей (массив \code{data}) должна полностью соблюдать порядок и тип данных, определён заголовком множество.

Возможные исключения: \ferror{IncompatibleData}.

\subsubsection{\lstinline|set.insert <object>obj : set|}

Вставит новый объект, поля объекта \code{obj} должна быть совместимы полностью с заголовком множество.

Возможные исключения: \ferror{IncompatibleObject}.

\subsubsection{\lstinline|set.insertBoolField <string>name <list>values : set|}

Создаёт новое поле типа \bool. \code{values} может быть пустым списком только если множество пустая. Значений массива распределяется так же как и при использовании \code{set.applicate}.

Возможные исключения: \ferror{FieldAlreadyExists}.

\subsubsection{\lstinline|set.insertDoubleField <string>name <list>values : set|}

Создаёт новое поле типа \double. \code{values} может быть пустым списком только если множество пустая. Значений массива распределяется так же как и при использовании \code{set.applicate}.

Возможные исключения: \ferror{FieldAlreadyExists}.

\subsubsection{\lstinline|set.insertIntField <string>name <list>values : set|}

Создаёт новое поле типа \integer. \code{values} может быть пустым списком только если множество пустая. Значений массива распределяется так же как и при использовании \code{set.applicate}.

Возможные исключения: \ferror{FieldAlreadyExists}.

\subsubsection{\lstinline|set.insertListField <string>name <list>value : set|}

Создаёт новое поле типа \bool. Все объекты множеству получат новое поле со значением \code{value}.

Возможные исключения: \ferror{FieldAlreadyExists}.

\subsubsection{\lstinline|set.insertStringField <string>name <list>values : set|}

Создаёт новое поле типа \str. \code{values} может быть пустым списком только если множество пустая. Значений массива распределяется так же как и при использовании \code{set.applicate}.

Возможные исключения: \ferror{FieldAlreadyExists}.

\subsubsection{\lstinline|set.remove <any ...>data : set|}

Удаляет объект, который подходит по описанию \code{data}.

\subsubsection{\lstinline|set.remove <object>obj : set|}

Удаляет объект \code{obj} из множества.

\subsubsection{\lstinline|set.removeField <string>name : set|}

Удаляет поле \code{name} с заголовка.

Возможные исключения: \ferror{FieldNotFound}.

%\newpage
