% !TeX spellcheck = ru_RU
\section{Строки}

{\bf Строки} (тип \str{}) позволяет хранить и манипулировать фрагменты текста. 

\subsection{Свойства}

В дальнейшем свойства будут отмечены следующим маркером: \mintinline{icl}{[r/w]} — доступно для чтения и записи, \mintinline{icl}{[r/o]} — доступно только для чтения, \mintinline{icl}{[w/o]} — доступно только для записи,  \mintinline{icl}{[r/*]} — доступно для записи только в режиме автоматизации, \mintinline{icl}{[*/*]} — доступ будет конкретизирован потом.

Строки имеют следующее дополнительные свойства:
\begin{icItems}
\item
	\mintinline{icl}{[r/o] string'empty : bool};
\item
	\mintinline{icl}{[r/o] string'length : int};
\item
	\mintinline{icl}{[r/o] string'last : string};
\item
    \mintinline{icl}{[r/o] string'(n : int) : string}.
\end{icItems}

\

На листинге \ref{stringprop} представлен код, использующий выше перечисленных свойства.

\subsubsection{\mintinline{icl}{[r/o] string'empty : bool}}

Строка считается пустой если количество содержащих символов равна нулю;

\begin{sourcecode}
    \captionof{listing}{Свойства класса string}
    \label{stringprop}
    \inputminted[linenos]{icl}{../sources/stringprop.icL}
\end{sourcecode}

\subsubsection{\mintinline{icl}{[r/o] string'length : int}}

Длина строки равна количеству символов в строке;

\subsubsection{\mintinline{icl}{[r/o] string'last : char}}

Последний символ строки, то же самое что и \mintinline{icl}{string.at (string'length - 1)}.

Возможные исключения: \ferror{EmptyString} (см. таб. \ref{errors}).

\subsubsection{\mintinline{icl}{[r/o] string'(n : int) : string}}

n-й символ строки. n должен быть целым литералов, примеры \mintinline{icl}{@str'0; @str'2}.

Возможные исключения: \ferror{OutOfBounds} (см. таб. \ref{errors}).

\subsection{Методы}

Строки имеют следующий набор дополнительных методов:
\begin{icItems}
\item
    \mintinline{icl}{string.append (str : string) : string};
\item
    \mintinline{icl}{string.at (i : int) : string};
\item
    \mintinline{icl}{string.beginsWith (str : string) : bool};
\item
    \mintinline{icl}{string.compare (str : string, caseSensitive = true) : bool};
\item
    \mintinline{icl}{string.count (str : string) : int};
\item
    \mintinline{icl}{string.endsWith (str : string) : bool};
\item
    \mintinline{icl}{string.indexOf (str : string, startPos = 0) : int};
\item
    \mintinline{icl}{string.insert (str : string, pos : int) : string};
\item
    \mintinline{icl}{string.lastIndexOf (str : string, startPos = -1) : int};
\item
    \mintinline{icl}{string.left (n : int) : string};
\item
    \mintinline{icl}{string.leftJustified (width : int, fillChar) : string, truncate = false :}\\*\mintinline{icl}{string};
\item
    \mintinline{icl}{string.mid (pos : int, n = -1) : string};
\item
    \mintinline{icl}{string.prepend (str : string) : string};
\item
    \mintinline{icl}{string.remove (pos : int, n : int) : string};
\item
    \mintinline{icl}{string.remove (str : string, caseSensitive = true) : string};
\item
    \mintinline{icl}{string.replace (pos : int, n : int, after : string) : string};
\item
    \mintinline{icl}{string.replace (before : string, after : string) : string};
\item
    \mintinline{icl}{string.right (n : int) : string};
\item
    \mintinline{icl}{string.rightJustified (width : int, fillChar : string, truncate = false)}\\*\mintinline{icl}{: string};
\item
    \mintinline{icl}{string.split (separator : string, keepEmptyParts = true, caseSensitive = true)}\\*\mintinline{icl}{: list};
\item
    \mintinline{icl}{string.substring (begin : int, end : int) : string};
\item
    \mintinline{icl}{string.trim (justWhitespace = true) : string}.
\end{icItems}

Некоторые методы пропущены, они будут представлены в главе \ref{regex}; Код, использующий выше перечисленных методов представлен на листинге \ref{stringmethods}.

\begin{sourcecode}
    \captionof{listing}{Методы класса string}
    \label{stringmethods}
    \inputminted[linenos]{icl}{../sources/stringmethods.icL}
\end{sourcecode}

\subsubsection{\mintinline{icl}{string.append (str : string) : string}}

Вставит строку \mintinline{icl}{str} в конце строки.

\subsubsection{\mintinline{icl}{string.at (i : int) : string}}

Возвращает ссылка на \mintinline{icl}{i}-й символ.

Возможные исключения: \ferror{OutOfBounds} (см. таб. \ref{errors}).

\subsubsection{\mintinline{icl}{string.beginsWith (str : string) : bool}}

Возвращает \true{}, если начало строки совпадает с \mintinline{icl}{str}, иначе \false{}.

\subsubsection{\mintinline{icl}{string.compare (str : string, caseSensitive = true) : bool}}

Сравнивает строки, возвращает \true{} если они равны, иначе \false{}. Аргумент \mintinline{icl}{caseSensitive} может быть пропущен, если установить его в \false{}, то регистр букв будет проигнорирован.

\subsubsection{\mintinline{icl}{string.count (str : string) : int}}

Считает сколько раз подстрока \mintinline{icl}{str} встречается в строке.

\subsubsection{\mintinline{icl}{string.endsWith (str : string) : bool}}

Возвращает \true{}, если конец строки совпадает с \mintinline{icl}{str}, иначе \false{}.

\subsubsection{\mintinline{icl}{string.indexOf (str : string, startPos = 0) : int}}

Возвращает индекс первой нахождения подстроки \mintinline{icl}{str} в строке, ища вперёд с позиции \mintinline{icl}{startPos}, если подстрока не найдено возвращает -1.

\subsubsection{\mintinline{icl}{string.insert (str : string, pos : int) : string}}

Вставит строку \mintinline{icl}{str} в позиции \mintinline{icl}{pos}.

\subsubsection{\mintinline{icl}{string.lastIndexOf (str : string, startPos = -1) : int}}

Возвращает индекс первой нахождения подстроки \mintinline{icl}{str} в строке, ища назад с позиции \mintinline{icl}{startPos}, если подстрока не найдена возвращает -1.

\subsubsection{\mintinline{icl}{string.left (n : int) : string}}

Возвращает подстроку содержащая первый \mintinline{icl}{n} символы.

\subsubsection{\mintinline{icl}{string.leftJustified (width : int, fillChar) : string, truncate = false :}\\* \mintinline{icl}{string}}

Возвращает строку длины \mintinline{icl}{width}, содержащую эту строку, в конце ставится \mintinline{icl}{width - .length} символов \mintinline{icl}{fillChar}. Если \mintinline{icl}{truncate == true} и \mintinline{icl}{width < .length}, то последние \mintinline{icl}{.length - width} символы будут удалены.

\subsubsection{\mintinline{icl}{string.mid (pos : int, n = -1) : string}}

Возвращает строку содержащую \mintinline{icl}{n} символы строки, начиная с позиции \mintinline{icl}{pos}. Будет возвращена пустая строка если установленный интервал выходит за рамки строки. Если \mintinline{icl}{n == -1} то будут возвращены все доступные символы начиная с позицией \mintinline{icl}{pos}.

\subsubsection{\mintinline{icl}{string.prepend (str : string) : string}}

Вставит строку \mintinline{icl}{str} в начале строки.

\subsubsection{\mintinline{icl}{string.remove (pos : int, n : int) : string}}

Удаляет \mintinline{icl}{n} символы строки начиная с позицией \mintinline{icl}{pos}.

\subsubsection{\mintinline{icl}{string.remove (str : string, caseSensitive = true) : string}}

Удаляет каждое нахождения подстроки \mintinline{icl}{str} в строке. Аргумент \mintinline{icl}{caseSensitive} может быть пропущен, если установить его в \false{} то регистр букв будет проигнорирован.

\subsubsection{\mintinline{icl}{string.replace (pos : int, n : int, after : string) : string}}

Заменяет интервал установленный позицией \mintinline{icl}{pos} и количеством символов \mintinline{icl}{n} строкой \mintinline{icl}{after}.

\subsubsection{\mintinline{icl}{string.replace (before : string, after : string) : string}}

Заменяет каждое нахождение подстроки \mintinline{icl}{before} подстрокой \mintinline{icl}{after}.

\subsubsection{\mintinline{icl}{string.right (n : int) : string}}

Возвращает строку содержащую последние \mintinline{icl}{n} символы строки.

\subsubsection{\mintinline{icl}{string.rightJustified (width : int, fillChar) : string, truncate = false :}\\* \mintinline{icl}{string};}

Возвращает строку длины \mintinline{icl}{width}, содержащую эту строку, в начале ставится \mintinline{icl}{width - .length} символов \mintinline{icl}{fillChar}. Если \mintinline{icl}{truncate == true} и \mintinline{icl}{width < .length}, то первые \mintinline{icl}{.length - width} символы будут удалены.

\subsubsection{\mintinline{icl}{string.split (separator : string, keepEmptyParts = true, caseSensiti : bool,}\\* \mintinline{icl}{ve = true) : list}}

Разрывает строку на подстроки при каждой нахождения подстроки \mintinline{icl}{separator} и собирает список из этих строк. Если \mintinline{icl}{separator} не встречается ни разу, будет возвращён список из одной строки — этой строки. Если \mintinline{icl}{keepEmptyParts == false} то пустые строки пропускаются.

\subsubsection{\mintinline{icl}{string.substring (begin : int, end : int) : string}}

Возвращает подстроку содержащую символы строки, от позиции \mintinline{icl}{begin} до \mintinline{icl}{end}.

\subsubsection{\mintinline{icl}{string.trim (justWhitespace = true) : string}}

Возвращает строку, копия текущей из которого удаляются пробельные символы с начало и конце строки. Если \mintinline{icl}{justWhitespace == false}, то будут удалены все знаки которые не является буквой или цифрой.

% \subsubsection{}

%\newpage
