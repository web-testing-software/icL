\section{Списки}

{\bf Списки} (тип \mintinline{icl}{list}) позволяет хранить несколько строк в одной переменной. Доступ к ним обеспечивается через индекс - номер по порядку строке в списке.

\subsection{Свойства}

Списки имеют следующее дополнительные свойства:
\begin{icItems}
\item
	\mintinline{icl}{[r/o] list'empty : bool};
\item
	\mintinline{icl}{[r/o] list'length : int};
\item
	\mintinline{icl}{[r/o] list'last : string};
\item
	\mintinline{icl}{[r/o] list'(n : int) : string}.
\end{icItems}

На листинге \ref{listprop} представлен код, использующий выше перечисленных свойства.

\begin{sourcecode}
	\captionof{listing}{Свойства класса list}
	\label{listprop}
    \inputminted[linenos]{icl}{../sources/listprop.icL}
\end{sourcecode}

\subsubsection{\mintinline{icl}{[r/o] list'empty : bool}}

Список считается пустым если он не содержит ни одну строку.

\subsubsection{\mintinline{icl}{[r/o] list'length : int}}

Длина списка равна количеству строк в списке.

\subsubsection{\mintinline{icl}{[r/o] list'last : string}}

Последняя строка в списке, то же соме что и \mintinline{icl}{list'at (list'length - 1)}.

Возможные исключения: \ferror{EmptyList} (см. таб. \ref{errors}).

\subsubsection{\mintinline{icl}{[r/o] list'(n : int) : string}}

n-я строка списка, n должен быть целым литералом.

Возможные исключения: \ferror{OutOfBounds} (см. таб. \ref{errors}).

\subsection{Методы}

Списки имеют следующий набор дополнительных методов:
\begin{icItems}
\item \mintinline{icl}{list.append (str : string) : list};
\item \mintinline{icl}{list.at (i : int) : string};
\item \mintinline{icl}{list.contains (str : string, caseSensitive = true) : bool};
\item \mintinline{icl}{list.clear () : list};
\item \mintinline{icl}{list.count (what : string) : int};
\item \mintinline{icl}{list.filter (str : string, caseSensitive = true) : bool};
\item \mintinline{icl}{list.indexOf (str : string, start = 0) : int};
\item \mintinline{icl}{list.insert (index : int, str : string) : list};
\item \mintinline{icl}{list.join (separator : string) : string};
\item \mintinline{icl}{list.lastIndexOf (str : string, start = -1) : int};
\item \mintinline{icl}{list.mid (pos : int, n = -1) : list};
\item \mintinline{icl}{list.prepend (str : string) : list};
\item \mintinline{icl}{list.move (from : int, to : int) : list};
\item \mintinline{icl}{list.removeAll (str : string) : list};
\item \mintinline{icl}{list.removeAt (i : int) : list};
\item \mintinline{icl}{list.removeDuplicates () : list};
\item \mintinline{icl}{list.removeFirst () : list};
\item \mintinline{icl}{list.removeLast () : list};
\item \mintinline{icl}{list.removeOne (str : string) : bool};
\item \mintinline{icl}{list.replaceInStrings (before : string, after : string) : list};
\item \mintinline{icl}{list.sort (caseSensitive = true) : list}.
\end{icItems}

Некоторые методы пропущены, они будут представлены в главе \ref{regex}; Код, использующий выше перечисленных методов представлен на листинге \ref{listmethods}. 

\begin{sourcecode}
	\captionof{listing}{Методы класса list}
	\label{listmethods}
    \inputminted[linenos]{icl}{../sources/listprop.icL}
\end{sourcecode}

Параметр \mintinline{icl}{caseSensitive} во всех функциях отвечает за чувствительность к регистру. Его установка гарантирует что регистр букв будет проигнорирован.

\subsubsection{\mintinline{icl}{list.append (str : string) : list}}

Вставит строку \mintinline{icl}{str} в конце списка.

\subsubsection{\mintinline{icl}{list.at (i : int) : string}}

Возвращает \mintinline{icl}{i}-й строка.

Возможные исключения: \ferror{OutOfBounds} (см. таб. \ref{errors}).

\subsubsection{\mintinline{icl}{list.contains (str : string, caseSensitive = true) : bool}}

Возвращает \true, если список содержит строка, ровна \mintinline{icl}{str}, иначе \false.

\subsubsection{\mintinline{icl}{list.clear () : list}}

Очищает список.

\subsubsection{\mintinline{icl}{list.count (what : string) : int}}

Возвращает сколько раз строка \mintinline{icl}{what} повторяется в списке.

\subsubsection{\mintinline{icl}{list.filter (str : string, caseSensitive = true) : bool}}

Возвращает новый список строк, содержащий только строки этого списка, содержащие подстроку \mintinline{icl}{str}. 

\subsubsection{\mintinline{icl}{list.indexOf (str : string, start = 0) : int}}

Возвращает индекс первой нахождения строки \mintinline{icl}{str} в списке, ища вперёд с позиции \mintinline{icl}{start}, если строка не найдено возвращает -1.

\subsubsection{\mintinline{icl}{list.insert (index : int, str : string) : list}}

Вставит строку \mintinline{icl}{str} в позиции \mintinline{icl}{index}, если \mintinline{icl}{index <= 0} то значение вставится в начале списка, если \mintinline{icl}{index >= list'length} то значение вставится в конце списка.

\subsubsection{\mintinline{icl}{list.join (separator : string) : string}}

Создаёт новую строку, из строк списка, путём последовательной конкатенации, между значениями вставится \mintinline{icl}{separator}, он может быть пустой строкой.

\subsubsection{\mintinline{icl}{list.lastIndexOf (str : string, start = -1) : int}}

Возвращает индекс первой нахождения строки \mintinline{icl}{str} в списке, ища назад с позиции \mintinline{icl}{start}, если строка не найдена возвращает -1.

\subsubsection{\mintinline{icl}{list.mid (pos : int, n = -1) : list}}

Возвращает новый список, содержащий \mintinline{icl}{n} строки, начиная со строкой с индексом \mintinline{icl}{pos}. Если \mintinline{icl}{n} имеет значение \mintinline{icl}{-1} то будут добавлены все значение до конца списка.

\subsubsection{\mintinline{icl}{list.prepend (str : string) : list}}

Вставит строку \mintinline{icl}{str} в начале списка.

\subsubsection{\mintinline{icl}{list.move (from : int, to : int) : list}}

Переставит строку с индексом \mintinline{icl}{from} в позиции с индексом \mintinline{icl}{to}.

Возможные исключения: \ferror{OutOfBounds} (см. таб. \ref{errors}).

\subsubsection{\mintinline{icl}{list.removeAll (str : string) : list}}

Удаляет все нахождения строке \mintinline{icl}{str}.

\subsubsection{\mintinline{icl}{list.removeAt (i : int) : list}}

Удаляет \mintinline{icl}{i}-я строка.

Возможные исключения: \ferror{OutOfBounds} (см. таб. \ref{errors}).

\subsubsection{\mintinline{icl}{list.removeDuplicates () : list}}

Удаляет повторы строк.

\subsubsection{\mintinline{icl}{list.removeFirst () : list}}

Удаляет первую строку.

Возможные исключения: \ferror{EmptyList} (см. таб. \ref{errors}).

\subsubsection{\mintinline{icl}{list.removeLast () : list}}

Удаляет последнюю строку.

Возможные исключения: \ferror{EmptyList} (см. таб. \ref{errors}).

\subsubsection{\mintinline{icl}{list.removeOne (str : string) : bool}}

Удаляет первое нахождение строке \mintinline{icl}{str}. Возвращает \true, если строка была найдена, иначе \false.

\subsubsection{\mintinline{icl}{list.replaceInStrings (before : string, after : string) : list}}

Заменяет подстроку \mintinline{icl}{before} с подстрокой \mintinline{icl}{after} в каждом строке.

\subsubsection{\mintinline{icl}{list.sort (caseSensitive = true) : list}}

Сортирует строки в алфавитном порядке.

%\newpage
