\section{Списки}

{\bf Списки} (тип \code{list}) позволяет хранить несколько строк в одной переменной. Доступ к ним обеспечивается через индекс - номер по порядку строке в списке.

\subsection{Свойства}

Списки имеют следующее дополнительные свойства:
\begin{icItems}
\item
	\lstinline|[r/o] list'empty : bool|;
\item
	\lstinline|[r/o] list'length : int|;
\item
	\lstinline|[r/o] list'last : string|;
\item
	\lstinline|[r/o] list'(n : int) : string|.
\end{icItems}

На листинге \ref{listprop} представлен код, использующий выше перечисленных свойства.

\begin{lstlisting}[caption=Свойства класса list, label=listprop]
@empty = [];
@fonts = ["Arial", "Helvetica", "Times", "Courier"];

@empty'empty; `` true
@fonts'empty; `` false

@empty'length; `` 0
@fonts'length; `` 4

@empty'last; `` error
@fonts'last; `` "Courier"

@empty'0; `` error
@fonts'0; `` "Arial"
\end{lstlisting}

\subsubsection{\lstinline|[r/o] list'empty : bool|}

Список считается пустым если он не содержит ни одну строку.

\subsubsection{\lstinline|[r/o] list'length : int|}

Длина списка равна количеству строк в списке.

\subsubsection{\lstinline|[r/o] list'last : string|}

Последняя строка в списке, то же соме что и \lstinline|list'at (list'length - 1)|.

Возможные исключения: \ferror{EmptyList}.

\subsubsection{\lstinline|[r/o] list'(n : int) : string|}

n-я строка списка, n должен быть целым литералом.

Возможные исключения: \ferror{OutOfBounds}.

\subsection{Методы}

Списки имеют следующий набор дополнительных методов:
\begin{icItems}
\item \lstinline|list.append (str : string) : list|;
\item \lstinline|list.at (i : int) : string|;
\item \lstinline|list.contains (str : string, caseSensitive = true) : bool|;
\item \lstinline|list.clear () : list|;
\item \lstinline|list.count (what : string) : int|;
\item \lstinline|list.filter (str : string, caseSensitive = true) : bool|;
\item \lstinline|list.indexOf (str : string, start = 0) : int|;
\item \lstinline|list.insert (index : int, str : string) : list|;
\item \lstinline|list.join (separator : string) : string|;
\item \lstinline|list.lastIndexOf (str : string, start = -1) : int|;
\item \lstinline|list.mid (pos : int, n = -1) : list|;
\item \lstinline|list.prepend (str : string) : list|;
\item \lstinline|list.move (from : int, to : int) : list|;
\item \lstinline|list.removeAll (str : string) : list|;
\item \lstinline|list.removeAt (i : int) : list|;
\item \lstinline|list.removeDuplicates () : list|;
\item \lstinline|list.removeFirst () : list|;
\item \lstinline|list.removeLast () : list|;
\item \lstinline|list.removeOne (str : string) : bool|;
\item \lstinline|list.replaceInStrings (before : string, after : string) : list|;
\item \lstinline|list.sort (caseSensitive = true) : list|.
\end{icItems}

Некоторые методы пропущены, они будут представлены в главе \ref{regex}; Код, использующий выше перечисленных методов представлен на листинге \ref{listmethods}. 

\begin{lstlisting}[caption=Методы класса list, label=listmethods]
@empty = [];
@fonts = ["Arial", "Helvetica", "Times", "Courier"];

@empty.append "";              `` [""]
@fonts.at 2;                   `` "Times"
@fonts.contains "arial";       `` false
@fonts.contains "arial" false; `` true
@empty.clear;                  `` []
@fonts.count "Arial";          `` 1

@fonts.filter "e";          `` ["Helvetica", "Times", "Courier"]
@fonts.indexOf "Times";     `` 2
@fonts.insert 1 "DejaVu";   `` ["Arial", "DejaVu", "Helvetica", "Times", "Courier"]
@fonts.join ", ";           `` "Arial, DejaVu, Helvetica, Times, Courier"
@fonts.lastIndexOf "Arial"; `` 0

@fonts.mid 2 2;          `` ["Times", "Courier"]
@fonts.prepend "DejaVu"; `` ["DejaVu", "Arial", "DejaVu" .. "Courier"]
@fonts.move 1 2;         `` ["DejaVu", "DejaVu", "Arial", "Helvetica" ..]

@fonts.removeAll "Helvetica"; `` ["DejaVu", "DejaVu", "Arial", "Times", "Courier"]
@fonts.removeAt 3;            `` ["DejaVu", "DejaVu", "Arial", "Courier"]
@fonts.removeDuplicates;      `` ["DejaVu", "Arial", "Courier"]
@fonts.removeFirst;           `` ["Arial", "Courier"]
@fonts.removeLast;            `` ["Arial"]
@fonts.removeOne "Arial";     `` []

@fonts = ["Arial", "Helvetica", "Times", "Courier"];

@fonts.replaceInStrings "e" "-"; `` ["Arial", "H-lv-tica", "Tim-s", "Couri-r"];
@fonts.sort;                     `` ["Arial", "Couri-r", "H-lv-tica", "Tim-s"];
\end{lstlisting}

Параметр \code{caseSensitive} во всех функциях отвечает за чувствительность к регистру. Его установка гарантирует что регистр букв будет проигнорирован.

\subsubsection{\lstinline|list.append (str : string) : list|}

Вставит строку \code{str} в конце списка.

\subsubsection{\lstinline|list.at (i : int) : string|}

Возвращает \code{i}-й строка.

Возможные исключения: \ferror{OutOfBounds}.

\subsubsection{\lstinline|list.contains (str : string, caseSensitive = true) : bool|}

Возвращает \true, если список содержит строка, ровна \code{str}, иначе \false.

\subsubsection{\lstinline|list.clear () : list|}

Очищает список.

\subsubsection{\lstinline|list.count (what : string) : int|}

Возвращает сколько раз строка \code{what} повторяется в списке.

\subsubsection{\lstinline|list.filter (str : string, caseSensitive = true) : bool|}

Возвращает новый список строк, содержащий только строки этого списка, содержащие подстроку \code{str}. 

\subsubsection{\lstinline|list.indexOf (str : string, start = 0) : int|}

Возвращает индекс первой нахождения строки \code{str} в списке, ища вперёд с позиции \code{start}, если строка не найдено возвращает -1.

\subsubsection{\lstinline|list.insert (index : int, str : string) : list|}

Вставит строку \code{str} в позиции \code{index}, если \code{index <= 0} то значение вставится в начале списка, если \code{index >= list'length} то значение вставится в конце списка.

\subsubsection{\lstinline|list.join (separator : string) : string|}

Создаёт новую строку, из строк списка, путём последовательной конкатенации, между значениями вставится \code{separator}, он может быть пустой строкой.

\subsubsection{\lstinline|list.lastIndexOf (str : string, start = -1) : int|}

Возвращает индекс первой нахождения строки \code{str} в списке, ища назад с позиции \code{start}, если строка не найдена возвращает -1.

\subsubsection{\lstinline|list.mid (pos : int, n = -1) : list|}

Возвращает новый список, содержащий \code{n} строки, начиная со строкой с индексом \code{pos}. Если \code{n} имеет значение \code{-1} то будут добавлены все значение до конца списка.

\subsubsection{\lstinline|list.prepend (str : string) : list|}

Вставит строку \code{str} в начале списка.

\subsubsection{\lstinline|list.move (from : int, to : int) : list|}

Переставит строку с индексом \code{from} в позиции с индексом \code{to}.

Возможные исключения: \ferror{OutOfBounds}.

\subsubsection{\lstinline|list.removeAll (str : string) : list|}

Удаляет все нахождения строке \code{str}.

\subsubsection{\lstinline|list.removeAt (i : int) : list|}

Удаляет \code{i}-я строка.

Возможные исключения: \ferror{OutOfBounds}.

\subsubsection{\lstinline|list.removeDuplicates () : list|}

Удаляет повторы строк.

\subsubsection{\lstinline|list.removeFirst () : list|}

Удаляет первую строку.

Возможные исключения: \ferror{EmptyList}.

\subsubsection{\lstinline|list.removeLast () : list|}

Удаляет последнюю строку.

Возможные исключения: \ferror{EmptyList}.

\subsubsection{\lstinline|list.removeOne (str : string) : bool|}

Удаляет первое нахождение строке \code{str}. Возвращает \true, если строка была найдена, иначе \false.

\subsubsection{\lstinline|list.replaceInStrings (before : string, after : string) : list|}

Заменяет подстроку \code{before} с подстрокой \code{after} в каждом строке.

\subsubsection{\lstinline|list.sort (caseSensitive = true) : list|}

Сортирует строки в алфавитном порядке.

%\newpage
