% !TeX spellcheck = ru_RU
\section{Литералы}

Константные значения, которые присутствуют в скрипте в виде части исходного кода, называются {\bf литералами}.

Литералы могут быть любыми из следующих типов:

\begin{icItems}
	\item
		логическое значение - \bool{};
	\item
		целое число - \integer{};
	\item
		дробное число - \double{};
	\item
		строка - \str{};
	\item
		список - \listtype{};
	\item
		объект - \object{};
	\item
		множество - \set{}.
\end{icItems}

\subsubsection{Логические значения}

Для представления {\bf логических значения} используется следующее литералы:
\begin{icItems}
	\item \true{} - логическое единица, истинно;
	\item \false{} - логической ноль, лож.
\end{icItems}

\subsubsection{Целые числа}

Для представления {\bf целых чисел} используется последовательность цифр, перед которым может присутствовать минус. Между минусом и последовательности цифр разделители должны отсутствовать, иначе минус будет интерпретирован как оператор.

\noindent Примеры:
\begin{lstlisting}[numbers=none]
23; -23; - 23; +23 + 3; 12 + -34; 15 - 24; 89--56; 2-3; `` ok
23-; 23+; -2А; 3f5; 23f; 23l; 12u; 89i; 2w1; 1q1; rt2;  `` error
\end{lstlisting}

\subsubsection{Дробные числа}

{\bf Дробный литерал} состоит из двух частей: целая и дробная. Они разделяются точкой. Каждая составная часть является целом числом. Дробная часть не может быть отрицательной.

\noindent Примеры:
\begin{lstlisting}[numbers=none]
23.233452; 29229992.2391; 100.0; -23.29199; -0.23; -0.45 - 1000.5;  `` ok
23.-4; 3а.34; 23-.44; 34.+23; -25.f; -23.5f; -w.45; -2.4e10; -2.E2; `` error
\end{lstlisting}

\subsubsection{Строки}

{\bf Строковым литералом} является последовательность символов, ограниченное с обеих сторон кавычками \lstinline`"`. Чтобы добавить \lstinline`"` используется \lstinline`\"`, символ табуляции - \lstinline`\t`, символ {\it новая строка} - \lstinline`\n`, символ {\it возврат на шаг} - \lstinline`\b`, \textbackslash \ - \lstinline`\\`.

\noindent Примеры:
\begin{lstlisting}[numbers=none]
"Hello \"to\" you!"; "Line1\nLine2"; "Tag1\n\bTag2\n\b"; "text"; "\\ \\ \n \\ \\";
\end{lstlisting}

\subsubsection{Списки}

{\bf Литералом списка} является последовательность строк (разделены запятой), ограниченное квадратными скобками.

\noindent Примеры:
\begin{lstlisting}[numbers=none]
@fruits = ["Apple", "Mango", "Banana", "Lime", "Lemon", "Olive"];
@vegetables = ["Cress", "Mustard", "Guar", "Soybean", "Leek", "Ahipa"];
\end{lstlisting}

\subsubsection{Объекты}

{\bf Объектом} является объединение нескольких переменных под общем именем, переменная объявлена внутри объекта называется {\it поле}. Литерал {\it поле} имеет следующий синтаксис \lstinline|name = value|, где  {\it name} - название и {\it value} - значение. В icL неинициализируемые поля, как и неинициализируемые переменные, нельзя объявить.

Также поле может быть описано в сокращённом форме \lstinline|@var|, где \lstinline|@var| любая переменная, в этом случае поле получит и название и значение переменной. Если количество полей превышает единицу и ни одну поле не объявляется явно, то такая семантическая последовательность \lstinline|[@var1, @var2]| будет воспринята как оператор объединения данных, чтобы указать что это объект укажите явно постое поле на первом месте \lstinline|[=, @var1, @var2]|.

\noindent Примеры:
\begin{lstlisting}[numbers=none]
@quotation = [author = "author", text = "text"];
@child = [age = 4, hasBrothers = true, hasParents = true];
@file = [isEmpty = false, size = 25220, readOnly = true];

@text = "la la la"; @file = "song.mp3";
@object = [=, @text, @file]; `` [text = @text, file = @file];
\end{lstlisting}

\subsubsection{Множества}

Только {\bf заголовок множества} можно описать литералом, литерал схож с литералом объекта, только вместо полей, указываются столбцы и они имеют следующий синтаксис - \lstinline|name : type|, где  {\it name} - название и {\it type} - тип значения.

Столбец также может быть описан в сокращённом виде, используя синтаксис \lstinline|type|, где \lstinline|type| имя типы данных, в этом случае столбец получит имя типа данных. В сравнения с объектами здесь конфликты отсутствуют.

\noindent Примеры:
\begin{lstlisting}[numbers=none]
@quotations = [author : string, text : string];
@children = [age : int, hasBrothers : bool, hasParents : bool];
@files = [isEmpty : bool, size : int, readOnly : bool];
\end{lstlisting}

%\newpage
