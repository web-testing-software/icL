% !TeX spellcheck = ru_RU
\section{Литералы}

Константные значения, которые присутствуют в скрипте в виде части исходного кода, называются {\bf литералами}.

Литералы могут быть любыми из следующих типов:

\begin{icItems}
	\item
		логическое значение — \bool{};
	\item
		целое число — \integer{};
	\item
		дробное число — \double{};
	\item
		строка — \str{};
	\item
		список — \listtype{};
	\item
		объект — \object{};
	\item
		множество — \set{}.
\end{icItems}

\subsubsection{Логические значения}

Для представления {\bf логических значения} используется следующее литералы:
\begin{icItems}
	\item \true{} — логическое единица, истинно;
	\item \false{} — логической ноль, лож.
\end{icItems}

\subsubsection{Целые числа}

Для представления {\bf целых чисел} используется последовательность цифр, перед которым может присутствовать минус. Между минусом и последовательности цифр разделители должны отсутствовать, иначе минус будет интерпретирован как оператор.

\noindent Примеры:
\begin{minted}{icl}
23; -23; - 23; +23 + 3; 12 + -34; 15 - 24; 89--56; 2-3; `` ok
23-; 23+; -2А; 3f5; 23f; 23l; 12u; 89i; 2w1; 1q1; rt2;  `` error
\end{minted}

\subsubsection{Действительные числа}

{\bf Действительный литерал} состоит из двух частей: целая и дробная. Они разделяются точкой. Каждая составная часть является целом числом. Дробная часть не может быть отрицательной.

\noindent Примеры:
\begin{minted}{icl}
23.233452; 29229992.2391; 100.0; -23.29199; -0.23; -0.45 - 1000.5;  `` ok
23.-4; 3а.34; 23-.44; 34.+23; -25.f; -23.5f; -w.45; -2.4e10; -2.E2; `` error
\end{minted}

\subsubsection{Строки}

{\bf Строковым литералом} является последовательность символов, ограниченное с обеих сторон кавычками \mintinline{icl}{"}. Чтобы добавить \mintinline{icl}{"} используется \mintinline{icl}{\"}, символ табуляции — \mintinline{icl}{\t}, символ {\it новая строка} — \mintinline{icl}{\n}, символ {\it возврат на шаг} — \mintinline{icl}{\b}, \textbackslash \ — \mintinline{icl}{\\}.

\noindent Примеры:
\begin{minted}{icl}
"Hello \"to\" you!"; "Line1\nLine2"; "Tag1\n\bTag2\n\b"; "text"; "\\ \\ \n \\ \\";
\end{minted}

\subsubsection{Списки}

{\bf Литералом списка} является последовательность строк (разделены запятой), ограниченное квадратными скобками.

\noindent Примеры:
\begin{minted}{icl}
@фрукты = ["Яблоко", "Манго", "Банан", "Лайм", "Лимон", "Маслина"];
@овощи = ["Кресс", "Горчица", "Гуар", "Соя", "Лук-порей", "Редис"];
\end{minted}

\subsubsection{Объекты}

{\bf Объектом} является объединение нескольких переменных под общем именем, переменная объявлена внутри объекта называется {\it полем}. Литерал {\it поле} имеет следующий синтаксис \mintinline{icl}{name = value}, где  {\it name} — название и {\it value} — значение. В icL неинициализируемые поля, как и неинициализируемые переменные, нельзя объявить.

Также поле может быть описано в сокращённом форме \mintinline{icl}{@var}, где \mintinline{icl}{@var} любая переменная, в этом случае поле получит и название и значение переменной. Если количество полей превышает единицу и ни одну поле не объявляется явно, то такая семантическая последовательность \mintinline{icl}{[@var1, @var2]} будет воспринята как оператор объединения данных, чтобы указать что это объект укажите явно постое поле на первом месте \mintinline{icl}{[=, @var1, @var2]}.

\noindent Примеры:
\begin{minted}{icl}
@quotation = [author = "author", text = "text"];
@child = [age = 4, hasBrothers = true, hasParents = true];
@file = [isEmpty = false, size = 25220, readOnly = true];

@text = "la la la"; @file = "song.mp3";
@object = [=, @text, @file]; `` [text = @text, file = @file];

@empty = [=];
\end{minted}

\subsubsection{Множества}

Только {\bf заголовок множества} можно описать литералом, литерал схож с литералом объекта, только вместо полей, указываются столбцы и они имеют следующий синтаксис — \mintinline{icl}{название : type}, где \mintinline{icl}{type} — тип значения.

Столбец также может быть описан в сокращённом виде, используя синтаксис \mintinline{icl}{type}, где \mintinline{icl}{type} имя типы данных, в этом случае столбец получит имя типа данных. В сравнения с объектами здесь конфликты отсутствуют.

\noindent Примеры:
\begin{minted}{icl}
@quotations = [author : string, text : string];
@children = [age : int, hasBrothers : bool, hasParents : bool];
@files = [isEmpty : bool, size : int, readOnly : bool];

@empty = [:];
\end{minted}

%\newpage
