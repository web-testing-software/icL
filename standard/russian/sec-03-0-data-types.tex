% !TeX spellcheck = ru_RU
\section{Типы данных}

В языке описания сценариев icL, типы данных выполняют очень важную роль, они определяют как данные и будут храниться, обработаться в процессе выполнения программы. Классификация типов данных представлена в таблице \ref{datatypeclasses}.

\stabletwo{3.5cm}{13.6cm}
{datatypeclasses}{Категории типов данных}
{Категория}{Описание}
{
	Базовые     & Типы данных описанных в предыдущем главе: \bool{}{} — логический, \integer{} — целое числа, \double{} — дробные числа. \\ \hline
	Сложные     & Типы данные содержащие несколько значений в контейнерах, примеры таких данных: строки, списки, множество. \\ \hline
	Системные   & Типы данных которых нельзя создать и хранить в пользовательских переменных, но доступ к ним можно получить через системных переменных, примеры таких данных: браузер, вкладка, история браузера, окно. \\ \hline
	Тип \void{} & Тип данных обозначающий отсутствия значения.
}

\subsection{Основные типы данных}

{\bf Основные типы данных} участвуют в логических и арифметических операциях языка icL. Также значения данных типов можно сравнивать. Логическими операциями являются конъюнкция, дизъюнкция, исключающие или, эквиваленция и инверсия. Арифметическими операциями являются сложение, вычитание, умножение, деление, извлечение корня и возведения в степень.

\subsection{Сложные типы данных}

{\bf Сложные типы данных} специализированные под хранения больших массивов данных. Строка позволяет хранить более двух миллиардов символов. Список позволяет хранить более двух миллиардов строк. А множество позволяет хранить большой объём данных ограниченный только оперативной памятью. Объект позволяет инкапсулировать данные разного рода под одном именем. Элемент позволяет хранить ссылку на один или несколько HTML-тэгов.

\subsection{Системные типы данных}

{\bf Системные типы данных} позволяют взаимодействовать с веб-страницей и браузером. Подробную информацию можно найти в главах \ref{webelments} и \ref{dataexchange}.

\subsection{Тип данных {\color{lightblue} void}}

{\bf Тип данных \void{}} означает отсутствия значения, он используется в самых различных целях:

\begin{icItems}
\item
	указать то что функция ничего не возвращает;
\item
	указать на то что в процессе работы функций произошла ошибка;
\item
	выбрать источник данных;
\item
	фильтровать данные;
\item
	и другие.
\end{icItems}


\subsection{Свойства}

Некоторые типы данных имеют свойства. Свойства позволяют получить характеристики хранимых данных. Вне зависимо от типа данные любой объект имеет следующие свойства (перечисленные свойства доступны только для чтения):

\begin{icItems}
\item
	\mintinline{icl}{any'typeName : string} — при чтении получаем строку содержащую название типа хранимых данных;
\item
	\mintinline{icl}{any'typeId : int} — при чтении получаем число содержащее идентификатор типа хранимых данных;
\item
	\mintinline{icl}{any'rValue : bool} — при чтении получаем \true{} если значение предназначено для правой части операции присваивания, иначе \false{};
\item
	\mintinline{icl}{any'readOnly : bool} — при чтении получаем \true{} если объект доступен только для чтения, иначе \false{};
\item
	\mintinline{icl}{any'lValue : bool} — при чтении получаем \true{} если объект доступен только для чтения и записи, иначе \false{};
\item
	\mintinline{icl}{any'link : bool} — при чтении получаем \true{} если значение объекта хранится во внешнем контейнере и изменение значении объекта будет изменить данные во внешней среде, иначе \false{}.
\end{icItems}

\noindent Примеры использования свойств —
\inputminted[linenos]{icl}{../sources/propertiesmain.icL}

\subsection{Методы}

Методы позволяют изменить состояние объекта. Все типы данных имеют только один общий метод \mintinline{icl}{any.ensureRValue}, который гарантирует что изменение значению объекту не будет изменить данные во вне текущего контекста.

\noindent Пример —
\inputminted[linenos]{icl}{../sources/anyensureRValue.icL}

%\newpage
