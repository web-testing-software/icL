% !TeX spellcheck = ru_RU
\section{Циклы}

Может возникнуть потребность в том, чтобы выполнить блок кода несколько раз. Для этого совсем не обязательно писать один то тоже код несколько раз, достаточно использовать цикл.

Язык icL предоставляет следующие типы циклов:

\begin{icItems}
\item
	\code{for} - параметрический цикл;
\item
	\code{while} - условное повторение кода;
\item
	\code{do while} - цикл с постусловий;
\item
	\code{for} - прохождение коллекций;
\item
	\code{filter} - выборочное прохождение коллекций;
\item
	\code{range} - частичное прохождение коллекций.
\end{icItems}


\subsubsection{\lstinline|for| - параметрический цикл}

{\bf Параметрический цикл} позволяет полностью управлять ходом выполнения кода. Он содержит:

\begin{icItems}
\item
	\code{initialization} - код отвечающий за инициализацию;
\item
	\code{condition} - условие, которое проверяется перед запуском цикла;
\item
	\code{step} - код, отвечающий за перехода на следующую итерацию цикла.
\item
	\code{commands} - тело цикла.
\end{icItems}

Параметрический цикл имеет следующий синтаксис -
\begin{lstlisting}[numbers=none]
for (initialization; condition; step) {
	commands
};
\end{lstlisting}

На листинге \ref{uniloopex} показан пример использование цикла \code{for}, цикл будет выводить числа от нуля до четырёх.

\begin{lstlisting}[caption=Параметрический цикл, label=uniloopex]
for (@i = 0; i < 5; @i = @i + 1) {
	Log.out @i;
};
\end{lstlisting}

\subsubsection{\lstinline|while| - условное повторение кода}

{\bf Условное повторение кода} будет многократно исполнять блок команд, пока условие остаётся истиной. Условие проверяется перед запуском цикла и если условие изначально - ложь, тело цикла не будет выполниться ни разу.

Главное его преимущества - простота. В сравнения с универсального цикла, в конструкции \code{while} отсутствуют фрагменты кода для инициализации и перехода на следующую итерацию. Он содержит только условие и тело, синтаксис -
\begin{lstlisting}[numbers=none]
while (condition) {
	commands
};
\end{lstlisting}

На листинге \ref{whileex} переставлен код, использующий цикл \code{while}. Переменная \code{@number} объявлено до цикла и будет доступа и после него. При завершении работы цикла \code{@number} будет равен 6-и.

\begin{lstlisting}[caption=Цикл while, label=whileex]
@number = 1;

while (@number <= 5) {
	@number = @number + 1;
};
\end{lstlisting}

\subsubsection{\lstinline|do while| - цикл с постусловием}

{\bf Цикл с постусловием} мало отличается от \code{while}, но он гарантирует что тело цикла будет выполнено минимум 1 раз. Также условие объявляется после тело цикла, указывая на то что она будет проверено после первой итерации. Синтаксис -
\begin{lstlisting}[numbers=none]
do {
	commands
} while (condition);
\end{lstlisting}

На листинге \ref{dowhileex} представлен код, использующий цикл \code{do while}. Несмотря на то что условие изначально \code{false}, первая итерация цикла будет выполнена, в результате чего \code{@number} станет равен 7-и.

\begin{lstlisting}[caption=Цикл do while, label=dowhileex]
@number = 6;

do {
	@number = @number + 1;
} while (@number < 5);
\end{lstlisting}

\subsubsection{\lstinline|for| - прохождение коллекций}

{Коллекция} - объект, содержащий несколько значений. Самый простой пример коллекций это список, он содержит несколько строк. Например, нужно выполнять операцию для каждой строки списка. Можно использовать параметрический цикл (см. листинг \ref{colluniloop}) или использовать упрощенное {\bf прохождение коллекций} (см. листинг \ref{collsimple}). Под коллекции имеется в виду следующие типы данных \code{list}, \code{set} и \code{element}.
\begin{lstlisting}[caption=Прохождение коллекций с помощью параметрического цикла, label=colluniloop]
@list = ["apple", "banana", "fruit"];

for (@i = 0; @i < @list'length; @i = @i + 1) {
	@list.at @i; `` get i-th value
}
\end{lstlisting}

\begin{lstlisting}[caption=Упрощённое прохождение коллекций, label=collsimple]
@list = ["apple", "banana", "fruit"];

for (@list) {
	@; `` get i-th value
}
\end{lstlisting}

\subsubsection{\lstinline|filter| - выборочное прохождение коллекций}

{\bf Конструкция \code{filter}} позволяет ставить условие, для каких элементов коллекций выполнять тело цикла, для каких нет.

В условия есть специальные переменные: \lstinline|@| - значение элемента коллекций и \lstinline|#| - индекс текущей значения.

Фильтр имеет следующий синтаксис -
\begin{lstlisting}[numbers=none]
filter (collection; condition) {
	commands
};
\end{lstlisting}

Фильтровать элементы можно по данных (см. листинг \ref{filterdataex}, выбирается те которые содержат букву а) или по индексам (см. листинг \ref{filterindexex}, выбирается чётные элементы списка).

\begin{lstlisting}[caption=Фильтрование по данных, label=filterdataex]
@list = ["apple", "banana", "fruit"];

filter (@list; @ << "a") {
	@; `` apple, banana
}
\end{lstlisting}

\subsubsection{\lstinline|range| - частичное прохождение коллекций}

{\bf Частичное прохождение} позволяет выполнить код для интервала коллекций. Интервал определяется одной или двумя условий. Первое условие используется для поиска начала интервала. Второе условие - для поиска конца, если оно отсутствует, будет использоваться первое условие. В отсутствия подходящих элементов, тела цикла не будет выполнена. Чтобы выбирать первый или последний элемент (в зависимость от предназначения условии), использует выражение \true{}.

Синтаксис для интервала с одним условием -
\begin{lstlisting}[numbers=none]
range (collection; condition) {
	commands
};
\end{lstlisting}

Синтаксис для интервала с отдельным условием для начало и конца -
\begin{lstlisting}[numbers=none]
range (collection; condition_for_begin; condition_for_end) {
	commands
};
\end{lstlisting}

\begin{lstlisting}[caption=Фильтрование по индексам, label=filterindexex]
@list = ["apple", "banana", "fruit"];

filter (@list; # \ 2 == 0) {
@; `` apple, fruit
}
\end{lstlisting}

Выбрать первый и последний элемент можно по значении (см. листинг \ref{bananakiwiex}, интервал от значений \code{banana} до \code{kiwi}) или по индексам (см. листинг \ref{seclastex}, интервал от второго до последнего).

\begin{lstlisting}[caption=Интервал: от banana до kiwi, label=bananakiwiex]
@list = ["apple", "banana", "fruit", "kiwi"];

range (@list; @ == "banana"; @ == "kiwi") {
	@; `` banana, fruit, kiwi
}
\end{lstlisting}

\begin{lstlisting}[caption=Интервал: от второго до последнего, label=seclastex]
@list = ["apple", "banana", "fruit", "kiwi"];

range (@list; # == 2; true) {
	@; `` fruit, kiwi
}
\end{lstlisting}

%\newpage
