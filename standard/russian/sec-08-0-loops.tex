% !TeX spellcheck = ru_RU
\section{Циклы}

Может возникнуть потребность в том, чтобы выполнить блок кода несколько раз. Для этого совсем не обязательно писать один то тоже код несколько раз, достаточно использовать цикл.

Язык icL предоставляет следующие типы циклов:

\begin{icItems}
\item
	\mintinline{icl}{for} - параметрический цикл;
\item
	\mintinline{icl}{while} - условное повторение кода;
\item
	\mintinline{icl}{do while} - цикл с постусловий;
\item
	\mintinline{icl}{for} - прохождение коллекций;
\item
	\mintinline{icl}{filter} - выборочное прохождение коллекций;
\item
	\mintinline{icl}{range} - частичное прохождение коллекций.
\end{icItems}


\subsubsection{\mintinline{icl}{for} - параметрический цикл}

{\bf Параметрический цикл} позволяет полностью управлять ходом выполнения кода. Он содержит:

\begin{icItems}
\item
	\mintinline{icl}{initialization} - код отвечающий за инициализацию;
\item
	\mintinline{icl}{condition} - условие, которое проверяется перед запуском цикла;
\item
	\mintinline{icl}{step} - код, отвечающий за перехода на следующую итерацию цикла.
\item
	\mintinline{icl}{commands} - тело цикла.
\end{icItems}

Параметрический цикл имеет следующий синтаксис -
\begin{minted}{icl}
for (initialization; condition; step) {
	commands
};
\end{minted}

На листинге \ref{uniloopex} показан пример использование цикла \mintinline{icl}{for}, цикл будет выводить числа от нуля до четырёх.

\begin{sourcecode}
	\captionof{listing}{Параметрический цикл}
	\label{uniloopex}
    \inputminted[linenos]{icl}{../sources/uniloopex.icL}
\end{sourcecode}

\subsubsection{\mintinline{icl}{while} - условное повторение кода}

{\bf Условное повторение кода} будет многократно исполнять блок команд, пока условие остаётся истиной. Условие проверяется перед запуском цикла и если условие изначально - ложь, тело цикла не будет выполниться ни разу.

Главное его преимущества - простота. В сравнения с универсального цикла, в конструкции \mintinline{icl}{while} отсутствуют фрагменты кода для инициализации и перехода на следующую итерацию. Он содержит только условие и тело, синтаксис -
\begin{minted}{icl}
while (condition) {
	commands
};
\end{minted}

На листинге \ref{whileex} переставлен код, использующий цикл \mintinline{icl}{while}. Переменная \mintinline{icl}{@number} объявлено до цикла и будет доступа и после него. При завершении работы цикла \mintinline{icl}{@number} будет равен 6-и.

\begin{sourcecode}
	\captionof{listing}{Цикл while}
	\label{whileex}
    \inputminted[linenos]{icl}{../sources/whileex.icL}
\end{sourcecode}

\subsubsection{\mintinline{icl}{do while} - цикл с постусловием}

{\bf Цикл с постусловием} мало отличается от \mintinline{icl}{while}, но он гарантирует что тело цикла будет выполнено минимум 1 раз. Также условие объявляется после тело цикла, указывая на то что она будет проверено после первой итерации. Синтаксис -
\begin{minted}{icl}
do {
	commands
} while (condition);
\end{minted}

На листинге \ref{dowhileex} представлен код, использующий цикл \mintinline{icl}{do while}. Несмотря на то что условие изначально \mintinline{icl}{false}, первая итерация цикла будет выполнена, в результате чего \mintinline{icl}{@number} станет равен 7-и.

\begin{sourcecode}
	\captionof{listing}{Цикл do while}
	\label{dowhileex}
    \inputminted[linenos]{icl}{../sources/dowhileex.icL}
\end{sourcecode}

\subsubsection{\mintinline{icl}{for} - прохождение коллекций}

{Коллекция} - объект, содержащий несколько значений. Самый простой пример коллекций это список, он содержит несколько строк. Например, нужно выполнять операцию для каждой строки списка. Можно использовать параметрический цикл (см. листинг \ref{colluniloop}) или использовать упрощенное {\bf прохождение коллекций} (см. листинг \ref{collsimple}). Под коллекции имеется в виду следующие типы данных \mintinline{icl}{list}, \mintinline{icl}{set} и \mintinline{icl}{element}.
\begin{sourcecode}
	\captionof{listing}{Прохождение коллекций с помощью параметрического цикла}
	\label{colluniloop}
    \inputminted[linenos]{icl}{../sources/colluniloop.icL}
\end{sourcecode}

\begin{sourcecode}
	\captionof{listing}{Упрощённое прохождение коллекций}
	\label{collsimple}
    \inputminted[linenos]{icl}{../sources/collsimple.icL}
\end{sourcecode}

\subsubsection{\mintinline{icl}{filter} - выборочное прохождение коллекций}

{\bf Конструкция \mintinline{icl}{filter}} позволяет ставить условие, для каких элементов коллекций выполнять тело цикла, для каких нет.

В условия есть специальные переменные: \mintinline{icl}{@} - значение элемента коллекций и \mintinline{icl}{#} - индекс текущей значения.

Фильтр имеет следующий синтаксис -
\begin{minted}{icl}
filter (collection; condition) {
	commands
};
\end{minted}

Фильтровать элементы можно по данных (см. листинг \ref{filterdataex}, выбирается те которые содержат букву а) или по индексам (см. листинг \ref{filterindexex}, выбирается чётные элементы списка).

\begin{sourcecode}
	\captionof{listing}{Фильтрование по данных}
	\label{filterdataex}
    \inputminted[linenos]{icl}{../sources/filterdataex.icL}
\end{sourcecode}

\subsubsection{\mintinline{icl}{range} - частичное прохождение коллекций}

{\bf Частичное прохождение} позволяет выполнить код для интервала коллекций. Интервал определяется одной или двумя условий. Первое условие используется для поиска начала интервала. Второе условие - для поиска конца, если оно отсутствует, будет использоваться первое условие. В отсутствия подходящих элементов, тела цикла не будет выполнена. Чтобы выбирать первый или последний элемент (в зависимость от предназначения условии), использует выражение \true{}.

Синтаксис для интервала с одним условием -
\begin{minted}{icl}
range (collection; condition) {
	commands
};
\end{minted}

Синтаксис для интервала с отдельным условием для начало и конца -
\begin{minted}{icl}
range (collection; condition_for_begin; condition_for_end) {
	commands
};
\end{minted}

\begin{sourcecode}
	\captionof{listing}{Фильтрование по индексам}
	\label{filterindexex}
    \inputminted[linenos]{icl}{../sources/filterindexex.icL}
\end{sourcecode}

Выбрать первый и последний элемент можно по значении (см. листинг \ref{bananakiwiex}, интервал от значений \mintinline{icl}{banana} до \mintinline{icl}{kiwi}) или по индексам (см. листинг \ref{seclastex}, интервал от второго до последнего).

\begin{sourcecode}
	\captionof{listing}{Интервал: от banana до kiwi}
	\label{bananakiwiex}
    \inputminted[linenos]{icl}{../sources/bananakiwiex.icL}
\end{sourcecode}

\begin{sourcecode}
	\captionof{listing}{Интервал: от второго до последнего}
	\label{seclastex}
    \inputminted[linenos]{icl}{../sources/seclastex.icL}
\end{sourcecode}

%\newpage
