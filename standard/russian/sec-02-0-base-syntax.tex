% !TeX spellcheck = ru_RU
% !TeX spellcheck = ru_RU
\part{Материал для продвинутых пользователях}

Данная часть предназначено для основных принципов построения языка icL. Рассчитана она на продвинутых пользователях с минимальными знаниями программирования.


\section{Базовый синтаксис}

icL достаточно прост в освоении, вставьте код с листинга \ref{first} в icL и выполняйте его. Первую программу можно уже сохранить в файле с расширением \textit{icL}.

\begin{sourcecode}
	\captionof{listing}{Первая программа}
	\label{first}
	\begin{minted}[linenos]{icl}
	Log.info "Тест!";
	\end{minted}
\end{sourcecode}

В консоль можем увидеть следующий \textbf{вывод программы}:

\begin{minted}{icl}
Тест!
\end{minted}

\subsection{Импорт в icL}

Все \textbf{стандартные библиотеки} встроены в языке, но можно импортировать \textbf{пользовательские}, с помощью:

\begin{icItems}
\item
	\mintinline{icl}{Import.none "путь/к/библиотеке.icL"} — выполнить код, который содержится в файле, ничего не импортировать.
\item
	\mintinline{icl}{Import.functions "путь/к/библиотеке.icL"} — выполнить код и импортировать функции; {\color{red}Важно:} импортированные функции не должны использовать глобальные переменные.
\item
	\mintinline{icl}{Import.all "путь/к/библиотеке.icL"} —  выполнить код, импортировать функции и глобальные переменные;
\item
	\mintinline{icl}{Import.run "путь/к/библиотеке.icL"} — выполнить код в текущем контексте, все функции и глобальные переменные импортируется и экспортируется;
\end{icItems}

\subsection{Токены в icL}

Программа на icL состоит из различных \textbf{токенов} (литералов, семантических конструкций, операторов, разделителях, ключевые слова), а токен может являться ключевым словом, идентификатором, константной, строковым литералом, либо символом. Например следующая команда состоит из четырёх токенов: \mintinline{icl}{Log.info "Привет, мир!";}

Отдельными токенами являются:

\begin{icItems}
\item
	\mintinline{icl}{Log} — идентификатор объекта;
\item
	\mintinline{icl}{.info} — идентификатор метода;
\item
	\mintinline{icl}{"Привет, мир!"} — строковый литерал;
\item
	\mintinline{icl}{;} — разделитель, конец команды.
\end{icItems}

\subsection{Комментарии}

\textbf{Комментарии} - это вспомогательный текст, который помогает понимать код, они полностью игнорируется командного процессора.

\textbf{Комментарии в линии} (\textit{inline}) являются строковым литералом ограниченным специальными кавычками \texttt{`}, как показано на листинге \ref{inlinecomment}.

\begin{sourcecode}
	\captionof{listing}{Комментарий в линии}
	\label{inlinecomment}
    \inputminted[linenos]{icl}{../sources/inlinecomment.icL}
\end{sourcecode}

\textbf{Одиночный комментарий} записывается с использованием символов \texttt{``} в начале, смотрите листинг \ref{linecomment}.

\begin{sourcecode}
	\captionof{listing}{Одиночный комментарий}
	\label{linecomment}
    \inputminted[linenos]{icl}{../sources/linecomment.icL}
\end{sourcecode}

\textbf{Многострочный комментарий} начинается и заканчивается с \texttt{```}, пример многострочного комментария приведён на листинге \ref{multilinecomment}.

\begin{sourcecode}
	\captionof{listing}{Многострочный комментарий}
	\label{multilinecomment}
    \inputminted[linenos]{icl}{../sources/multilinecomment.icL}
\end{sourcecode}

\subsection{Идентификторы}

\textbf{Идентификатор} в icL — это имя, используемое для идентификации переменной, функций, методов и свойств. Идентификатор начинается с символов обозначающий его предназначение(\mintinline{icl}{@}, \mintinline{icl}{#}, \mintinline{icl}{.}, \mintinline{icl}{'} или с буквой верхнего или нижнего регистра), за которым следует от 0 до 32 букв(английского или национального алфавита), цифр (от 0 до 9) или нижнее подчёркивание.

icL — чувствительный к регистру язык. Таким образом \mintinline{icl}{@var} и \mintinline{icl}{@Var} являются двумя разными идентификаторами. Вот несколько примеров допустимых идентификаторов:

\begin{minted}{icl}
#loop      Tab        .append  'length  Doc   @i   @VAR
@variable  sumPoints  #global  .merge   .get  #01  sin
\end{minted}

\subsection{Ключевые слова}

В icL \textbf{ключевые слова} зарезервированные. Их всего 18: \mintinline{icl}{if}, \mintinline{icl}{else} \mintinline{icl}{for}, \mintinline{icl}{filter}, \mintinline{icl}{range}, \mintinline{icl}{exists}, \mintinline{icl}{while}, \mintinline{icl}{do}, \mintinline{icl}{any}, \mintinline{icl}{emit}, \mintinline{icl}{emiter}, \mintinline{icl}{slot}, \mintinline{icl}{jammer}, \mintinline{icl}{listen}, \mintinline{icl}{wait}, \mintinline{icl}{switch}, \mintinline{icl}{case} и \mintinline{icl}{assert}. В данном документе они выделены синим цветом.

\subsection{Пробельные символы и разделители}

\textbf{Пробельный символ} (\textit{whitespace}) — этот термин используется в icL для описания пробелов, символов табуляции, символов новой строкой и комментариев. Пробельные символы не обязательные, они используются для улучшения читабельности кода. На листинге \ref{unreadable} показан пример кода без пробельных символов, а на листинге \ref{readable} с пробельными символами.

\begin{sourcecode}
	\captionof{listing}{Koд без пробельных символов}
	\label{unreadable}
    \inputminted[linenos]{icl}{../sources/unreadable.icL}
\end{sourcecode}

В icL присутствует \textbf{разделитель команд} \mintinline{icl}{;}, его можно пропустить, но не рекомендуется. Команда — это набор токенов, распределенных в определённом порядке и характеризующее действие. Примеры команд: открыть сайт — \mintinline{icl}{Tab.get "URL"}, закрыть вкладку — \mintinline{icl}{Tab.close}.

При описании последовательности действий, их надо разделить, например последовательность вышеперечисленных команд описывается так:
\begin{minted}{icl}
Tab.get "URL"; Tab.close()
\end{minted}

Таким образом из команд собираем сценарии.

\begin{sourcecode}
	\captionof{listing}{Koд с пробельных символов}
	\label{readable}
    \inputminted[linenos]{icl}{../sources/readable.icL}
\end{sourcecode}
