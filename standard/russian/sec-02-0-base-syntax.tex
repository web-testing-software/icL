% !TeX spellcheck = ru_RU
\section{Базовый синтаксис}

icL достаточно прост в освоении, вставите код с листинга \ref{first} в icL и выполняете его. Первую программу можно уже сохранить в файле с расширением icl.

\begin{lstlisting}[caption=Первая программа, label=first]
Log.info "Test!";
\end{lstlisting}

В консоль можем увидеть следующий \textbf{вывод программы}:

\begin{lstlisting}[numbers=none]
Test!
\end{lstlisting}

\subsection{Импорт в icL}

Все \textbf{стандартные библиотеки} встроены в языке, но можно написать импортировать \textbf{пользовательские}, с помощью:

\begin{icItems}
\item
	\lstinline|Import.none "path/to/file.iclib"| - выполнить код, который содержатся в файле, ничего не импортировать.
\item
	\lstinline|Import.functions "path/to/file.iclib"| - выполнить код и импортировать функций; {\color{red}Важно:} импортированные функций не должны использовать глобальные переменные.
\item
	\lstinline|Import.all "path/to/file.iclib"| -  выполнить код, импортировать функций и глобальные переменные;
\item
	\lstinline|Import.run "path/to/file.iclib"| - выполнить код в текущем контексте, все функции и глобальные переменные импортируется и экспортируется;
\end{icItems}

\subsection{Токены в icL}

Программа на icL состоит из различных \textbf{токенов} (литералов, семантических конструкциях), а токен может являться ключевым словом, идентификатором, константной, строковым литералом, либо символом. Например следующая команда состоит из четырёх токенов: \lstinline`Log.info "Hello world!";`

Отдельными токенами являются:

\begin{icItems}
\item
	\lstinline`Log` - идентификатор объекта;
\item
	\lstinline`.info` - идентификатор метода;
\item
	\lstinline`"Hello world!"` - строковый литерал;
\item
	\lstinline`;` - разделитель, конец команды.
\end{icItems}

\subsection{Комментарии}

\textbf{Комментарии} - это вспомогательный текст, который помогает понимать написанных сценариях, они полностью игнорируется командного процессора.

\textbf{Комментарии в линии} (\textit{inline}) являются строковым литералом ограниченным специальными кавычками \texttt{`}, как показано на листинге \ref{inlinecomment}.

\begin{lstlisting}[caption=Комментарий в линии,label=inlinecomment]
No comment `comment` no comment
\end{lstlisting}

\textbf{Одиночный комментарий} записывается с использованием символов \texttt{``} в начале, смотрите листинг \ref{linecomment}.

\begin{lstlisting}[caption=Одиночный комментарий,label=linecomment]
No comment `` comment
\end{lstlisting}

\textbf{Многострочный комментарий} начинается и заканчивается с \texttt{```}, пример многострочного комментария приведён на листинге \ref{multilinecomment}.

\begin{lstlisting}[caption=Многострочный комментарий,label=multilinecomment]
No comment
``` comment 1
	comment 2
	comment 3
``` No comment
\end{lstlisting}

\subsection{Идентификторы}

\textbf{Идентификатор} в icL - это имя, используемое для идентификации переменной, функций, методов и свойств. Идентификатор начинается с символов обозначающий его предназначение(\lstinline`@`, \lstinline`#`, \lstinline`!`, \lstinline`.`, \lstinline`'` или с буквой верхнего или нижнего регистра), за которым следует от 0 до 32 букв(английского или национального алфавита) и цифр (от 0 до 9).

icL - чувствительный к регистру язык. Таким образом \textit{@var} и \textit{@Var} являются двумя разными идентификаторами. Вот несколько примеров допустимых идентификаторов:

\begin{lstlisting}[numbers=none]
#loop		Tab		    .append		'length	DOM	    @i	 	@VAR
@variable	sumPoints	#global		.merge	.get	#01		sin
\end{lstlisting}

\subsection{Ключевые слова}

В icL \textbf{ключевые слова} не зарезервированные. Их всего 15: \lstinline`if`, \lstinline|else| \lstinline`for`, \lstinline`filter`, \lstinline`range`, \lstinline`exists`, \lstinline`while`, \lstinline`do`, \lstinline`any`, \lstinline`emit`, \lstinline`emiter`, \lstinline`slot`, \lstinline|jammer|, \lstinline|reverse| и \lstinline|assert|. В данном документе они выделены синим цветом.

\subsection{Пробельные символы и разделители}

\textbf{Пробельный символ} (\textit{whitespace}) - этот термин используется в icL для описания пробелов, символов табуляции, символов новой строкой и комментариев. Пробельные символы не обязательные, они используются для улучшения читабельности кода. На листинге \ref{unreadable} показан пример кода без пробельных символов, а на листинге \ref{readable} с пробельными символами.

\begin{lstlisting}[caption=Koд без пробельных символов,label=unreadable]
if(Tab.get"mai.ru"){(DOM.query"button").click;}else{Log.error"The site mai.ru is unaviable";};
\end{lstlisting}

В icL присутствует только один разделитель - \textbf{разделитель команд} \lstinline`;`. Команда - это набор токенов, распределенных в определённом порядке и характеризующее действие. Примеры команд: открыть сайт - \lstinline`Tab.get "URL"`, закрыть вкладку - \lstinline`Tab.close`.

При описании последовательности действий, их надо разделить, например последовательность вышеперечисленных команд описывается так:

\begin{lstlisting}[numbers=none]
Tab.get "URL"; Tab.close
\end{lstlisting}

\begin{lstlisting}[caption=Koд с пробельных символов,label=readable]
`` the begin of program

`` try to go to mai.ru
if (Tab.get "mai.ru") {
	`` site loaded successfull
	`` click the button
	DOM.query("button").click;
}
else {
	`` try again later
	`` now log the error
	Log.error "The site mai.ru is unaviable";
};

`` end of the program
\end{lstlisting}

Таким образом из команд собираем сценарии. Перед закрывающих скобок ставить разделитель команд опционально.