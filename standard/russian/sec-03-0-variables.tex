\section{Переменные}

\textbf{Переменная} - название области хранения, который могут манипулировать сценария. Каждая переменная в icL имеет область видимость (фрагмент кода где можно её использовать) и тип, который определяет размер и способ размещения памяти переменной; диапазон значений можно применить к переменной.

Имя переменной является идентификатором, начинающийся с \lstinline`@` или {\color{blue2}\lstinline`#`}.
\textbf{Основные типы} переменных показаны в таблице \ref{variablestypes}.

\stabletwo{2cm}{15.1cm}{variablestypes}{Типы переменных}%
{Тип}{Описание}%
{
	bool   & Логическое значения, имеет в состояние: истинно или лож. 					 \\ \hline
	int    & Целое число, позволяет хранить значения от -2.147.483.648 до 2.147.483.647. \\ \hline
	double & Дробное число, позволяет хранить вещественные числа. 						 \\
}

icL также присутствуют и \textbf{сложные типы} переменных, такие как строки, списки, множества, объекты, которые мы посмотрим в полеживающих главах. В этой главе изучаем только основные типы.

\subsection{Объявления и инициализация переменных}

\textbf{Объявления и инициализация} переменных, также операция \textit{присваивание} в icL неотличимы, и имеют общий вид \lstinline`a = b`, где \lstinline`a` новая или существующая переменная, а \lstinline`b` - значение. В тех случаев когда переменная \textbf{встречается первый раз}, мы её объявляем и инициализируем. В противном случае мы присваиваем ей значение.

На листинге \ref{initexample} показаны несколько пример объявлений и инициализации переменных, обратите внимание что дробные числа пишется через точку, а не через запятую как принято в Европе и Российском Федераций.

\begin{lstlisting}[caption=Пример объявлений и инициализации переменных,label=initexample]
@bool = false;
@catched = true;
@int = 234;
@double = 23.4;
#pi = 3.14;
#negative = -100.0;
\end{lstlisting}

\subsection{Локальные переменные}

\textbf{Локальные переменные} имею узкую область видимость, ограниченные фигурными скобками которые их охватывает, и только после их объявлений.

{\bf Идентификаторы} локальных переменных начинается с символом \lstinline`@`.

На листинге \ref{localvars} показана область видимость переменной \lstinline`@var`, в точках объявления переменных \lstinline`@test1`, \lstinline`@test2` и \lstinline`@test6` - она не видна, когда в точках объявления переменных \lstinline`@test3`, \lstinline`@test4` и \lstinline`@test5` - да.
\begin{lstlisting}[caption=Область видимости локальных перемен, label=localvars]
`` error
@test1 = @var;
if (@) {
	`` error
	@test2 = @var;
	`` initialization
	@var = 0;
	`` ok
	@test3 = @var;
	if (@) {
		`` ok
		@test4 = @var;
	}
	`` ok
	@test5 = @var;
}
`` error
@test6 = @var;
\end{lstlisting}

\subsection{Глобальные переменные}

\textbf{Глобальные переменные} имеют самую широкую область видимости, их видны из любой точке программы после их инициализации. Использовать глобальные переменные не рекомендуется, так как они могут привести к серьезных ошибок.

{\bf Идентификаторы} глобальных переменных начинается с символом {\color{blue2}\lstinline`#`}. Локальные переменные с одинаковым названием могут быть несколько, когда глобальные - нет. Идентификатор глобальной переменной - уникальный.

Как указано на листинге \ref{globalvars}, в точках объявления переменных \lstinline`@test1`, \lstinline`@test2` и \lstinline`@test3` переменная \lstinline`@var` не видна, когда в точках объявления переменных \lstinline`@test4`, \lstinline`@test5` и \lstinline`@test6` - да.

\

\begin{lstlisting}[caption=Область видимости глобальных перемен, label=globalvars]
`` error
@test1 = @var;
if (@) {
	`` error
	@test2 = @var;
	if (@) {
		`` error
		@test3 = @var;
		`` initialization
		@var = 0;
		`` ok
		@test4 = @var;
	}
	`` ok
	@test5 = @var;
}
`` ok
@test6 = @var;
\end{lstlisting}

\subsection{Левые и правые значения в icL}

В icL присутствуют 3 типа значении:

\begin{icEnum}
\item
	левые значения ({\it lvalue}) - переменные;
\item
	правые значения ({\it rvalue}) - переменные и константы;
\item
	javascript значения ({\it jsvalue}) - их будем рассматривать позже.
\end{icEnum}

Левые значения могут находиться с обеих сторон знака {\it присваивание}, когда правые - только справа. Примеры правильно и неправильно кода иллюстрированы на листинге \ref{rlvalues};

\begin{lstlisting}[caption=Левые и правые значения, label=rlvalues]
@a1 = @a2; @a3 = 123; `` ok
123 = @a1; 125 = 456; `` error
\end{lstlisting}

\subsection{Выводы}

{\bf Работа с переменными в icL} - максимально проста, но продвинутых пользователях без знания в программировании рекомендуется не использовать глобальные переменные. Для написания сценариев средней и низкой сложности, локальные переменные лучше подходят.
