% !TeX spellcheck = ru_RU
\section{Переменные}

\textbf{Переменная} — это пара "имя" – "значения". Каждая переменная в icL имеет область видимость (фрагмент кода где можно её использовать) и тип, который определяет диапазон и способ размещения значении переменной.

Имя переменной является идентификатором, начинающийся с \mintinline{icl}{@} или {\color{blue2}\mintinline{icl}{#}}.

\subsection{Объявления и инициализация переменных}

\textbf{Объявления и инициализация} переменных реализуется путём определений пар "имя" — "значение", имя от значений разделяется токеном \mintinline{icl}{=} и имеют общий вид \mintinline{icl}{имя = значение}.

На листинге \ref{initexample} показаны несколько пример объявлений и инициализации переменных, обратите внимание что дробные числа пишется через точку, а не через запятую как принято в Европе и Российском Федераций.

\begin{sourcecode}
	\captionof{listing}{Пример объявлений и инициализации переменных}
	\label{initexample}
    \inputminted[linenos]{icl}{../sources/initexample.icL}
\end{sourcecode}

\subsection{Локальные переменные}

\textbf{Локальные переменные} имеют узкую область видимости, ограниченную фигурными скобками которые их охватывает, и только после их объявления.

{\bf Идентификаторы} локальных переменных начинается с символом \mintinline{icl}{@}.

На листинге \ref{localvars} показана область видимость переменной \mintinline{icl}{@var}, в точках объявления переменных \mintinline{icl}{@test1}, \mintinline{icl}{@test2} и \mintinline{icl}{@test6} — она не видна, когда в точках объявления переменных \mintinline{icl}{@test3}, \mintinline{icl}{@test4} и \mintinline{icl}{@test5} — да.
\begin{sourcecode}
	\captionof{listing}{Область видимости локальных перемен}
	\label{localvars}
    \inputminted[linenos]{icl}{../sources/localvars.icL}
\end{sourcecode}

\subsection{Глобальные переменные}

\textbf{Глобальные переменные} имеют самую широкую область видимости, их видны из любой точки программы после их инициализации. Использовать глобальные переменные не рекомендуется, так как они могут привести к серьезных ошибок.

{\bf Идентификаторы} глобальных переменных начинается с символом {\color{blue2}\mintinline{icl}{#}}. Локальные переменные с одинаковым названием могут быть несколько, когда глобальные — нет. Идентификатор глобальной переменной — уникальный.

Как указано на листинге \ref{globalvars}, в точках объявления переменных \mintinline{icl}{@test1}, \mintinline{icl}{@test2} и \mintinline{icl}{@test3} переменная \mintinline{icl}{@var} не видна, когда в точках объявления переменных \mintinline{icl}{@test4}, \mintinline{icl}{@test5} и \mintinline{icl}{@test6} — да.

\

\begin{sourcecode}
	\captionof{listing}{Область видимости глобальных перемен}
	\label{globalvars}
	\inputminted[linenos]{icl}{../sources/globalvars.icL}
\end{sourcecode}

\subsection{Левые и правые значения в icL}

В icL присутствуют 3 типа значении:

\begin{icEnum}
\item
	левые значения ({\it lvalue}) — переменные;
\item
	правые значения ({\it rvalue}) — переменные и константы;
\item
	javascript значения ({\it jsvalue}) — их будем рассматривать позже.
\end{icEnum}

Левые значения могут находиться с обеих сторон знака {\it присваивание}, когда правые — только справа. Примеры правильно и неправильно кода иллюстрированы на листинге \ref{rlvalues};

\begin{sourcecode}
	\captionof{listing}{Левые и правые значения}
	\label{rlvalues}
	\inputminted[linenos]{icl}{../sources/rlvalues.icL}
\end{sourcecode}

\subsection{Запакованные значения}

{\bf Запакованные значения} используются для специальных операторов. Запаковать значения можно изолируя их в круглых скобок и разделяя запятой, самый простой пример использования запакованных значения это замена значения переменных — \mintinline{icl}{(@a, @b) = (@b.ensureRValue, @a.ensureRValue)}. Значения привязано к переменной, по этому мы должны преобразовать в правые значения (они независимые).

\subsection{Переменные @ и \#}

{Переменная \mintinline{icl}{@}} — акроним для \mintinline{icl}{@stack} и \mintinline{icl}{Stack'stack}. Системная переменная которая служит, для того чтобы передать данные с родительского стекового контейнера дочернего и наоборот. 

{Переменная \mintinline{icl}{#}} — акроним для \mintinline{icl}{@console}, результат выполнения команды сохраняется в данной переменной и возвращается в консоль или передаётся следующей команде.

\subsection{Выводы}

{\bf Работа с переменными в icL} — максимально проста, но продвинутых пользователях без знания в программировании рекомендуется не использовать глобальные переменные. Для написания сценариев средней и низкой сложности, локальные переменные лучше подходят.
