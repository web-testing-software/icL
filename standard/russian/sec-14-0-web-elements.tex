% !TeX spellcheck = ru_RU
\section{Веб-элементы}
\label{webelments}

{\bf Веб-элементы} (тип данных \element) представляет собой ссылки на HTML-теги веб-страницы.

Элемент считается единственным если он получен с помощью функций \lstinline|Doc.query| или \lstinline|element.query|. Элемент считается коллекцией если он получен с помощью функций \lstinline|Doc.queryAll| или \lstinline|element.queryAll|. При получении элемента по индексу, результат также будет единственным.

\subsection{Литерал}

Веб-элементы могут быть описаны литералом \lstinline|method@target[selector]|, где \lstinline|method| - одно из следующих фиксируемых слов: \lstinline|css|, \lstinline|xpath|, \lstinline|link|, \lstinline|links|, \lstinline|tag|, \lstinline|tags|. \lstinline|@target| - локальная переменная содержащая веб-элемент. \lstinline|selector| - селектор, набор последовательных символов, по которому определяется какие элементы будут выбраны, для каждого метода нужен специфический селектор.

Метод \lstinline|css| получает в качестве селектора - селектор CSS. Данный метод будет возвращать первый подходящий элемент. Чтобы получить все используете \lstinline|css:all|, где \lstinline|all| - модификатор, подробное информация о модификаторах представлена в главе \ref{sec-modifiers}.

Метод \lstinline|xpath| получает в качестве селектора - путь XPath. Данный метод будет возвращать первый подходящий элемент. Чтобы получить все используете \lstinline|xpath:all|, где \lstinline|all| - модификатор, подробное информация о модификаторах представлена в главе \ref{sec-modifiers}.

Метод \lstinline|link| получает в качестве селектора текст ссылки и возвращает соответствующую ссылку.

Метод \lstinline|links| получает в качестве селектора текст ссылки и возвращает все подходящие ссылки.

Метод \lstinline|tag| получает в качестве селектора название тега и возвращает первый элемент соответствующего тега.

Метод \lstinline|tags| получает в качестве селектора название тега и возвращает все элементы соответствующего тега.

Примеры литералов -
\begin{lstlisting}[numbers=none]
@div = tag[div];
@a = tag@div[a];
`` @a == css[div a]
`` @a != css[div > a]

css[#content > div:nth-of-type(3)]
css[header ul > li:nth-child(even)]
\end{lstlisting}

 
\subsection{Свойства}

Веб-элементы имеют следующее свойства:
\begin{icItems}
\item \lstinline|[r/*] element'attr-* : string|;
\item \lstinline|[r/*] element'css-* : string|;
\item \lstinline|[r/o] element'empty : bool|;
\item \lstinline|[r/o] element'enabled : bool|;
\item \lstinline|[r/o] element'length : int|;
\item \lstinline|[*/*] element'prop-* : any|;
\item \lstinline|[r/o] element'rect : object|;
\item \lstinline|[r/o] element'selected : bool|;
\item \lstinline|[r/o] element'tag : string|;
\item \lstinline|[r/o] element'text : string|;
\item \lstinline|[r/o] element'(<int> n) : element|;
\end{icItems} 

\subsubsection{\lstinline|[r/*] element'attr-* : string|}

\code{[w3c]} Возвращает значения атрибута \code{*}.

\code{[icL]} Возвращает JS-значение для запрашиваемого атрибута.

Возможные исключения: \ferror{NoSessions}, \ferror{EmptyElement}, \ferror{MultiElement}, \ferror{NoSuchWindow} и \ferror{StaleElementReference} (см. таб. \ref{errors}).

\subsubsection{\lstinline|[r/*] element'css-* : string|}

\code{[w3c]} Возвращает значения свойства CSS \code{*}.

\code{[icL]} Возвращает JS-значение для запрашиваемой свойства CSS.

Возможные исключения: \ferror{NoSessions}, \ferror{EmptyElement}, \ferror{MultiElement}, \ferror{NoSuchWindow} и \ferror{StaleElementReference} (см. таб. \ref{errors}).

\subsubsection{\lstinline|[r/o] element'empty : bool|}

Возвращает \true, если коллекция не содержит ни одного элемента, иначе \false.

\subsubsection{\lstinline|[r/o] element'enabled : bool|}

Возвращает \false, если элемент является элементом формы и он отключен, иначе \true. 

Возможные исключения: \ferror{NoSessions}, \ferror{EmptyElement}, \ferror{MultiElement}, \ferror{NoSuchWindow} и \ferror{StaleElementReference} (см. таб. \ref{errors}).

\subsubsection{\lstinline|[r/o] element'length : int|}

Возвращает количество элементов в коллекции.

\subsubsection{\lstinline|[*/*] element'prop-* : string|}

\code{[w3c]} Возвращает значения свойства \code{*}.

\code{[icL]}  Возвращает JS-значение для запрашиваемой свойства. Тип JS-значений не определён, это можно сводить к проблемам в динамическом анализаторе кода. При использовании свойств неопределённых в главе \ref{elements:predefined:properties}, используете преобразование для указания какого типа будет результат. Если переменная уже инициализируемая, то никаких проблем не должны быть. Новых перемен инициализируете одним из следующим способом -
\begin{lstlisting}[numbers=none]
@var = element'prop-userdefined : string;
@var:string = element'prop-userdefined;
\end{lstlisting}

Возможные исключения: \ferror{NoSessions}, \ferror{EmptyElement}, \ferror{MultiElement}, \ferror{NoSuchWindow} и \ferror{StaleElementReference} (см. таб. \ref{errors}).

\subsubsection{\lstinline|[r/o] element'rect : obj|}

Возвращает положение размер элемента на экране в CSS пикселей, а точнее объект содержащий следующее поля:
\begin{icItems}
	\item \code{x : double} - координата x относительно левого края;
	\item \code{y : double} - координата y относительно верхнего края;
	\item \code{width : double} - ширина;
	\item \code{height : double} - высота;
\end{icItems}

\code{[icL]} Возвращает \set, в случае коллекции.

Возможные исключения: \ferror{NoSessions}, \ferror{EmptyElement}, \ferror{MultiElement}, \ferror{NoSuchWindow} и \ferror{StaleElementReference} (см. таб. \ref{errors}).

\subsubsection{\lstinline|[r/o] element'selected : bool|}

Возвращает \true, если элемент является включённым чекбоксом, отмеченной радио кнопкой или выбранной опцией, иначе \false.

Возможные исключения: \ferror{NoSessions}, \ferror{EmptyElement}, \ferror{MultiElement}, \ferror{NoSuchWindow} и \ferror{StaleElementReference} (см. таб. \ref{errors}).

\subsubsection{\lstinline|[r/o] element'tag : string|}

Возвращает название тега.

\code{[icL]} Возвращает \listtype, в случае коллекции.

Возможные исключения: \ferror{NoSessions}, \ferror{EmptyElement}, \ferror{MultiElement}, \ferror{NoSuchWindow} и \ferror{StaleElementReference} (см. таб. \ref{errors}).

\subsubsection{\lstinline|[r/o] element'text : string|}

Возвращает текст элемента видимый на экран. 

\code{[icL]} Возвращает \listtype, в случае коллекции.

Возможные исключения: \ferror{NoSessions}, \ferror{EmptyElement} и \ferror{MultiElement}, \ferror{NoSuchWindow} и \ferror{StaleElementReference} (см. таб. \ref{errors}).

\subsubsection{\lstinline|[r/o] element'(<int> n) : element|}

Возвращает новую единый элемент содержащий n-й элемент.

Возможные исключения: \ferror{OutOfBounds}, \ferror{StaleElementReference} (см. таб. \ref{errors}).

\subsection{Операторы}

В контексте веб-элементов появляются новый оператор \lstinline|[element ...] : element|.

\subsubsection{\lstinline|[element ...] : element|}

Возвращает новую коллекцию, содержащая всех представленных элементов.

\subsection{Методы}

{\bf Методы} класса \code{element} делится на 2 категории: базовые и расширенные. Базовые методы определены стандартом W3C WebDriver. Расширенные методы определяются стандартом языка icL и могут быть изменены в следующих версиях языка.

\subsection{Базовые методы}

Перечень базовых методов:
\begin{icItems}
    \item \lstinline|element.clear () : element|;
	\item \lstinline|element.click () : element|;
	\item \lstinline|element.query (cssSelector : string) : element|;
	\item \lstinline|element.query (by : int, selector : string) : element|;
	\item \lstinline|element.queryAll (cssSelector : string) : element|;
	\item \lstinline|element.queryAll (by : int, selector : string) : element|;
	\item \lstinline|element.queryAllByXPath (xpath : string) : element|;
	\item \lstinline|element.queryByXPath (xpath : string) : element|;
	\item \lstinline|element.queryLink (name : string, isFragment = false) : element|;
	\item \lstinline|element.queryLinks (name : string, isFragment = false) : element|;
	\item \lstinline|element.queryTag (name : string) : element|;
	\item \lstinline|element.queryTags (name : string) : element|;
	\item \lstinline|element.screenshot () : string|;
	\item \lstinline|element.sendKeys (modifiers : int, text : string) : element|;
\end{icItems}

\subsubsection{\lstinline|element.clear () : element|}

Если элемент является полем ввода данных, то ого значение очищается. Если элемент редактируемый, то его свойство \code{innerHtml} присваивается пустая строка.

\code{[icL]} Очищены будут все элементы контейнера.

Возможные исключения: \ferror{NoSessions}, \ferror{InvalidArrgument}, \ferror{EmptyElement}, \ferror{MultiElement}, \ferror{NoSuchWindow}, \ferror{StaleElementReference}, \ferror{InvalidElementState} (см. таб. \ref{errors}).

\subsubsection{\lstinline|element.click () : element|}

Симулирует клик по центре элемента, и ждёт загрузка страницы.

\code{[icL]} Клик будет симулирован на все элементы контейнера.

Возможные исключения: \ferror{NoSessions}, \ferror{EmptyElement}, \ferror{MultiElement}, \ferror{NoSuchWindow}, \ferror{InvalidElement}, \ferror{ElementNotInteractable}, \ferror{ElementClickIntercepted} и \ferror{StaleElementReference} (см. таб. \ref{errors}).

\subsubsection{\lstinline|element.query (by = By'cssSelector, selector : string) : element|}

Параметр \code{by} получает одно из следующих значений:
\begin{icItems}
	\item \lstinline|[r/o] By'cssSelector : 1| - селектор CSS;
	\item \lstinline|[r/o] By'linkText : 2| - текст ссылки;
	\item \lstinline|[r/o] By'partialLinkText : 3| - фрагмент текста ссылки;
	\item \lstinline|[r/o] By'tagName : 4| - название тега;
	\item \lstinline|[r/o] By'xPath : 5| - XPath.
\end{icItems}

Параметр \code{selector} получает селектор CSS, текст ссылки, фрагмент текста ссылки, название тега или XPath, в зависимость от значений первого параметра.

Метод возвращает новый \element{} содержащий первый элемент найден по нужному методу в текущем элементе.

\code{[icL]} Поиск ведётся во всех элементов контейнера.

Возможные исключения: \ferror{NoSessions}, \ferror{EmptyElement}, \ferror{ElementNotFound}, \ferror{MultiElement}, \ferror{NoSuchWindow}, \ferror{NoSuchElement}, \ferror{InvalidSelector}, \ferror{StaleElementReference} (см. таб. \ref{errors}).

\subsubsection{\lstinline|element.queryAll (by = By'cssSelector, selector : string) : element|}

Параметры получают такие же значения, как и в случае с \code{element.query}, только данный метод возвращает коллекцию найденных элементов или пустую коллекцию если ничего не нашлось.

\code{[icL]} Поиск ведётся во всех элементов контейнера.

Возможные исключения: \ferror{NoSessions}, \ferror{EmptyElement}, \ferror{MultiElement}, \ferror{NoSuchWindow} и \ferror{StaleElementReference} (см. таб. \ref{errors}).

\subsubsection{\lstinline|element.queryAllByXPath (xpath : string) : element|}

Акроним для \lstinline|element.queryAll (By'xPath, @xpath)|;

\subsubsection{\lstinline|element.queryByXPath (xpath : string) : element|}

Акроним для \lstinline|element.query (By'xPath, @xpath)|;

\subsubsection{\lstinline|element.queryLink (name : string, isFragment = false) : element|}

Акроним для:
\begin{icItems}
	\item \lstinline|element.query (By'linkText, @name)|;
	\item \lstinline|element.query (By'partialLinkText, @name)|.
\end{icItems}

\subsubsection{\lstinline|element.queryLinks (name : string, isFragment = false) : element|}

Акроним для:
\begin{icItems}
	\item \lstinline|element.queryAll (By'linkText, @name)|;
	\item \lstinline|element.queryAll (By'partialLinkText, @name)|.
\end{icItems}

\subsubsection{\lstinline|element.queryTag (name : string) : element|}

Акроним для \lstinline|element.query (By'tagName, @name)|;

\subsubsection{\lstinline|element.queryTags (name : string) : element|}

Акроним для \lstinline|element.queryAll (By'tagName, @name)|;

\subsubsection{\lstinline|element.screenshot () : string|}

Возвращает строку, содержащая код base64 скриншота элемента. Его можно сохранить как изображение используя \lstinline|Make.image (base64 : string, path : string) : void|.

Возможные исключения: \ferror{NoSessions}, \ferror{EmptyElement}, \ferror{MultiElement}, \ferror{NoSuchWindow} и \ferror{StaleElementReference} (см. таб. \ref{errors}).

\subsubsection{\lstinline|element.sendKeys (modifiers : int, text : string) : element|}

Параметр \code{modifiers} получает один из следующих параметров (или сумма нескольких из них):
\begin{icItems}
	\item \lstinline|[r/o] Key'ctrl : 1| - Control;
	\item \lstinline|[r/o] Key'shift : 2| - Shift;
	\item \lstinline|[r/o] Key'alt : 3| - Alt;
\end{icItems}

Параметр \code{text} получает текст, будущим напечатанным на клавиатуре.

Возможные исключения: \ferror{NoSessions}, \ferror{EmptyElement}, \ferror{MultiElement}, \ferror{ElementNotIntractable} \ferror{NoSuchWindow} и \ferror{StaleElementReference} (см. таб. \ref{errors}).

\subsection{Расширенные методы}

Расширенные методы могут работать медленно в режиме тестирования при использовании внешнего браузера. При содержании нескольких элементов в коллекции, операции принимаются на каждого элемента, и возвращена будет коллекция, а не единый элемент.

Перечень расширенных методов:
\begin{icItems}
    \item \lstinline|element.add (other : element) : element|;
	\item \lstinline|element.child (index : int) : element|;
	\item \lstinline|element.closest (cssSelector : string) : element|;
	\item \lstinline|element.contains (text : string, asFragment = false) : element|;
	\item \lstinline|element.copy () : element|;
	\item \lstinline|element.get (i : int) : element|;
	\item \lstinline|element.filter (cssSelector : string) : element|;
	\item \lstinline|element.next () : element|;
	\item \lstinline|element.prev () : element|;
	\item \lstinline|element.parent () : element|;
\end{icItems}

\subsubsection{\lstinline|element.add (other : element) : element|}

Добавит все элементы коллекции \code{other}.

Возможные исключения: \ferror{NoSessions}, \ferror{NoSuchWindow}, \ferror{StaleElementReference} (см. таб. \ref{errors}).

\subsubsection{\lstinline|element.child (i : int) : element|}

Возвращает новый элемент, содержащая \code{i}-й дочерний элемент.

Возможные исключения: \ferror{NoSessions}, \ferror{NoSuchWindow}, \ferror{StaleElementReference}, \ferror{OutOfBounds} (см. таб. \ref{errors}).

\subsubsection{\lstinline|element.closest (cssSelector : string) : element|}

Возвращает новый элемент, содержащая ближайший родительский элемент, который подходит по селектору \code{cssSelector}.

Возможные исключения: \ferror{NoSessions}, \ferror{NoSuchWindow}, \ferror{StaleElementReference} (см. таб. \ref{errors}).

\subsubsection{\lstinline|element.contains (text : string, asFragment = false) : element|}

Возвращает новая коллекция, содержащая все элементы, которые содержат текст/фрагмент \code{text}.

Возможные исключения: \ferror{NoSessions}, \ferror{NoSuchWindow}, \ferror{StaleElementReference} (см. таб. \ref{errors}).

\subsubsection{\lstinline|element.copy () : element|}

Возвращает новая коллекция, содержащая все элементы текущей коллекций.

Возможные исключения: \ferror{NoSessions}, \ferror{NoSuchWindow}, \ferror{StaleElementReference} (см. таб. \ref{errors}).

\subsubsection{\lstinline|element.get (i : int) : element|}

Возвращает i-й элемент коллекции.

Возможные исключения: \ferror{NoSessions}, \ferror{NoSuchWindow}, \ferror{StaleElementReference}, \ferror{OutOfBounds} (см. таб. \ref{errors}).

\subsubsection{\lstinline|element.filter (cssSelector : string) : element|}

Возвращает новая коллекция, содержащая все элементы, которые подходят по селектору \code{cssSelector}.

Возможные исключения: \ferror{NoSessions}, \ferror{NoSuchWindow}, \ferror{StaleElementReference} (см. таб. \ref{errors}).

\subsubsection{\lstinline|element.next () : element|}

Возвращает новый элемент, содержащая следующий дочерний элемент.

Возможные исключения: \ferror{NoSessions}, \ferror{NoSuchWindow}, \ferror{StaleElementReference} (см. таб. \ref{errors}).

\subsubsection{\lstinline|element.prev () : element|}

Возвращает новый элемент, содержащая предыдущий дочерний элемент.

Возможные исключения: \ferror{NoSessions}, \ferror{NoSuchWindow}, \ferror{StaleElementReference} (см. таб. \ref{errors}).

\subsubsection{\lstinline|element.parent () : element|}

Возвращает новый элемент, содержащая родительский элемент.

Возможные исключения: \ferror{NoSessions}, \ferror{NoSuchWindow}, \ferror{StaleElementReference} (см. таб. \ref{errors}).

\subsection{Предопределённый набор свойств}
\label{elements:predefined:properties}

В icL предопределены только те свойства, которые имеют тип данных присутствующий в icL:
\begin{icItems}
	\item Node:	
	\begin{icItems}
		\item \lstinline|[r/o] element'prop-childNodes : element [c]|;
		\item \lstinline|[r/o] element'prop-firstChild : element [i]|;
		\item \lstinline|[r/o] element'prop-innerText : string|;
		\item \lstinline|[r/o] element'prop-isConnected : bool|;
		\item \lstinline|[r/o] element'prop-lastChild : element [i]|;
		\item \lstinline|[r/o] element'prop-nodeName : string|;
		\item \lstinline|[r/o] element'prop-nodeType : int|;
		\item \lstinline|[r/*] element'prop-nodeValue : string|;
		\item \lstinline|[r/o] element'prop-parentElement : element [i]|;
		\item \lstinline|[r/o] element'prop-textContent : string|;
	\end{icItems}
	
	\item Element:
	\begin{icItems}
		\item \lstinline|[r/*] element'prop-className : string|;
		\item \lstinline|[r/o] element'prop-clientHeight : double|;
		\item \lstinline|[r/o] element'prop-clientLeft : double|;
		\item \lstinline|[r/o] element'prop-clientTop : double|;
		\item \lstinline|[r/o] element'prop-clientWidth : double|;
		\item \lstinline|[r/o] element'prop-computedName : string|;
		\item \lstinline|[r/o] element'prop-computedRole : string|;
		\item \lstinline|[r/*] element'prop-id : string|;
		\item \lstinline|[r/*] element'prop-innerHTML : string|;
		\item \lstinline|[r/o] element'prop-localName : string|;
		\item \lstinline|[r/o] element'prop-nextElementSibling : element [i]|;
		\item \lstinline|[r/*] element'prop-outerHTML : string|;
		\item \lstinline|[r/o] element'prop-prefix : string|;
		\item \lstinline|[r/o] element'prop-previousElementSibling : element [i]|;
		\item \lstinline|[r/*] element'prop-scrollHeight : double|;
		\item \lstinline|[r/*] element'prop-scrollLeft : double|;
		\item \lstinline|[r/*] element'prop-scrollTop : double|;
		\item \lstinline|[r/*] element'prop-scrollWidth : double|;
		\item \lstinline|[r/o] element'prop-tagName : string|;
		\item \lstinline|[r/o] element'prop-baseURI : string|;
	\end{icItems}
	
	\item HTMLElement:
	\begin{icItems}
		\item \lstinline|[r/*] element'prop-accessKey : string|;
		\item \lstinline|[r/o] element'prop-accessKeyLabel : string|;
		\item \lstinline|[r/*] element'prop-contentEditable : string|;
		\item \lstinline|[r/o] element'prop-isContentEditable : bool|;
		\item \lstinline|[r/o] element'prop-dataset : object|;
		\item \lstinline|[r/*] element'prop-dir : string|;
		\item \lstinline|[r/*] element'prop-draggable : bool|;
		\item \lstinline|[r/*] element'prop-hidden : bool|;
		\item \lstinline|[r/*] element'prop-inert : bool|;
		\item \lstinline|[r/*] element'prop-lang : string|;
		\item \lstinline|[r/o] element'prop-offsetHeight : double|;
		\item \lstinline|[r/o] element'prop-offsetLeft : double|;
		\item \lstinline|[r/o] element'prop-offsetParent : element [i]|;
		\item \lstinline|[r/o] element'prop-offsetTop : double|;
		\item \lstinline|[r/o] element'prop-offsetWidth : double|;
		\item \lstinline|[r/*] element'prop-spellcheck : double|;
		\item \lstinline|[r/*] element'prop-title : string|;
	\end{icItems}
	
	\item HTMLAnchorElement:
	\begin{icItems}
		\item \lstinline|[r/*] element'prop-download : string|;
		\item \lstinline|[r/*] element'prop-hash : string|;
		\item \lstinline|[r/*] element'prop-host : string|;
		\item \lstinline|[r/*] element'prop-hostname : string|;
		\item \lstinline|[r/*] element'prop-href : string|;
		\item \lstinline|[r/*] element'prop-hreflang : string|;
		\item \lstinline|[r/*] element'prop-media : string|;
		\item \lstinline|[r/*] element'prop-password : string|;
		\item \lstinline|[r/o] element'prop-origin : string|;
		\item \lstinline|[r/*] element'prop-pathname : string|;
		\item \lstinline|[r/*] element'prop-port : string|;
		\item \lstinline|[r/*] element'prop-protocol : string|;
		\item \lstinline|[r/*] element'prop-rel : string|;
		\item \lstinline|[r/*] element'prop-search : string|;
		\item \lstinline|[r/*] element'prop-target : string|;
		\item \lstinline|[r/*] element'prop-text : string|;
		\item \lstinline|[r/*] element'prop-type : string|;
		\item \lstinline|[r/*] element'prop-username : string|;
	\end{icItems}
	
	\item HTMLAreaElement:
	\begin{icItems}
		\item \lstinline|[r/*] element'prop-alt : string|;
		\item \lstinline|[r/*] element'prop-coords : string|;
	\end{icItems}
	
	\item HTMLButtonElement:
	\begin{icItems}
		\item \lstinline|[r/*] element'prop-autofocus : bool|;
		\item \lstinline|[r/*] element'prop-disabled : bool|;
		\item \lstinline|[r/o] element'prop-form : element [i]|;
		\item \lstinline|[r/*] element'prop-formAction : string|;
		\item \lstinline|[r/*] element'prop-formEnctype : string|;
		\item \lstinline|[r/*] element'prop-formMethod : string|;
		\item \lstinline|[r/*] element'prop-formNoValidate : bool|;
		\item \lstinline|[r/*] element'prop-formTarget : string|;
		\item \lstinline|[r/o] element'prop-labels : element [c]|;
		\item \lstinline|[r/*] element'prop-name : string|;
		\item \lstinline|[r/*] element'prop-value : string|;
		\item \lstinline|[r/o] element'prop-willValidate  : bool|;
	\end{icItems}
	
	\item HTMLCanvasElement:
	\begin{icItems}
		\item \lstinline|[r/*] element'prop-height : int|;
		\item \lstinline|[r/*] element'prop-width : int|;
	\end{icItems}
	
	\item HTMLDataListElement: \lstinline|[r/o] element'prop-options : element [c]|;
	
	\item HTMLFormElement:
	\begin{icItems}
		\item \lstinline|[r/o] element'prop-elements : element [c]|;
		\item \lstinline|[r/o] element'prop-length : int|;
		\item \lstinline|[r/*] element'prop-action : string|;
		\item \lstinline|[r/*] element'prop-encoding : string|;
		\item \lstinline|[r/*] element'prop-enctype : string|;
		\item \lstinline|[r/*] element'prop-acceptCharset : string|;
		\item \lstinline|[r/*] element'prop-autocomplete : string|;
		\item \lstinline|[r/*] element'prop-noValidate : string|;
	\end{icItems}
	
	\item HTMLIFrameElement: \lstinline|[r/*] element'prop-allowPaymentRequest|;
	
	\item HTMLImageElement:
	\begin{icItems}
		\item \lstinline|[r/o] element'prop-complete : bool|;
		\item \lstinline|[r/o] element'prop-crossOrigin : string|;
		\item \lstinline|[r/o] element'prop-isMap : bool|;
		\item \lstinline|[r/o] element'prop-naturalHeight : int|;
		\item \lstinline|[r/o] element'prop-naturalWidth : int|;
		\item \lstinline|[r/*] element'prop-src : string|;
		\item \lstinline|[r/*] element'prop-useMap : string|;
	\end{icItems}
	
	\item HTMLInputElement:
	\begin{icItems}
		\item \lstinline|[r/*] element'prop-accept : string|;
		\item \lstinline|[r/*] element'prop-checked : bool|;
		\item \lstinline|[r/*] element'prop-defaultChecked : bool|;
		\item \lstinline|[r/*] element'prop-defaultValue : string|;
		\item \lstinline|[r/*] element'prop-dirName : string|;
		\item \lstinline|[r/*] element'prop-indeterminate : bool|;
		\item \lstinline|[r/*] element'prop-list : element [i]|;
		\item \lstinline|[r/*] element'prop-min : string|;
		\item \lstinline|[r/*] element'prop-max : string|;
		\item \lstinline|[r/*] element'prop-maxLength : int|;
		\item \lstinline|[r/*] element'prop-multiple : bool|;
		\item \lstinline|[r/*] element'prop-pattern : string|;
		\item \lstinline|[r/*] element'prop-placeholder : string|;
		\item \lstinline|[r/*] element'prop-readOnly : bool|;
		\item \lstinline|[r/*] element'prop-required : bool|;
		\item \lstinline|[r/*] element'prop-selectionStart : int|;
		\item \lstinline|[r/*] element'prop-selectionEnd : int|;
		\item \lstinline|[r/*] element'prop-selectionDirection : string|;
		\item \lstinline|[r/*] element'prop-size : int|;
		\item \lstinline|[r/*] element'prop-step : string|;
		\item \lstinline|[r/o] element'prop-validity : bool|;
		\item \lstinline|[r/o] element'prop-validationMessage : string|;
		\item \lstinline|[r/*] element'prop-valueAsNumber : double|;
	\end{icItems}
	
	\item HTMLLabelElement:
	\begin{icItems}
		\item \lstinline|[r/o] element'prop-control : element [i]|;
		\item \lstinline|[r/*] element'prop-htmlFor : string|;
	\end{icItems}
	
	\item HTMLLinkElement: \lstinline|[r/*] element'prop-as : string|;
	\item HTMLMapElement: \lstinline|[r/o] element'prop-areas : element [c]|;
	
	\item HTMLMediaElement:
	\begin{icItems}
		\item \lstinline|[r/*] element'prop-autoplay : bool|;
		\item \lstinline|[r/*] element'prop-controls : bool|;
		\item \lstinline|[r/o] element'prop-currentSrc : string|;
		\item \lstinline|[r/*] element'prop-currentTime : double|;
		\item \lstinline|[r/*] element'prop-defaultMuted : bool|;
		\item \lstinline|[r/*] element'prop-defaultPlaybackRate : bool|;
		\item \lstinline|[r/*] element'prop-disableRemotePlayback : bool|;
		\item \lstinline|[r/o] element'prop-duration : double|;
		\item \lstinline|[r/o] element'prop-ended : bool|;
		\item \lstinline|[r/*] element'prop-loop : bool|;
		\item \lstinline|[r/*] element'prop-mediaGroup : string|;
		\item \lstinline|[r/*] element'prop-muted : bool|;
		\item \lstinline|[r/o] element'prop-networkState : int|;
		\item \lstinline|[r/o] element'prop-paused : bool|;
		\item \lstinline|[r/*] element'prop-playbackRate : double|;
		\item \lstinline|[r/*] element'prop-preload : string|;
		\item \lstinline|[r/o] element'prop-readyState : int|;
		\item \lstinline|[r/o] element'prop-seeking : bool|;
		\item \lstinline|[r/*] element'prop-volume : double|;
	\end{icItems}
	
	\item HTMLMetaElement:
	\begin{icItems}
		\item \lstinline|[r/*] element'prop-content : string|;
		\item \lstinline|[r/*] element'prop-httpEquiv : string|;
	\end{icItems}
	
	\item HTMLMeterElement:
	\begin{icItems}
		\item \lstinline|[r/*] element'prop-high : double|;
		\item \lstinline|[r/*] element'prop-low : double|;
	\end{icItems}
	
	\item HTMLModElement: \lstinline|[r/*] element'prop-cite : string|;
	
	\item HTMLOListElement:
	\begin{icItems}
		\item \lstinline|[r/*] element'prop-reversed : bool|;
		\item \lstinline|[r/*] element'prop-start : int|;
	\end{icItems}
	
	\item HTMLOptionElement:
	\begin{icItems}
		\item \lstinline|[r/*] element'prop-defaultSelected : bool|;
		\item \lstinline|[r/o] element'prop-index : int|;
		\item \lstinline|[r/*] element'prop-label : string|;
		\item \lstinline|[r/*] element'prop-selected : bool|;
	\end{icItems}
	
	\item HTMLProgressElement: \lstinline|[r/o] element'prop-position : double|;
	
	\item HTMLScriptElement:
	\begin{icItems}
		\item \lstinline|[r/*] element'prop-charset : string|;
		\item \lstinline|[r/*] element'prop-async : bool|;
		\item \lstinline|[r/*] element'prop-defer : bool|;
		\item \lstinline|[r/*] element'prop-noModule : bool|;
	\end{icItems}
	
	\item HTMLSelectElement:
	\begin{icItems}
		\item \lstinline|[r/*] element'prop-selectedIndex : int|;
		\item \lstinline|[r/o] element'prop-selectedOptions : element [c]|;
	\end{icItems}
	
	\item HTMLTableCellElement:
	\begin{icItems}
		\item \lstinline|[r/*] element'prop-abbr : string|;
		\item \lstinline|[r/o] element'prop-cellIndex : int|;
		\item \lstinline|[r/*] element'prop-colSpan : int|;
		\item \lstinline|[r/*] element'prop-rowSpan : int|;
		\item \lstinline|[r/*] element'prop-scope : string|;
	\end{icItems}
	
	\item HTMLTableColElement: \lstinline|[r/*] element'prop-span : int|;
	
	\item HTMLTableElement:
	\begin{icItems}
		\item \lstinline|[r/o] element'prop-caption : element [i]|;
		\item \lstinline|[r/o] element'prop-tBodies : element [c]|;
		\item \lstinline|[r/o] element'prop-tHead : element [i]|;
		\item \lstinline|[r/o] element'prop-tFoot : element [i]|;
	\end{icItems}
	
	\item HTMLTableRowElement:
	\begin{icItems}
		\item \lstinline|[r/o] element'prop-cells : element [c]|;
		\item \lstinline|[r/o] element'prop-rowIndex : int|;
	\end{icItems}
	
	\item HTMLTextAreaElement:
	\begin{icItems}
		\item \lstinline|[r/*] element'prop-cols : int|;
		\item \lstinline|[r/o] element'prop-textLength : int|;
		\item \lstinline|[r/*] element'prop-wrap : string|;
	\end{icItems}
	
	\item HTMLTimeElement: \lstinline|[r/*] element'prop-dateTime : string|;
	
	\item HTMLTrackElement:
	\begin{icItems}
		\item \lstinline|[r/*] element'prop-kind : string|;
		\item \lstinline|[r/*] element'prop-srclang : string|;
		\item \lstinline|[r/*] element'prop-label : string|;
		\item \lstinline|[r/*] element'prop-default : bool|;
	\end{icItems}
	
	\item HTMLVideoElement:
	\begin{icItems}
		\item \lstinline|[r/*] element'prop-poster : string|;
		\item \lstinline|[r/*] element'prop-videoHeight : int|;
		\item \lstinline|[r/*] element'prop-videoWidth : int|;
	\end{icItems}
	
	% \item \lstinline|[r/w] element'prop-|;
\end{icItems}

\code{element [c]} будет всегда коллекцией, \code{element [i]} - варьирует в зависимости от субъекта, если субъект - элемент, результат также будет элементом, если субъект - коллекция, результат также будет коллекцией. 

Напоминаю что стандарт W3C WebDriver предусматривает чтения свойств только для элементов, и в режиме тестирования все свойства будут доступны только для чтения.

Свойства \code{rows} не указано в данном списке, потому что оно повторяется у разных элементов имея разные типы.

%\newpage
