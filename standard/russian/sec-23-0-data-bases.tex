% !TeX spellcheck = ru_RU
\section{Базы данных}

Модуль баз данных определяет следующее глобальные объекты: \mintinline{icl}{DBManager} — менеджер баз данных, \mintinline{icl}{DB} — последняя открытая база данных, \mintinline{icl}{Query} — последний запрос.

Модуль баз данных обеспечивает следующий набор возможностей:
\begin{icItems}
    \item \mintinline{icl}{DBManager.openSQLite (path : string) : DB};
    \item \mintinline{icl}{DBManager.connect (server : string, user : string, password : string) : DB};
    \item \mintinline{icl}{DBManager.connect (server : string) : DB};
	\item \mintinline{icl}{DB.query (q : Code) : Query};
	\item \mintinline{icl}{DB.close () : void};
	\item \mintinline{icl}{[w/o] Query'(name : string) : any};
	\item \mintinline{icl}{Query.set (field : string, value : any) : void};
	\item \mintinline{icl}{Query.exec () : bool};
	\item Доступны после выполнения:
	\begin{icItems}
	    \item \mintinline{icl}{[r/o] Query'(name : string) : any};
		\item \mintinline{icl}{Query.getRowsAffected () : int};
		\item \mintinline{icl}{Query.getError () : bool};
		\item \mintinline{icl}{Query.getLength () : int};
		\item \mintinline{icl}{Query.get (field : string) : any};
		\item \mintinline{icl}{Query.next () : bool};
		\item \mintinline{icl}{Query.previous () : bool};
		\item \mintinline{icl}{Query.first () : bool};
		\item \mintinline{icl}{Query.last () : bool};
		\item \mintinline{icl}{Query.seek (i : int, relative = false) : bool};
	\end{icItems}
\end{icItems}

\subsubsection{\mintinline{icl}{DBManager.openSQLite (path : string) : DB}}

Открывает новое подключение. \mintinline{icl}{path} — путь к файлу базы данных.

Возможные исключения: \ferror{NoSuchDatabase} (см. таб. \ref{errors}).

\subsubsection{\mintinline{icl}{DBManager.connect (server : string, user : string, password : string) : DB}}

Создаёт соединение к icL DB Client через WebSockets, клиентов можно писать на любые языки программирование, для написания собсвенных клиентов познакомитесь с \textit{icL DataBase Share over WebSockets}, описан в \textit{icL Wire Protocol}.

Возможные исключения: \ferror{NoSuchServer} и \ferror{WrongUserPassword} (см. таб. \ref{errors}).

\subsubsection{\mintinline{icl}{DBManager.connect (server : string) : DB}}

Создаёт соединение к icL DB Client через HTTP, клиентов можно писать на любые языки программирование, для написания собсвенных клиентов познакомитесь с \textit{icL DataBase Share over HTTP}, описан в \textit{icL Wire Protocol}.

Возможные исключения: \ferror{NoSuchServer} (см. таб. \ref{errors}).

\subsubsection{\mintinline{icl}{DB.query (q : Code) : Query}}

Создаёт запрос, на основе кода SQL, изолированный в фигурных скобках.

\subsubsection{\mintinline{icl}{DB.close () : void}}

Закрывает подключение к базе данных.

Возможные исключения: \ferror{NoSuchDatabase} (см. таб. \ref{errors}).

\subsubsection{\mintinline{icl}{[w/o] Query'(name : string) : any}}

Возвращает объект, позволяющий заменить заменитель на значение через присваивание. Заменители в коде имеют следующий синтаксис \mintinline{icl}{:name}.

Возможные исключения: \ferror{NoSuchPlaceholder} (см. таб. \ref{errors}).

\subsubsection{\mintinline{icl}{Query.set (field : string, value : any) : void}}

Установит значения заменителя с именем \mintinline{icl}{field}.

Возможные исключения: \ferror{NoSuchField} (см. таб. \ref{errors}).

\subsubsection{\mintinline{icl}{Query.exec () : bool}}

Возвращает \true, если выполнения запроса была успешной, иначе \false.

\subsubsection{\mintinline{icl}{Query.getError () : string}}

Возвращает текст ошибки, если при выполнении кода произошла ошибка.

\subsubsection{\mintinline{icl}{[r/o] Query'(name : string) : any}}

После вызова функции \mintinline{icl}{Query.exec} свойства будут возвращать значения запрошенных полей. Если такое поле отсутствует, будет возвращать \void.

\subsubsection{\mintinline{icl}{Query.getRowsAffected () : int}}

Количество обновлённых/добавленных строк в базе данных.

Возможные исключения: \ferror{QueryNotExecutedYet} (см. таб. \ref{errors}).

\subsubsection{\mintinline{icl}{Query.getLength () : int}}

Количество результатов полученных при выполнении команды \mintinline{icl}{SELECT}.

Возможные исключения: \ferror{QueryNotExecutedYet} (см. таб. \ref{errors}).

\subsubsection{\mintinline{icl}{Query.get (field : string) : any}}

Возвращает значение поле или \void{} если столбец \mintinline{icl}{field} отсутствует.

Возможные исключения: \ferror{QueryNotExecutedYet} (см. таб. \ref{errors}).

\subsubsection{\mintinline{icl}{Query.next () : bool}}

Возвращает \true, если получилось перейти на следующую запись, иначе \false.

Возможные исключения: \ferror{QueryNotExecutedYet} (см. таб. \ref{errors}).

\subsubsection{\mintinline{icl}{Query.previous () : bool}}

Возвращает \true, если получилось перейти на предыдущую запись, иначе \false.

Возможные исключения: \ferror{QueryNotExecutedYet} (см. таб. \ref{errors}).

\subsubsection{\mintinline{icl}{Query.first () : bool}}

Возвращает \true, если получилось перейти на первую запись, иначе \false.

Возможные исключения: \ferror{QueryNotExecutedYet} (см. таб. \ref{errors}).

\subsubsection{\mintinline{icl}{Query.last () : bool}}

Возвращает \true, если получилось перейти на последнюю запись, иначе \false.

Возможные исключения: \ferror{QueryNotExecutedYet} (см. таб. \ref{errors}).

\subsubsection{\mintinline{icl}{Query.seek (i : int, relative = false) : bool}}

Возвращает \true, если получилось перейти на \mintinline{icl}{i}-ю запись или сдвинуть курсор на \mintinline{icl}{i}-e количество шагов (при условии \mintinline{icl}{@relative == true}), иначе \false.

Возможные исключения: \ferror{QueryNotExecutedYet} (см. таб. \ref{errors}).

\subsubsection{Пример}

Пример кода, использующий базы данных представлен на листинге \ref{dbexample}. Так же как и при использовании кода на языке Javascript, в коде можно встроить переменные icL, локальные переменный имеет синтаксис \mintinline{icl}{@:name}, а глобальные — \mintinline{icl}{#:name}.

\begin{sourcecode}
	\captionof{listing}{Пример кода использующий базу данных}
	\label{dbexample}
    \inputminted[linenos]{icl}{../sources/dbexample.icL}
\end{sourcecode}

%\newpage
