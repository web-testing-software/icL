% !TeX spellcheck = ru_RU
\section{Автозавершение команд}

\textbf{Разделитель команд} в icL может быть пропущен, но пропускать его не рекомендуется. Для улучшения читабельности кода разделитель лучше пропустить после конструкции \code{if}, \code{else}, \code{while}, \code{do while}, \code{for}, \code{filter}, \code{range}. Для наглядности можете сравнить листинг \ref{nodelimiterskipping} с листингом \ref{delimiterskipping}.

\begin{lstlisting}[caption=Без пропущенных разделители, label=nodelimiterskipping]
fun = () : int { @ = 23; };

if (fun() == 12) { Log.out 12345678; } 
else             { Log.out 87654321; };

emiter { func(@x); }
slot:Error1 { @x = 45; };
\end{lstlisting}

\begin{lstlisting}[caption=С пропущенными разделителями, label=delimiterskipping]
fun = () : int { @ = 23 }

if (fun() == 12) { Log.out 12345678 } 
else             { Log.out 87654321 }

emiter { func(@x) }
slot:Error1 { @x = 45 }
\end{lstlisting}

Команда может быть завершена автоматически после следующих семантических конструкции:
\begin{icItems}
	\item Литерал;
	\item Значение;
	\item Свойство;
	\item Запакованные данные;
	\item Блок кода.
\end{icItems}