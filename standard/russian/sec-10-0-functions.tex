% !TeX spellcheck = ru_RU
\section{Функции}

{\bf Функции} в icL позволяют повторное использовать код, и структурировать код. Все функций в icL - глобальные.

\subsubsection{Определение функции}

{\bf Определение функции} состоит из заголовка и тела функции.

\noindent Синтаксис -
\begin{lstlisting}[numbers=none]
name = parameters : return_type {
	commands
}
\end{lstlisting}

Описание всех частей функции:
\begin{icItems}
\item
	\code{name} - {\bf название функции};
\item
	\code{parameters} - {\bf список параметров}. Когда функция вызывается вы передаёте значения параметру. Список параметров определяет тип, количество и порядок параметров. Список параметров необязателен (см. листинги \ref{fullfunc} и \ref{noargsfunc});
\item
	\code{return_type} - {\bf тип данных которых} функция будет возвращать. Если функция ничего не возвращает тип данных можно пропустить или указать явно \void{}. Чтобы возвращать значение используется команда \code{@ = value} (см. листинги \ref{fullfunc}, \ref{notypefunc} и \ref{minfunc});
\item
	\code{commands} - {\bf тело функции} содержит команды, которые определяют что делает функция.
\end{icItems}

\begin{lstlisting}[caption=Полноценная функция, label=fullfunc]
sum = <int>a <int>b : int {
	@ = @a + @b;
};
\end{lstlisting}

\begin{lstlisting}[caption=Функция без аргументов, label=noargsfunc]
pi = double {
	@ = 3.14;
};
\end{lstlisting}

\begin{lstlisting}[caption=Функция без типа возвращаемой значений, label=notypefunc]
out = <int>a <int>b {
	Log.out @a @b;
};
\end{lstlisting}

\subsubsection{Вызов функции}

Вы можете вызвать функцию следующим образом -
\begin{lstlisting}[numbers=none]
name arguments;
\end{lstlisting}
где \code{name} - {\bf название функции}, \code{arguments} - {\bf список аргументов} (он должен быть совместим со списком параметров функции). Как вызвать функций, объявленные на листингов \ref{fullfunc} - \ref{minfunc}, показано на листинге \ref{callfunc}.

\begin{lstlisting}[caption=Функция без аргументов и тип возвращаемой значений, label=minfunc]
do = {
	Log.out "It's work!";
};
\end{lstlisting}

\begin{lstlisting}[caption=Вызов функций, label=callfunc]
sum 2 3; 	`` returns 5
pi; 		`` returns 3.14
out 4 5; 	`` returns void
do; 		`` returns void

sum (sum (1, 2), 3);	`` returns 6
out (1, sum (,4 5));	`` returns void
\end{lstlisting}

%\newpage
