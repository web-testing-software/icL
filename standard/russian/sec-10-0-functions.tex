% !TeX spellcheck = ru_RU
\section{Функции}

{\bf Функции} в icL позволяют повторное использовать код, и структурировать код. Все функций в icL - глобальные.

\subsubsection{Определение функции}

{\bf Определение функции} состоит из заголовка и тела функции.

\noindent Синтаксис -
\begin{lstlisting}[numbers=none]
name = (parameters) : return_type {
	commands
}
\end{lstlisting}

Описание всех частей функции:
\begin{icItems}
\item
	\code{name} - {\bf название функции};
\item
	\code{parameters} - {\bf список параметров}. Когда функция вызывается вы передаёте значения параметру. Список параметров определяет тип, количество и порядок параметров. Список параметров необязателен (см. листинги \ref{fullfunc} и \ref{noargsfunc});
\item
	\code{return_type} - {\bf тип данных которых} функция будет возвращать. Если функция ничего не возвращает тип данных можно пропустить или указать явно \void{}. Чтобы возвращать значение используется команда \code{@ = value} или \lstinline|Stack.return value| (см. листинги \ref{fullfunc}, \ref{notypefunc} и \ref{minfunc});
\item
	\code{commands} - {\bf тело функции} содержит команды, которые определяют что делает функция.
\end{icItems}

Между параметрами ставится разделитель - запятая. Даже если параметр только один, скобки нельзя пропускать.

Параметр описывается следующим синтаксисом \lstinline|@name : type|, где \lstinline|@name| - название, \lstinline|type| - тип данных. Можно указать только тип данных (без двоеточия), в этом случае название параметра будет название типа данных.

Параметр имеющий значение по умолчанию описывается следующим способом \lstinline|@name = value|, где \lstinline|@name| - название, \lstinline|value| - значение по умолчанию, она может быть константном литералом, переменной или выражений. Значения будут вычислены один раз - про объявлений функций. Значение по умолчанию могут иметь любые параметры, не только последнее.

\begin{lstlisting}[caption=Полноценная функция, label=fullfunc]
sum = (@a : int, @b : int) : int {
	@ = @a + @b;
};
\end{lstlisting}

\begin{lstlisting}[caption=Функция без аргументов, label=noargsfunc]
pi = () : double {
	@ = 3.14;
};
\end{lstlisting}

\begin{lstlisting}[caption=Функция без типа возвращаемой значений, label=notypefunc]
out = (@a : int, @b : int) {
	Log.out @a @b;
};
\end{lstlisting}

\begin{lstlisting}[caption=Функция без аргументов и тип возвращаемой значений, label=minfunc]
do = () {
Log.out "It's work!";
};
\end{lstlisting}

\subsubsection{Вызов функции}

Вы можете вызвать функцию следующим образом -
\begin{lstlisting}[numbers=none]
name (arguments);
\end{lstlisting}
где \code{name} - {\bf название функции}, \code{arguments} - {\bf список аргументов} (он должен быть совместим со списком параметров функции). Как вызвать функций, объявленные на листингов \ref{fullfunc} - \ref{minfunc}, показано на листинге \ref{callfunc}. 

Последнее параметры имеющие значение по умолчанию могут быть пропущены, для остальных при желаниях использовать значение по умолчанию нужно передать \void{} значение, её можно описать одним символом - тильда (\lstinline|~|). Например функция \lstinline|func = (@var1 = 3, @var2 : int) {}|, можно вызвать следующим образом \lstinline|fun (~, 3)|. Значения по умолчанию могут сократить код, см. листинг \ref{defaultparametrs}.

Функций имеющие всего 1 обязательным параметром, можно вызвать используя акроним \lstinline|name arg| или \lstinline|name (arg)|. Положение обязательного параметра не имеет значение (в начале, в середине или в конце списка параметров).

\begin{lstlisting}[caption=Вызов функций, label=callfunc]
sum (2, 3); 	`` returns 5
pi;				`` returns 3.14
out (4, 5); 	`` returns void
do;				`` returns void

sum (sum (1, 2), 3);	`` returns 6
out (1, sum (,4 5));	`` returns void
\end{lstlisting}

\begin{lstlisting}[caption=Вызов функций, label=callfunc]
sum (2, 3); 	`` returns 5
pi;				`` returns 3.14
out (4, 5); 	`` returns void
do;				`` returns void

sum (sum (1, 2), 3);	`` returns 6
out (1, sum (,4 5));	`` returns void
\end{lstlisting}

\begin{lstlisting}[caption=Значения по умольчанию, label=defaultparametrs]
fun = (@a = 21, @b = 23) {}

`` the line 
fun(exists(@a, # > 3), exists(@b, # < 5));
`` replace the next code
if (@a > 3) {
	if (@b < 5) {
		fun (@a, @b);
	}
	else {
		fun (@a, 23);
	}
}
else {
	if (@b < 5) {
		fun (21, @b);
	}
	else {
		fun (21, 23);
	}
}
\end{lstlisting}

%\newpage
