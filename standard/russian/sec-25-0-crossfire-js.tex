% !TeX spellcheck = ru_RU
\section{crossfire.js}

Технология {\bf crossfire.js} позволяет вызвать функцию icL из веб-страницы. Функция будет выполнено асинхронно.

Вставить функцию icL в Javascript можно с помощью заменителя \mintinline{icl}{!{name}}. При вызове функций icL, параметры по типу данных должны подходить, иначе будет сгенерирован сигнал, который остановит в последствия выполнение скрипта. Для корректность восприятия данных рекомендуется использовать следующее функций Javascript:
\begin{icItems}
	\item \mintinline{icl}{crossfire.bool(arg)};
	\item \mintinline{icl}{crossfire.int(arg)};
	\item \mintinline{icl}{crossfire.double(arg)};
	\item \mintinline{icl}{crossfire.string(arg)};
	\item \mintinline{icl}{crossfire.list(arg)};
	\item \mintinline{icl}{crossfire.object(arg)};
	\item \mintinline{icl}{crossfire.set(arg)};
	\item \mintinline{icl}{crossfire.element(arg)};
\end{icItems}

Элементарный пример как вызвать правильно функцию icL представлен на листинге \ref{crossfireexample}. Функция \mintinline{icl}{onclick} получает один параметр, но \mintinline{icl}{func} - нет. Соответственно нужен интерфейс, легче всего определить анонимную функцию.

\begin{sourcecode}
\captionof{listing}{Пример вызова функций icL}
\label{crossfireexample}
\begin{minted}[linenos]{icl}
@el = Doc.query "button";
!func = { Log.info "It's working!" };

$run {
	@{el}[0].onlick = function (ev) { !{func}() }
};
\end{minted}
\end{sourcecode}

Использование технологий \textit{crossfire.js} в скрипте приводит к переходу программы в режиме ожидания. В этом режиме программа не будет сама закрываться никогда, а будет ждать внешний вызов функций, чтобы остановить программу используете сигнал \mintinline{icl}{Exit}.

Список технологии \textit{icL} доступен по ссылке: \ferror{https://gitlab.com/lixcode/icL/tree/standard/technologies\#technologies}.
