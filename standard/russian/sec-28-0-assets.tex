% !TeX spellcheck = ru_RU
\section{Утверждения}

{\bf Утверждения} позволяет оценить на сколько успешно прошел тест. Утверждения в icL имеют 2 синтаксисы: \mintinline{icl}{assert (condition)} и \mintinline{icl}{assert (condition, message)}. В первом присутствует только условие \mintinline{icl}{condition}, если оно равно \false{} или \void, то icL будет утверждать что есть проблема в тестированным ПО, отправляя сообщение \mintinline{icl}{message}, если оно отсутствует - условие будет отправлено в качестве сообщений. Как правильно отправлять одно и то же сообщение нескольких утверждений показано на листинге \ref{assertexample}. Результат утверждений будет сохранён только если оно определенно внутри шага теста, если утверждением оказывается ложным, программа останавливаться.

\begin{sourcecode}
\captionof{listing}{Пример использования утверждениях}
\label{assertexample}
\begin{minted}[linenos]{icl}
`` wrong
assert (@val != 2 || @val != 4 || @val != 127, "Message");

`` correct
@message = "Message";
assert (@val != 2,   @message);
assert (@val != 4,   @message);
assert (@val != 127, @message);

`` without using a variable
for any ("Message") {
	assert (@val != 2,   @);
	assert (@val != 4,   @);
	assert (@val != 127, @);
}
\end{minted}
\end{sourcecode}


