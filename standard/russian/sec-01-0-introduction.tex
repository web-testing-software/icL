% !TeX spellcheck = ru_RU
\section{Введение}

\textbf{icL} — \textbf{язык описания сценариев}, оптимизирован под описания сценариев тестирования веб-приложений. Данный документ (так же как и программное обеспечение icL) распространяется под лицензий GNU GPLv3.

Язык icL имеет свой особенности и построен на философию icL. Согласно её код должен выглядеть максимально простым, быть коротким, обеспечить нужный функционал и быть устойчив к ошибкам (см. главу \ref{errorless-sec}).

\subsection{Читатели}

Этот документ предназначен для всех тех людей, которые ищут отравную точку, откуда можно начать изучать язык icL. Также данный документ используется при разработке командного процессора, его поведение  во всех ситуациях, неописанных в данном документе считается неопределённым.

Чувствуйте себя свободным в указаниях на ошибки и представлении новых идей и точек зрения, жду ваших письмах по адресу icl@vivaldi.net.

\subsection{Что вы должны уметь}

Прежде чем приступать к изучению этого языка, Вам желательно иметь базовое представление о компьютерном программировании.

\subsection{Обзор языка icL}

icL — язык сценариев тестирования веб-приложений. Его разработка началась в 2017 году, и первый выпуск планируется к 2020 году. В настоящий момент он находится в активной разработке.

icL — \textbf{язык с С-подобным синтаксисом}, который использует статическую типизацию. В icL нельзя определить собственные типы данных, так как он разработан не для программистов, знания полученные в школе на уроках информатики должны быть достаточны. Язык icL поддерживает только одну парадигму программирования - процедурную. При необходимости обработки данных, можно использовать экспорт/импорт в/из cvs и базы данных.

\subsection{Пример кода}

В icL точка входа в программе — начало файла, программа \textit{Hello world!} иллюстрирована на листинге \ref{example0}.

\begin{sourcecode}
	\captionof{listing}{Пример}
	\label{example0}
    \inputminted[linenos]{icl}{../sources/helloworld.icL}
\end{sourcecode}

\subsection{Изучение icL}

Самое важное при изучении icL — это сосредоточиться на идеях и не потеряться в технических деталях его реализации.

\subsection{Области применения icL}

Язык icL является частью \textbf{программы icL}, с его помощью можно управлять браузером, а именно:
\begin{icItems}
\item
	открыть вкладку;
\item
	закрыть вкладку;
\item
	перейти на веб-страницу;
\item
	симулировать события клавиатуры и мышки;
\item
	взаимодействовать с веб-страницей;
\item
	выполнить код на языке javascript;
\item
	управлять веб-страницей;
\item
	обменять информацию с веб-страничкой;
\item
	сделать screenshot;
\item
	управлять памятью;
\item
	экспортировать данные в csv файл;
\item
	импортировать данные из csv файла;
\item
	выполнить запросы на языке SQL.
\end{icItems}

\subsection{Начало работы}

Чтобы начать работать достаточно установить и запустить программу icL.
