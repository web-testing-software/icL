\section{crossfire.js}

Технология {\bf crossfire.js} позволяет вызвать функцию icL из веб-страницы. Функция будет выполнено асинхронно.

Вставить функцию icL в Javascript можно с помощью заменителя \lstinline|!{name}|. При вызове функций icL, параметры по типу данных должны подходить, иначе будет сгенерирован сигнал, который остановит в последствия выполнение скрипта. Для корректность восприятия данных рекомендуется использовать следующее функций Javascript:
\begin{icItems}
	\item \lstinline|crossfire.bool(arg)|;
	\item \lstinline|crossfire.int(arg)|;
	\item \lstinline|crossfire.double(arg)|;
	\item \lstinline|crossfire.string(arg)|;
	\item \lstinline|crossfire.list(arg)|;
	\item \lstinline|crossfire.object(arg)|;
	\item \lstinline|crossfire.set(arg)|;
	\item \lstinline|crossfire.element(arg)|;
\end{icItems}

Элементарный пример как вызвать правильно функцию icL представлен на листинге \ref{crossfireexample}. Функция \code{onclick} получает один параметр, но \code{func} - нет. Соответственно нужен интерфейс, легче всего определить анонимную функцию.

\begin{lstlisting}[caption=Пример вызова функций icL, label=crossfireexample]
@el = Doc.query "button";
!func = { Log.info "It's working!" };

$run {
	@{el}[0].onlick = function (ev) { !{func}() }
};
\end{lstlisting}

