% !TeX spellcheck = ru_RU
\section{Операторы управления}

Операторы управления имеют условие, которое будет вычислено. А также и блоки команд, чья выполнение зависит от значения условии.

В icL приличествуют следующее операторы управления:
\begin{icItems}
	\item \code{if};
	\item \code{if else};
	\item каскадное \code{if else};
	\item \code{switch case};
\end{icItems}

\subsubsection{\lstinline|if|}

\code{if} позволяет ставить условие выполнения блока команд. Он имеет следующий синтаксис -
\begin{lstlisting}[numbers=none]
if (condition) {
	commands
};
\end{lstlisting}
где \code{condition} любое выражение возвращающее значение типа \bool{} и \code{commands} любой набор команд.

Также допускаются пропустить круглые скобки, синтаксис -
\begin{lstlisting}[numbers=none]
if true  { commands };
if !true { commands };
if @var  { commands };
if !@var { commands };
\end{lstlisting}

\subsubsection{\lstinline|if else|}

\code{if else} позволяет выбирать между двух блоков команд, в случае когда условие истинно - выполняется первый блок, иначе второй блок.

Конструкция \code{if else} имеет следующий синтаксис -
\begin{lstlisting}[numbers=none]
if (condition) {
	commands1
}
else {
	commands2
};
\end{lstlisting}

\subsubsection{Каскадное \lstinline|if else|}

Каскадное \code{if else} позволяет выбирать между n блоков команд, но для это нужно предъявить n-1 условии.

Каскадное \code{if else} имеет следующий синтаксис -
\begin{lstlisting}[numbers=none]
if (condition1) {
	commands1
} else if (condition2) {
	commands2
} else {
	commands3
};
\end{lstlisting}

\subsubsection{\lstinline|switch case|}

\code{switch case} позволяет выбирать между n блоков команд, но для это нужно предъявить n условии.

Синтаксис -
\begin{lstlisting}[numbers=none]
switch (value) {
	case (caseValue) { ```code``` }
	`` more cases
}
\end{lstlisting}

\code{switch} получает значения \code{value}, она может иметь любой тип кроме \bool. Дальше внутри \code{switch}-а имеет несколько \code{case}-ов. Если значения \code{case}-а - \code{caseValue} совпадает с \code{value}, то \code{case} будет выполнен.

\code{case} может иметь несколько значений разделённые запятой.

Он может получить в качестве значения \void, выглядит это следующим образом \lstinline|case (~) { ```code``` }|, такой \code{case} будет выполнен только если ни один из предыдущих не был выполнен.

Ещё одна специальное значение - \code{#}, через неё обозначается что текущий \code{case} нужно выполнить при условии что предыдущий был выполнен.

Разбираем на примере все возможность \code{switch case}, пример представлен на листинге \ref{switchcaseex}.

\begin{lstlisting}[caption=Использование switch case, label=switchcaseex]
@v = 0 | 1 | 2 | 3 | 4 | 5;

switch (@v) {
	case (0) { ```code1``` }
	case (1, 2) { ```code2``` }
	case (#, 3) { ```code3``` }
	case (4) { ```code4``` }
	case (~) { ```code5``` }
}
\end{lstlisting}

Переменная \code{@v} имеет значения в интервале от 0 до 6.

Разбираем какие фрагменты кода будут выполнены для каждой возможной значении:
\begin{icItems}
	\item 0 - будет выполнен \code{code1};
	\item 1 - будет выполнен \code{code2} и \code{code3};
	\item 2 - будет выполнен \code{code2} и \code{code3};
	\item 3 - будет выполнен \code{code3};
	\item 4 - будет выполнен \code{code4};
	\item 5 - будет выполнен \code{code5};
	\item 6 - будет выполнен \code{code5}.
\end{icItems}

%\newpage
