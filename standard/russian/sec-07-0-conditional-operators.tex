% !TeX spellcheck = ru_RU
\section{Операторы управления}

Операторы управления имеют условие, которое будет вычислено. А также и блоки команд, чья выполнение зависит от значения условии.

В icL приличествуют следующее операторы управления:
\begin{icItems}
	\item \mintinline{icl}{if};
	\item \mintinline{icl}{if else};
	\item каскадное \mintinline{icl}{if else};
	\item \mintinline{icl}{switch case};
\end{icItems}

\subsubsection{\mintinline{icl}{if}}

\mintinline{icl}{if} позволяет ставить условие выполнения блока команд. Он имеет следующий синтаксис —
\begin{minted}{icl}
if (условие) {
	команды
};
\end{minted}
где \mintinline{icl}{условие} любое выражение возвращающее значение типа \bool{} и \mintinline{icl}{команды} любой набор команд.

Также допускаются пропустить круглые скобки, синтаксис —
\begin{minted}{icl}
if true  { команды };
if !true { команды };
if @var  { команды };
if !@var { команды };
\end{minted}

\subsubsection{\mintinline{icl}{if else}}

\mintinline{icl}{if else} позволяет выбирать между двух блоков команд, в случае когда условие истинно — выполняется первый блок, иначе второй блок.

Конструкция \mintinline{icl}{if else} имеет следующий синтаксис —
\begin{minted}{icl}
if (условие) {
	команды1
}
else {
	команды2
};
\end{minted}

\subsubsection{Каскадное \mintinline{icl}{if else}}

Каскадное \mintinline{icl}{if else} позволяет выбирать между n блоков команд, но для это нужно предъявить n-1 условии.

Каскадное \mintinline{icl}{if else} имеет следующий синтаксис —
\begin{minted}{icl}
if (условие1) {
	команды1
} else if (условие2) {
	команды2
} else {
	команды3
};
\end{minted}

\subsubsection{\mintinline{icl}{switch case}}

\mintinline{icl}{switch case} позволяет выбирать между n блоков команд, но для это нужно предъявить n условии.

Синтаксис —
\begin{minted}{icl}
switch (значение) {
	case (значениеСлучая) { `код` }
	`` больше case-ов
}
\end{minted}

\mintinline{icl}{switch} получает значения \mintinline{icl}{значение}, она может иметь любой тип кроме \bool. Дальше внутри \mintinline{icl}{switch}-а имеет несколько \mintinline{icl}{case}-ов. Если значения \mintinline{icl}{case}-а — \mintinline{icl}{значениеСлучая} совпадает с \mintinline{icl}{значение}, то \mintinline{icl}{case} будет выполнен.

\mintinline{icl}{case} может иметь несколько значений разделённые запятой.

Он может получить в качестве значения \void, выглядит это следующим образом \mintinline{icl}{case (~) { `код` }}, такой \mintinline{icl}{case} будет выполнен только если ни один из предыдущих не был выполнен.

Ещё одна специальное значение - \mintinline{icl}{#}, через неё обозначается что текущий \mintinline{icl}{case} нужно выполнить при условии что предыдущий был выполнен.

Разбираем на примере все возможность \mintinline{icl}{switch case}, пример представлен на листинге \ref{switchcaseex}.

\begin{sourcecode}
	\captionof{listing}{Использование switch case}
	\label{switchcaseex}
    \inputminted[linenos]{icl}{../sources/switchcaseex.icL}
\end{sourcecode}

Переменная \mintinline{icl}{@v} имеет значения в интервале от 0 до 6.

Разбираем какие фрагменты кода будут выполнены для каждой возможной значении:
\begin{icItems}
	\item 0 - будет выполнен \mintinline{icl}{code1};
	\item 1 - будет выполнен \mintinline{icl}{code2} и \mintinline{icl}{code3};
	\item 2 - будет выполнен \mintinline{icl}{code2} и \mintinline{icl}{code3};
	\item 3 - будет выполнен \mintinline{icl}{code3};
	\item 4 - будет выполнен \mintinline{icl}{code4};
	\item 5 - будет выполнен \mintinline{icl}{code5};
	\item 6 - будет выполнен \mintinline{icl}{code5}.
\end{icItems}

%\newpage
