\section{Условные операторы}

Условные структуры имеют условие, которое будет вычислено. А также и блоки команд, чья выполнение зависит от значения условии.

В icL приличествуют следующее условные операторы:
\begin{icItems}
	\item \code{if};
	\item \code{if else};
	\item каскадное \code{if else};
	\item \code{exists};
	\item \code{if exists};
	\item \code{for any}.
\end{icItems}

\subsubsection{\lstinline`if`}

\code{if} позволяет ставить условия выполнения блока команд. Он имеет следующий синтаксис -
\begin{lstlisting}[numbers=none]
if (condition) {
	commands
};
\end{lstlisting}
где \code{condition} любое выражение возвращающее значение типа \bool{} и \code{commands} любой набор команд.

Также допускаются пропустить круглые скобки, синтаксис -
\begin{lstlisting}[numbers=none]
if true  { commands };
if @var  { commands };
\end{lstlisting}

\subsubsection{\lstinline`if else`}

\code{if else} позволяет выбирать между двух блоков команд, в случае когда условие истинно - выполняется первый блок, иначе второй блок.

Конструкция \code{if else} имеет следующий синтаксис -
\begin{lstlisting}[numbers=none]
if (condition) {
	commands1
}
else {
	commands2
};
\end{lstlisting}

\subsubsection{Каскадное \lstinline`if else`}

Каскадное \code{if else} позволяет выбирать между n блоков команд, но для это нужно предъявить n-1 условии.

Каскадное \code{if else} имеет следующий синтаксис -
\begin{lstlisting}[numbers=none]
if (condition1) {
	commands1
} else if (condition2) {
	commands2
} else {
	commands3
};
\end{lstlisting}

\subsubsection{\lstinline`exists`}

\code{exists} позволяет условно возвращать данные, если они подходят под определённому критерия.

Условия по умолчанию:
\begin{icItems}
\item
	для \bool{} - \code{# == true};
\item
	для \integer{} - \code{# != 0};
\item
	для \double{} - \code{# != 0.0};
\item
	для \str{} - \code{!#'empty};
\item
	для \listtype{} - \code{!#'empty};
\item
	для \set{} - \code{!#'empty};
\item
	для \element{} - \code{#'empty}.
\end{icItems}

Если использовать условию по умолчанию, используется следующий синтаксис - \lstinline|exists(expression)|.
При необходимости задать своё условие, синтаксис чуть изменяется -\lstinline|exists(expression, condition)|.

\subsubsection{\lstinline`if exists`}

Конструкция \code{if exists} позволяет выполнять блок команд в зависимости от результата работы конструкции \code{exists}. Также повторно использовать значения выражения. В случае когда условие конструкции \code{exists} истинно, то блок команд выполняется и в нём передаётся данный полученные от \code{exists} под именем \code{@}.
Простой и элементарный пример представлен на листинге \ref{ifexistsex}.

\begin{lstlisting}[caption=Использование if exist, label=ifexistsex]
if exists(23 + 3, # > 20) {
	Log.out "@ = " @; `` @ = 26
};
\end{lstlisting}

\subsubsection{\lstinline`for any`}

\code{for any} позволяет повторно использовать любое значение. Пример использования представлен на листинге \ref{foranyex}.

\begin{lstlisting}[caption=Использование for any, label=foranyex]
for any(23 + 3) {
	Log.out "@ = " @; `` @ = 26
};
\end{lstlisting}

%\newpage
