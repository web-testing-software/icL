% !TeX spellcheck = ru_RU
% !TeX spellcheck = ro_RO
\part{Material pentru programatori}

Pentru a înțelege acest material trebuiesc cunoștințe în programare. În ea nu vor fi explicați termini tehnici. Necătînd la aceea că limbajul icL este simplu, în el se poate de realizat scripturi destul de complicate, de lucrat cu baze de date, expresii regulare și multe altele, se așteaptă dezvoltare intensă a acestei părți în următoarele versiuni ale limbajului icL.


\section{Дополнительные возможности}

{\bf Дополнительные возможности} присутствуют в icL для полного перекрытия стандарта W3C WebDriver и для решения определённых проблем, которые могут появиться при написании скриптов.

Данные возможности не обязательно использовать, так как при запуске скрипта создаётся сессия и открывается новая вкладка, и при остановке сессия закрывается автоматически.

% \subsection{Возможности W3C WebDriver}

\subsection{{\color{orange} Sessions}}

Объект \sessions{} имеет следующие свойства:
\begin{icItems}
	\item \mintinline{icl}{[r/o] Sessions'length : int};
	\item \mintinline{icl}{[r/o] Sessions'(i : int) : Session}.
\end{icItems}

И следующие методы:
\begin{icItems}
	\item \mintinline{icl}{Sessions.closeAll () : void};
	\item \mintinline{icl}{Sessions.get (i : int) : Session};
	\item \mintinline{icl}{Sessions.new () : Session}.
\end{icItems}

\subsubsection{\mintinline{icl}{[r/o] Sessions'length : int}}

Возвращает количество отрытых сессии.

\subsubsection{\mintinline{icl}{[r/o] Sessions'(i : int) : Session}}

Возвращает \mintinline{icl}{i}-ю сессию.

Возможные исключения: \ferror{OutOfBounds} (см. таб. \ref{errors}).

\subsubsection{\mintinline{icl}{Sessions.closeAll () : void}}

Закрывает всех сессий.

\subsubsection{\mintinline{icl}{Sessions.get (i : int) : Session}}

Возвращает \mintinline{icl}{i}-ю сессию.

Возможные исключения: \ferror{OutOfBounds} (см. таб. \ref{errors}).

\subsubsection{\mintinline{icl}{Sessions.new () : Session}}

Отрывает новую сессию.

Возможные исключения: \ferror{SessionNotCreated} (см. таб. \ref{errors}).

\subsection{{\color{orange} Session}}

Команда \session{} возвращает указатель на текущую сессию.

Объект \session{} имеет следующие свойства:
\begin{icItems}
	\item \mintinline{icl}{[r/w] Session'implicitTimeout : int};
	\item \mintinline{icl}{[r/w] Session'pageLoadTimeout : int};
	\item \mintinline{icl}{[r/w] Session'scriptTimeout : int};
	\item \mintinline{icl}{[r/o] Session'source : string};
	\item \mintinline{icl}{[r/o] Session'title : string};
	\item \mintinline{icl}{[r/w] Session'url : string}.
\end{icItems}

И следующие методы:
\begin{icItems}
	\item \mintinline{icl}{Session.back () : Session};
	\item \mintinline{icl}{Session.close () : void};
	\item \mintinline{icl}{Session.forward () : Session};
	\item \mintinline{icl}{Session.refresh () : Session};
	\item \mintinline{icl}{Session.screenshot () : string};
	\item \mintinline{icl}{Session.switchTo () : Session}.
\end{icItems}

\subsubsection{\mintinline{icl}{[r/w] Session'implicitTimeout : int}}

Время ожидания в миллисекундах нахождения элемента и перехода элемента в состоянии доступности для взаимодействий с ним. По умолчанию равна нулю.

Возможные исключения: \ferror{NoSessions}, \ferror{InvalidArgument} (значение для установки меньше нуля или больше максимальной безопасной целей значений) см. таб. \ref{errors}.

\subsubsection{\mintinline{icl}{[r/w] Session'pageLoadTimeout : int}}

Время ожидания загрузке страницы. По умолчанию 300 000 миллисекунд.

Возможные исключения: \ferror{NoSessions}, \ferror{InvalidArgument} (см. таб. \ref{errors}).

\subsubsection{\mintinline{icl}{[r/w] Session'scriptTimeout : int}}

Время ожидания синхронией выполнения кода на языке Javascript. По умолчанию 30 000 миллисекунд.

Возможные исключения: \ferror{NoSessions}, \ferror{InvalidArgument} (см. таб. \ref{errors}).

\subsubsection{\mintinline{icl}{[r/o] Session'source : string}}

Исходный код страницы.

Возможные исключения: \ferror{NoSessions}, \ferror{NoSuchWindow} (см. таб. \ref{errors}).

\subsubsection{\mintinline{icl}{[r/o] Session'title : string}}

Название страницы.

Возможные исключения: \ferror{NoSessions}, \ferror{NoSuchWindow} (см. таб. \ref{errors}).

\subsubsection{\mintinline{icl}{[r/w] Session'url : string}}

Адрес текущей странице.

Возможные исключения: \ferror{NoSessions}, \ferror{NoSuchWindow} (см. таб. \ref{errors}).

\subsubsection{\mintinline{icl}{Session.back () : Session}}

Переходит на предыдущую страницу, как при нажатии на кнопке браузера "Назад".

Возможные исключения: \ferror{NoSessions}, \ferror{NoSuchWindow}, \ferror{Timeout} (см. таб. \ref{errors}).

\subsubsection{\mintinline{icl}{Session.close () : void}}

Прекращает сессию. Все вкладки будут закрыты.

Возможные исключения: \ferror{NoSessions}, \ferror{NoSessions}, \ferror{NoSuchSession} (см. таб. \ref{errors}).

\subsubsection{\mintinline{icl}{Session.forward () : Session}}

Вернётся на следующую страницу, как при нажатии на кнопке браузера "Вперёд".

Возможные исключения: \ferror{NoSessions}, \ferror{NoSuchWindow}, \ferror{Timeout} (см. таб. \ref{errors}).

\subsubsection{\mintinline{icl}{Session.refresh () : Session}}

Перезагружает страницу.

Возможные исключения: \ferror{NoSessions}, \ferror{NoSuchWindow}, \ferror{Timeout} (см. таб. \ref{errors}).

\subsubsection{\mintinline{icl}{Session.screenshot () : string}}

Возвращает скриншот, изображение кодируется в строке base64.

Возможные исключения: \ferror{NoSessions}, \ferror{NoSuchWindow}, \ferror{UnableToCaptureScreen} (см. таб. \ref{errors}).

\subsubsection{\mintinline{icl}{Session.switchTo () : Session}}

Переключает активную сессию.

Возможные исключения: \ferror{NoSuchSession} (см. таб. \ref{errors}).

%\subsubsection{}

\subsection{{\color{orange} Windows}}

Объект \windows{} имеет следующие свойства:
\begin{icItems}
	\item \mintinline{icl}{[r/o] Windows'length : int};
	\item \mintinline{icl}{[r/o] Windows'(i : int) : Window}.
\end{icItems}

И следующий метод \mintinline{icl}{Windows.get (i : int) : Window}.

\subsubsection{\mintinline{icl}{[r/o] Windows'length : int}}

Возвращает количество окон в текущей сессии.

Возможные исключения: \ferror{NoSessions} (см. таб. \ref{errors}).

\subsubsection{\mintinline{icl}{[r/o] Windows'(i : int) : Window}}

Возвращает \mintinline{icl}{i}-е окно.

Возможные исключения: \ferror{NoSessions}, \ferror{OutOfBounds} (см. таб. \ref{errors}).

\subsubsection{\mintinline{icl}{Windows.get (i : int) : Window}}

Возвращает \mintinline{icl}{i}-е окно.

Возможные исключения: \ferror{NoSessions}, \ferror{OutOfBounds} (см. таб. \ref{errors}).

\subsection{{\color{orange} Window}}

Объект \window{} имеет следующие свойства:
\begin{icItems}
	\item \mintinline{icl}{[r/w] Window'height : int};
	\item \mintinline{icl}{[r/w] Window'width : int};
	\item \mintinline{icl}{[r/w] Window'x : int};
	\item \mintinline{icl}{[r/w] Window'y : int}.
\end{icItems}

И следующие методы:
\begin{icItems}
	\item \mintinline{icl}{Window.close () : void};
	\item \mintinline{icl}{Window.focus () : Window};
	\item \mintinline{icl}{Window.fullscreen () : Window};
	\item \mintinline{icl}{Window.maximize () : Window};
	\item \mintinline{icl}{Window.minimize () : Window};
	\item \mintinline{icl}{Window.restore () : Window};
	\item \mintinline{icl}{Window.switchToDefault () : Window};
	\item \mintinline{icl}{Window.switchToFrame (i : int) : Window};
	\item \mintinline{icl}{Window.switchToFrame (el : element) : Window};
	\item \mintinline{icl}{Window.switchToParent () : Window}.
\end{icItems}

\subsubsection{\mintinline{icl}{[r/w] Window'height : int}}

Высота текущей окно в пикселях.

Возможные исключения: \ferror{NoSessions}, \ferror{NoSuchWindow}, \ferror{InvalidArgument}, \ferror{UnsupportedOperation} (см. таб. \ref{errors}).

\subsubsection{\mintinline{icl}{[r/w] Window'width : int}}

Ширина текущей окно в пикселях.

Возможные исключения: \ferror{NoSessions}, \ferror{NoSuchWindow}, \ferror{InvalidArgument}, \ferror{UnsupportedOperation} (см. таб. \ref{errors}).

\subsubsection{\mintinline{icl}{[r/w] Window'x : int}}

Координата $x$ текущей окно в пикселях.

Возможные исключения: \ferror{NoSessions}, \ferror{NoSuchWindow}, \ferror{InvalidArgument}, \ferror{UnsupportedOperation} (см. таб. \ref{errors}).

\subsubsection{\mintinline{icl}{[r/w] Window'y : int}}

Координата $y$ текущей окно в пикселях.

Возможные исключения: \ferror{NoSessions}, \ferror{NoSuchWindow}, \ferror{InvalidArgument}, \ferror{UnsupportedOperation} (см. таб. \ref{errors}).

\subsubsection{\mintinline{icl}{Window.close () : void}}

Закрывает окну, если она последняя закрывается и сессию.

Возможные исключения: \ferror{NoSessions}, \ferror{NoSuchWindow} (см. таб. \ref{errors}).

\subsubsection{\mintinline{icl}{Window.focus () : Window}}

Делает окно активным.

Возможные исключения: \ferror{NoSessions}, \ferror{NoSuchWindow} (см. таб. \ref{errors}).

\subsubsection{\mintinline{icl}{Window.fullscreen () : Window}}

Переключает режим отображения окна на полный экран.

Возможные исключения: \ferror{NoSessions}, \ferror{NoSuchWindow} (см. таб. \ref{errors}).

\subsubsection{\mintinline{icl}{Window.maximize () : Window}}

Распахнет окно.

Возможные исключения: \ferror{NoSessions}, \ferror{NoSuchWindow} (см. таб. \ref{errors}).

\subsubsection{\mintinline{icl}{Window.minimize () : Window}}

Свернёт окно.

Возможные исключения: \ferror{NoSessions}, \ferror{NoSuchWindow} (см. таб. \ref{errors}).

\subsubsection{\mintinline{icl}{Window.restore () : Window}}

Восстановит окно в нормальном режиме.

Возможные исключения: \ferror{NoSessions}, \ferror{NoSuchWindow} (см. таб. \ref{errors}).

\subsubsection{\mintinline{icl}{Window.switchToDefault () : Window}}

Переключает фокус на главный frame.

Возможные исключения: \ferror{NoSessions}, \ferror{NoSuchWindow} (см. таб. \ref{errors}).

\subsubsection{\mintinline{icl}{Window.switchToFrame (i : int) : Window}}

Переключает фокус на frame с индексом \mintinline{icl}{i}.

Возможные исключения: \ferror{NoSessions}, \ferror{NoSuchWindow}, \ferror{NoSuchFrame} (см. таб. \ref{errors}).

\subsubsection{\mintinline{icl}{Window.switchToFrame (el : element) : Window}}

Переключает фокус на frame элемента \mintinline{icl}{el}. Элемент должен быть обязательно тэгом frame или iframe.

Возможные исключения: \ferror{NoSessions}, \ferror{NoSuchWindow}, \ferror{NoSuchFrame}, \ferror{StaleElementReference} (см. таб. \ref{errors}).

\subsubsection{\mintinline{icl}{Window.switchToParent () : Window}}

Переключает фокус на родительский frame.

Возможные исключения: \ferror{NoSessions}, \ferror{NoSuchWindow} (см. таб. \ref{errors}).

\subsection{{\color{orange} Cookies}}

Объект \cookies{} имеет следующее свойство: \mintinline{icl}{[r/o] Cookies'name : string :}\\* \mintinline{icl}{Cookie}.

И следующие методы: 
\begin{icItems}
	\item \mintinline{icl}{Cookies.deleteAll () : void};
	\item \mintinline{icl}{Cookies.get (name : string) : Cookie}.
\end{icItems}

\subsubsection{\mintinline{icl}{[r/o] Cookies'(name : string) : Cookie}}

Возвращает \cookie{} с нужным названием.

Возможные исключения: \ferror{NoSessions}, \ferror{NoSuchWindow}, \ferror{NoSuchCookie} (см. таб. \ref{errors}).

\subsubsection{\mintinline{icl}{Cookies.deleteAll () : void}}

Удаляет все существующие cookie.

\subsubsection{\mintinline{icl}{Cookies.get (name : string) : Cookie}}

Возвращает \cookie{} с нужным названием.

Возможные исключения: \ferror{NoSessions}, \ferror{NoSuchWindow}, \ferror{NoSuchCookie} (см. таб. \ref{errors}).

\subsection{{\color{orange} Cookie}}

Объект \cookie{} имеет следующие свойства:
\begin{icItems}
	\item \mintinline{icl}{[r/w] Cookie'domain : string};
	\item \mintinline{icl}{[r/w] Cookie'expiry : int};
	\item \mintinline{icl}{[r/w] Cookie'httpOnly : bool};
	\item \mintinline{icl}{[r/w] Cookie'name : string};
	\item \mintinline{icl}{[r/w] Cookie'path : string};
	\item \mintinline{icl}{[r/w] Cookie'secure : bool};
	\item \mintinline{icl}{[r/w] Cookie'value : string}.
\end{icItems}

И следующие методы:
\begin{icItems}
	\item \mintinline{icl}{Cookie.add (years : int, months : int, days : int, hours = 0, minutes = 0,}\\* \mintinline{icl}{seconds = 0) : Cookie};
	\item \mintinline{icl}{Cookie.load () : Cookie};
	\item \mintinline{icl}{Cookie.resetTime () : Cookie};
	\item \mintinline{icl}{Cookie.save () : Cookie}.;
	\item \mintinline{icl}{Cookie.delete () : void}.
\end{icItems}

\subsubsection{\mintinline{icl}{[r/w] Cookie'domain : string}}

Доменное имя, на которым cookie доступен.

\subsubsection{\mintinline{icl}{[r/w] Cookie'expiry : int}}

Время указывающая срок годности cookie. По умолчанию -1, указывает на то что cookie будет удален при окончания сессии.

\subsubsection{\mintinline{icl}{[r/w] Cookie'httpOnly : bool}}

Значение по умолчанию - \false.

\subsubsection{\mintinline{icl}{[r/w] Cookie'name : string}}

Название, обязательное для заполнения.

\subsubsection{\mintinline{icl}{[r/w] Cookie'path : string}}

Значение по умолчанию - \mintinline{icl}{"/"}.

\subsubsection{\mintinline{icl}{[r/w] Cookie'secure : bool}}

Значение по умолчанию - \false.

\subsubsection{\mintinline{icl}{[r/w] Cookie'value : string}}

Значение самого cookie, обязательное для заполнения.

\subsubsection{\mintinline{icl}{Cookie.add (years : int, months : int, days : int, hours = 0, minutes = 0}\\*\noindent\mintinline{icl}{seconds = 0) : Cookie}}

Добавляет к сроку годности ножное количество лет, месяцев, дней, часов, минут и секунд.

\subsubsection{\mintinline{icl}{Cookie.load () : Cookie}}

Загружает данные о cookie с браузера.

Возможные исключения: \ferror{NoSessions}, \ferror{NoSuchWindow}, \ferror{NoSuchCookie} (см. таб. \ref{errors}).

\subsubsection{\mintinline{icl}{Cookie.resetTime () : Cookie}}

Сбросит строк на текущее время. Например установить срок годности год: \mintinline{icl}{Cookie.resetTime().add (1, 0, 0)}.

\subsubsection{\mintinline{icl}{Cookie.save () : Cookie}}

Сохраняет изменения в браузере.

Возможные исключения: \ferror{NoSessions}, \ferror{NoSuchWindow}, \ferror{InvalidArgument}, \ferror{UnableToSetCookie}, \ferror{InvalidCookieDomain} (см. таб. \ref{errors}).

\subsubsection{\mintinline{icl}{Cookie.delete () : void}}

Удаляет файл cookie.

Возможные исключения: \ferror{NoSessions}, \ferror{NoSuchWindow}.

\subsubsection{Создать новый cookie}

На листинге \ref{newcookies}, представлен правильный способ создания новых cookie.

\begin{sourcecode}
	\captionof{listing}{Создания новых cookie}
	\label{newcookies}
    \inputminted[linenos]{icl}{../sources/newcookies.icL}
\end{sourcecode}


\subsection{{\color{orange} Alert}}

Объект \alert{} имеет следующее свойство: \mintinline{icl}{[r/o] Alert'text : string}.

И следующие методы:
\begin{icItems}
	\item \mintinline{icl}{Alert.accept () : void};
	\item \mintinline{icl}{Alert.dismiss () : void};
	\item \mintinline{icl}{Alert.sendKeys (keys : string) : void}.
\end{icItems}

\subsubsection{\mintinline{icl}{[r/o] Alert'text : string}}

Возвращает текст предупреждений.

Возможные исключения: \ferror{NoSessions}, \ferror{NoSuchWindow}, \ferror{NoSuchAlert} (см. таб. \ref{errors}).

\subsubsection{\mintinline{icl}{Alert.accept () : void}}

Принимает предупреждению.

Возможные исключения: \ferror{NoSessions}, \ferror{NoSuchWindow}, \ferror{NoSuchAlert} (см. таб. \ref{errors}).

\subsubsection{\mintinline{icl}{Alert.dismiss () : void}}

Отменяет предупреждению.

Возможные исключения: \ferror{NoSessions}, \ferror{NoSuchWindow}, \ferror{NoSuchAlert} (см. таб. \ref{errors}).

\subsubsection{\mintinline{icl}{Alert.sendKeys (keys : string) : void}}

Заполняет форму текстом \mintinline{icl}{keys} и подтверждает ввод.

Возможные исключения: \ferror{NoSessions}, \ferror{NoSuchWindow}, \ferror{NoSuchAlert}, \ferror{ElementNotIn\-teractable} (см. таб. \ref{errors}).

% \subsection{Возможности icL}

\subsection{{\color{orange} Tabs}}

Объект \tabs{} имеет следующие свойства:
\begin{icItems}
	\item \mintinline{icl}{[r/o] Tabs'current : Tab};
	\item \mintinline{icl}{[r/o] Tabs'first : Tab};
	\item \mintinline{icl}{[r/o] Tabs'last : Tab};
	\item \mintinline{icl}{[r/o] Tabs'length : int};
	\item \mintinline{icl}{[r/o] Tabs'next : Tab};
	\item \mintinline{icl}{[r/o] Tabs'previous : Tab};
	\item \mintinline{icl}{[r/o] Tabs'(i : int) : Tab}.
	% \item \mintinline{icl}{Tabs'};
\end{icItems}

И следующие методы:
\begin{icItems}
	\item \mintinline{icl}{Tabs.close (template : string) : int};
	\item \mintinline{icl}{Tabs.close (url : regex) : int};
	\item \mintinline{icl}{Tabs.closeByTitle (template : string) : int};
	\item \mintinline{icl}{Tabs.closeByTitle (title : regex) : int};
	\item \mintinline{icl}{Tabs.closeOthers () : int};
	\item \mintinline{icl}{Tabs.closeToLeft () : int};
	\item \mintinline{icl}{Tabs.closeToRight () : int};
	\item \mintinline{icl}{Tabs.find (template : string) : Tab};
	\item \mintinline{icl}{Tabs.find (url : regex) : Tab};
	\item \mintinline{icl}{Tabs.findByTitle (template : string) : Tab};
	\item \mintinline{icl}{Tabs.findByTitle (title : regex) : Tab}.
	% \item \mintinline{icl}{Tabs.};
\end{icItems}

В режиме тестирования вкладки будут в произвольном порядке. В режиме автоматизации в строгом порядке.

\subsubsection{\mintinline{icl}{[r/o] Tabs'current : Tab}}

Текущая вкладка.

Возможные исключения: \ferror{NoSessions} (см. таб. \ref{errors}).

\subsubsection{\mintinline{icl}{[r/o] Tabs'first : Tab}}

Первая вкладка.

Возможные исключения: \ferror{NoSessions} (см. таб. \ref{errors}).

\subsubsection{\mintinline{icl}{[r/o] Tabs'last : Tab}}

Последняя вкладка.

Возможные исключения: \ferror{NoSessions} (см. таб. \ref{errors}).

\subsubsection{\mintinline{icl}{[r/o] Tabs'length : int}}

Количество вкладок.

Возможные исключения: \ferror{NoSessions} (см. таб. \ref{errors}).

\subsubsection{\mintinline{icl}{[r/o] Tabs'next : Tab}}

Следующая вкладка.

Возможные исключения: \ferror{NoSessions}, \ferror{NoSuchTab} (см. таб. \ref{errors}).

\subsubsection{\mintinline{icl}{[r/o] Tabs'previous : Tab}}

Предыдущая вкладка.

Возможные исключения: \ferror{NoSessions}, \ferror{NoSuchTab} (см. таб. \ref{errors}).

\subsubsection{\mintinline{icl}{[r/o] Tabs'(i : int) : Tab}}

\mintinline{icl}{i}-я вкладка.

Возможные исключения: \ferror{NoSessions}, \ferror{OutOfBounds}.

\subsubsection{\mintinline{icl}{Tabs.close (template : string) : int}}

Закрывает все вкладки, у которых URL подходит по шаблону. Возвращает количество закрытых вкладок.

Возможные исключения: \ferror{NoSessions} (см. таб. \ref{errors}).

\subsubsection{\mintinline{icl}{Tabs.close (url : regex) : int}}

Закрывает все вкладки, имеющее URL подходящий по регулярному выражению \mintinline{icl}{url}. Возвращает количество закрытых вкладок.

Возможные исключения: \ferror{NoSessions} (см. таб. \ref{errors}).

\subsubsection{\mintinline{icl}{Tabs.closeByTitle (template : string) : int}}

Закрывает все вкладки, у которых название подходит по шаблону. Возвращает количество закрытых вкладок.

Возможные исключения: \ferror{NoSessions} (см. таб. \ref{errors}).

\subsubsection{\mintinline{icl}{Tabs.closeByTitle (title : regex) : int}}

Закрывает все вкладки, имеющее название подходяще по регулярному выражению \mintinline{icl}{url}. Возвращает количество закрытых вкладок.

Возможные исключения: \ferror{NoSessions} (см. таб. \ref{errors}).

\subsubsection{\mintinline{icl}{Tabs.closeOthers () : int}}

Закрывает все вкладки кроме текущей. Возвращает количество закрытых вкладок.

Возможные исключения: \ferror{NoSessions} (см. таб. \ref{errors}).

\subsubsection{\mintinline{icl}{Tabs.closeToLeft () : int}}

Закрывает все вкладки которые находится с лева от текущей. Возвращает количество закрытых вкладок.

Возможные исключения: \ferror{NoSessions} (см. таб. \ref{errors}).

\subsubsection{\mintinline{icl}{Tabs.closeToRight () : int}}

Закрывает все вкладки которые находится с права от текущей. Возвращает количество закрытых вкладок.

Возможные исключения: \ferror{NoSessions} (см. таб. \ref{errors}).

\subsubsection{\mintinline{icl}{Tabs.find (template : string) : Tab}}

Возвращает первая вкладка, у которой URL подходит по шаблону.

Возможные исключения: \ferror{NoSessions} (см. таб. \ref{errors}).

\subsubsection{\mintinline{icl}{Tabs.find (url : regex) : Tab}}

Возвращает первая вкладка, у которой URL подходит по регулярному выражению.

Возможные исключения: \ferror{NoSessions} (см. таб. \ref{errors}).

\subsubsection{\mintinline{icl}{Tabs.findByTitle (template : string) : Tab}}

Возвращает первая вкладка, у которой название подходит по шаблону.

Возможные исключения: \ferror{NoSessions} (см. таб. \ref{errors}).

\subsubsection{\mintinline{icl}{Tabs.findByTitle (title : regex) : Tab}}

Возвращает первая вкладка, у которой название подходит по регулярному выражению.

Возможные исключения: \ferror{NoSessions} (см. таб. \ref{errors}).

\subsection{{\color{orange} Tab}}

Объект \tab{} имеет следующие свойства:
\begin{icItems}
	\item \mintinline{icl}{[icL] [r/o] Tab'canGoBack : bool};
	\item \mintinline{icl}{[icL] [r/o] Tab'canGoForward : bool};
	\item \mintinline{icl}{[r/o] Tab'screenshot : string};
	\item \mintinline{icl}{[r/o] Tab'source : string};
	\item \mintinline{icl}{[r/*] Tab'title};
	\item \mintinline{icl}{[r/w] Tab'url}.
\end{icItems}

И следующие методы:
\begin{icItems}
	\item \mintinline{icl}{Tab.back () : void};
	\item \mintinline{icl}{Tab.close () : void};
	\item \mintinline{icl}{Tab.focus () : void};
	\item \mintinline{icl}{Tab.forward () : void};
	\item \mintinline{icl}{Tab.get (url : string) : bool};
	\item \mintinline{icl}{Tab.load (url : string) : bool}.
\end{icItems}

\subsubsection{\mintinline{icl}{[icL] [r/o] Tab'canGoBack : bool}}

Возвращает \true, если можно вернутся на предыдущую страницу, иначе \false.

\subsubsection{\mintinline{icl}{[icL] [r/o] Tab'canGoForward : bool}}

Возвращает \true, если можно вернутся на следующую страницу, иначе \false.

\subsubsection{\mintinline{icl}{[r/o] Tab'screenshot : string}}

Возвращает скриншот вкладки, кодирован в base64.

Возможные исключения: \ferror{NoSessions}, \ferror{NoSuchWindow}, \ferror{UnableToCaptureScreen} (см. таб. \ref{errors}).

\subsubsection{\mintinline{icl}{[r/o] Tab'source : string}}

Исходник страницы, открытой в указанной вкладке.

Возможные исключения: \ferror{NoSessions}, \ferror{NoSuchWindow} (см. таб. \ref{errors}).

\subsubsection{\mintinline{icl}{[r/*] Tab'title}}

Название страницы, открытой в указанной вкладке.

Возможные исключения: \ferror{NoSessions}, \ferror{NoSuchWindow} (см. таб. \ref{errors}).

\subsubsection{\mintinline{icl}{[r/w] Tab'url}}

URL страницы, открытой в указанной вкладке.

Возможные исключения: \ferror{NoSessions}, \ferror{NoSuchWindow} (см. таб. \ref{errors}).

\subsubsection{\mintinline{icl}{Tab.back () : void}}

Переходит на предыдущую страницу, как при нажатии на кнопке браузера "Назад".

Возможные исключения: \ferror{NoSessions}, \ferror{NoSuchWindow}, \ferror{Timeout} (см. таб. \ref{errors}).

\subsubsection{\mintinline{icl}{Tab.close () : void}}

Закрывает окну, если она последняя закрывается и сессию.

Возможные исключения: \ferror{NoSessions}, \ferror{NoSuchWindow} (см. таб. \ref{errors}).

\subsubsection{\mintinline{icl}{Tab.focus () : void}}

Переключает фокус на вкладку. Фокус можно менять только внутри активной сессии.

Возможные исключения: \ferror{NoSessions} (см. таб. \ref{errors}).

\subsubsection{\mintinline{icl}{Tab.forward () : void}}

Вернётся на следующую страницу, как при нажатии на кнопке браузера "Вперёд".

Возможные исключения: \ferror{NoSessions}, \ferror{NoSuchWindow}, \ferror{Timeout} (см. таб. \ref{errors}).

\subsubsection{\mintinline{icl}{Tab.get (url : string) : bool}}

Загружает страницу, URL должен быть абсолютным. Возвращает \true{} при удачной загрузке, иначе \false.

\subsubsection{\mintinline{icl}{Tab.load (url : string) : void}}

Загружает страницу, URL должен быть абсолютным. При неудаче генерирует ошибку.

Возможные исключения: \ferror{NoSessions}, \ferror{NoSuchWindow}, \ferror{InvalidArgument}, \ferror{Timeout}, \ferror{InsecureCertificate} (см. таб. \ref{errors}).

\subsection{{\color{orange} Doc}}

Объект \dom{} имеет следующие методы:
\begin{icItems}
	\item \mintinline{icl}{Doc.query (by = By'cssSelector, selector : string) : element};
	\item \mintinline{icl}{Doc.queryAll (by = By'cssSelector, selector : string) : element};
	\item \mintinline{icl}{Doc.queryAllByXPath (xpath : string) : element};
	\item \mintinline{icl}{Doc.queryByXPath (xpath : string) : element};
	\item \mintinline{icl}{Doc.queryLink (name : string, isFragment = false) : element};
	\item \mintinline{icl}{Doc.queryLinks (name : string, isFragment = false) : element};
	\item \mintinline{icl}{Doc.queryTag (name : string) : element};
	\item \mintinline{icl}{Doc.queryTags (name : string) : element}.
\end{icItems}

\subsubsection{\mintinline{icl}{Doc.query (by = By'cssSelector, selector : string) : element}}

Принимает то же самые параметры, как и \mintinline{icl}{element.query}, только поиск ведётся по всему документу.

Возможные исключения: \ferror{NoSessions}, \ferror{Timeout} (см. таб. \ref{errors}).

\subsubsection{\mintinline{icl}{Doc.queryAll (by = By'cssSelector, selector : string) : element}}

Принимает то же самые параметры, как и \mintinline{icl}{element.queryAll}, только поиск ведётся по всему документу. 

Возможные исключения: \ferror{NoSessions} (см. таб. \ref{errors}).

\subsubsection{\mintinline{icl}{Doc.queryAllByXPath (xpath : string) : element}}

Акроним для \mintinline{icl}{Doc.queryAll (By'xPath, @xpath)}.

\subsubsection{\mintinline{icl}{Doc.queryByXPath (xpath : string) : element}}

Акроним для \mintinline{icl}{Doc.query (By'xPath, @xpath)}.

\subsubsection{\mintinline{icl}{Doc.queryLink (name : string, isFragment = false) : element}}

Акроним для:
\begin{icItems}
	\item \mintinline{icl}{Doc.query (By'linkText, @name)};
	\item \mintinline{icl}{Doc.query (By'partialLinkText, @name)};
\end{icItems}

\subsubsection{\mintinline{icl}{Doc.queryLinks (name : string, isFragment = false) : element}}

Акроним для:
\begin{icItems}
	\item \mintinline{icl}{Doc.queryAll (By'linkText, @name)};
	\item \mintinline{icl}{Doc.queryAll (By'partialLinkText, @name)};
\end{icItems}

\subsubsection{\mintinline{icl}{Doc.queryTag (name : string) : element}}

Акроним для \mintinline{icl}{Doc.query (By'tagName, @name)}.

\subsubsection{\mintinline{icl}{Doc.queryTags (name : string) : element}}

Акроним для \mintinline{icl}{Doc.queryAll (By'tagName, @name)}.

\subsection{{\color{orange} Files}}

Объект \files{} имеет следующие методы:
\begin{icItems}
	\item \mintinline{icl}{Files.open (path : string) : File};
	\item \mintinline{icl}{Files.create (path : string) : File};
	\item \mintinline{icl}{Files.createDir (path : string) : void};
	\item \mintinline{icl}{Files.createPath (path : string) : void}.
\end{icItems}

\subsubsection{\mintinline{icl}{Files.open (path : string) : File}}

Открывает файл.

Возможные исключения: \ferror{FileNotFound} (см. таб. \ref{errors}).

\subsubsection{\mintinline{icl}{Files.create (path : string) : File}}

Открывает файл, если он не существует, он создаётся.

Возможные исключения: \ferror{FolderNotFound} (см. таб. \ref{errors}).

\subsubsection{\mintinline{icl}{Files.createDir (path : string) : void}}

Создаёт папку, папка в которым она создаётся должна уже существовать.

Возможные исключения: \ferror{FolderNotFound} (см. таб. \ref{errors}).

\subsubsection{\mintinline{icl}{Files.createPath (path : string) : void}}

Создаёт все несуществующие папки пути.

\subsection{{\color{orange} File}}

Объект \file{} имеет следующие свойства:
\begin{icItems}
	\item \mintinline{icl}{[r/o] File'CSV : 1};
	\item \mintinline{icl}{[r/w] File'format : int};
	\item \mintinline{icl}{[r/o] File'none : 0};
	\item \mintinline{icl}{[r/o] File'TSV : 2};
	\item \mintinline{icl}{[r/o] File'valid : bool}.
\end{icItems}

И следующие методы:
\begin{icItems}
	\item \mintinline{icl}{File.close () : void};
	\item \mintinline{icl}{File.delete () : void}.
\end{icItems}

\subsubsection{\mintinline{icl}{[r/o] File'CSV : 1}}

Формат CSV.

\subsubsection{\mintinline{icl}{[r/w] File'format : int}}

Возвращает формат файла.

\subsubsection{\mintinline{icl}{[r/o] File'none : 0}}

Не инициализированный файл.

\subsubsection{\mintinline{icl}{[r/o] File'TSV : 2}}

Формат TSV.

\subsubsection{\mintinline{icl}{[r/o] File'valid : bool}}

Возвращает \true, если файл инициализированный, иначе \false.

\subsubsection{\mintinline{icl}{File.close () : void}}

Закрывает файл.

\subsubsection{\mintinline{icl}{File.delete () : void}}

Удаляет файл.

\subsection{{\color{orange} Make}}

Объект \make{} имеет следующие метод: \mintinline{icl}{Make.image (base64 : string, path :}\\*\mintinline{icl}{string)}.

\subsubsection{\mintinline{icl}{Make.image (base64 : string, path : string) : void}}

Сохраняет скриншот на жёстком диске.

\subsection{{\color{orange} Log}}

Объект \logtype{} имеет следующие методы:
\begin{icItems}
	\item \mintinline{icl}{Log.error (message : string) : void};
	\item \mintinline{icl}{Log.info (message : string) : void};
	\item \mintinline{icl}{Log.out (args : any ...) : void};
	\item \mintinline{icl}{Log.stack (var : any) : void};
	\item \mintinline{icl}{Log.state (var : any) : void}.
\end{icItems}

\subsubsection{\mintinline{icl}{Log.error (message : string) : void}}

Выводит сообщение об ошибке.

\subsubsection{\mintinline{icl}{Log.info (message : string) : void}}

Выводит информационное сообщение.

\subsubsection{\mintinline{icl}{Log.out (args : any ...) : void}}

Выводит отладочное сообщение, принимает несколько параметров любого типа. При получении переменных выводит исходник, название перемены, тип данных и значение. При получении константных только значение. При получении результата выполнения функции - тип и значение. 

\subsubsection{\mintinline{icl}{Log.stack (var : any) : void}}

Выводит список всех стеков, указывая в каких из них встречается переменная с указанным именем, и какие значения имеются.

\subsubsection{\mintinline{icl}{Log.state (var : any) : void}}

Выводит список всех состояний, указывая в каких из них встречается переменная с указанным именем, и какие значения имеются.

\subsection{{\color{orange} Numbers}}

Объект \numbers{} имеет следующие свойства:
\begin{icItems}
	\item \mintinline{icl}{[r/o] Numbers'max : 4};
	\item \mintinline{icl}{[r/o] Numbers'min : 3};
	\item \mintinline{icl}{[r/o] Numbers'product : 2};
	\item \mintinline{icl}{[r/o] Numbers'process : int};
	\item \mintinline{icl}{[r/o] Numbers'sum : 1}.
\end{icItems}

И следующие методы:
\begin{icItems}
	\item \mintinline{icl}{Numbers.process (a : int, b : int) : int};
	\item \mintinline{icl}{Numbers.process (a : double, b : double) : double};
	\item \mintinline{icl}{Numbers.restoreProcess () : void};
	\item \mintinline{icl}{Numbers.setProcess (proc : int) : void}.
\end{icItems}

\subsubsection{\mintinline{icl}{[r/o] Numbers'max : 4}}

Выбирать максимум.

\subsubsection{\mintinline{icl}{[r/o] Numbers'min : 3}}

Выбирать минимум.

\subsubsection{\mintinline{icl}{[r/o] Numbers'product : 2}}

Умножить числа.

\subsubsection{\mintinline{icl}{[r/o] Numbers'process : int}}

Текущий способ обработки чисел.

\subsubsection{\mintinline{icl}{[r/o] Numbers'sum : 1}}

Складывать числа.

\subsubsection{\mintinline{icl}{Numbers.process (a : int, b : int) : int}}

Обработать целых чисел текущим методом.

\subsubsection{\mintinline{icl}{Numbers.process (a : double, b : double) : double}}

Обработать дробных чисел текучим методом.

\subsubsection{\mintinline{icl}{Numbers.restoreProcess () : void}}

Удаляет последняя запись стека способов обработки.

\subsubsection{\mintinline{icl}{Numbers.setProcess (proc : int) : void}}

Добавляет новая запись в стеке способов обработки.

\subsection{{\color{orange} Math}}

Объект \mintinline{icl}{Math} имеет следующие свойства:
\begin{icItems}
	\item \mintinline{icl}{[r/o] Math'1divPi : double};
	\item \mintinline{icl}{[r/o] Math'1divSqrt2 : double};
	\item \mintinline{icl}{[r/o] Math'2divPi : double};
	\item \mintinline{icl}{[r/o] Math'2divSqrtPi : double};
	\item \mintinline{icl}{[r/o] Math'e : double};
	\item \mintinline{icl}{[r/o] Math'ln2 : double};
	\item \mintinline{icl}{[r/o] Math'ln10 : double};
	\item \mintinline{icl}{[r/o] Math'log2e : double};
	\item \mintinline{icl}{[r/o] Math'log10e : double};
	\item \mintinline{icl}{[r/o] Math'pi : double};
	\item \mintinline{icl}{[r/o] Math'piDiv2 : double};
	\item \mintinline{icl}{[r/o] Math'piDiv4 : double};
	\item \mintinline{icl}{[r/o] Math'sqrt2 : double}.
\end{icItems}

И следующие методы:
\begin{icItems}
	\item \mintinline{icl}{Math.acos (v : double) : double};
	\item \mintinline{icl}{Math.asin (v : double) : double};
	\item \mintinline{icl}{Math.atan (v : double) : double};
	\item \mintinline{icl}{Math.ceil (v : double) : int};
	\item \mintinline{icl}{Math.cos (v : double) : double};
	\item \mintinline{icl}{Math.degreesToRadians (v : double) : double};
	\item \mintinline{icl}{Math.exp (v : double) : double};
	\item \mintinline{icl}{Math.floor (v : double) : int};
	\item \mintinline{icl}{Math.ln (v : double) : double};
	\item \mintinline{icl}{Math.min (arr : int ...) : int};
	\item \mintinline{icl}{Math.min (arr : double ...) : double};
	\item \mintinline{icl}{Math.max (arr : int ...) : int};
	\item \mintinline{icl}{Math.max (arr : double ...) : double};
	\item \mintinline{icl}{Math.radiansToDegrees (v : double) : double};
	\item \mintinline{icl}{Math.round (<double>) : int};
	\item \mintinline{icl}{Math.sin (v : double) : double};
	\item \mintinline{icl}{Math.tan (v : double) : double}.
\end{icItems}

\subsubsection{\mintinline{icl}{[r/o] Math'1divPi : double}}

1 делить на пи ($\frac{1}{\pi}$).

\subsubsection{\mintinline{icl}{[r/o] Math'1divSqrt2 : double}}

1 делить на корень из числа 2 ($\frac{1}{\sqrt{2}}$).

\subsubsection{\mintinline{icl}{[r/o] Math'2divPi : double}}

2 делить на пи ($\frac{2}{\pi}$).

\subsubsection{\mintinline{icl}{[r/o] Math'2divSqrtPi : double}}

2 делить на корень из числа пи ($\frac{2}{\sqrt{\pi}}$).

\subsubsection{\mintinline{icl}{[r/o] Math'e : double}}

Число ($e$).

\subsubsection{\mintinline{icl}{[r/o] Math'ln2 : double}}

Натуральный логарифм числа 2 ($\ln{2}$).

\subsubsection{\mintinline{icl}{[r/o] Math'ln10 : double}}

Натуральный логарифм числа 10 ($\ln_{10}$).

\subsubsection{\mintinline{icl}{[r/o] Math'log2e : double}}

Логарифм числа е по основанию 2 ($\log_{2}{e}$).

\subsubsection{\mintinline{icl}{[r/o] Math'log10e : double}}

Логарифм числа е по основанию 10 ($\log_{10}{e}$).

\subsubsection{\mintinline{icl}{[r/o] Math'pi : double}}

Число пи ($\pi$).

\subsubsection{\mintinline{icl}{[r/o] Math'piDiv2 : double}}

Пи по полам ($\frac{\pi}{2}$).

\subsubsection{\mintinline{icl}{[r/o] Math'piDiv4 : double}}

Пи на 4 ($\frac{\pi}{4}$).

\subsubsection{\mintinline{icl}{[r/o] Math'sqrt2 : double}}

Корень из числа 2 ($\sqrt{2}$).

\subsubsection{\mintinline{icl}{Math.acos (v : double) : double}}

Арккосинус ($\arccos{v}$).

\subsubsection{\mintinline{icl}{Math.asin (v : double) : double}}

Арксинус ($\arcsin{v}$).

\subsubsection{\mintinline{icl}{Math.atan (v : double) : double}}

Арктангенс ($\arctan{v}$).

\subsubsection{\mintinline{icl}{Math.ceil (v : double) : int}}

Наименьшее целое число больше или равна \mintinline{icl}{v}.

\subsubsection{\mintinline{icl}{Math.cos (v : double) : double}}

Косинус ($\cos{v}$).

\subsubsection{\mintinline{icl}{Math.degreesToRadians (v : double) : double}}

Преобразует градусы в радианы.

\subsubsection{\mintinline{icl}{Math.exp (v : double) : double}}

Функция экспонент ($\exp{v}$).

\subsubsection{\mintinline{icl}{Math.floor (v : double) : int}}

Наибольшее целое число меньше или равна \mintinline{icl}{v}.

\subsubsection{\mintinline{icl}{Math.ln (v : double) : double}}

Натуральный логарифм ($\ln{v}$).

\subsubsection{\mintinline{icl}{Math.min (arr : int ...) : int}}

Возвращает наименьшее целое число.

\subsubsection{\mintinline{icl}{Math.min (arr : double ...) : double}}

Возвращает наименьшее дробное число.

\subsubsection{\mintinline{icl}{Math.max (arr : int ...) : int}}

Возвращает наибольшее целое число.

\subsubsection{\mintinline{icl}{Math.max (arr : double ...) : double}}

Возвращает наибольшее дробное число.

\subsubsection{\mintinline{icl}{Math.radiansToDegrees (v : double) : double}}

Преобразует радианы в градусы.

\subsubsection{\mintinline{icl}{Math.round (<double>) : int}}

Возвращает ближайшее целое число.

\subsubsection{\mintinline{icl}{Math.sin (v : double) : double}}

Синус ($\sin{v}$).

\subsubsection{\mintinline{icl}{Math.tan (v : double) : double}}

Тангенс ($\tan{v}$).

\subsection{{\color{orange} Import}}

Объект \mintinline{icl}{Import} имеет следующие методы:
\begin{icItems}
	\item \mintinline{icl}{Import.none (data = [=], path : string) : void};
	\item \mintinline{icl}{Import.functions (data = [=], path : string) : void};
	\item \mintinline{icl}{Import.all (data = [=], path : string) : void};
	\item \mintinline{icl}{Import.run (path : string) : void}.
\end{icItems}

Объект \mintinline{icl}{data} позволяет передать данные в изолированном контексте, они там будут доступны как глобальные переменные. Это позволяет в одном файле хранить несколько версий библиотеки например, и при использовании указать какую версию загрузить.

\subsubsection{\mintinline{icl}{Import.none (data = [=], path : string) : void}}

Создаёт изолируемый контекст, в нём выполняет файл.

\subsubsection{\mintinline{icl}{Import.functions (data = [=], path : string) : void}}

Создаёт изолируемый контекст, в нём выполняет файл, потом импортирует все функции в текущем контексте.

\subsubsection{\mintinline{icl}{Import.all (data = [=], path : string) : void}}

Создаёт изолируемый контекст, в нём выполняет файл, потом импортирует все функции и глобальные переменный в текущем контексте.

\subsubsection{\mintinline{icl}{Import.run (path : string) : void}}

Выполняет файл в текущем контексте.

%\newpage
