% !TeX spellcheck = ru_RU
\section{Дата и время}

Временная метка (тип \mintinline{icl}{datetime}) не может быть описаны литералом, а только преобразованы из строки используя определённый формат.

\subsection{Операторы}

Для временной метки доступны следующие операторы первого ранга:
\begin{icItems}
	\item \mintinline{icl}{string : datetime};
	\item \mintinline{icl}{(dt : string) : (datetime, format : string)};
	\item \mintinline{icl}{datetime : string};
	\item \mintinline{icl}{datetime : (string, format : string)};
\end{icItems}

\subsubsection{\mintinline{icl}{string : datetime}}

Преобразует строку в \mintinline{icl}{datetime}, используя формат \\* \mintinline{icl}{yyyy-MM-ddTHH:mm:ss[Z|[+|-]HH:mm]}.

\subsubsection{\mintinline{icl}{(dt : string) : (datetime, format : string)}}

Преобразует строку \mintinline{icl}{dt} в \mintinline{icl}{datetime}, используя указанный формат, например: \mintinline{icl}{"21.01.2019" : (datetime, "dd.MM.yyyy")}. Список доступных заменителей:
\begin{icItems}
	\item \mintinline{icl}{d} - число;
	\item \mintinline{icl}{dd} - число с ведущем нулём;
	\item \mintinline{icl}{ddd} - сокращение названий дня;
	\item \mintinline{icl}{dddd} - полное название дня;
	\item \mintinline{icl}{M} - месяц;
	\item \mintinline{icl}{MM} - месяц с ведущем нулём;
	\item \mintinline{icl}{MMM} - сокращение названий месяца;
	\item \mintinline{icl}{MMMM} - полное название месяца;
	\item \mintinline{icl}{yy} - год как двухзначное число;
	\item \mintinline{icl}{yyyy} - год как четырёхзначное число;
	\item \mintinline{icl}{h} - час;
	\item \mintinline{icl}{hh} - час с ведущем нулём;
	\item \mintinline{icl}{m} - минута;
	\item \mintinline{icl}{mm} - минута с ведущем нулём;
	\item \mintinline{icl}{s} - секунда;
	\item \mintinline{icl}{ss} - секунда с ведущем нулём;
	\item \mintinline{icl}{ap} - am или pm;
	\item \mintinline{icl}{AP} - AP или PM;
\end{icItems}

\subsubsection{\mintinline{icl}{datetime : string}}

Преобразует дату и время в строке формата \mintinline{icl}{yyyy-MM-ddTHH:mm:ss[Z|[+|-]HH:mm]}.

\subsubsection{\mintinline{icl}{datetime : (string, format : string)}}

Пробрезует дату и время в строке с нужным форматом.

\subsection{Свойства}

Временная метка имеет следующие свойства:
\begin{icItems}
	\item \mintinline{icl}{[r/w] datetime'year : int};
	\item \mintinline{icl}{[r/w] datetime'month : int};
	\item \mintinline{icl}{[r/w] datetime'day : int};
	\item \mintinline{icl}{[r/w] datetime'hour : int};
	\item \mintinline{icl}{[r/w] datetime'minute : int};
	\item \mintinline{icl}{[r/w] datetime'second : int};
	\item \mintinline{icl}{[r/o] datetime'valid : bool}.
\end{icItems}

\subsubsection{\mintinline{icl}{[r/w] datetime'year : int}}

Год временной метки.

\subsubsection{\mintinline{icl}{[r/w] datetime'month : int}}

Месяц временной метки.

\subsubsection{\mintinline{icl}{[r/w] datetime'day : int}}

Число временной метки.

\subsubsection{\mintinline{icl}{[r/w] datetime'hour : int}}

Час временной метки.

\subsubsection{\mintinline{icl}{[r/w] datetime'minute : int}}

Минута временной метки.

\subsubsection{\mintinline{icl}{[r/w] datetime'second : int}}

Секунда временной метки.

\subsubsection{\mintinline{icl}{[r/o] datetime'valid : bool}}

\true, если временная метка правильная, иначе \false.

\subsection{Методы}

Временная метка имеет следующие методы:
\begin{icItems}
	\item \mintinline{icl}{datetime.addSecs (secs : int) : datetime};
	\item \mintinline{icl}{datetime.addDays (days : int) : datetime};
	\item \mintinline{icl}{datetime.addMonths (months : int) : datetime};
	\item \mintinline{icl}{datetime.addYears (months : int) : datetime};
	\item \mintinline{icl}{datetime.secsTo (dt : datetime) : int};
	\item \mintinline{icl}{datetime.daysTo (dt : datetime) : int};
	\item \mintinline{icl}{datetime.toUTC () : datetime};
	\item \mintinline{icl}{datetime.toTimeZone (hours : int, minutes : int) : datetime}.
\end{icItems}

\subsubsection{\mintinline{icl}{datetime.addSecs (secs : int) : datetime}}

Добавляет нужное количество секунд к времени.

\subsubsection{\mintinline{icl}{datetime.addDays (days : int) : datetime}}

Добавляет нужное количество дней к времени.

\subsubsection{\mintinline{icl}{datetime.addMonth (months : int) : datetime}}

Добавляет нужное количество месяцев к времени.

\subsubsection{\mintinline{icl}{datetime.addYears (months : int) : datetime}}

Добавляет нужное количество лет к времени.

\subsubsection{\mintinline{icl}{datetime.secsTo (dt : datetime) : int}}

Вычисляет количество секунд от первой временной метки до второй.

\subsubsection{\mintinline{icl}{datetime.daysTo (dt : datetime) : int}}

Вычисляет количество дней от первой временной метки до второй.

\subsubsection{\mintinline{icl}{datetime.toUTC () : datetime}}

Возвращает новую временную метку, представляя всемирное координированное время.

\subsubsection{\mintinline{icl}{datetime.toTimeZone (hours : int, minutes : int) : datetime}}

Возвращает новую временную метку, представляя время в нужном часовом поясе.

\subsection{Текущее время}

Для получение текущей времени предусматриваются следующие методы:
\begin{icItems}
	\item \mintinline{icl}{Datetime.current () : datetime};
	\item \mintinline{icl}{Datetime.currentUTC () : datetime}.
\end{icItems}

\subsubsection{\mintinline{icl}{Datetime.current () : datetime}}

Текущее локальное время.

\subsubsection{\mintinline{icl}{Datetime.currentUTC () : datetime}}

Текущее UTC время.
