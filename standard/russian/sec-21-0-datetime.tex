% !TeX spellcheck = ru_RU
\section{Дата и время}

Временная метка (тип \code{datetime}) не может быть описаны литералом, а только преобразованы из строки используя определённый формат.

\subsection{Операторы}

Для временной метки доступны следующие операторы первого ранга:
\begin{icItems}
	\item \lstinline|string : datetime|;
	\item \lstinline|(dt : string) : (datetime, format : string)|;
	\item \lstinline|datetime : string|;
	\item \lstinline|datetime : (string, format : string)|;
\end{icItems}

\subsubsection{\lstinline|string : datetime|}

Преобразует строку в \code{datetime}, используя формат \\* \lstinline`yyyy-MM-ddTHH:mm:ss[Z|[+|-]HH:mm]`.

\subsubsection{\lstinline|(dt : string) : (datetime, format : string)|}

Преобразует строку \code{dt} в \code{datetime}, используя указанный формат, например: \lstinline|"21.01.2019" : (datetime, "dd.MM.yyyy")|. Список доступных заменителей:
\begin{icItems}
	\item \code{d} - число;
	\item \code{dd} - число с ведущем нулём;
	\item \code{ddd} - сокращение названий дня;
	\item \code{dddd} - полное название дня;
	\item \code{M} - месяц;
	\item \code{MM} - месяц с ведущем нулём;
	\item \code{MMM} - сокращение названий месяца;
	\item \code{MMMM} - полное название месяца;
	\item \code{yy} - год как двухзначное число;
	\item \code{yyyy} - год как четырёхзначное число;
	\item \code{h} - час;
	\item \code{hh} - час с ведущем нулём;
	\item \code{m} - минута;
	\item \code{mm} - минута с ведущем нулём;
	\item \code{s} - секунда;
	\item \code{ss} - секунда с ведущем нулём;
	\item \code{ap} - am или pm;
	\item \code{AP} - AP или PM;
\end{icItems}

\subsubsection{\lstinline|datetime : string|}

Преобразует дату и время в строке формата \lstinline`yyyy-MM-ddTHH:mm:ss[Z|[+|-]HH:mm]`.

\subsubsection{\lstinline|datetime : (string, format : string)|}

Пробрезует дату и время в строке с нужным форматом.

\subsection{Свойства}

Временная метка имеет следующие свойства:
\begin{icItems}
	\item \lstinline|[r/w] datetime'year : int|;
	\item \lstinline|[r/w] datetime'month : int|;
	\item \lstinline|[r/w] datetime'day : int|;
	\item \lstinline|[r/w] datetime'hour : int|;
	\item \lstinline|[r/w] datetime'minute : int|;
	\item \lstinline|[r/w] datetime'second : int|;
	\item \lstinline|[r/o] datetime'valid : bool|.
\end{icItems}

\subsubsection{\lstinline|[r/w] datetime'year : int|}

Год временной метки.

\subsubsection{\lstinline|[r/w] datetime'month : int|};

Месяц временной метки.

\subsubsection{\lstinline|[r/w] datetime'day : int|}

Число временной метки.

\subsubsection{\lstinline|[r/w] datetime'hour : int|}

Час временной метки.

\subsubsection{\lstinline|[r/w] datetime'minute : int|}

Минута временной метки.

\subsubsection{\lstinline|[r/w] datetime'second : int|}

Секунда временной метки.

\subsubsection{\lstinline|[r/o] datetime'valid : bool|}

\true, если временная метка правильная, иначе \false.

\subsection{Методы}

Временная метка имеет следующие методы:
\begin{icItems}
	\item \lstinline|datetime.addSecs (secs : int) : datetime|;
	\item \lstinline|datetime.addDays (days : int) : datetime|;
	\item \lstinline|datetime.addMonths (months : int) : datetime|;
	\item \lstinline|datetime.addYears (months : int) : datetime|;
	\item \lstinline|datetime.secsTo (dt : datetime) : int|;
	\item \lstinline|datetime.daysTo (dt : datetime) : int|;
	\item \lstinline|datetime.toUTC () : datetime|;
	\item \lstinline|datetime.toTimeZone (hours : int, minutes : int) : datetime|.
\end{icItems}

\subsubsection{\lstinline|datetime.addSecs (secs : int) : datetime|}

Добавляет нужное количество секунд к времени.

\subsubsection{\lstinline|datetime.addDays (days : int) : datetime|}

Добавляет нужное количество дней к времени.

\subsubsection{\lstinline|datetime.addMonth (months : int) : datetime|}

Добавляет нужное количество месяцев к времени.

\subsubsection{\lstinline|datetime.addYears (months : int) : datetime|}

Добавляет нужное количество лет к времени.

\subsubsection{\lstinline|datetime.secsTo (dt : datetime) : int|}

Вычисляет количество секунд от первой временной метки до второй.

\subsubsection{\lstinline|datetime.daysTo (dt : datetime) : int|}

Вычисляет количество дней от первой временной метки до второй.

\subsubsection{\lstinline|datetime.toUTC () : datetime|}

Возвращает новую временную метку, представляя всемирное координированное время.

\subsubsection{\lstinline|datetime.toTimeZone (hours : int, minutes : int) : datetime|}

Возвращает новую временную метку, представляя время в нужном часовом поясе.

\subsection{Текущее время}

Для получение текущей времени предусматриваются следующие функций:
\begin{icItems}
	\item \lstinline|Datetime.current () : datetime|;
	\item \lstinline|Datetime.currentUTC () : datetime|.
\end{icItems}

\subsubsection{\lstinline|Datetime.current () : datetime|}

Текущее локальное время.

\subsubsection{\lstinline|Datetime.currentUTC () : datetime|}

Текущее UTC время.
