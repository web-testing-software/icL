% !TeX spellcheck = ru_RU
\section{Преобразование типов}

Для преобразования данных в icL присутствуют 4 операторы первого ранга:
\begin{icItems}
	\item \lstinline|data : type : type|;
	\item \lstinline|data :: type : bool|;
	\item \lstinline|data :* type : bool|;
	\item \lstinline|data :? type : type|;
\end{icItems}

И 1 оператор 7-ого ранга \lstinline|date :! type : type|.

\subsubsection{\lstinline|data : type|}

Преобразует данные в нужном типа данных. Доступен следующее вариации:
\begin{icItems}
	\item \lstinline|value : type|;
	\item \lstinline|(value...) : type|;
	\item \lstinline|(value...) : (type...)|;
\end{icItems}

\lstinline|value : type| - преобразует значение в нужном типа данных.

\lstinline|(value...) : type| - преобразует запакованные значения в запакованных значении нужного типа. Возвращает запакованные значения, они могут быть использованы для вызова функции, например \lstinline|func ((2.0, 3.2) : int)|.

\lstinline|(value...) : (type...)| - преобразует каждое запакованное значение в нужном типа данных, количество значений в обоих пакетов должна совпадать. Возвращает также запакованное значения.

Возможные исключения: \ferror{UnrealCast}, \ferror{ParsingFailed}, \ferror{EmptyList}, \ferror{MultiList}, \ferror{IncompatibleObject}, \ferror{IncompatibleRoot} и \ferror{ComplexField} (см. таб. \ref{errors}).

\subsubsection{\lstinline|data :: type : bool|}

Возвращает \true, если данные имеют нужный тип, иначе \false. Имеет следующие вариации:
\begin{icItems}
	\item \lstinline|value :: type|;
	\item \lstinline|(value...) :: type|;
	\item \lstinline|(value...) :: (type...)|;
\end{icItems}

\subsubsection{\lstinline|data :* type : bool|}

Возвращает \true, если данные могут быть преобразованы в нужном типе, иначе \false. Имеет следующие вариации:
\begin{icItems}
	\item \lstinline|value :* type|;
	\item \lstinline|(value...) :* type|;
	\item \lstinline|(value...) :* (type...)|;
\end{icItems}

\subsubsection{\lstinline|data :? type : type|}

Преобразует данные в нужном тип данных. Особенность данного оператора, то что он не генерирует исключений, а возвращает \void{} при неудаче.

\subsubsection{\lstinline|data :! type : type|}

\lstinline|data :! type : type| отличается от \lstinline|data : type : type| только рангом, обычное преобразование имеет ранг 1 и она выполняется в последнюю очередь, а этот оператор гарантирует немедленное преобразование.

\subsection{Возможные преобразования}

В таблице \ref{castingtable} представлены совместимость преобразовании данных. Минусом отмечены невозможные преобразования, они всегда генерируют ошибку \ferror{UnrealCast}. Плюсом отмечены преобразования, которые никогда не генерируют ошибки. Звёздочкой отмечены возможные преобразования, но они также могут генерировать ошибки. В названии столбцов указан изначальный тип данных, в названии строки указан желаемый тип данных.

\begin{table}[htb]
	\caption{Таблица преобразований}
	\label{castingtable}
	\begin{tabular}{|l|c|c|c|c|c|c|c|c|c|}
		\hline
		          & \void & \bool & \integer & \double & \str & \listtype & \object & \set & \element \\ \hline
		\void     & +     & -     & -        & -       & -    & -         & -       & -    & -        \\ \hline
		\bool     & +     & +     & +        & +       & +    & +         & +       & +    & +        \\ \hline
		\integer  & +     & +     & +        & +       & *    & -         & -       & -    & -        \\ \hline
		\double   & +     & +     & +        & +       & *    & -         & -       & -    & -        \\ \hline
		\str      & +     & +     & +        & +       & +    & *         & +       & +    & +        \\ \hline
		\listtype & +     & -     & -        & -       & +    & +         & -       & +    & -        \\ \hline
		\object   & +     & -     & -        & -       & *    & -         & +       & *    & -        \\ \hline
		\set      & +     & -     & -        & -       & *    & *         & -       & +    & -        \\ \hline
		\element  & +     & -     & -        & -       & -    & -         & -       & -    & +        \\ \hline
	\end{tabular}
\end{table}

\subsubsection{\lstinline|void : void|}

Возвращает исходный объект.

\subsubsection{\lstinline|void : bool|}

Возвращает \false.

\subsubsection{\lstinline|void : int|}

Возвращает \lstinline|0|.

\subsubsection{\lstinline|void : double|}

Возвращает \lstinline|0.0|.

\subsubsection{\lstinline|void : string|}

Возвращает \lstinline|""|.

\subsubsection{\lstinline|void : list|}

Возвращает \lstinline|[]|.

\subsubsection{\lstinline|void : object|}

Возвращает \lstinline|[=]|.

\subsubsection{\lstinline|void : set|}

Возвращает \lstinline|[:]|.

\subsubsection{\lstinline|void : element|}

Возвращает пустую коллекцию.

\subsubsection{\lstinline|bool : bool|}

Возвращает исходный объект.

\subsubsection{\lstinline|bool : int|}

Возвращает \lstinline|1|, если логическое значение - единица, иначе \lstinline|0|.

\subsubsection{\lstinline|bool : double|}

Возвращает \lstinline|1.0|, если логическое значение - единица, иначе \lstinline|0.0|.

\subsubsection{\lstinline|bool : string|}

Возвращает \lstinline|"true"|, если логическое значение - единица, иначе \lstinline|"false"|.

\subsubsection{\lstinline|int : bool|}

Возвращает значение выражений \lstinline|int != 0|.

\subsubsection{\lstinline|int : int|}

Возвращает исходный объект.

\subsubsection{\lstinline|int : double|}

Возвращает вещественное число, имеющее целая часть равна \integer, а дробная равна нулю.

\subsubsection{\lstinline|int : string|}

Возвращает строку, содержащая строковое представление числа \integer.

\subsubsection{\lstinline|double : bool|}

Возвращает значение выражений \lstinline|double != 0.0|.

\subsubsection{\lstinline|double : int|}

Возвращает целое число, равна целой части числа \double. Дробная часть пропадёт.

\subsubsection{\lstinline|double : double|}

Возвращает исходный объект.

\subsubsection{\lstinline|double : string|}

Возвращает строку, содержащая строковое представление числа \double.

\subsubsection{\lstinline|string : bool|}

Возвращает значение выражений \lstinline|!string'empty|.

\subsubsection{\lstinline|string : int|}

Возвращает целое число, являющейся результат парсинга строки \str.

Возможные исключения: \ferror{ParsingFailed} (см. таб. \ref{errors}).

\subsubsection{\lstinline|string : double|}

Возвращает вещественное число, являющейся результат парсинга строки \str. Целая часть от дробной делить надо точкой, а не запятой.

Возможные исключения: \ferror{ParsingFailed} (см. таб. \ref{errors}).

\subsubsection{\lstinline|string : string|}

Возвращает исходный объект.

\subsubsection{\lstinline|string : list|}

Возвращает \lstinline|[string]|.

\subsubsection{\lstinline|string : object|}

Возвращает объект, являющийся результат парсинга JSON-строки \str. JSON-объект должен иметь поля только следующих типов: \bool, \integer, \double, \str{} и \listtype.

Возможные исключения: \ferror{ParsingFailed}, \ferror{IncompatibleRoot} и \ferror{ComplexField} (см. таб. \ref{errors}).

\subsubsection{\lstinline|string : set|}

Возвращает множество, являющийся результат парсинга JSON-строки \str, которая должна содержать массив объект, а каждый объект должен соответствовать критерия оператора \lstinline|string : object|.

Возможные исключения: \ferror{ParsingFailed}, \ferror{IncompatibleObject}, \ferror{IncompatibleRoot} и \ferror{ComplexField} (см. таб. \ref{errors}).

\subsubsection{\lstinline|list : bool|}

Возвращает значение выражений \lstinline|!list'empty|.

\subsubsection{\lstinline|list : string|}

Возвращает первую строку списка, если список состоит только из одной строки, иначе генерирует исключение.

Возможные исключения: \ferror{EmptyList} и \ferror{MultiList} (см. таб. \ref{errors}).

\subsubsection{\lstinline|list : list|}

Возвращает исходный объект.

\subsubsection{\lstinline|list : set|}

Каждая строка списка будет преобразовано в объекте оператором \lstinline|string : object|, получены объекты будут группированы в множестве, полученная множество будет возвращена оператором.

\subsubsection{\lstinline|object : bool|}

Возвращает \true, если объект хранит хотя бы одну переменную, иначе \false.

\subsubsection{\lstinline|object : string|}

Возвращает JSON-строку, описывающая объект.

\subsubsection{\lstinline|object : object|}

Возвращает исходный объект.

\subsubsection{\lstinline|set : bool|}

Возвращает значение выражений \lstinline|!set'empty|.

\subsubsection{\lstinline|set : string|}

Возвращает JSON-строку, описывающая множеству.

\subsubsection{\lstinline|set : object|}

Возвращает единственный объект, если множество содержит только один объект, иначе генерирует исключение.

Возможные исключения: \ferror{EmptySet} и \ferror{MultiSet} (см. таб. \ref{errors}).

\subsubsection{\lstinline|set : list|}

Возвращает список JSON-строк, где каждая строка описывает объект множества.

\subsubsection{\lstinline|set : set|}

Возвращает исходный объект.

\subsubsection{\lstinline|element : bool|}

Возвращает значение выражений \lstinline|!element'empty|.

\subsubsection{\lstinline|element : string|}

Возвращает строка, описывающая последовательность действий, при которых была получена данная коллекция.

\subsubsection{\lstinline|element : element|}

Возвращает исходный объект.
