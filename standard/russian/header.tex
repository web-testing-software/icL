 %\renewcommand{\rmdefault}{ftm}
\counterwithin{lstlisting}{section}
\counterwithin{table}{section}

\renewcommand{\lstlistingname}{Листинг}

\setlength\abovecaptionskip{2pt}
\setlength\belowcaptionskip{1pt}

\pagestyle{fancy}
\fancyhead[]{}
\fancyhead[L]{\textmd{Релиз-кандидат}}
\fancyhead[C]{}
\fancyhead[R]{\textmd{\thepage{} из \pageref{LastPage}}}
\fancyfoot[]{}
\renewcommand{\headrulewidth}{0pt}

\lstset{
	language=icL,
	extendedchars=true,
	basicstyle=\footnotesize\ttfamily,
	showstringspaces=false,
	showspaces=false,
	showtabs=false,
	numbers=left,
	stepnumber=1,
	tabsize=4,
	breaklines=true,
	breakatwhitespace=true,
	backgroundcolor=\color{codebg},
	style=framed,
	lineskip=0pt,
	aboveskip=0pt,
	autogobble=true,
	frame=l,
	framesep=6mm,
	framexleftmargin=2.5mm,
	fillcolor=\color{codeheaderbg},
	rulecolor=\color{blue}
}

\setlength{\LTleft}{0pt}
\onehalfspacing

\newcommand{\greycell}[1]{\cellcolor{lightgray}\centering\textbf{#1}}

\newcommand{\code}[1]{ \lstinline|#1| }
\newcommand{\true}{\lstinline|true|}
\newcommand{\false}{\lstinline|false|}
\newcommand{\bool}{\lstinline|bool|}
\newcommand{\integer}{\lstinline|int|}
\newcommand{\double}{\lstinline|double|}
\newcommand{\str}{\lstinline|string|}
\newcommand{\listtype}{\lstinline|list|}
\newcommand{\element}{\lstinline|element|}
\newcommand{\set}{\lstinline|set|}
\newcommand{\setitem}{\lstinline|item|}
\newcommand{\object}{\lstinline|object|}
\newcommand{\function}{\lstinline|function|}
\newcommand{\void}{\lstinline|void|}
\newcommand{\request}{\lstinline|request|}
\newcommand{\chartype}{\lstinline|char|}
\newcommand{\regex}{\lstinline|regex|}
% \newcommand{}{ \lstinline|| }

\newcommand{\sessions}{\lstinline|Sessions|}
\newcommand{\session}{\lstinline|Session|}
\newcommand{\windows}{\lstinline|Windows|}
\newcommand{\window}{\lstinline|Window|}
\newcommand{\cookies}{\lstinline|Cookies|}
\newcommand{\cookie}{\lstinline|Cookie|}
\newcommand{\alert}{\lstinline|Alert|}
\newcommand{\tabs}{\lstinline|Tabs|}
\newcommand{\tab}{\lstinline|Tab|}
\newcommand{\dom}{\lstinline|DOM|}
\newcommand{\files}{\lstinline|Files|}
\newcommand{\file}{\lstinline|File|}
\newcommand{\make}{\lstinline|Make|}
\newcommand{\logtype}{\lstinline|Log|}
\newcommand{\numbers}{\lstinline|Numbers|}

% row1 width, row2 width, label, name,
% row1 name, row2 name, body
\newcommand{\tabletwo}[7]{
	\begin{longtable}[h]{|p{#1}|p{#2}|}
	\caption*{Таблица \thetable{}: #4} \label{#3} \\

	\hline
	\multicolumn{1}{|p{#1}|}{\greycell{#5}} &
	\multicolumn{1}{p{#2}|}{\greycell{#6}} \\
	\hline
	\multicolumn{1}{|p{#1}|}{\greycell{1}} &
	\multicolumn{1}{p{#2}|}{\greycell{2}} \\
	\hline
	\endfirsthead

	\multicolumn{2}{l}%
	{{Продолжение таблицы \thetable{}: #4}} \\
	\hline
	\multicolumn{1}{|p{#1}|}{\greycell{1}} &
	\multicolumn{1}{p{#2}|}{\greycell{2}} \\
	\hline
	\endhead

	\hline
	\endfoot

	\hline
	\endlastfoot
#7
	\end{longtable}
}

% row1 width, row2 width, label, name,
% row1 name, row2 name, body
\newcommand{\stabletwo}[7]{
	\begin{longtable}[h]{|p{#1}|p{#2}|}
	\caption*{Таблица \thetable{}: #4} \label{#3} \\

	\hline
	\multicolumn{1}{|p{#1}|}{\greycell{#5}} &
	\multicolumn{1}{p{#2}|}{\greycell{#6}} \\
	\hline
	\endfirsthead

	\multicolumn{2}{l}%
	{{Продолжение таблицы \thetable{}: #4}} \\
	\hline
	\multicolumn{1}{|p{#1}|}{\greycell{#5}} &
	\multicolumn{1}{p{#2}|}{\greycell{#6}} \\
	\hline
	\endhead

	\hline
	\endfoot

	\hline
	\endlastfoot
#7
	\end{longtable}
}

% row1 width, row2 width, row3 width, label, name,
% row1 name, row2 name, row3 name, body
\newcommand{\tablethree}[9]{
	\begin{longtable}[h]{|p{#1}|p{#2}|p{#3}|}
	\caption*{Таблица \thetable{}: #5} \label{#4} \\

	\hline
	\multicolumn{1}{|p{#1}|}{\greycell{#6}} &
	\multicolumn{1}{p{#2}|}{\greycell{#7}} &
	\multicolumn{1}{p{#3}|}{\greycell{#8}} \\
	\hline
	\multicolumn{1}{|p{#1}|}{\greycell{1}} &
	\multicolumn{1}{p{#2}|}{\greycell{2}} &
	\multicolumn{1}{p{#3}|}{\greycell{3}} \\
	\hline
	\endfirsthead

	\multicolumn{3}{l}%
	{{Продолжение таблицы \thetable{}: #5}} \\
	\hline
	\multicolumn{1}{|p{#1}|}{\greycell{1}} &
	\multicolumn{1}{p{#2}|}{\greycell{2}} &
	\multicolumn{1}{p{#3}|}{\greycell{3}} \\
	\hline
	\endhead

	\hline
	\endfoot

	\hline
	\endlastfoot
#9
	\end{longtable}
}

% row1 width, row2 width, row3 width, label, name,
% row1 name, row2 name, row3 name, body
\newcommand{\stablethree}[9]{
	\begin{longtable}[h]{|p{#1}|p{#2}|p{#3}|}
	\caption*{Таблица \thetable{}: #5} \label{#4} \\

	\hline
	\multicolumn{1}{|p{#1}|}{\greycell{#6}} &
	\multicolumn{1}{p{#2}|}{\greycell{#7}} &
	\multicolumn{1}{p{#3}|}{\greycell{#8}} \\
	\hline
	\endfirsthead

	\multicolumn{3}{l}%
	{{Продолжение таблицы \thetable{}: #5}} \\
	\hline
	\multicolumn{1}{|p{#1}|}{\greycell{#6}} &
	\multicolumn{1}{p{#2}|}{\greycell{#7}} &
	\multicolumn{1}{p{#3}|}{\greycell{#8}} \\
	\hline
	\endhead

	\hline
	\endfoot

	\hline
	\endlastfoot
#9
	\end{longtable}
}

% rows, header, repeated, label, name, body
\newcommand{\tableuni}[6]{
	\begin{longtable}[h]{#1}
	\caption*{Таблица \thetable{}: #5} \label{#4} \\

	\hline
	#2
	#3
	\endfirsthead

	\multicolumn{3}{l}%
	{{Продолжение таблицы \thetable{}: #5}} \\
	#3
	\endhead

	\hline
	\endfoot

	\hline
	\endlastfoot
#6
	\end{longtable}
}

% rows, header, repeated, label, name, body
\newcommand{\stableuni}[6]{
	\begin{longtable}[h]{#1}
	\caption*{Таблица \thetable{}: #4} \label{#3} \\

	\hline
	#2
	\hline
	\endfirsthead

	\multicolumn{3}{l}%
	{{Продолжение таблицы \thetable{}: #4}} \\
	\hline
	#2
	\hline
	\endhead

	\hline
	\endfoot

	\hline
	\endlastfoot
#5
	\end{longtable}
}

