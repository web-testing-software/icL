% !TeX spellcheck = ru_RU
\section{Механизм состояний}

Все переменные icL хранятся в контейнерах 2 типов: стек и состояние.

Стек представляет собой стек контейнеров, каждый контейнер строго привязан к фрагменту кода icL, при вводе курсора в область контейнер создаётся, при выходе разрушается.

Состояние представляет собой двусвязный список контейнеров, они не привязываются к коду и доступны глобально. По умолчанию создаётся лишь один контейнер, но новые могут быть созданы пользователем.

\subsection{Управление стеками}

Для управления стеками, используется следующий набор функций и свойств:
\begin{icItems}
	\item \lstinline|[r/o] _stacks'(<string>name) : _stack|;
	\item \lstinline|_stacks.loop <int>i = 0 : _stack|;
	\item \lstinline|_stacks.while <int>i = 0 : _stack|;
	\item \lstinline|_stacks.dowhile <int>i = 0 : _stack|;
	\item \lstinline|_stacks.for <int>i = 0 : _stack|;
	\item \lstinline|_stacks.forany <int>i = 0 : |;
	\item \lstinline|_stacks.range <int>i = 0 : _stack|;
	\item \lstinline|_stacks.filter <int>i = 0 : _stack|;
	\item \lstinline|_stack.clear : void|;
	\item \lstinline|_stack.continue : void|;
	\item \lstinline|_stack.break : void|.
\end{icItems}

\subsubsection{[r/o] \lstinline|_stacks'(<string>name) : _stack|}

Возвращает контейнер имеющий имя \code{name}. Наименование контейнера имеет следующий синтаксис -
\begin{lstlisting}[numbers=none]
{:name
	`` content
}
\end{lstlisting}
Могут быть именованными только блоки команд (код ограничен круглыми скобками). Код ограничен круглыми скобками не может иметь название. Вызов функций создаёт контейнер имеющий название функций.

\subsubsection{\lstinline|_stacks.loop <int>i = 0 : _stack|}

Возвращает \code{i}-й \code{while}, \code{dowhile}, \code{for}, \code{range} или \code{filter} контейнер.

Возможные исключения: \ferror{OutOfBounds}.

\subsubsection{\lstinline|_stacks.while <int>i = 0 : _stack|}

Возвращает \code{i}-й \code{while} контейнер.

Возможные исключения: \ferror{OutOfBounds}.

\subsubsection{\lstinline|_stacks.dowhile <int>i = 0 : _stack|}

Возвращает \code{i}-й \code{do while} контейнер.

Возможные исключения: \ferror{OutOfBounds}.

\subsubsection{\lstinline|_stacks.for <int>i = 0 : _stack|}

Возвращает \code{i}-й \code{for} контейнер.

Возможные исключения: \ferror{OutOfBounds}.

\subsubsection{\lstinline|_stacks.forany <int>i = 0 : |}

Возвращает \code{i}-й \code{for any} контейнер.

Возможные исключения: \ferror{OutOfBounds}.

\subsubsection{\lstinline|_stacks.range <int>i = 0 : _stack|}

Возвращает \code{i}-й \code{range} контейнер.

Возможные исключения: \ferror{OutOfBounds}.

\subsubsection{\lstinline|_stacks.filter <int>i = 0 : _stack|}

Возвращает \code{i}-й \code{filter} контейнер.

Возможные исключения: \ferror{OutOfBounds}.

\subsubsection{\lstinline|_stack'(<string>name) : any|}

Возвращает переменную, находящаяся именно в этом контейнере. Например: уже существует переменная \code{i} в родительском контейнере, нужно определить переменная с таким же именем в текущем контексте. Проблема решается следующим образом: \lstinline|_stack'i = 0| вместо \lstinline|@i = 0|.

\subsubsection{\lstinline|_stack.clear : void|}

Удаляет все переменные, которые хранится в контейнере.

\subsubsection{\lstinline|_stack.continue : void|}

Переход на следующую итерацию, доступно только для циклов.

Возможные исключения: \ferror{InvalidArgument}.

\subsubsection{\lstinline|_stack.break : void|}

Прекращает немедленно работу цикла или остановит выполнение функций. Доступно только для циклов и функций.

Возможные исключения: \ferror{InvalidArgument}.

\subsection{Управление состояниями}

Для управления состояниями, используется следующий набор функций и свойств:
\begin{icItems}
	\item \lstinline|_states.new : void|;
	\item \lstinline|_states.newAtEnd : void|;
	\item \lstinline|_states.toNext : void|;
	\item \lstinline|_states.toPrev : void|;
	\item \lstinline|_states.clear : void|;
	\item \lstinline|_states.delete : void|;
	\item \lstinline|_states.toFirst : void|;
	\item \lstinline|_states.toLast : void|.
\end{icItems}

Объект \lstinline|_states| является коллекцией, а значит её можно посматривать через \lstinline|for (_states) {}|. Поддержка других циклов не предусматриваться.

\subsubsection{\lstinline|_states.new <object>data = [<>] : void|}

Создаёт новое состояние после текущей и переключает фокус на ней. Все данные объекта \code{data} добавляются в новом контейнере.

\subsubsection{\lstinline|_states.newAtEnd <object>data = [<>] : void|}

Создаёт новое состояние в конце, не переключает фокус на ней. Все данные объекта \code{data} добавляются в новом контейнере.

\subsubsection{\lstinline|_states.toNext : bool|}

Возвращает \true, если существует следующее состояние и фокус был перемещён на неё, иначе \false.

\subsubsection{\lstinline|_states.toPrev : bool|}

Возвращает \true, если существует предыдущее состояние и фокус был перемещён на неё, иначе \false.

\subsubsection{\lstinline|_states.clear : void|}

Удаляет все переменные, которые хранится в контейнере.

\subsubsection{\lstinline|_states.delete : void|}

Удаляет состояние, если существует следующее то фокус переходит на ней, иначе если существует предыдущее то фокус переходит на ней, в противном случае текущее состояние - последняя и она не удаляется, а только очищается.

\subsubsection{\lstinline|_states.toFirst : void|}

Переходит на первое состояние.

\subsubsection{\lstinline|_states.toLast : void|}

Переходит на последнее состояние.

\subsubsection{\lstinline|for (_states) \{\}|}

Выполняет тело цикла для каждой состояний.

%\newpage
