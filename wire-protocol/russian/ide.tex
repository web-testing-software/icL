% !TeX spellcheck = ru_RU
\section{Среда разработки}

В этом главе рассматриваются запросы которые может генерировать среда разработки при обращении к серверу (Forwards). Также и запросы которые среда разработки может получить от сервера (Backwards).

\subsection{Forwards}

Список запросов к серверу:
\begin{icItems}
	\item \jsinline{open-session};
	\item \jsinline{close-session};
	\item \jsinline{create-project};
	\item \jsinline{r-w-open-project};
	\item \jsinline{r-o-open-project};
	\item \jsinline{release-project};
	\item \jsinline{get-session};
	\item \jsinline{get-project};
	\item \jsinline{get-file};
	\item \jsinline{update-project};
	\item \jsinline{draft-project};
	\item \jsinline{upload-file};
	\item \jsinline{end-transaction};
	\item \jsinline{load-draft};
	\item \jsinline{get-versions};
	\item \jsinline{get-version};
	\item \jsinline{remove-project}.
\end{icItems}

\subsubsection{\jsinline{open-session}}

Открывает сессию, пользователь получает доступ к проектам внутри сессий.

Принимает только параметр \jsinline{name} (строка) — название сессий.

\jsinline{payload} ответа содержит поле \jsinline{id} — идентификатор сессии.

\subsubsection{\jsinline{close-session}}

Закрывает сессия, отключает доступ пользователя к проектам сессий без закрытия сокета.

Принимает параметр \jsinline{id} (целое число) — идентификатор сессий.

\subsubsection{\jsinline{create-project}}

Создаёт проект на сервере и возвращает его.

Принимает один параметр \jsinline{name} (строка) — название проекта.

Объект проекта имеет следующие поля:

\begin{icItems}
	\item \jsinline{name} (строка) — название проекта;
	\item \jsinline{main} (строка) — полный путь к выполняемую файла, включая имя сессий, проекта, версия и название;
	\item \jsinline{lib} — список строк, содержащий полный путь к библиотекам на сервере;
	\item \jsinline{res} — список строк, содержащий полный путь к ресурсам на сервере.
\end{icItems}

\subsubsection{\jsinline{r-w-open-project}}

Если проект не занят, то сервер его резервирует и возвращает объект проекта.

Принимает параметр \jsinline{id} (целое число) — идентификатор проекта.

\subsubsection{\jsinline{r-o-open-project}}

Возвращает объект проекта.

Принимает параметр \jsinline{id} (целое число) — идентификатор проекта.

\subsubsection{\jsinline{release-project}}

Удаляет бронь с проекта, делая его доступным для редактирования другими разработчиками.

Принимает параметр \jsinline{id} (целое число) — идентификатор проекта.

\subsubsection{\jsinline{get-session}}

Получает список проектов сессий, в виде списка строк как поле \jsinline{projects} полезных данных ответа.

Принимает параметр \jsinline{id} (целое число) — идентификатор сессии.

\subsubsection{\jsinline{get-project}}

Получает объект проекта, включая его версию, использоваться будет при выполнении теста для проверки обновления зависимостях.

Принимает параметр \jsinline{id} (целое число) — идентификатор проекта.

\subsubsection{\jsinline{get-file}}

Получения содержимого файла.

Имеет параметр \jsinline{path} (строка) — полный путь файла на сервере.
 
\jsinline{payload} ответа имеет следующее поля:

\begin{icItems}
	\item \jsinline{type} (строка): \jsinline{"text"} — текстовый файл, \jsinline{"base65"} — бинарный файл зашифрован в base65;
	\item \jsinline{content} (строка) — содержимое файла.
\end{icItems}

\subsubsection{\jsinline{update-project}}

Начинает транзакцию загрузки проекта на сервер.

Принимает параметр \jsinline{id} (целое число) — идентификатор проекта.

\subsubsection{\jsinline{draft-project}}

Начинает транзакция загрузки черновика на сервер.

Принимает параметр \jsinline{id} (целое число) — идентификатор проекта.

\subsubsection{\jsinline{upload-file}}

Загружает файл на сервер (нужна открытая транзакция). 

Принимает следующее параметры:

\begin{icItems}
	\item \jsinline{path} (строка) — полный локальный путь;
	\item \jsinline{type} (строка) — тип файла \jsinline{"text"} или \jsinline{"base65"};
	\item \jsinline{content} (строка) — содержимое файла.
\end{icItems}

\subsubsection{\jsinline{end-transaction}}

Заканчивает транзакцию.

Принимает параметр \jsinline{project} (целое число) — идентификатор проекта.

\subsubsection{\jsinline{cancel-transaction}}

Отменяет транзакцию.

Принимает параметр \jsinline{project} (целое число) — идентификатор проекта.

\subsubsection{\jsinline{load-draft}}

Получает объект проекта, содержащий ссылки на черновые файлы. 

Принимает параметр \jsinline{project} (целое число) — идентификатор проекта.

\subsubsection{\jsinline{get-versions}}

Получает список версий проекта.

Принимает параметр \jsinline{project} (целое число) — идентификатор проекта.

\subsubsection{\jsinline{get-version}}

Получает объект проекта, содержащий ссылки на определённую версию проекта.

Принимает следующие параметры:

\begin{icItems}
	\item \jsinline{project} (целое число) — идентификатор проекта;
	\item \jsinline{version} (целое число) — нужная версия.
\end{icItems}

\subsubsection{\jsinline{remove-project}}

Принимает параметр \jsinline{id} (целое число) — идентификатор проекта.

Переводит проект в состояния недоступности. Ничего не удаляет из базу данных или с файловой системы, только менеджер имеет право удалить проект полностью, а также восстановить их после удаления разработчиком.

\subsection{Backwards}

Список запросов от сервера:
\begin{icItems}
	\item \jsinline{mark-busy};
	\item \jsinline{mark-unbusy};
	\item \jsinline{add-project};
	\item \jsinline{update-project};
	\item \jsinline{drop-project}.
\end{icItems}

\subsubsection{\jsinline{mark-busy}}

Сервер указывает отметить проект как зарезервированный.

Получает аргумент \jsinline{project} (целое число) — идентификатор проекта.

\subsubsection{\jsinline{mark-unbusy}}

Сервер указывает отметить проект как незарезервированный.

Получает аргумент \jsinline{project} (целое число) — идентификатор проекта.

\subsubsection{\jsinline{add-project}}

Сервер указывает добавить проект в сессии.

Получает аргумент \jsinline{project} (целое число) — идентификатор проекта.

\subsubsection{\jsinline{update-project}}

Сервер указывает обновить проект.

Получает аргумент \jsinline{project} (целое число) — идентификатор проекта.

\subsubsection{\jsinline{drop-project}}

Сервер указывает удалить проект из сессии.

Получает аргумент \jsinline{project} (целое число) — идентификатор проекта.
