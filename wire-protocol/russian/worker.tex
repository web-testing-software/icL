% !TeX spellcheck = ru_RU
\section{Исполнитель}

В этом главе рассматриваются запросы, которых может генерировать исполнитель при обращении к серверу (Forwards). Также и запросы которые исполнитель может получить от сервера (Backwards).

\section{Forwards}

Список запросов к серверу:

\begin{icItems}
	\item \jsinline{get-next};
	\item \jsinline{mark}.
\end{icItems}

Получения содержимое проекта реализуется через запросы, описанные в протоколе среды разработки.

\subsubsection{\jsinline{get-next}}

Просит получить следующую задачу.

Принимает параметры:
\begin{icItems}
	\item \jsinline{system} (строка) — тип операционной системы: \jsinline{"unix"}, \jsinline{"windows"} или \jsinline{"mac"};
	\item \jsinline{browsers} (список строк) — список настроенных браузеров.
\end{icItems}

\jsinline{payload} ответа содержит следующее поля;
\begin{icItems}
	\item \jsinline{project} (число) — идентификатор проекта;
	\item \jsinline{browser} (строка) — идентификатор браузера;
	\item \jsinline{task-id} (число) — идентификатор задачи.
\end{icItems}

\subsubsection{\jsinline{mark}}

Отмечает результат работы.

Принимает параметры:
\begin{icItems}
	\item \jsinline{task-id} (число) — идентификатор задачи;
	\item \jsinline{status} (строка) — статус задачи:
	\begin{icItems}
		\item \jsinline{"successful"} — пройден удачно;
		\item \jsinline{"test-failed"} — тест упал из-за ошибки в коде;
		\item \jsinline{"assert-failed"} — тест упал при проверке утверждений;
		\item \jsinline{"unknown"} — неизвестный результат;
	\end{icItems}
	\item \jsinline{log} (строка) — лог;
	\item \jsinline{error-log} (строка) — лог ошибок;
	\item \jsinline{steps} (массив) — массив шагов, каждый шаг имеет:
	\begin{icItems}
		\item \jsinline{name} (строка) — название;
		\item \jsinline{description} (строка) — описание;
		\item \jsinline{status} (логическое значения) — если \jsinline{true} то пройден успешно, иначе \jsinline{false};
		\item \jsinline{steps} (массив) — перечень подшагов.
	\end{icItems}
\end{icItems}

\subsection{Backward}

Исполнитель ожидает только одну команду от сервера \jsinline{start-working}, не имеющая параметры и не требует ответа. Таким образом сервер сообщает исполнителю что пора работать.
