% !TeX spellcheck = ru_RU
\section{Базы данных}

Клиент базы данных может общается с icL Share через протокол \textit{WebSockets} и \textit{HTTP}.

\subsection{Базы данных через WebSockets}

Клиент является среда разработки или исполнитель, который выполняет скрипт icL. Сервер является ваша собственная реализация сервиса базы данных.

При использовании протокола WebSockets клиент будет просить стандартную аутентификацию icL, описана в первом главе, также будет указан роль, по которому можно настроить данные, на пример для среды разработки возвращать только первую тестовую запись, а при выполнении на исполнителе все.

Клиент к серверу оформляет только один запрос \jsinline{run-sql}, имеющий один параметр \jsinline{sql} — запрос описан на языке SQL.

\jsinline{payload} ответа должен представлять собой объект имеющий следующие поля:

\begin{icItems}
	\item \jsinline{head} (массив строк) — заголовок таблицы;
	\item \jsinline{body} (массив объектов) — каждый объект должен содержать все данные записи в соответствия с заголовком.
\end{icItems}

\subsection{Базы данных через HTTP}

В этом случае сервер не нуждается в идентификации.

REST API должен поддерживать следующий метод:

\mint{icl}{POST /run-sql}

Тело запроса является JSON-объектом и содержит следующее поля:

\begin{icItems}
	\item \jsinline{role} — роль клиента, такой же как и при аутентификации;
	\item \jsinline{sql} — запрос описан на языке SQL.
\end{icItems}

Тело ответа должна быть JSON-объектом и содержать следующее поля:

\begin{icItems}
	\item \jsinline{head} (массив строк) - заголовок таблицы;
	\item \jsinline{body} (массив объектов) - каждый объект должен содержать все данные записи в соответствия с заголовком.
\end{icItems}
